%!TEX root=main.tex

\section{Jeux d'atteignabilité}
\label{sect:jeuxAtt}

Un jeu d'atteignabilité est un jeu sur graphe particulier. Chaque joueur possède un ensemble objectif (un sous ensemble de sommets du graphe) qu'il souhaite atteindre. Le déroulement d'une partie d'un jeu d'atteignabilité se déroule comme un jeu sur graphe (cf \ref{derPar}) sauf que le but de chaque joueur est d'atteindre un élément de son ensemble objectif. On peut considérer les jeux d'atteignabilité selon deux points de vue: les jeux qualitatifs et les jeux quantitatifs. Nous développons dans les sections suivantes ces deux notions.


%!TEX root=main.tex


\subsection{Jeux qualitatifs}

%DEFINITION: jeu d'atteignabilité à objectif qualitatif

Nous nous attardons dans un premier temps sur les \emph{jeux d'atteignabilité à objectif qualitatif}. Ce sont des jeux sur graphe tels que chaque joueur tente d'atteindre un élément de son ensemble objectif. Nous nous intéressons alors à savoir si à partir d'un noeud du graphe un joueur peut toujours s'assurer d'atteindre un élément de son ensemble objectif et ce quelle que soit les stratégies adoptées par les autres joueurs. La réponse à cette question est alors binaire: soit oui, soit non. Dans le cas où celle-ci est positive, nous disons que ce joueur possède une \emph{stratégie gagnante} et que ce noeud de départ est un \emph{état gagnant}.
	
	\begin{defi}[Jeu d'atteignabilité à objectif qualitatif]
		Un \textit{jeu d'atteignabilité à objectif qualitatif} est un jeu sur graphe $\mathcal{G} = (\mathcal{A}, (Goal_{i})_{i \in \Pi},(\Omega _{i})_{i \in \Pi})$ où :
		\begin{enumerate}
			\item[$\bullet$] $\mathcal{A} = (\Pi,V,(V_{i})_{i \in \Pi}, E)$ est l'arène d'un jeu sur graphe,
			\item[$\bullet$] Pour tout $i \in \Pi$, $Goal_{i} \subseteq V $ est l'ensemble des sommets de $V$ que $J_{i}$ essaie d'atteindre,
			\item[$\bullet$] Pour tout $i \in \Pi$, $\Omega _{i} = \{(u_{j})_{j \in \mathbb{N}}\in V^{\omega}| \exists k \in \mathbb{N}$  tel que $u_{k}\in Goal_{i}\}$. C'est l'ensemble des jeux $\rho$ sur $\mathcal{G}$ pour lesquels $J_{i}$ atteint son objectif.
		\end{enumerate}	
	\end{defi}
	
% DEFINITION: stratégie gagnante
	\label{strategieGagnante}
	\begin{defi}[Stratégie gagnante]
		Soit $v \in V$, soit $\sigma _{i}$ une stratégie du joueur $i$, on dit que $\sigma _{i}$ est \textit{gagnante pour $J_{i}$} à partir de $v$ si 
		$$ \forall \sigma_{-i}, \, \langle \sigma_i, \sigma_{-i} \rangle_v \in \Omega_i.$$
	\end{defi}
	
% DEFINITION: ensemble des états gagnants		
	
	\begin{defi}
		Soit $\mathcal{G} = (\Pi,(V,E),(V_{i})_{i \in \Pi}, (Goal_{i})_{i \in \Pi},(\Omega _{i})_{i \in \Pi})$,\\
		$W_{i} = \{ v_{j} |v_j \in V$ et $\exists \sigma _{i}$ une stratégie gagnantes pour $J_{i}$ à partir de $v_{j}\}$ est \textit{l'ensemble des états gagnants} de $J_{i}$. C'est l'ensemble des sommets de $\mathcal{G}$ à partir desquels $J_{i}$ est assuré d'atteindre son objectif.
	\end{defi}
	
	
	
	Une fois le concept de jeu d'atteignabilité clairement établi, nous pouvons nous poser les questions suivantes : \og Quels joueurs peuvent gagner le jeu?\fg~et \og Quelle stratégie doivent adopter les joueurs pour atteindre leur objectif quelle que soit la stratégie jouée par les autres joueurs?\fg~. \\
	
Dans le cadre de ce travail nous abordons uniquement le cas des jeux qualitatifs à deux joueurs et à sommes nulles.
	
%-------------------------------------
%Cas des jeux à deux joueurs et à somme nulle
%-------------------------------------
	
	\subsubsection*{Jeux à deux joueurs et à somme nulle}
	Nous sommes intéressés par l'étude des jeux d'atteignabilité à objectif qualitatif dans le cadre des jeux à deux joueurs. Dans ce cadre, nous notons $\Pi = \{ 1,2\}$. Nous supposons que $\Omega _{2} = V^{\omega}\backslash \Omega _{1}$, on dit alors que le jeu est \textit{à somme nulle}. Ceci signifie que dans le cas du jeu d'atteignabilité à deux joueurs le but de $J_{2}$ est d'empêcher $J_{1}$ d'atteindre son objectif. Nous allons expliciter une méthode permettant de déterminer à partir de quels sommets $J_{1}$ (respectivement $J_{2}$) est assuré de gagner le jeu (respectivement d'empêcher $J_{1}$ d'atteindre son objectif). Dans ce cas, nous posons $F$ l'ensemble des sommets objectifs de $J_{1}$.
	
	Nous commençons en énonçant et en prouvant quelques propriétés qui permettent d'élaborer un processus algorithmique afin de trouver les états gagnants de chacun de deux joueurs.
	
%PROPRIETE	
	\begin{propriete}
		\label{Wempty}
		
		Soit $\mathcal{G}$ un jeu, on a : $W_{1}\cap W_{2} = \emptyset$.
	\end{propriete}
	\begin{demonstration}
		Supposons au contraire que $W_{1}\cap W_{2} \neq \emptyset$. Cela signifie qu'il existe $s \in W_{1}$ tel que $s \in W_{2}$.\\
		$s \in W_{1}$ si et seulement si il existe $\sigma _{1}$ une stratégie de $J_{1}$ telle que pour toute $\sigma _{2}$ stratégie de $J_{2}$ nous avons :$ \langle \sigma _{1},\sigma _{2} \rangle_s \in \Omega _{1}$.\\
		$s \in W_{2}$ si et seulement si il existe $\tilde{\sigma} _{2}$ une stratégie de $J_{2}$ telle que pour toute $\tilde{\sigma}_{1}$ stratégie de $J_{1}$ nous avons :$\langle \tilde{\sigma}_{1},\tilde{\sigma}_{2} \rangle_s \in \Omega _{2}$.\\
		Dès lors, on obtient : $\langle\sigma _{1},\tilde{\sigma}_{2}\rangle_s \in \Omega _{1} \cap \Omega _{2}$. Or $\Omega _{1} \cap \Omega _{2} = \emptyset$, ce qui amène la contradiction.\\
	\end{demonstration}

Cette première propriété signifie simplement qu'un noeud du jeu ne peut pas être un état gagnant pour les deux joueurs. En couplant cette propriété avec la définition suivante, nous constatons que dans le cas des jeux déterminés tels que $W_1 \cap W_2 = \emptyset$, les ensembles des états gagnants de $J_1$ et de $J_2$ forment une partition de $V$. De surcroit, si l'on connait un procédé permettant de déterminer $W_1$ alors on connait également $W_2$.

%DEFINITION: jeu déterminé	
	\begin{defi}[Jeu déterminé]
		Soit $\mathcal{G}$ un jeu, on dit que ce jeu est \textit{déterminé} si et seulement si $W_{1} = V \backslash W_{2}$.
	\end{defi}

%DEFINITION: predecesseur + ensembles attracteurs	
	\begin{defi}
		\label{def:predecesseur}
		 Soit $X \subseteq V$.\\ 
		Posons $Pre(X) = \{ v \in V_{1}| \exists v'((v,v')\in E) \wedge (v' \in X)\} \cup \{ v \in V_{2}|\forall v' ((v,v')\in E) \Rightarrow (v' \in X)\}$.
		Définissons $(X_{k})_{k \in \mathbb{N}}$ la suite de sous-ensembles de $V$ suivante: \\
		\begin{center}
			$
			  \begin{cases}
			   X_{0} = F \\
			   X_{k+1} = X_{k} \cup Pre(X_{k})
		       \end{cases}
			. $
		\end{center}
		
	\end{defi}
	
%PROPRIETE
	
	\begin{propriete}
		\label{prop:suiteUltConst}$ $\\
		La suite $(X_{k})_{k \in \mathbb{N}}$ est ultimement constante (\emph{i.e.,} $\exists n \in \mathbb{N} \, \forall n \geq n X_k = X_n$). 
	\end{propriete}
	\begin{demonstration}
		
		Premièrement, nous avons clairement que  $\forall k \in \mathbb{N}, X_{k} \subseteq X_{k+1}$.\\
		Deuxièmement, nous avons : $\forall k \in \mathbb{N}, |X_{k}| \leq |V| $.\\
		Dès lors, vu que la suite $(X_{k})_{k \in \mathbb{N}}$ est une suite croissante dont la cardinalité des ensembles est bornée par celle de $V$, elle est ultimement constante.\\
		
	\end{demonstration}
	
	
% DEFINITION: attracteur
	
	\begin{defi}
	 L'\textit{attracteur de F} , noté $Attr(F)$, est défini de la manière suivante :$Attr(F) = \bigcup_{k \in \mathbb{N}} X_k$.
	\end{defi}
	
	
	$Pre(X)$ est l'ensemble des noeuds à partir desquels $J_1$ est certain d'atteindre un élément de $X$ en une étape (en empruntant un arc du graphe). Calculer $X_k$ revient donc à déterminer à partir de quels états $J_1$ est assuré d'atteindre son objectif en au plus $k$ étapes. Dès lors, comme $Attr(F)$ est la limite de la suite $(X_k)_{k \in \mathbb{N}}$, trouver $Attr(F)$ revient à trouver $W_1$. C'est le résultat que nous énonçons et démontrons ci-dessous.
	
% PROPRIETE

	\begin{propriete}
		\label{prop:attracteur}
	\begin{equation}
		W_{1} = Attr(F) \label{line1}
	\end{equation}
	\begin{equation}
		W_{2} = V \backslash Attr(F) \label{line2}
	\end{equation}
		
	\end{propriete}
	\begin{demonstration}
		Pour prouver ~\eqref{line1} et ~\eqref{line2} nous allons procéder en plusieurs étapes en prouvant chaque inclusion séparément.
		Nous commençons par montrer que $Attr(F) \subseteq W_{1}$ puis que $V \backslash Attr(F) \subseteq W_{2}$ et enfin nous concluons en prouvant que $W_{1} \subseteq Attr(F)$ et $W_{2} \subseteq V\backslash Attr(F)$. De ces quatre inclusions nous pouvons alors conclure les égalités recherchées. \\
		
		\noindent$\mathbf{Attr(F) \subseteq W_{1}}$: Soit $v \in Attr(F)$ alors par la propriété \ref{prop:suiteUltConst} on a \linebreak : $Attr(F) = X_{N}$ pour un certain $N \in \mathbb{N}$. Montrons par récurrence sur $n$ que dans ce cas, pour tout $n \in \mathbb{N}$ tel que $X_{n} \subseteq Attr(F)$ on peut construire une stratégie $\sigma _{1}$ pour $J_{1}$ telle que pour tout $\sigma_2 \in \Sigma_2$, $\langle \sigma _{1},\sigma _{2}\rangle_v \in \Omega _{1}$.
		\begin{enumerate}
			\item[$\star$] Pour $n=0$: alors $v \in X_{0} = F$ et l'objectif est atteint par $J_{1}$. $\sigma_1$ peut alors être définie de n'importe quelle manière puisque l'objectif est atteint.
			
			\item[$\star$] Supposons que la propriété soit vérifiée pour tout $ 0 \leq n \leq k $ et montrons qu'elle est toujours satisfaite pour $n = k + 1 \leq N$. \\
			Soit $v \in X_{k+1} = X_{k} \cup Pre(X_{k})$. \\
			Si $v \in X_{k}$ alors par hypothèse de récurrence, il existe $\sigma _{1}$ telle que pour tout $\sigma_2$, $ \langle \sigma _{1},\sigma _{2} \rangle_v \in \Omega _{1}$.\\
			Si $v \in Pre(X_{k})\backslash X_k$, alors si $v \in V_{1}$ par définition de $Pre(X_{k})$ on sait qu'il existe $v'\in X_{k}$ tel que $(v,v')\in E$. De plus, comme $v' \in X_k$ , par hypothèse de récurrence, on sait qu'il existe $\tilde{\sigma}_1$ telle que $\forall \sigma_2 \in \Sigma_2$ $\langle \sigma_1, \sigma_2 \rangle_{v'} \in \Omega_1$. Ainsi, on définit $\sigma _{1}(u) =\begin{cases} v' & \text{si } u = v \\ \tilde{\sigma_1}(u) & \text{sinon} \end{cases}$. Il en découle que $\forall \sigma_2 \in \Sigma_2$: $\langle \sigma_1, \sigma_2 \rangle_v \in \Omega_1$.
			 Tandis que si $v \in V_{2}$, par définition de $Pre(X_k)$, quelle que soit la stratégie $\sigma _{2}$ adoptée par $J_{2}$  nous sommes assurés que $\sigma _{2}(v) \in X_{k}$. Donc par hypothèse de récurrence, on sait qu'il existe $\tilde{\sigma}_1 \in \Sigma_1$ telle que pour toute stratégie $\tilde{\sigma_2} \in \Sigma_2$ on ait : $\langle \tilde{\sigma}_1, \tilde{\sigma}_2 \rangle_{\sigma_2(v)} \in \Omega_1$. Il suffit donc de prendre $\sigma_1  = \tilde{\sigma}_1$.\\
			
			 Dès lors, le résultat : $\exists \sigma_1 \in \Sigma_1$ telle que $\forall \sigma_2 \in \Sigma_2$ on ait \linebreak $ \langle \sigma _{1},\sigma _{2} \rangle_v \in \Omega _{1}$ est assuré.
		\end{enumerate}
		Par cette preuve par récurrence, l'assertion est bien vérifiée.\\
		
		\noindent$\mathbf{V \backslash Attr(F) \subseteq W_{2}}$: Soit $v \in V \backslash Attr(F)$. Une stratégie gagnante pour $J_{2}$ est une stratégie telle que à chaque tour de jeu le sommet $s$ considéré soit dans l'ensemble $V\backslash Attr(F)$. En effet, puisque $F \subseteq Attr(F)$, en s'assurant de rester en dehors de l'attracteur on est certain de ne pas atteindre un élément de $F$.\\
		
		Si $v \in V_1$ alors cela signifie que $\forall v'$ tel que $(v,v') \in E$ on a $v' \in V\backslash Attr(F)$. Tandis que si $v \in V_2$ alors $ \exists v'$ tel que $ (v, v') \in E$ et $ v' \in V\backslash Attr(F)$. On définit donc $\sigma_2(v) = v'$. Pour construire la stratégie gagnante $\sigma_2$ de $J_2$, on réitère cet argument.\\
		
		
		\noindent $\mathbf{W_{1} \subseteq Attr(F)}$: Supposons au contraire : $W_{1} \not\subseteq Attr(F)$. Cela signifie qu'il existe $v \in W_{1}$ tel que $v \notin Attr(F)$. D'où $v \in V\backslash Attr(F)$ et comme $V \backslash Attr(F) \subseteq W_{2}$, on a $v \in W_{2}$. Or par la propriété \ref{Wempty}, $W_{1} \cap W_{2} = \emptyset $ et ici $v \in W_{1}$ et $v \in W_{2}$. Ce qui amène la contradiction.\\
		
		\noindent $\mathbf{W_{2} \subseteq V\backslash Attr(F)}$ : La preuve est similaire à celle de $W_{1} \subseteq Attr(F)$.\\
		
		Ces quatre inclusions d'ensemble démontrent donc ~\eqref{line1} et ~\eqref{line2}.
	\end{demonstration}
	\begin{rem}
		Cette propriété nous montre que les jeux d'atteignabilité à objectif qualitatif et à deux joueurs sont déterminés.
	\end{rem}
	
%EXEMPLE
Illustrons maintenant le calcul de l'attracteur sur un simple exemple.

\begin{exemple}
Soit $\mathcal{G} = ((V,E),V_{1},V_{2},\Omega _{1}, \Omega _{2},F)$ le jeu d'atteignabilité représenté par le graphe ci-dessous. On a: $V = \{ v_{0},v_{1},v_{2},v_{3},v_{4},v_{5},v_{6},v_{7},v_{8} \}$, $E$ est l'ensemble des arcs représentés sur le graphe, $V_{1}$ est représenté par les sommets de forme ronde, $V_{2}$ est représenté par les sommets de forme carrée et $F$ est l'ensemble des sommets grisés.\\
Appliquons sur l'exemple ci-dessous, le principe de l'attracteur.

%!TEX root=main.tex

\begin{figure}[!ht]

	\centering

	\begin{tikzpicture}
		
		\node[nR] (v5)at(2,4){$v_{5}$};
		\node[nC] (v4) at (4,4){$v_{4}$};
		\node[nC] (v2) at (6,4){$v_{2}$};
		\node[nR] (v1) at (8,4){$v_{1}$};
		\node[nRG] (v0) at (8,2){$v_{0}$};
		\node[nR] (v3) at (6,2){$v_{3}$};
		\node[nC] (v6) at (2,2){$v_{6}$};
		\node[nR] (v7) at (3.5,0){$v_{7}$};
		\node[nR] (v8) at (0.5,0){$v_{8}$};
	
		\draw[fleche] (v5)--(v4);
		\draw[fleche] (v4)--(v2);
		\draw[fleche] (v2)--(v1);
		\draw[fleche] (v1)--(v0);
		\draw[->,>=latex] (v0.south) to [out=-95,in=-45](v3.south);
		\draw[fleche] (v3)--(v0);
		\draw[fleche] (v3)--(v2);
		\draw[fleche] (v4)--(v3);
		\draw[fleche] (v5)--(v4);
		\draw[fleche] (v6)--(v8);
		\draw[fleche] (v5)--(v6);
		\draw[fleche] (v8)--(v7);
		\draw[fleche] (v7)--(v6);

	\end{tikzpicture}
	
	
	\caption{$X_{0} = \{ v_{0} \}$ -Situation initiale, le seul état grisé est l'état objectif.}

\end{figure}

\begin{figure}[!ht]
	\centering

	\begin{tikzpicture}
		
		\node[nR] (v5)at(2,4){$v_{5}$};
		\node[nC] (v4) at (4,4){$v_{4}$};
		\node[nC] (v2) at (6,4){$v_{2}$};
		\node[nRG] (v1) at (8,4){$v_{1}$};
		\node[nRG] (v0) at (8,2){$v_{0}$};
		\node[nRG] (v3) at (6,2){$v_{3}$};
		\node[nC] (v6) at (2,2){$v_{6}$};
		\node[nR] (v7) at (3.5,0){$v_{7}$};
		\node[nR] (v8) at (0.5,0){$v_{8}$};
	
		\draw[fleche] (v5)--(v4);
		\draw[fleche] (v4)--(v2);
		\draw[fleche] (v2)--(v1);
		\draw[fleche] (v1)--(v0);
		\draw[->,>=latex] (v0.south) to [out=-95,in=-45](v3.south);
		\draw[fleche] (v3)--(v0);
		\draw[fleche] (v3)--(v2);
		\draw[fleche] (v4)--(v3);
		\draw[fleche] (v5)--(v4);
		\draw[fleche] (v6)--(v8);
		\draw[fleche] (v5)--(v6);
		\draw[fleche] (v8)--(v7);
		\draw[fleche] (v7)--(v6);

	\end{tikzpicture}
	
	
	\caption{$X_{1} = \{ v_{0},v_{1},v_{3}\}$ -- Première étape, $v_{1},v_{3} \in Pre(X_{0})$ car $v_{1},v_{3} \in V_{1}$ et il existe un arc entre $v_{1}$ et $v_{0}$ et entre $v_{3}$ et $v_{0}$.}

\end{figure}


\begin{figure}[!ht]
	\centering

	\begin{tikzpicture}
		
		\node[nR] (v5)at(2,4){$v_{5}$};
		\node[nC] (v4) at (4,4){$v_{4}$};
		\node[nCG] (v2) at (6,4){$v_{2}$};
		\node[nRG] (v1) at (8,4){$v_{1}$};
		\node[nRG] (v0) at (8,2){$v_{0}$};
		\node[nRG] (v3) at (6,2){$v_{3}$};
		\node[nC] (v6) at (2,2){$v_{6}$};
		\node[nR] (v7) at (3.5,0){$v_{7}$};
		\node[nR] (v8) at (0.5,0){$v_{8}$};
	
		\draw[fleche] (v5)--(v4);
		\draw[fleche] (v4)--(v2);
		\draw[fleche] (v2)--(v1);
		\draw[fleche] (v1)--(v0);
		\draw[->,>=latex] (v0.south) to [out=-95,in=-45](v3.south);
		\draw[fleche] (v3)--(v0);
		\draw[fleche] (v3)--(v2);
		\draw[fleche] (v4)--(v3);
		\draw[fleche] (v5)--(v4);
		\draw[fleche] (v6)--(v8);
		\draw[fleche] (v5)--(v6);
		\draw[fleche] (v8)--(v7);
		\draw[fleche] (v7)--(v6);

	\end{tikzpicture}
	
	
	\caption{$X_{2} = \{ v_{0},v_{1},v_{2},v_{3} \}$ -- Deuxième étape, $v_{2} \in Pre(X_{1})$ car $v_{2} \in V_{2}$ et tous les arcs sortant de $v_{2}$ atteignent un état de $X_{1}$.}

\end{figure}

\begin{figure}[!ht]
	\centering

	\begin{tikzpicture}
		
		\node[nR] (v5)at(2,4){$v_{5}$};
		\node[nCG] (v4) at (4,4){$v_{4}$};
		\node[nCG] (v2) at (6,4){$v_{2}$};
		\node[nRG] (v1) at (8,4){$v_{1}$};
		\node[nRG] (v0) at (8,2){$v_{0}$};
		\node[nRG] (v3) at (6,2){$v_{3}$};
		\node[nC] (v6) at (2,2){$v_{6}$};
		\node[nR] (v7) at (3.5,0){$v_{7}$};
		\node[nR] (v8) at (0.5,0){$v_{8}$};
	
		\draw[fleche] (v5)--(v4);
		\draw[fleche] (v4)--(v2);
		\draw[fleche] (v2)--(v1);
		\draw[fleche] (v1)--(v0);
		\draw[->,>=latex] (v0.south) to [out=-95,in=-45](v3.south);
		\draw[fleche] (v3)--(v0);
		\draw[fleche] (v3)--(v2);
		\draw[fleche] (v4)--(v3);
		\draw[fleche] (v5)--(v4);
		\draw[fleche] (v6)--(v8);
		\draw[fleche] (v5)--(v6);
		\draw[fleche] (v8)--(v7);
		\draw[fleche] (v7)--(v6);

	\end{tikzpicture}
	
	
	\caption{$X_{3} = \{ v_{0},v_{1},v_{2},v_{3},v_{4}\} $ -- Troisième étape, $v_{4} \in Pre(X_{2})$ car $v_{4} \in V_{2}$ et tous les arcs sortants de $v_{2}$ atteignent un état de $X_{2}$ ( $v_{2}$ et $v_{3}$).}

\end{figure}


\begin{figure}[!ht]
	\centering

	\begin{tikzpicture}
		
		\node[nRG] (v5)at(2,4){$v_{5}$};
		\node[nCG] (v4) at (4,4){$v_{4}$};
		\node[nCG] (v2) at (6,4){$v_{2}$};
		\node[nRG] (v1) at (8,4){$v_{1}$};
		\node[nRG] (v0) at (8,2){$v_{0}$};
		\node[nRG] (v3) at (6,2){$v_{3}$};
		\node[nC] (v6) at (2,2){$v_{6}$};
		\node[nR] (v7) at (3.5,0){$v_{7}$};
		\node[nR] (v8) at (0.5,0){$v_{8}$};
	
		\draw[fleche] (v5)--(v4);
		\draw[fleche] (v4)--(v2);
		\draw[fleche] (v2)--(v1);
		\draw[fleche] (v1)--(v0);
		\draw[->,>=latex] (v0.south) to [out=-95,in=-45](v3.south);
		\draw[fleche] (v3)--(v0);
		\draw[fleche] (v3)--(v2);
		\draw[fleche] (v4)--(v3);
		\draw[fleche] (v5)--(v4);
		\draw[fleche] (v6)--(v8);
		\draw[fleche] (v5)--(v6);
		\draw[fleche] (v8)--(v7);
		\draw[fleche] (v7)--(v6);

	\end{tikzpicture}
	
	
	\caption{$X_{4}= Attr(F) = \{ v_{0},v_{1},v_{2},v_{3},v_{4}, v_{5} \} $ -- Dernière étape, $v_{5} \in Pre(X_{3})$ car $v_{4} \in V_{1}$ et il existe un arc sortant de $v_{5}$ vers $v_{4} \in Pre(X_{3})$.}

\end{figure}









\FloatBarrier
\end{exemple}

\noindent\textbf{Implémentation}:\\

Au vu de la définition de l'attracteur d'un ensemble, l'implémentation de la résolution d'un jeu d'atteignabilité à deux joueurs avec objectif qualitatif en découle aisément. En effet, si nous possédons un algorithme permettant de calculer $Pre(X)$ pour tout sous-ensemble $X$ de sommets du graphe, il suffit alors d'appeler plusieurs fois cet algorithme afin de générer la suite $(X_k)_{k \in \mathbb{N}}$ conformément à la définition~\ref{def:predecesseur} jusqu'à ce que celle-ci soit constante. Nous donnons ci-après le pseudo-code d'un algorithme permettant de résoudre ce type de jeu. \\

Soit $\mathcal{G} = ((V,E),V_{1},V_{2}, F, \Omega _{1}, \Omega _{2})$, nous supposons que $V$ comprend $n$ sommets numérotés de $0$ à $n-1$ et que le graphe $G = (V,E)$ est encodé sur base de sa \textit{liste d'adjacence}: adj - \emph{i.e.,} pour chaque sommet indicé par $k$ on possède une liste des sommets $v_l$ tels que $0 \leq l \leq n-1$ et $(v_k, v_l) \in E$.\\

Un premier algorithme (algorithme~\ref{algo:preX}) calcule à partir d'un sous ensemble $X \subseteq V$ de $V$ l'ensemble $Pre(X)$ comme défini en \ref{def:predecesseur}. Le principe de cet algorithme est le suivant: pour tout sommet $v\in V_{1}$ on teste s'il existe un arc sortant de $v$ vers un sommet de $X$. Si c'est le cas, alors $v\in Pre(X)$. On traite ensuite tous les sommets $v \in V_{2}$. Pour qu'un tel sommet $v$ appartienne à $Pre(X)$ il faut que tous les arcs sortant de $v$ atteignent un sommet de $X$.

\begin{notations}
	Pour l'écriture de cet algorithme nous adoptons les conventions de notation suivantes:
	\begin{itemize}
		\item[$\bullet$] On note $adj$ la liste d'adjacence. Pour récupérer la liste des successeurs du noeud $v_1$, on note donc $adj[1]$ et pour récupérer le premier successeur de $v_1$ on écrit donc $adj[1][0]$ (on suppose que les listes utilisées sont indicées à partir de 0).
		\item[$\bullet$] La notation $|adj[n]|$ désigne le nombre de successeurs que possède le noeud indicé par $n$ (\emph{i.e.,} la longueur de la liste $adj[n]$).
	\end{itemize}
\end{notations}

\begin{algorithm}
	\caption{PreX}
	\label{algo:preX}
	\begin{algorithmic}[1]
		\REQUIRE Un sous-ensemble $X$ de sommets de $V$.
		\ENSURE Pre(X)
		
		\STATE preX $\leftarrow$ un ensemble vide
		
		\FORALL { $v_{i} \in V_{1}$}	
			\STATE ind $\leftarrow$ 0
			\STATE existeArc $\leftarrow$ faux
			\WHILE{$\neg existeArc$ et $ind \leq |adj[i]|$}
				\IF {$adj[i][ind] \in X$}
					\STATE existeArc $\leftarrow$ vrai
				\ELSE
					\STATE ind $\leftarrow$ ind + 1
				\ENDIF
			\ENDWHILE
			\IF {(existeArc = vrai)}
				\STATE ajouter $v_{i}$ à $preX$
			\ENDIF
		\ENDFOR
		
		\FORALL { $v_{i}\in V_{2}$}
			\STATE ind $\leftarrow$ 0
			\STATE tousArcs $\leftarrow$ vrai
			\WHILE{tousArcs et $ind \leq |adj[i]|$}
				\IF {$adj[i][ind] \in X$}
					\STATE ind $\leftarrow$ ind + 1
				\ELSE
					\STATE  tousArcs $\leftarrow$ faux
				\ENDIF
			\ENDWHILE
			\IF {(tousArcs = vrai)}
				\STATE ajouter $v_{i}$ à $preX$
			\ENDIF
		\ENDFOR
		
		\RETURN preX
			
\end{algorithmic}
		
\end{algorithm}

Un second algorithme (algorithme~\ref{algo:attrF}) permet de calculer les états gagnants de $J_{1}$ en utilisant les résultats \ref{prop:suiteUltConst} et \ref{prop:attracteur}. En effet, on y construit itérativement la suite $(X_{k})_{k \in \mathbb{N}}$ jusqu'à ce qu'elle devienne ultimement constante (\emph{i.e.,} jusqu'à ce que la cardinalité des ensembles $X_{k}$ ne varie plus).

\begin{algorithm}
	\caption{Attr(F)}
	\label{algo:attrF}
	\begin{algorithmic}[1]
		\REQUIRE $F$ l'ensemble des états objectifs de $J_{1}$
		\ENSURE $Attr(F)$ l'ensemble des états gagnants de $J_{1}$
		
		\STATE $X_{k}$ $\leftarrow$ F
		\STATE tailleDiff $\leftarrow$ vrai
		
		\WHILE {(tailleDiff = vrai)}
			\STATE ancCard $\leftarrow$ $|X_{k}|$
			\STATE $preX_{k}$ $\leftarrow PreX(X_{k})$
			\STATE Ajouter à $X_{k}$ tous les éléments de $preX_{k}$
			\STATE nouvCard $\leftarrow$ $|X_{k}|$
			
			\IF {ancCard = nouvCard}
				\STATE tailleDiff $\leftarrow$ faux
			\ENDIF
		\ENDWHILE
		
		\RETURN $X_{k}$
\end{algorithmic}
\end{algorithm}
$ $\\

\noindent\textbf{Complexité}

Terminons en abordant brièvement la complexité de ces algorithmes. Posons $n$ (respectivement $m$) le nombre de sommets du graphe (respectivement le nombre d'arcs).
\begin{description}
	\item[Algorithme~\ref{algo:preX}] Pour chaque noeud du graphe, on parcourt sa liste de successeurs. Dans le pire des cas on doit parcourir toute cette liste et donc effectuer un test pour chaque successeur de chaque noeud. Chacun de ses tests et des opérations qui en découlent sont en $\mathcal{O}(1)$ et sont effectuées dans le pire cas $m$ fois. La complexité de cet algorithme est donc en $\mathcal{O}(m)$.
	
	\item[Algorithme~\ref{algo:attrF}] Supposons que nous utilisons une structure de données telle que l'on puisse récupérer la taille des ensembles en $\mathcal{O}(1)$, alors la complexité de cet algorithme est en $\mathcal{O}(m.n)$. En effet, dans le pire cas on part d'un ensemble $F$ de cardinalité égale à 1 et on ne rajoute qu'un seul élément à la suite $(X_k)_{k \in \mathbb{N}}$ à chaque étape du calcul de $Pre(X_k)$ jusqu'à obtenir un ensemble de cardinalité $n$. Dès lors, dans ce cas on fait appel $n-1$ fois à l'algorithme~\ref{algo:preX} et on obtient bien un algorithme en $\mathcal{O}(m.n)$.

	
\end{description}

Dans la section suivante nous ne nous contentons plus d'une réponse binaire à la question qui est de savoir si un joueur est assuré d'atteindre ou non son objectif. Nous désirons quantifier la réalisation de cet objectif. 

\FloatBarrier
%\clearpage




%!TEX root=main.tex

\subsection{Jeux quantitatifs }

Contrairement aux jeux à objectif qualitatif pour lesquels l'objectif d'un joueur est d'assurer qu'une certaine propriété soit vérifiée, on associe aux \textit{jeux à objectif quantitatif} une certaine valeur quantifiée. Le but d'un joueur est donc de \textit{maximiser} ou de \textit{minimiser} cette valeur afin que sa satisfaction soit maximale.\\

Nous introduisons cette section par un exemple, ensuite nous abordons les notions essentielles aux jeux quantitatifs en distinguant les \textit{jeux multijoueurs} et les \textit{jeux à deux joueurs}.\\
Les définitions et les notions sont inspirées de l'article de Brihaye \emph{at al.} \cite{DBLP:conf/lfcs/BrihayePS13}.\\

\noindent\textbf {Exemple introductif} \\
\indent Antoine et Thomas désirent se rendre à l'école à pied. Le chemin étant long, Antoine propose à Thomas un jeu. A chaque carrefour et à tour de rôle un des deux garçons choisit la route à emprunter. A chaque mètre parcouru, Thomas devra donner un bonbon à Antoine. L'objectif de Thomas est donc d'emprunter le plus court chemin jusque l'école afin de minimiser son coût. Tandis que l'objectif d'Antoine est d'emprunter le plus long chemin pour obtenir le plus de bonbons possible.\\

Au vu de cet exemple, il est clair que le modèle des jeux qualitatifs n'est pas suffisant pour modéliser cette situation. En effet, il ne permet pas de caractériser le fait que plus Antoine obtiendra de bonbons, plus il sera satisfait et inversement pour Thomas. Pour ce faire, nous devons introduire les concepts de \textit{fonction de gain} (ou \textit{fonction de coût} en fonction du point de vue duquel on se place) ainsi qu'un nouveau concept de solution pour ces jeux appelés \textit{jeu avec coût (cost games)} : les \textit{équilibres de Nash}. Nous supposons que dans de tels jeux les joueurs sont \textit{rationnels} c'est-à-dire qu'ils jouent de telle sorte à maximiser leur gain ou minimiser leur coût.\\

Un moyen de modéliser notre exemple est de considérer un jeu Min-Max à somme nulle. Nous commençons donc en abordant les notions de jeux Min-Max avec coût ainsi que de stratégies optimales et nous continuons en traitant le cas des jeux multijoueurs avec coût et le concept d'équilibre de Nash.\\

%----------------------------------------
% Jeu avec coût MIN-MAX
%---------------------------------------
%!TEX root=main.tex

\subsubsection{Jeux Min-Max avec coût}

% DEFINITION: jeu Min-Max avec coût

\begin{defi}[Jeu Min-Max avec coût] $\text{ }$\\
	Soit une arène $\mathcal{A} = (V, (V_{Min}, V_{Max}), E) $
	Un \textit{jeu Min-Max avec coût} est un tuple $\mathcal{G} = (\mathcal{A}, Cost_{Min}, Gain_{Max})$, où
	\begin{enumerate}
		\item[$\bullet$] $Cost_{Min}: Plays \rightarrow \mathbb{R} \cup \{+ \infty, -\infty \}$ est la \textit{fonction de coût} du joueur \textit{Min}.
		\item[$\bullet$] $Gain_{Max}: Plays \rightarrow \mathbb{R} \cup \{ + \infty, -\infty \}$ est la \textit{fonction de gain} du joueur \textit{Max}.
	\end{enumerate}
		
\end{defi}

\begin{rem}
	Dans cette définition, on sous-entend que $\Pi = \{ Min, Max \}$.
\end{rem}

Pour chaque $\rho \in Plays$, $Cost_{Min}(\rho)$ représente le montant que \textit{Min} perd quand le jeu $\rho$ est joué et $Gain_{Max}(\rho)$ représente le gain que \textit{Max} gagne quand le jeu $\rho$ est joué.
Le but de \textit{Min} (resp. \textit{Max}) est donc de \textbf{minimiser} (resp. \textbf{maximiser}) sa fonction de coût (resp. fonction de gain). Ce qui explique le choix des noms des joueurs.\\

%DEFINITION: jeu Min-Max à somme nulle

\begin{defi}[Jeu à somme nulle]
	Un jeu Min-Max avec coût est dit \textit{à somme nulle} si $Gain_{Max} = Cost_{Min}$.
\end{defi}

% DEFINITION: garantir le paiement

\begin{defi}[Garantir le paiement]$\text{}$\\
	
	\begin{center}On dit que le joueur \textit{Max} \textit{garantit le paiement} $d \in \mathbb{R}$\\ 
		ssi \\ 
	$\exists \sigma _{1}\Sigma _{Max}$ tq $\forall \sigma _{2} \in \Sigma _{Min}$ $ Gain_{Max}(\sigma _{1},\sigma _{2}) \geq d$\\\end{center}
	
	\begin{center} 
		On dit que le joueur \textit{Min} \textit{garantit le paiement} $d \in \mathbb{R}$\\		
		ssi	\\
		$\exists \sigma _{1}\in \Sigma _{Min}$ tq $\forall \sigma _{2} \in \Sigma _{Max}$ $ Cout_{Min}(\sigma _{1},\sigma _{2}) \leq d$\\
		\end{center}

\end{defi}
		

% DEFINITION: Valeur supérieure et valeur inférieure

\begin{defi}[Valeur inférieure/supérieure]
	
	Soit $\mathcal{G}$ un jeu Min-Max avec coût, on définit pour chaque sommet $v \in V$: 
	\begin{enumerate}
		\item[$\bullet$]\textit{Valeur supérieure:} $\overline{Val}(v) = \inf\limits_{\sigma _{1} \in \Sigma _{Min}} \sup\limits_{\sigma _{2} \in \Sigma_{Max}} Cost_{Min}(\rho)$ où $\rho = Outcome(v,(\sigma _{1},\sigma _{2}))$
		
		\item[$\bullet$]\textit{Valeur inférieure:} $\underline{Val}(v) = \sup\limits_{\sigma _{2} \in \Sigma_{Max}}  \inf\limits_{\sigma _{1} \in \Sigma _{Min}} Gain_{Max}(\rho)$  où $\rho = Outcome(v,(\sigma _{1},\sigma _{2}))$
	\end{enumerate}
\end{defi}
\begin{rem}
	La \textit{valeur supérieure}  $\overline{Val}(v)$ est la plus grande valeur que $J_{Min}$ peut perdre et la \textit{valeur inférieure} $\underline{Val}(v) $ est la plus grande valeur que $J_{Max}$ peut gagner.
\end{rem}

%PROPRIETE 
\begin{propriete}
	Pour tout $v \in V$, on a : $\underline{Val}(v) \leq \overline{Val}(v)$
\end{propriete}

%DEFINITION: jeu déterminé et valeur d'un jeu

\begin{defi}[Jeu déterminé et valeur d'un jeu]
		Soit $\mathcal{G}$ un jeu Min-Max avec coût, on dit que $\mathcal{G}$ est \textit{déterminé} si pour tout $v \in V$, $\overline{Val}(v) = \underline{Val}(v)$. On dit alors que le jeu $\mathcal{G}$ a une \textit{valeur} et pour tout $v \in V$ on note $Val(v) = \overline{Val}(v) = \underline{Val}(v)$.
\end{defi}

%DEFINITION: Stratégie optimale

\begin{defi}[Stratégie $\epsilon$-optimale]
	Soit $\epsilon > 0$,
	\begin{enumerate}
	\item[$\bullet$] $\sigma _{1} \in \Sigma _{Max}$ est une \textit{stratégie $\epsilon$-optimale} ssi $\forall v \in V $ $ \forall \sigma _{2}\in \Sigma_{Min}$ $ Gain_{Max}(\rho) \geq \underline{Val}(v) + \epsilon  $\\ où $\rho = Outcome(v, (\sigma _{1},\sigma _{2}))$.
	\item[$\bullet$] $\sigma _{2} \in \Sigma _{Min}$ est une \textit{stratégie $\epsilon$-optimale} ssi $\forall v \in V $ $ \forall \sigma _{1}\in \Sigma_{Max}$ $Cost_{Min}(\rho) \leq \overline{Val}(v) + \epsilon $\\ où $\rho = Outcome(v, (\sigma _{1},\sigma _{2}))$.
	\item[$\bullet$] Si $\epsilon = 0$, on dit que la stratégie $\sigma _{i}$ est \textit{optimale}
	\end{enumerate}
\end{defi}

%DEFINITION: reachability-price game

\begin{defi}[Reachability-price game]
	Soit $\mathcal{A} = (V, (V_{Min}, V_{Max}), E) $,soit $w: E \rightarrow \mathbb{R}$ une fonction de poids,
	un \textit{"reachability-price game"} est un jeu Min-Max avec coût $\mathcal{G} = (\mathcal{A},RP_{Min},RP_{Max})$\\ avec un objectif donné $Goal \subseteq V$, où pour tout $\rho \in Plays$ tq $\rho = \rho _{0}\rho _{1}...$:\\
	
	$RP_{Min}(\rho)=RP_{Max}(\rho) =$ $\begin{cases}
									\sum_{i = 0}^{n-1} w(\rho_{i},\rho_{i+1}) & \text{ si } n \text{ est le plus petit indice tq } \rho_{n}\in 					  Goal\\
									+\infty & \text{sinon}
									\end{cases}$ \\
									
  \noindent Ce jeu est un jeu à somme nulle.
\end{defi}

% EXEMPLE : reachability-price game + jeu déterminé + valeur + stratégie optimale

%!TEX root=main.tex

\begin{exemple}
	Soit $\mathcal{G} = (V,(V_{Min},V_{Max}),E,RP_{Min},RP_{Max})$ le \og\textit{reachability-price game}\fg décrit par la figure \ref{ex:reachPriceGame1} où les sommets contrôlés par $J_{Min}$ sont les sommets ronds et les sommets contrôlés par $J_{Max}$ sont les sommets carrés et $Goal = \{ v_{3} \}$.
	
	\begin{figure}[ht!]
		\centering

		\begin{tikzpicture}

			\node[nRG] (v3) at (4,-2){$v_{3}$};
			\node[nC] (v2) at (4,0){$v_{2}$};
			\node[nR] (v1) at (2,0){$v_{1}$};
			\node[nR] (v0) at (0,0){$v_{0}$};
			\node[nR] (v4) at (6,0){$v_{4}$};

			\draw[->,>=latex] (v0.north) to [out=95,in= 80] node[midway,above]{$1$}(v1.north);
			\draw[->,>=latex] (v1) to node[midway,above]{$1$} (v2);
			\draw[->,>=latex] (v1) to node[midway,above]{$1$} (v0);
			\draw[->,>=latex] (v2) to node[midway,above]{$1$} (v4);
			

			\draw[->,>=latex] (v2) to node[midway,right]{$1$} (v3);
			\draw[->,>=latex] (v4) to node[midway,right]{$1$} (v3);
			
			\draw[->,>=latex] (v3) to node[midway,above]{$1$} (v0);





		\end{tikzpicture}


		\caption{reachability-price game}
		\label{ex:reachPriceGame1}


	\end{figure}
	
Dans un premier temps, montrons que $\mathcal{G}$ est déterminé. Nous avons : $\underline{Val}(v_{0})=\overline{Val}(v_{0})=4$,  $\underline{Val}(v_{1})=\overline{Val}(v_{1}) =3$, $\underline{Val}(v_{2})=\overline{Val}(v_{2})= 2$,  $\underline{Val}(v_{3})=\overline{Val}(v_{3})= 0$, $\underline{Val}(v_{4})=\overline{Val}(v_{4})= 1$.\\


Ensuite, exhibons une stratégie optimale pour le joueur \textit{Min} et une stratégie optimale pour le joueur \textit{Max}.\\

\noindent Prenons la stratégie sans mémoire de $J_{Min}$ définie comme suit : $\sigma _{1}(v) = $ $\begin{cases}
																						v_{0} & \text{si }v = v_{3}\\
																						v_{1} & \text{si }v = v_{0}\\
																						v_{2} & \text{si }v = v_{1}\\
																						v_{3} & \text{si }v = v_{4}
																					\end{cases}$.

\noindent Pour $J_{Max}$ considérons la stratégie sans mémoire suivante : 																					$\sigma _{2}(v) = $ $\begin{cases}
					v_{4} & \text{si } v = v_{2}
																																										
																																										\end{cases}$.\\
																																										
						
\end{exemple}

Dans ~\cite{DBLP:conf/lfcs/BrihayePS13} le théorème suivant est énoncé:

\begin{thm}
	\label{thm:1}
	Les \og\textit{reachability-price games}\fg sont déterminés et ont des stratégies optimales sans mémoire.
\end{thm}


Comme dans le cas des jeux d'atteignabilité à objectif qualitatif et au vu du résultat ~\ref{thm:1} précédant nous nous interrogeons quant à la façon d'implémenter un algorithme pour résoudre les \textit \og \textit{reachability-price games} \fg.L'objectif de cet algorithme est de trouver une stratégie optimale pour chaque joueur ainsi que la valeur associée à chaque sommet du graphe. L'idée est la suivante: le joueur \textit{Min} a pour but d'emprunter un \textbf{plus court chemin} possible allant d'un noeud initial $v_{0}$ vers un noeud $v \in Goal$. On trouve dans la littérature différents algorithmes qui résolvent les problèmes de plus court chemin dans les graphes orientés.Toutefois, il n'y a pas de deuxième joueur qui agit de manière \textbf{antagoniste} face au joueur \textit{Min} qui entre en compte dans ces algorithmes. C'est pourquoi nous avons tenté une adaptation de l'algorithme de \textit{Dijkstra} (que nous rappelons dans l'annexe A (p.\pageref{algo:dijkstra})).

Comme dans le cas de l'algorithme de Dijkstra, la fonction de coût associée au graphe doit être de la forme $w : E \rightarrow \mathbb{R}^{+}$. Le but de l'algorithme est de calculer le paiement minimum que le joueur \textit{Min} peut garantir quelle que soit la stratégie adoptée par le joueur \textit{Max}. Du fait que le joueur \textit{Max} joue de manière antagoniste par rapport au joueur \textit{Min}, à chaque fois que c'est au joueur \textit{Max} de prendre une décision il voudra maximiser son gain. Pour ce faire, on aura besoin de connaître le chemin le plus coûteux allant du sommet du joueur \textit{Max} en cours de traitement vers un certain état objectif $o \in Goal$. Dès lors, contrairement à l'algorithme classique de Dijkstra qui part d'une source, l'algorithme s'exécutera à rebours à partir des états $o \in Goal$. Ci-dessous nous reprenons les idées essentielles de l'algorithme proposé ainsi que son pseudo-code.\\

\noindent \textbf{Idées de l'algorithme}\\

\begin{enumerate}
	
	\item[$\bullet$] A tout sommet $v \in V$ on associe une valeur $d$ qui représente l'estimation de la valeur $Val(v)$ (qui existe par le théorème ~\ref{thm:1}). Cette valeur est mise à jour en cours d'exécution de l'algorithme de sorte qu'à la fin de celle-ci on ait pour tout $v \in V$ $Val(v) = d$. Comme pour l'algorithme de Dijkstra on initialise la valeur des sommets à $+\infty$ sauf pour les sommets objectifs $o \in Goal$ pour lesquels on initialise la valeur à 0.
	
	\item[$\bullet$] De plus, pour tout sommet $v \in V$, on associe une \textit{file de priorité} $S$ dans laquelle on stocke chaque successeur $s$ de $v$ déjà relaxé (ie. dont on a fini le traitement). A chacun de ces successeurs on joint l'estimation de  $Val(v)$ si l'$Outcome$ associé à cette estimation est de la forme $v s ... o$ pour $o \in Goal$. On note les éléments de $S$: $(val,succ)$.
	
	\item[$\bullet$] Pour tout sommet $v \in V_{Max}$, on associe le nombre de successeurs que ce sommet possède. On note ce nombre : $nbrSucc$. De plus, si l'on désire reconstituer la stratégie optimale pour le joueur \textit{Max} on stockera dans une structure de données adéquate la liste des successeurs déjà testés pour la recherche du chemin le plus long.
	
		\item[$\bullet$] On utilise une \textit{file de priorité} $Q$ (structure de données permettant de stocker des éléments en fonction de la valeur d'une clef) qui permet de stocker les sommets classés par leur valeur $d$. Sur cette file de priorité on peut effectuer les opérations suivantes: insertion d'un élément, extraction d'un élément ayant la clef de la plus petit valeur,lecture de l'élément ayant la clef de la plus petite valeur,test de la présence ou non d'élément dans $Q$, augmentation ou diminution de la valeur de la clef associée à un sommet. A l'initialisation de l'algorithme les sommets présents dans $Q$ sont tous ceux présents dans $V$.
	
	\item[$\bullet$] On maintient $T \subseteq V$ un ensemble de sommets qui vérifient la propriété suivante: pour tout sommet $v \in V$ $Val(v) = d$. Il s'agit donc de l'ensemble des sommets dont le traitement est terminé. A l'initialisation de l'algorithme $T = \emptyset$.
	
	\item[$\bullet$] Tant que $Q \neq \emptyset$, l'algorithme procède de la manière suivante: 
	\begin{enumerate}
		\item On regarde la valeur du minimum de $Q$ et le sommet $s$ associé.
		\item Si cette valeur est $+\infty$, alors cela signifie qu'on a fini de traiter tous les sommets qui pouvaient atteindre un sommet objectif. On ajoute donc tous les sommets restants de $Q$ dans $T$.
		\item S'il s'agit d'un sommet du joueur \textit{Min} ou d'un sommet objectif $o$ alors on extrait le minimum de $Q$ et on l'ajoute à $T$, on \textit{relaxe} tous les arcs entrant de $s$.
		\item S'il s'agit d'un sommet du joueur \textit{Max}, on regarde la valeur de $nbrSucc$ si elle est égale à 1 alors le joueur \textit{Max} n'a pas le choix du chemin à emprunter. On ajoute donc $s$ à $T$ et on \textit{relaxe} tous les arcs entrant de $s$. Si $nbrSucc \neq 1$, alors \textit{Max} a plusieurs choix de chemin. Pour maximiser son gain, il ne choisira pas de passer par le successeur qui admet la plus petite valeur de $Val(s)$. On retire donc la plus petite valeur $S$, on met à joueur la clef de $s$ dans $Q$ et on décrémente $nbrSucc$ ainsi on sait qu'il y a un successeur de moins à traiter et on ajoute ce successeur à l'ensemble des successeurs déjà traités. En effet, au bout du compte, on désire que la dernière valeur restante dans $s.Q$ soit celle qui assure la maximisation du gain de \textit{Max}.
	\end{enumerate}
	
	\item[$\bullet$] La \textit{relaxation} des arcs entrants de $s$ consiste en la méthode suivante: pour tout $(p,s)\in E$ on lit le minimum de $s.S$(la file de priorité des valeurs associées aux successeurs du sommet $s$), on calcule la nouvelle estimation de la valeur du sommet $p$, on insère dans $p.S$ la valeur calculée associée à $s$; on décrémente la clef associée à $p$ dans $Q$.
	
\end{enumerate}

\noindent \textbf{Pseudo-code}\\

Soient $\mathcal{A} = (V, (V_{Min}, V_{Max}), E) $ une arène et $\mathcal{G} = (\mathcal{A},RP_{Min},RP_{Max})$ un \og \textit{reachability-prince game} \fg. Nous utilisons les notations suivante : 
\begin{enumerate} 
	\item[$\bullet$]$Q$ et $S$ sont les files de priorité décrites ci-dessus.
	\item[$\bullet$]$s.S$  correspond à la file de priorité $S$ du sommet $s$.
	\item[$\bullet$]$T$ est le sous-ensemble de $V$ décrit ci-dessus.
	\item[$\bullet$]Pour tout $v \in V$, $Pred(v)$ est l'ensemble des prédécesseurs de $v$.
	\item[$\bullet$]Pour tout $v \in V$, $v.d$ est l'estimation de $Val(v)$.
	\item[$\bullet$]Pour tout $v \in V_{Max}$, $v.nbrSucc$ est le nombre de successeurs de $v$ qu'il reste à traiter, $v.t$ est l'ensemble des successeurs de $v$ déjà testés.
\end{enumerate}

L'algorithme en lui même est explicité par l'algorithme ~\ref{algo:dijkMinMax} et les algorithmes \ref{algo:initS},\ref{algo:initQ},\ref{algo:relaxerMinMax},\ref{algo:traiterMax} sont appelés au sein de l'algorithme ~\ref{algo:dijkMinMax}.
Comme pour l'algorithme de Dijkstra, pour une meilleure complexité les files de priorités sont supposées implémentées par une structure de donnée telle que chaques opérations \textsc{Extraire-Min}() et \textsc{Lire-Min()} est en $\mathcal{O}(\log V)$ ainsi que chaque \textsc{DécrémenterClef}($Clef(v),nouvVal$) alors l'agorithme est en $\mathcal{O}((V + E) \log V)$.\\ 

%ALGO: DijkstraMinMax

\begin{algorithm}
	\caption{\textsc {DijkstraMinMax}(G,w,o)}
	 \label{algo:dijkMinMax}
	\begin{algorithmic}[1]
		\REQUIRE $G = (V,E)$ un graphe orienté pondéré où $E$ est représenté par sa matrice d'adjacence, $w: E \rightarrow \mathbb{R}^{+}$ une fonction de poids,$o$ le sommet objectif.
		\ENSURE / \textbf{\textsc{Effet(s) de bord :}} Calcule pour chaque noeud la valeur de ce noeud.
		
		\STATE $Q \leftarrow$\textsc{Initialiser-Q}($G,o$)
		\STATE $S \leftarrow$\textsc{Initialiser-S}($G,o$)
		
		\STATE $T \leftarrow \emptyset$
		\WHILE {$Q \neq \emptyset$}
			\STATE $s \leftarrow Q.\textsc{Lire-Min()}$
			\STATE $(val,succ) \leftarrow s.S.$\textsc{Lire-Min}$()$
			\IF{$val = +\infty$}
				\WHILE{$Q \neq \emptyset$}
					\STATE $u \leftarrow Q.$\textsc{Extraire-Min}$()$
					\STATE $T.$\textsc{Insérer}$(u)$
				\ENDWHILE
			
			
			\ELSE
				\IF{$s \in V_{Min} \cup \{ o \}$ }
					\STATE $s \leftarrow Q.$\textsc{Extraire-Min}$()$
					\STATE $T.$\textsc{Insérer}$(s)$
					\FORALL{$p \in Pred(s)$}
						\STATE \textsc{Relaxer}$(p,s,w)$
					\ENDFOR
				
				
				\ELSE
					\STATE \textsc{Traiter-Max}$(s)$
				\ENDIF
			\ENDIF
		\ENDWHILE
				
			
\end{algorithmic}
		
\end{algorithm}

%ALGO: Initialiser-S

\begin{algorithm}
	\caption{\textsc {Initialiser-S}($G,o)$}
	 \label{algo:initS}
	\begin{algorithmic}[1]
		\REQUIRE $G$ un graphe orienté pondéré et le noeud objectif $o$.
		\ENSURE / \textbf{\textsc{Effet(s) de bord :}} initialise les files de priorités de tous les sommets.
		
		\FORALL{$v \in G.V\backslash \{ Goal \}$}
			\STATE $S \leftarrow$ nouveau tas
			\STATE $v.S \leftarrow S$
			\STATE $v.S.$\textsc{Insérer}$(+\infty, NULL)$
		\ENDFOR
		\FORALL{$o \in Goal$}
			\STATE $S \leftarrow$ nouveau tas
			\STATE $o.S \leftarrow S$
			\STATE $o.S.$\textsc{Insérer}$(0, NULL)$
		\ENDFOR
	
			
\end{algorithmic}
		
\end{algorithm}

%ALGO: Initialiser-Q


\begin{algorithm}
	\caption{\textsc {Initialiser-Q}$(G,o)$}
	 \label{algo:initQ}
	\begin{algorithmic}[1]
		\REQUIRE $G$ un graphe orienté pondéré, l'état objectif $o$.
		\ENSURE Un tas $Q$ qui comprend tous les sommets de $V$ classés par leur valeur $d$.
		
		\STATE $Q \leftarrow$ nouveau tas 
		\FORALL{$v \in G.V \backslash \{ Goal \}$}
			\STATE $v.d \leftarrow +\infty$
			\STATE $Q.$\textsc{Insérer}(v)
		\ENDFOR
		\FORALL{$o \in Goal$}
			\STATE $o.d \leftarrow 0$
			\STATE $Q.$\textsc{Insérer}(o)	
		\ENDFOR
		
		\RETURN $Q$
	
			
\end{algorithmic}
		
\end{algorithm}

%ALGO:Relaxer

\begin{algorithm}
	\caption{\textsc {Relaxer}$(p,s,w)$}
	 \label{algo:relaxerMinMax}
	\begin{algorithmic}[1]
		\REQUIRE deux sommets $p$ et $s$, une fonction de poids $w : E \rightarrow \mathbb{R}^{+}$.
		\ENSURE / \textbf{\textsc{Effet(s) de bord :}} Ajoute à p.S une valeur pour $p$ et son successeur.
		
		\STATE $(sVal,succ)$ $\leftarrow$ \textsc{Lire-Min(s.S)}
		\STATE $pVal \leftarrow w(p,s) + sVal$
		\STATE $p.S.$\textsc{Insérer}$((pVal,s))$
		\STATE $Q.$\textsc{Décrémenter-Clef}$(p,pVal)$
			
\end{algorithmic}
		
\end{algorithm}

%ALGO:Traiter-Max

\begin{algorithm}
	\caption{\textsc {Traiter-Max}$(s)$}
	 \label{algo:traiterMax}
	\begin{algorithmic}[1]
		\REQUIRE un sommet $s$ appartenant au joueur \textit{Max}.
		\ENSURE / \textbf{\textsc{Effet(s) de bord :}} traite le sommet $s$ soit en relaxant ses arcs entrants soit en supprimant la plus petite valeur de s.S.
		
		\IF {$s.nbrSucc = 1$}
			\STATE $T.$\textsc{Insérer}(s)
			\FORALL{( $p \in Pred(s)$)}
				\STATE \textsc{Relaxer}$(p,s,w)$
			\ENDFOR
			
		\ELSE
			\STATE $(val,succ) \leftarrow$ $s.S.$\textsc{Extraire-Min}()
			\STATE $(nouvVal,nouvSucc) \leftarrow$ $s.S.$\textsc{Lire-Min}()
			\STATE $Q.$\textsc{Incrémenter-Clef}$(s,nouvVal)$
			\STATE $s.nbrSucc \leftarrow s.nbrSucc - 1 $
			\STATE $s.t.$\textsc{Insérer}$(succ)$
		\ENDIF
			
				
			
\end{algorithmic}
		
\end{algorithm}

On remarque que l'algorithme \textsc{DijkstraMinMax}  est un algorithme à effet de bord mais ne renvoie aucune valeur. Pour récupérer les valeurs de chaque sommet $s$, ainsi qu'une stratégie optimale pour $J_{Min}$ et une strategie optimale pour $J_{max}$ il suffit de récupérer la racine de $s.S$ qui comprend $Val(s)$ ainsi que le successeur qu'il faut emprunter pour obtenir cette valeur. On obtient alors les algorithmes \textsc{RécupéréStratégies} (qui récupère une stratégie optimale pour chaque joueur) et \textsc{RécupérerValeurs} qui récupère la valeur de chaque noeud).

\begin{rem}
	
	Dans le cas où $Val(v) = +\infty$, nous n'avons pas de successeur à notre disposition dans la file de priorité. Comme les fonctions de stratégie sont des fonctions totales, il faut toutefois définir $\sigma _{i} (v)$.
\begin{enumerate}
	\item[$1^{er} cas:$] Si $v \in V_{Min}$ alors cela signifie que pour tout $s \in V$ tq $(v,s) \in E$ on a : $Val(s) = +\infty$. En effet, s'il existe $s' \in V$ tq $(v,s')\in E$ et $Val(v') < +\infty$ alors $J_{Min}$ a tout intérêt à jouer $\sigma _{Min}(v) = s'$.
	\item[$2^{eme} cas:$] Si $v \in V_{Max}$ alors cela signifie qu'il existe $s \in V$ tq $(v,s) \in E$ tq : $Val(s) = +\infty$. En effet, sinon pour tout $s' \in V$ tq $(v,s') \in E$ et $Val(v') < +\infty$ alors $Val(v) \neq +\infty$. 
	
\end{enumerate}

\end{rem}

\begin{algorithm}
	\caption{\textsc{RécupérerStratégies}($G$)}
	\label{algo:recupStrat}
	\begin{algorithmic}[1]
	
	\STATE \textsc{Afficher("Stratégie optimale pour $J_{Max}$")}
	\FORALL{$v \in V_{Max}$} 
		\IF{$Val(v) = +\infty$}
			\STATE Choisir $s \in Succ(v) \backslash v.t$
			\STATE \textsc{Afficher( $v " \rightarrow " s$)}
		\ELSE
			\STATE $(val,succ) \leftarrow v.S.$\textsc{Lire-Min}$()$
			\STATE \textsc{Afficher( $v " \rightarrow " succ$)}
		\ENDIF
	\ENDFOR
	
	\STATE \textsc{Afficher("Stratégie optimale pour $J_{Min}$")}
	\FORALL{$v \in V_{Min}$} 
		\IF{$Val(v) = +\infty$}
			\STATE Choisir $s \in Succ(v)$
			\STATE \textsc{Afficher( $v " \rightarrow " s$)}
		\ELSE	
			\STATE $(val,succ) \leftarrow v.S.$\textsc{Lire-Min}$()$
			\STATE \textsc{Afficher( $v $"$ \rightarrow $" $succ$)}
		\ENDIF
	\ENDFOR
	
	\end{algorithmic}
\end{algorithm}

\begin{algorithm}
	\caption{\textsc{RécupérerValeurs}($G$)}
	\label{algo:recupVal}
	\begin{algorithmic}[1]
	
	\FORALL{$v \in V$} 
		\STATE $(val,succ) \leftarrow v.S.$\textsc{Lire-Min}$()$
		\STATE \textsc{Afficher( "La valeur de" $v$ "est " $val$)}
	\ENDFOR
	
	
	\end{algorithmic}
\end{algorithm}


\FloatBarrier
	

%EXEMPLE: application de l'algorithme DijkstraMinMax sur un exemple 

%!TEX root=main.tex

\begin{exemple}

	Soient $\mathcal{A} = (V, (V_{Min}, V_{Max}), E) $, $w: E \rightarrow \mathbb{R}$ une fonction de poids et $\mathcal{G} = (\mathcal{A}, g, Goal)$ le \og \textit{reachability-price game}\fg  associé, l'arène et la fonction de poids sont représentés sur le graphe de la figure~\ref{ex:reachPrice1} et $V_{Min}$ (resp. $V_{Max}$) est représenté par les sommets ronds (resp. les sommets carrés). $Goal =\{ v_{0} \}$. Pour les figures [9-15] , les états grisés représentent les états entièrement traités (i.e. les états dans $T$) et à l'intérieur de ceux-ci se trouve la valeur associée à l'état. De plus la table~\ref{tab:filePrior} (p.~\pageref{tab:filePrior}) reprend pour chaque étape et chaque noeud le contenu de la file de priorité $S$.
	
	\begin{figure}[ht!]
		\centering

		\begin{tikzpicture}

			\node[nC] (v7) at (0,0){$v_{7}$};
			\node[nR] (v6) at (2,0){$v_{6}$};
			\node[nC] (v5) at (4,0){$v_{5}$};
			\node[nR] (v4) at (6,0){$v_{4}$};
			\node[nC] (v2) at (8,0){$v_{2}$};
			\node[nC] (v0) at (10,0){$v_{0}$};
			\node[nR]  (v1) at (8,2){$v_{1}$};
			\node[nR] (v3) at (8,-2){$v_{3}$};

			\draw[->,>=latex] (v7) to [bend right] node[midway,above]{$1$} (v6);
			\draw[->,>=latex] (v6) to [bend right] node[midway,above]{$1$} (v7);
			\draw[->,>=latex] (v5) to node[midway,above]{$1$} (v6);
			\draw[->,>=latex] (v5) to node[midway,above]{$1$} (v4);


			\draw[->,>=latex] (v4) to node[midway,above]{$5$} (v2);
			\draw[->,>=latex] (v4) to node[midway,left]{$1$} (v3);

			\draw[->,>=latex] (v3) to node[midway,below]{$5$} (v0);
			\draw[->,>=latex] (v3) to node[midway,left]{$1$} (v2);

			\draw[->,>=latex] (v2) to node[midway,above]{$1$} (v0);
			\draw[->,>=latex] (v2) to node[midway,left]{$1$} (v1);

			\draw[->,>=latex] (v1) to node[midway,right]{$1$} (v0);

			\draw[->,>=latex] (v0) to [loop right] node[midway,right]{$1$} (v0);





		\end{tikzpicture}


		\caption{Arène du \og reachability-price game \fg }
		\label{ex:reachPrice1}


	\end{figure}

%ETAPE 1	
	\begin{figure}[ht!]
		\centering

		\begin{tikzpicture}

			\node[nC] (v7) at (0,0){$+\infty$};
			\node[nR] (v6) at (2,0){$+\infty$};
			\node[nC] (v5) at (4,0){$+\infty$};
			\node[nR] (v4) at (6,0){$+\infty$};
			\node[nC] (v2) at (8,0){$1$};
			\node[nCG] (v0) at (10,0){$0$};
			\node[nR]  (v1) at (8,2){$1$};
			\node[nR] (v3) at (8,-2){$5$};

			\draw[->,>=latex] (v7) to [bend right] node[midway,above]{$1$} (v6);
			\draw[->,>=latex] (v6) to [bend right] node[midway,above]{$1$} (v7);
			\draw[->,>=latex] (v5) to node[midway,above]{$1$} (v6);
			\draw[->,>=latex] (v5) to node[midway,above]{$1$} (v4);
			

			\draw[->,>=latex] (v4) to node[midway,above]{$5$} (v2);
			\draw[->,>=latex] (v4) to node[midway,left]{$1$} (v3);
			
			\draw[->,>=latex] (v3) to node[midway,below]{$5$} (v0);
			\draw[->,>=latex] (v3) to node[midway,left]{$1$} (v2);
			
			\draw[->,>=latex] (v2) to node[midway,above]{$1$} (v0);
			\draw[->,>=latex] (v2) to node[midway,left]{$1$} (v1);
			
			\draw[->,>=latex] (v1) to node[midway,right]{$1$} (v0);
			
			\draw[->,>=latex] (v0) to [loop right] node[midway,right]{$1$} (v0);


		\end{tikzpicture}


		\caption{Première étape -- Traitement du noeud $v_{0}$. On relaxe :($v_{1},v_{0}$), ($v_{2},v_{0}$), ($v_{3},v_{0}$) et ($v_{0z},v_{0}$). On grise $v_{0}$.}
		\label{ex:reachPrice2}


	\end{figure}
	
%ETAPE 2	
		\begin{figure}[ht!]
			\centering

			\begin{tikzpicture}

				\node[nC] (v7) at (0,0){$+\infty$};
				\node[nR] (v6) at (2,0){$+\infty$};
				\node[nC] (v5) at (4,0){$+\infty$};
				\node[nR] (v4) at (6,0){$+\infty$};
				\node[nC] (v2) at (8,0){$+\infty$};
				\node[nCG] (v0) at (10,0){$0$};
				\node[nR]  (v1) at (8,2){$1$};
				\node[nR] (v3) at (8,-2){$5$};

				\draw[->,>=latex] (v7) to [bend right] node[midway,above]{$1$} (v6);
				\draw[->,>=latex] (v6) to [bend right] node[midway,above]{$1$} (v7);
				\draw[->,>=latex] (v5) to node[midway,above]{$1$} (v6);
				\draw[->,>=latex] (v5) to node[midway,above]{$1$} (v4);


				\draw[->,>=latex] (v4) to node[midway,above]{$5$} (v2);
				\draw[->,>=latex] (v4) to node[midway,left]{$1$} (v3);

				\draw[->,>=latex] (v3) to node[midway,below]{$5$} (v0);
				\draw[->,>=latex] (v3) to node[midway,left]{$1$} (v2);

				\draw[->,>=latex] (v2) to node[midway,above]{$1$} (v0);
				\draw[->,>=latex] (v2) to node[midway,left]{$1$} (v1);

				\draw[->,>=latex] (v1) to node[midway,right]{$1$} (v0);

				\draw[->,>=latex] (v0) to [loop right] node[midway,right]{$1$} (v0);


			\end{tikzpicture}


			\caption{Deuxième étape -- Traitement du noeud $v_{2}$. $nbrSucc = 2$, on n'a donc pas encore testé tous les chemins possibles à partir de ce noeud. Décrémentation de $nbrSucc$, retrait de la plus petit valeur de $S$ et ajout de $v_{0}$ dans les successeurs déjà testés. }
			\label{ex:reachPrice3}


		\end{figure}
%ETAPE 3	
		\begin{figure}[ht!]
		\centering

		\begin{tikzpicture}

			\node[nC] (v7) at (0,0){$+\infty$};
			\node[nR] (v6) at (2,0){$+\infty$};
			\node[nC] (v5) at (4,0){$5$};
			\node[nR] (v4) at (6,0){$+\infty$};
			\node[nC] (v2) at (8,0){$2$};
			\node[nCG] (v0) at (10,0){$0$};
			\node[nRG]  (v1) at (8,2){$1$};
			\node[nR] (v3) at (8,-2){$+\infty$};

			\draw[->,>=latex] (v7) to [bend right] node[midway,above]{$1$} (v6);
			\draw[->,>=latex] (v6) to [bend right] node[midway,above]{$1$} (v7);
			\draw[->,>=latex] (v5) to node[midway,above]{$1$} (v6);
			\draw[->,>=latex] (v5) to node[midway,above]{$1$} (v4);

			\draw[->,>=latex] (v4) to node[midway,above]{$5$} (v2);
			\draw[->,>=latex] (v4) to node[midway,left]{$1$} (v3);

			\draw[->,>=latex] (v3) to node[midway,below]{$5$} (v0);
			\draw[->,>=latex] (v3) to node[midway,left]{$1$} (v2);

     		\draw[->,>=latex] (v2) to node[midway,above]{$1$} (v0);
			\draw[->,>=latex] (v2) to node[midway,left]{$1$} (v1);

			\draw[->,>=latex] (v1) to node[midway,right]{$1$} (v0);

			\draw[->,>=latex] (v0) to [loop right] node[midway,right]{$1$} (v0);


		\end{tikzpicture}


		\caption{Troisième étape -- Traitement du noeud $v_{1}$. On relaxe $(v_{2},v_{1})$. On grise $v_{1}$. }
		\label{ex:reachPrice4}


				\end{figure}
				
%ETAPE 4	
	\begin{figure}[ht!]
	\centering
	\begin{tikzpicture}

	\node[nC] (v7) at (0,0){$+\infty$};
	\node[nR] (v6) at (2,0){$+\infty$};
	\node[nC] (v5) at (4,0){$+\infty$};
	\node[nR] (v4) at (6,0){$7$};
	\node[nCG] (v2) at (8,0){$2$};
	\node[nCG] (v0) at (10,0){$0$};
	\node[nRG]  (v1) at (8,2){$1$};
	\node[nR] (v3) at (8,-2){$3$};

	\draw[->,>=latex] (v7) to [bend right] node[midway,above]{$1$} (v6);
	\draw[->,>=latex] (v6) to [bend right] node[midway,above]{$1$} (v7);
	\draw[->,>=latex] (v5) to node[midway,above]{$1$} (v6);
	\draw[->,>=latex] (v5) to node[midway,above]{$1$} (v4);

	\draw[->,>=latex] (v4) to node[midway,above]{$5$} (v2);
	\draw[->,>=latex] (v4) to node[midway,left]{$1$} (v3);

	\draw[->,>=latex] (v3) to node[midway,below]{$5$} (v0);
	\draw[->,>=latex] (v3) to node[midway,left]{$1$} (v2);

	\draw[->,>=latex] (v2) to node[midway,above]{$1$} (v0);
	\draw[->,>=latex] (v2) to node[midway,left]{$1$} (v1);

	\draw[->,>=latex] (v1) to node[midway,right]{$1$} (v0);

	\draw[->,>=latex] (v0) to [loop right] node[midway,right]{$1$} (v0);


	\end{tikzpicture}

	\caption{Quatrième étape -- Traitement du noeud $v_{2}$. $nbrSucc = 1$. On relaxe $(v_{4},v_{2})$ et $(v_{3},v_{2})$. On grise $v_{2}$. }
	\label{ex:reachPrice5}


	\end{figure}
	
	
%ETAPE 5	
		\begin{figure}[ht!]
		\centering
		\begin{tikzpicture}

		\node[nC] (v7) at (0,0){$+\infty$};
		\node[nR] (v6) at (2,0){$+\infty$};
		\node[nC] (v5) at (4,0){$+\infty$};
		\node[nR] (v4) at (6,0){$4$};
		\node[nCG] (v2) at (8,0){$2$};
		\node[nCG] (v0) at (10,0){$0$};
		\node[nRG]  (v1) at (8,2){$1$};
		\node[nRG] (v3) at (8,-2){$3$};

		\draw[->,>=latex] (v7) to [bend right] node[midway,above]{$1$} (v6);
		\draw[->,>=latex] (v6) to [bend right] node[midway,above]{$1$} (v7);
		\draw[->,>=latex] (v5) to node[midway,above]{$1$} (v6);
		\draw[->,>=latex] (v5) to node[midway,above]{$1$} (v4);

		\draw[->,>=latex] (v4) to node[midway,above]{$5$} (v2);
		\draw[->,>=latex] (v4) to node[midway,left]{$1$} (v3);

		\draw[->,>=latex] (v3) to node[midway,below]{$5$} (v0);
		\draw[->,>=latex] (v3) to node[midway,left]{$1$} (v2);

		\draw[->,>=latex] (v2) to node[midway,above]{$1$} (v0);
		\draw[->,>=latex] (v2) to node[midway,left]{$1$} (v1);

		\draw[->,>=latex] (v1) to node[midway,right]{$1$} (v0);

		\draw[->,>=latex] (v0) to [loop right] node[midway,right]{$1$} (v0);


		\end{tikzpicture}

		\caption{Cinquième étape -- Traitement du noeud $v_{3}$. On relaxe $(v_{4},v_{3})$. On grise $v_{3}$. }
		\label{ex:reachPrice6}


		\end{figure}
		
%ETAPE 6	
\begin{figure}[ht!]
	\centering
	\begin{tikzpicture}
	\node[nC] (v7) at (0,0){$+\infty$};
	\node[nR] (v6) at (2,0){$+\infty$};
	\node[nC] (v5) at (4,0){$5$};
	\node[nRG] (v4) at (6,0){$4$};
	\node[nCG] (v2) at (8,0){$2$};
	\node[nCG] (v0) at (10,0){$0$};
	\node[nRG]  (v1) at (8,2){$1$};
	\node[nRG] (v3) at (8,-2){$3$};

	\draw[->,>=latex] (v7) to [bend right] node[midway,above]{$1$} (v6);
	\draw[->,>=latex] (v6) to [bend right] node[midway,above]{$1$} (v7);
	\draw[->,>=latex] (v5) to node[midway,above]{$1$} (v6);
	\draw[->,>=latex] (v5) to node[midway,above]{$1$} (v4);

	\draw[->,>=latex] (v4) to node[midway,above]{$5$} (v2);
	\draw[->,>=latex] (v4) to node[midway,left]{$1$} (v3);

	\draw[->,>=latex] (v3) to node[midway,below]{$5$} (v0);
	\draw[->,>=latex] (v3) to node[midway,left]{$1$} (v2);

	\draw[->,>=latex] (v2) to node[midway,above]{$1$} (v0);
	\draw[->,>=latex] (v2) to node[midway,left]{$1$} (v1);

	\draw[->,>=latex] (v1) to node[midway,right]{$1$} (v0);

	\draw[->,>=latex] (v0) to [loop right] node[midway,right]{$1$} (v0);


	\end{tikzpicture}

	\caption{Sixième étape -- Traitement du noeud $v_{4}$. On relaxe $(v_{5},v_{4})$. On grise $v_{4}$. }
	\label{ex:reachPrice7}


\end{figure}

%ETAPE 7	
\begin{figure}[ht!]
	\centering
	\begin{tikzpicture}
	\node[nC] (v7) at (0,0){$+\infty$};
	\node[nR] (v6) at (2,0){$+\infty$};
	\node[nC] (v5) at (4,0){$+\infty$};
	\node[nRG] (v4) at (6,0){$4$};
	\node[nCG] (v2) at (8,0){$2$};
	\node[nCG] (v0) at (10,0){$0$};
	\node[nRG]  (v1) at (8,2){$1$};
	\node[nRG] (v3) at (8,-2){$3$};

	\draw[->,>=latex] (v7) to [bend right] node[midway,above]{$1$} (v6);
	\draw[->,>=latex] (v6) to [bend right] node[midway,above]{$1$} (v7);
	\draw[->,>=latex] (v5) to node[midway,above]{$1$} (v6);
	\draw[->,>=latex] (v5) to node[midway,above]{$1$} (v4);

	\draw[->,>=latex] (v4) to node[midway,above]{$5$} (v2);
	\draw[->,>=latex] (v4) to node[midway,left]{$1$} (v3);

	\draw[->,>=latex] (v3) to node[midway,below]{$5$} (v0);
	\draw[->,>=latex] (v3) to node[midway,left]{$1$} (v2);

	\draw[->,>=latex] (v2) to node[midway,above]{$1$} (v0);
	\draw[->,>=latex] (v2) to node[midway,left]{$1$} (v1);

	\draw[->,>=latex] (v1) to node[midway,right]{$1$} (v0);

	\draw[->,>=latex] (v0) to [loop right] node[midway,right]{$1$} (v0);


	\end{tikzpicture}

	\caption{Dernière étape -- Traitement du noeud $v_{4}$. $nbrSucc =2$.On retire la plus petite valeur de $S$. }
	\label{ex:reachPrice8}


\end{figure}

	
%Tableau des files de priorités$

\begin{table}[]
	
\caption{Sommets associés à leur file de priorité $S$ pour chaque étape de l'algorithme}
\label{tab:filePrior}

\begin{tabular}{l|l|l|l|l|l|l|l|l|}
\cline{2-5}
                               & Sommets &    &    &      \\ \hline
\multicolumn{1}{|l|}{}         & $v_{0}$ &$v_{1}$  & $v_{2}$ &$v_{3}$  \\ \hline
\multicolumn{1}{|l|}{Etape 1:} & $(0,null),(1,v_{0})$   &$(+\infty,null),(1,v_{0})$     &$(+\infty,null),(1,v_{0})$    &$(+\infty,null),(5,v_{0})$    \\ \hline
\multicolumn{1}{|l|}{Etape 2:} &$(0,null),(1,v_{0})$         &$(+\infty,null),(1,v_{0})$    &$(+\infty,null)$    &$(+\infty,null),(5,v_{0})$     \\ \hline
\multicolumn{1}{|l|}{Etape 3:} &$(0,null),(1,v_{0})$         &$(+\infty,null),(1,v_{0})$    &$(+\infty,null),(2,v_{1})$    &$(+\infty,null),(5,v_{0})$  \\ \hline
\multicolumn{1}{|l|}{Etape 4:} &$(0,null),(1,v_{0})$         &$(+\infty,null),(1,v_{0})$    &$(+\infty,null),(2,v_{1})$    &$(+\infty,null),(5,v_{0}),(3,v_{2})$   \\ \hline
\multicolumn{1}{|l|}{Etape 5:}   &$(0,null),(1,v_{0})$         &$(+\infty,null),(1,v_{0})$    &$(+\infty,null),(2,v_{1})$    &$(+\infty,null),(5,v_{0}),(3,v_{2})$    \\ \hline 
\multicolumn{1}{|l|}{Etape 6:}   &$(0,null),(1,v_{0})$         &$(+\infty,null),(1,v_{0})$    &$(+\infty,null),(2,v_{1})$    &$(+\infty,null),(5,v_{0}),(3,v_{2})$     \\ \hline
\multicolumn{1}{|l|}{Etape 7:}   &$(0,null),(1,v_{0})$         &$(+\infty,null),(1,v_{0})$    &$(+\infty,null),(2,v_{1})$    &$(+\infty,null),(5,v_{0}),(3,v_{2})$   \\ \hline

\multicolumn{1}{|l|}{}  		&  &    &    &      \\ \hline
\multicolumn{1}{|l|}{}         & $v_{4}$ &$v_{5}$  & $v_{6}$ &$v_{7}$  \\ \hline
\multicolumn{1}{|l|}{Etape 1:} & $(+\infty,null)$   &$(+\infty,null)$     &$(+\infty,null)$    &$(+\infty,null)$    \\ \hline
\multicolumn{1}{|l|}{Etape 2:} &$(+\infty,null)$    &$(+\infty,null)$    &$(+\infty,null)$    &$(+\infty,null)$    \\ \hline
\multicolumn{1}{|l|}{Etape 3:} &$(+\infty,null)$    &$(+\infty,null)$    &$(+\infty,null)$    &$(+\infty,null)$    \\ \hline
\multicolumn{1}{|l|}{Etape 4:} &$(+\infty,null),(7,v_{2})$    &$(+\infty,null)$    &$(+\infty,null)$    &$(+\infty,null)$    \\ \hline
\multicolumn{1}{|l|}{Etape 5:} &$(+\infty,null),(7,v_{2}),(4,v_{3})$    &$(+\infty,null)$    &$(+\infty,null)$    &$(+\infty,null)$    \\ \hline
\multicolumn{1}{|l|}{Etape 6:} &$(+\infty,null),(7,v_{2}),(4,v_{3})$     &$(+\infty,null),(5,v_{4})$    &$(+\infty,null)$    &$(+\infty,null)$    \\ \hline
\multicolumn{1}{|l|}{Etape 7:} &$(+\infty,null),(7,v_{2}),(4,v_{3})$     &$(+\infty,null)$    &$(+\infty,null)$    &$(+\infty,null)$    \\ \hline

\end{tabular}




\end{table}
	
\end{exemple}	 

\clearpage




 

%----------------------------
%Jeu multijoueur avec coût
%----------------------------
%!TEX root=main.tex


\subsubsection{Jeux multijoueurs avec coût}
% DEFINITION: mutliplayer cost game

\begin{defi}[Jeu multijoueur avec coût]
	Soit $\mathcal{A} = (\Pi, V, (V_{i})_{i \in \Pi},E)$ une arène,
	un \textit{jeu multijoueur avec coût} est un tuple $\mathcal{G} = (\mathcal{A},(Cost_{i})_{i \in \Pi})$ où
	\begin{enumerate}
		\item[$\bullet$] $\mathcal{A} = (\Pi ,V ,(V_{i})_{i \in \Pi} ,E )$ est l'arène d'un jeu sur graphe.
		\item[$\bullet$] $Cost_{i}: Plays \rightarrow \mathbb{R} \cup \{ +\infty , -\infty \} $ est la \textit{fonction de coût} de $J_{i}$. 
	\end{enumerate}
\end{defi}


	Pour chaque $\rho \in Plays$, $Cost_{i}(\rho)$ représente le montant que $J_{i}$ perd quand le jeu $\rho$ est joué.
	Le but de chaque joueur est donc de \textbf{minimiser} sa fonction de coût.

%EXEMPLE: fonctions de coût.
\begin{exemple}[Fonctions de coût]
	\label{ex:fonctionsCout}
  Dans le cadre de ce projet, nous nous intéressons aux jeux sur graphe tels que l'objectif des joueurs est un objectif quantitatif. De plus, nous souhaitons que l'objectif des joueurs soit atteint le plus rapidement possible. Les fonctions de coût qui nous intéressent sont donc les suivantes: \\
	
	Pour tout  $\rho = \rho _{0} \rho _{1} \rho _{2} \ldots $ où $\rho \in Plays$ on définit:
	\begin{enumerate}
	\item $Cost_{i}(\rho) = $ $\begin{cases} 
									\min \{ i | \rho _{i} \in Goal_{i} \} & \text{si } \exists i \text{ tq } \rho _{i} \in Goal_{i} \\
									+\infty & \text{ sinon}
									\end{cases}$
	\item $\varphi _{i}(\rho) = $ $\begin{cases}
									\sum_{i = 0}^{n-1} w(\rho_{i},\rho_{i+1}) & \text{ si } n \text{ est le plus petit indice tq } \rho_{n}\in 					  Goal_{i}\\
									+\infty & \text{sinon}
									\end{cases}$ \\
									où $w$ est une \textit{fonction de poids} (cf. définition \ref{def:fonctionPoids}).
	\end{enumerate}
\end{exemple}

\begin{rem}
	L'exemple 1 est un cas particulier de l'exemple 2 avec $w(\rho_{i},\rho_{i+1}) = 1$ pour tout $i$.
\end{rem}
%DEFINITION: poids d'un arc

\begin{defi}[Fonction de poids]
	\label{def:fonctionPoids}
	A chaque arc d'un graphe $G = (V,E)$ on peut y associer un \textit{poids} (\emph{i.e.,} une valeur chiffrée). On associe donc à $G$ une \textit{fonction de poids}  $w : E \rightarrow \mathbb{R}$. On dit alors que $G$ est un graphe \textit{pondéré}.
\end{defi}

\begin{exemple}
	Si chaque $v \in V$ représente une ville sur une carte, alors on peut imaginer une fonction de poids représentant une des valeurs suivantes:
	\begin{enumerate}
		\item [$\bullet$] le nombre de kilomètres entre deux villes,
		\item [$\bullet$] le temps pour aller d'une ville à l'autre,
		\item [$\bullet$] la consommation d'essence pour aller d'une ville à l'autre.
	\end{enumerate}
\end{exemple}


%DEFINITION: jeu d'atteignabilité multijoueur à objectif quantitatif

\begin{defi}[Jeu d'atteignabilité multijoueur à objectif quantitatif]
	
	Un \textit{jeu d'atteignabilité multijoueur à objectif quantitatif} est un jeu multijoueur avec coût $\mathcal{G} = (\Pi ,V ,(V_{i})_{i \in \Pi} ,E ,(Cost_{i})_{i \in \Pi})$ tel que pour tout joueur $i \in \Pi$ $Cost_{i} = \varphi _{i}$ pour un certain $Goal _{i} \subseteq V$.
	On note ces jeux $\mathcal{G} = (\mathcal{A},(\varphi _{i})_{i\in \Pi},(Goal_{i})_{i \in \Pi})$.
\end{defi}
	



%EXEMPLE: graphe pondéré



%!TEX root=main.tex

\begin{exemple}[Jeu avec un graphe pondéré]
	\label{ex:graphePond}
	
Soit un jeu $\mathcal{G} = ( \Pi, V, (V_{1},V_{2}), E, (Goal_{1},Goal_{2}))$ tel que $\Pi = {1,2}$, $V_{1} = \{ v_{0}, v_{1}, v_{3} \}$, $V_{2} = \{ v_{2}\}$, $Goal_{1} = \{ v_{3}\}$, $Goal_{2} = \{ v_{0} \}$, $w(v_{i},v_{j}) = 1$ pour tout $0 \leq i,j \leq 3 , i \neq j$. Cet exemple est illustré par la figure \ref{ex:graphePond1}.

\begin{figure}[ht!]
	\centering

	\begin{tikzpicture}
		
		\node[nRB] (v3) at (4,-2){$v_{3}$};
		\node[nC] (v2) at (4,0){$v_{2}$};
		\node[nR] (v1) at (2,0){$v_{1}$};
		\node[nRG] (v0) at (0,0){$v_{0}$};
	
		\draw[->,>=latex] (v0.north) to [out=95,in= 80] node[midway,above]{$1$}(v1.north);
		\draw[->,>=latex] (v1) to node[midway,above]{$1$} (v2);
		\draw[->,>=latex] (v1) to node[midway,above]{$1$} (v0);
		
		\draw[->,>=latex] (v2) to node[midway,right]{$1$} (v3);
		\draw[->,>=latex] (v3) to node[midway,above]{$1$} (v0);
		
	\end{tikzpicture}
	
	
	\caption{Jeu avec un graphe pondéré }
	\label{ex:graphePond1}
	

\end{figure}
\end{exemple}

\FloatBarrier


%DEFINITION: équilibre de Nash

\begin{defi}[Equilibre de Nash]
	
	Soit $(\mathcal{G}, v_{0})$ un \textit{jeu multijoueur avec coût et initialisé}, un profil de stratégie $(\sigma _{i})_{i \in \Pi}$ est un \textit{équilibre de Nash} dans $(\mathcal{G}, v_{0})$ si, pour chaque joueur $j \in \Pi$ et pour chaque stratégie $\tilde{\sigma}_{j}$ du joueur $j$, on a :
	\begin{center}$ Cost_{j}(\rho) \leq Cost_{j}(\tilde{\rho})$ \end{center}
	où $\rho = \langle (\sigma _{i})_{i \in \Pi}\rangle_{v_0}$ et $\tilde{\rho} = \langle \tilde{\sigma} _{j} ,\sigma _{-j}\rangle_{v_0}$.
\end{defi}	


%DEFINITION: déviation profitable

\begin{defi}[Déviation profitable]
	
	Soit $(\mathcal{G}, v_{0})$ un \textit{jeu multijoueur avec coût et initialisé}, soit $(\sigma _{i})_{i \in \Pi}$ un profil de stratégie, $\tilde{\sigma _{j}}$ est une \textit{déviation profitable} pour le joueur $j$ relativement à $(\sigma _{i})_{i \in \Pi}$ si:
	\begin{center} $ Cost_{j}(\rho) > Cost_{j}(\tilde{\rho})$ \end{center}
	où $\rho = \langle (\sigma _{i})_{i \in \Pi} \rangle_{v_0}$ et $\tilde{\rho} = \langle \tilde{\sigma} _{j} ,\sigma _{-j} \rangle_{v_0}$. 
\end{defi}

%REMARQUE: signification de la def d'EN
\begin{rem}
	$(\sigma _{i})_{i\in \Pi}$ est un équilibre de Nash si tout joueur $j \in \Pi$ n'a aucun intérêt à dévier de sa stratégie $\sigma _{j}$ si les autres joueurs suivent $\sigma _{-j}$. C'est-à-dire qu'aucun joueur n'a de déviation profitable. 
\end{rem}

%EXEMPLE D'EN

%!TEX root=main.tex

\begin{exemple}
	Considérons le jeu décrit dans l'exemple \ref{ex:graphePond} muni pour $J_{1}$ et $J_{2}$ de la fonction de coût $\varphi _{i}$ de l'exemple \ref{ex:fonctionsCout} (p.\pageref{ex:fonctionsCout}) et donnons un exemple d'équilibre de Nash de ce jeu.\\
	
	 \begin{minipage}[c]{0.30\linewidth}Soit $\sigma _{1}(v) =$ $\begin{cases}
						 v_{1} & \text{ si } v = v_{0}\\
						 v_{2} & \text{ si } v = v_{1}\\
						 v_{0} & \text{ si } v = v_{3}
						\end{cases}$ \end{minipage} \hfill
	\begin{minipage}[c]{0.30\linewidth}\center{et}\end{minipage} \hfill \begin{minipage}[c]{0.30\linewidth}	\center{$\sigma _{2}(v) = v_{3}$ si $v = v_{2}$} \end{minipage} \newline
		
\noindent	(rem: ces deux stratégies sont des stratégies sans mémoire), $(\sigma _{1}, \sigma _{2})$ est un équilibre de Nash de $(\mathcal{G}, v_{0})$. De plus, soit $\rho = Outcome(v_{0},(\sigma _{1},\sigma _{2}))$ nous avons : $Cost_{1}(\rho) = 3$ et $Cost_{2}(\rho) = 0$.\\
\begin{demonstration}	
	Les seules déviations à considérer sont celles quand $J_{1}$ est en $v_{1}$ car il a le choix de se rendre en $v_{0}$ ou en $v_{2}$.\\ Considérons $\tilde{\sigma _{1}}(v) = $ $\begin{cases}
						 v_{1} & \text{ si } v = v_{0}\\
						 v_{0} & \text{ si } v = v_{1}\\
						 v_{0} & \text{ si } v = v_{3}
						\end{cases}$ .
Dans ce cas, soit $\tilde{\rho} = Outcome(v_{0},(\tilde{\sigma _{1}},\sigma _{2}))$, nous avons : \mbox{$Cost_{1}(\tilde{\rho}) = + \infty$}. Ce n'est donc pas une déviation profitable.\\
Considérons maintenant la famille $(\sigma _{1}^{n})_{n \in \mathbb{N}}$ de stratégie de $J_{1}$ qui décrit le fait que $J_{1}$ fait passer $n$ fois le "jeton" par l'état $v_{1}$ avant de le faire glisser vers l'état $v_{2}$. \\

$\sigma _{1}^{n}(h) = $ $\begin{cases}
					 v_{1} & \text{ si } h = h'v_{0}\\
					 v_{0} & \text{ si } h = h'v_{3}\\
					 v_{2} & \text{ si } h = h'v_{1} \text{ et } \exists h_{1},\ldots,h_{n} \in h' \text{ tq } h_{1},\ldots,h_{n} = v_{1} \\
					v_{0} & \text{ sinon}
					
					\end{cases}$ .
					
\noindent Nous avons alors: soit $ p = Outcome(v_{0},(\sigma _{1}^{n},\sigma _{2}))$, $Cost_{1}(p) = 2n+1$. Et pour tout $n > 1$, on a que $3=Cost_{1}(\rho) < Cost_{1}(p) = 2n +1$. $\sigma _{1}^{n}$ n'est donc pas une déviation profitable pour $J_{1}$ et ce pour tout $n \in  \mathbb{N}$.\\

Comme $\sigma _{2}$ est la seule stratégie possible pour $J_{2}$, nous pouvons déduire qu'aucun des deux joueurs n'a de déviation profitable. Donc $(\sigma _{1}, \sigma _{2})$ est un équilibre de Nash de $(\mathcal{G},v_{0})$.

\end{demonstration}			 		
    
\end{exemple}



% Résultat d'existence d'un équilibre de Nash à petite mémoire

 Dans~\cite{DBLP:conf/lfcs/BrihayePS13}, le théorème suivant est énoncé et prouvé:

\begin{thm}
	Soient $\mathcal{A} = (\Pi, V, (V_{i})_{i \in \Pi}, E)$ et $\mathcal{G} = (\mathcal{A},(\varphi _{i})_{i \in \Pi}, (Goal_{i})_{i \in \Pi})$ un jeu d'atteignabilité multijoueur à objectif quantitatif, il existe un équilibre de Nash dans tout jeu initialisé ($\mathcal{G},v_{0}$) avec $v_{0}\in V$. De plus, cet équilibre possède une mémoire d'au plus $|V| + |\Pi|$.
\end{thm}






