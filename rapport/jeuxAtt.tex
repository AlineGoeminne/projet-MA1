%!TEX root=main.tex

\section{Jeux d'atteignabilité}

Un jeu d'atteignabilité est un jeu sur graphe particulier. Chaque joueur possède un ensemble objectif qu'il souhaite atteindre. Le déroulement d'une partie d'un jeu d'atteignabilité se déroule comme un jeu sur graphe (cf \ref{derPar}) sauf que le but de chaque joueur est d'atteindre un élément de son ensemble objectif. On peut considérer les jeux d'atteignabilité selon deux points de vue: les jeux qualitatifs et les jeux quantitatifs. Nous développons dans les sections suivantes ces deux notions.

%-----------------------------
%-----------------------------
%Jeux qualtitatifs
%-----------------------------
%-----------------------------

\subsection{Jeux qualitatifs}

%DEFINITION: jeu d'atteignabilité à objectif qualitatif
	
	\begin{defi}[Jeu d'atteignabilité à objectif qualitatif]
		Un \textit{jeu d'atteignabilité à objectif qualitatif} est un jeu sur graphe $\mathcal{G} = (\Pi,(V,E),(V_{i})_{i \in \Pi}, (Goal_{i})_{i},(\Omega _{i})_{i \in \Pi})$ où :
		\begin{enumerate}
			\item[$\bullet$] Pour tout $i \in \Pi$, $Goal_{i\in \Pi} \subseteq V $ est l'ensemble des sommets de $V$ que $J_{i}$ essaie d'atteindre.
			\item[$\bullet$] Pour tout $i \in \Pi$, $\Omega _{i} = \{(u_{j})_{j \in \mathbb{N}}\in V^{\omega}| \exists k \in \mathbb{N}$  tel que $u_{k}\in Goal_{i}\}$. C'est l'ensemble des jeux $\rho$ sur $\mathcal{G}$ pour lesquels $J_{i}$ gagne le jeu.
		\end{enumerate}	
	\end{defi}
	
% DEFINITION: stratégie gagnante
	\label{strategieGagnante}
	\begin{defi}[Stratégie gagnante]
		Soit $v \in V$, soit $\sigma _{i}$ une stratégie du joueur $i$, on dit que $\sigma _{i}$ est \textit{gagnante pour $J_{i}$} à partir de $v$ si $Outcome(v,(\sigma _{i}, \sigma _{-i})) \subseteq \Omega _{1}$.
	\end{defi}
	
	\begin{rem}
		Dans le cadre de la définition \ref{strategieGagnante} , $Outcome(v,(\sigma _{i}, \sigma _{-i}))$ ne représente pas un seul jeu mais bien un ensemble de jeux. En effet, dans ce cas $\sigma _{-i}$ n'est pas fixé.
	\end{rem}
	
% DEFINITION: ensemble des états gagnants		
	
	\begin{defi}[Ensemble des états gagnants]
		Soit $\mathcal{G} = (\Pi,(V,E),(V_{i})_{i \in \Pi}, (Goal_{i})_{i \in \Pi},(\Omega _{i})_{i \in \Pi})$,\\
		\mbox{$W_{i} = \{ u_{j} |j\in \mathbb{N}$ et il existe une stratégie gagnante $\sigma _{i}$ pour $J_{i}$ à partir de $u_{j}\}$} est \textit{l'ensemble des états gagnants} de $J_{i}$. C'est l'ensemble des sommets de $\mathcal{G}$ à partir desquels $J_{i}$ est assuré de gagner.
	\end{defi}
	
	
	
	Une fois le concept de jeu d'atteignabilité clairement établi, nous pouvons nous poser les questions suivantes : "Quels joueurs peuvent-ils gagner le jeu?" et "Quelle stratégie doivent adopter les joueurs pour atteindre leur objectif quelle que soit la stratégie jouée par les autres joueurs?". \\
	
	Intéressons nous au cas de ces jeux restreints à deux joueurs.
%-------------------------------------
%Cas des jeux à deux joueurs
%-------------------------------------
	
	\subsubsection{Cas particulier des jeux à deux joueurs}
	Nous sommes intéressés à étudier les jeux d'atteignabilité à objectif qualitatif dans le cadre des jeux à deux joueurs. Dans ce cadre, nous notons $\Pi = \{1,2\}$ et nous avons que $\Omega _{2} = V^{\omega}\backslash \Omega _{1}$. Ceci signifie que dans le cas du jeu d'atteignabilité à deux joueurs le but de $J_{2}$ est d'empêcher $J_{1}$ d'atteindre son objectif. Nous allons expliciter une méthode permettant de déterminer à partir de quels sommets $J_{1}$ (respectivement $J_{2}$) est assuré de gagner le jeu (respectivement d'empêcher $J_{1}$ d'atteindre son objectif).Dans ce cas nous posons $F$ l'ensemble des sommets objectifs de $J_{1}$.
	
	\begin{rem}
		Ce jeu est un exemple de \textit{jeu combinatoire}.
	\end{rem}

%PROPRIETE	
	\begin{propriete}
		Soit $\mathcal{G}$ un jeu, on a : $W_{1}\cap W_{2} = \emptyset$.
	\end{propriete}
	\begin{demonstration}
		Supposons au contraire que $W_{1}\cap W_{2} \neq \emptyset$. Cela signifie qu'il existe $s \in W_{1}$ tel que $s \in W_{2}$.\\
		$s \in W_{1}$ si et seulement si il existe $\sigma _{1}$ une stratégie de $J_{1}$ telle que pour toute $\sigma {2}$ stratégie de $J_{2}$ nous avons : $Outcome(s,(\sigma _{1},\sigma _{2})) \in \Omega _{1}$.\\
		$s \in W_{2}$ si et seulement si il existe $\tilde{\sigma} _{2}$ une stratégie de $J_{2}$ telle que pour toute $\tilde{\sigma}_{1}$ stratégie de $J_{1}$ nous avons : $Outcome(s,(\tilde{\sigma}_{1},\tilde{\sigma}_{2})) \in \Omega _{2}$.\\
		Dès lors, on obtient : $Outcome(s,(\sigma _{1},\tilde{\sigma}_{2})) \in \Omega _{1} \cap \Omega _{2}$. Or $\Omega _{1} \cap \Omega _{2} = \emptyset$, ce qui amène la contradiction.\\
	\end{demonstration}

%DEFINITION: jeu déterminé	
	\begin{defi}[Jeu déterminé]
		Soit $\mathcal{G}$ un jeu, on dit que ce jeu est \textit{déterminé} ssi $W_{1} = V \backslash W_{2}$.
	\end{defi}

%DEFINITION: predecesseur + ensembles attracteurs	
	\begin{defi}
		 Soit $X \subseteq V$.\\ 
		Posons $Pre(X) = \{ v \in V_{1}| \exists v'((v,v')\in E) \wedge (v' \in X)\} \cup \{ v \in V_{2}|\forall v' ((v,v')\in E) \Rightarrow (v' \in X)\}$
		Définissons $(X_{k})_{k \in \mathbb{N}}$ la suite de sous-ensembles de $V$ suivante: \\
		
			$$\left\lbrace
			  \begin{array}{c}
			   X_{0} = F \\
			   X_{k+1} = X_{k} \cup Pre(X_{k})
		       \end{array}
			\right. $$
		
	\end{defi}
	
%PROPRIETE
	
	\begin{propriete}
		
		La suite $(X_{k})_{k \in \mathbb{N}}$ est ultimement constante. 
	\end{propriete}
	\begin{demonstration}
		Premièrement, nous avons clairement que  $\forall k \in \mathbb{N}, X_{k} \subseteq X_{k+1}$.\\
		Deuxièmement, nous avons : $\forall k \in \mathbb{N}, |X_{k}| \leq |V| $.\\
		Dès lors, vu que la suite $(X_{k})_{k \in \mathbb{N}}$ est une suite croissante dont la cardinalité des ensembles est bornée par celle de $V$, elle est ultimement constante.\\
		
	\end{demonstration}
	
% DEFINITION: attracteur
	
	\begin{defi}
		La limite de la suite $(X_{k})_{k \in \mathbb{N}}$ est appelée \textit{attracteur de F} et sera notée $Attr(F)$.
	\end{defi}
		
		
\subsection{Jeux quantitatifs }
	blabla
