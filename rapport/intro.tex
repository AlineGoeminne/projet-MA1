%!TEX root=main.tex

\section{Introduction}

La théorie des jeux est une discipline mathématique visant à modéliser des situations où différents agents interagissent de manière stratégique. Elle rencontre de nombreuses applications dans divers domaines tels que l'économie, l'informatique et la biologie par exemple.
A une situation on associe un \emph{jeu} tel que les \emph{joueurs} représentent les agents et on s'interroge alors sur la meilleure façon dont doivent agir chaque joueur, c'est-à-dire sur la \emph{stratégie} qu'ils adoptent.

Chaque agent est indépendant et vise à atteindre son propre objectif. Le théorie a d'abord étudié le cas des jeux à deux joueurs à \emph{objectif qualitatif} tel qu'un joueur tente d'atteindre son objectif alors que l'autre veut l'en empêcher. Cette représentation est utilisée pour modéliser la situation où un contrôleur lutte contre l'environnement pour atteindre son objectif. Cette situation a été étendue à des \emph{objectifs quantitatifs} où le contrôleur essaie de maximiser son gain tandis que l'environnement tente de la minimiser. Dans cette situation on veut déterminer quelle est la \emph{stratégie optimale} que doit choisir chaque joueur.

Cette modélisation n'étant pas suffisante pour des situations plus complexes avec plus de deux intervenants. C'est pourquoi il est intéressant d'étudier les jeux multijoueur à somme non nulle. Dans ce contexte, nous nous intéressons au concept de solution que sont les \emph{équilibres de Nash}. Un équilibre de Nash est un profil de stratégies - \emph{i.e.,} la donnée d'une stratégie pour chaque joueur- tel qu'aucun joueur n'a intérêt à dévier de sa stratégie si les autres joueurs se conforment à la stratégie qui leur correspond dans ce profil.
Nous sommes donc particulièrement intéressés par l'étude des équilibres de Nash pour des jeux multijoueur à somme non nulle avec des objectifs quantitatifs que l'on peut représenter grâce à l'utilisation de \emph{graphes} et où chaque joueur joue tour à tour.\\


Dans le cadre de notre travail, l'objectif quantitatif qui nous intéresse est celui d'\emph{atteignabilité}: chaque joueur tente de minimiser le coût du chemin dans le graphe qui lui permet d'atteindre son objectif. Dans l'article~\cite{DBLP:conf/lfcs/BrihayePS13}, Brihaye \emph{et al.} prouvent qu'il existe toujours un équilibre de Nash dans un tel jeu. Toutefois, quand il existe plusieurs équilibres de Nash nous sommes intéressés par celui qui est le plus pertinent. Nous entendons pas équilibres pertinent  celui pour lequel un maximum de joueurs atteignent leur objectif et tel que le coût de chaque joueur pour atteindre son objectif est minimal.

Le but de notre travail est d'expliciter et d'implémenter un processus algorithmique rapide pour trouver un équilibre de Nash pertinent. Parcourir tous les chemins possibles du graphe, en extraire tous les équilibres de Nash et puis seulement sélectionner le plus pertinent n'est pas une approche exploitable car elle est d'une complexité exponentielle. Nous avons donc testé plusieurs approches afin de guider notre exploration du graphe.\\


Notre travaille se présente de la manière suivante: nous commençons par expliquer certaines notions générales de la théorie des jeux dont nous aurons besoin, nous nous attarderons sur la description des jeux à somme nulle et sur les jeux multijoueurs à somme non nulle (sections~\refrangeconj{section:conceptsFond}{sect:jeuxAtt}). Ensuite, nous expliquerons ce qu'est, pour nous, un équilibre de Nash pertinent et nous énoncerons une propriété qui nous permettra de tester si un certain \emph{outcome} du jeu (un chemin infini dans le graphe du jeu) correspond à celui d'un équilibre de Nash (section~\ref{section:equilibrePert}). Nous explicitons ensuite, quels procédés nous avons testés afin d'arriver à notre objectif . Nous terminons en expliquant l'implémentation de ces procédés que nous avons effectuée (section~\ref{section:implementation}).
