%!TEX root=main.tex

\section{Concepts fondamentaux de théorie des jeux}
Dans le cadre de notre projet, nous sommes intéressés par les jeux sur graphes où tous les joueurs ont un \textit{objectif d'atteignabilité}.
Dans cette section, nous abordons la notion de jeux sur graphe de manière générale ainsi que les concepts fondamentaux liés à la théorie des jeux tels que les notions de stratégies, d'équilibres de Nash, ...\\
Cette section sera essentiellement inspirée de l'article de Brihaye et al. \cite{DBLP:conf/lfcs/BrihayePS13}.

\subsection{Jeux sur graphe}


% Définition d'arène


\begin{defi}[Arène]
	Soit $\Pi$ un ensemble (fini) de joueurs. On appelle \textit{arène} le tuple suivant :\\
	 $\mathcal{A} = (\Pi,V , (V_{i})_{i\in{\Pi}}, E )$ où :
	\begin{enumerate}
		\item[$\bullet$] $G = (V,E)$ est un \textit{graphe orienté}  dont $V$ est l'ensemble (fini) de ses sommets (\textit{vertex}) et $E \subseteq V \times V$ est l'ensemble de ses arcs (\textit{edge}). De plus, pour tout $ v\in V $ il existe $v'\in V$ tel que $(v,v') \in E$ (ie: pour tout sommet dans le graphe, il existe un arc sortant de ce sommet).
		\item[$\bullet$] $(V_{i})_{i\in\Pi}$ est une partition de l'ensemble des sommets du graphe $G$ telle que $V_{i}$ est l'ensemble des sommets du joueur $i$.
	\end{enumerate}
\end{defi}

%Définition de jeu sur graphe

\begin{defi}[Jeu sur graphe]
	Un \textit{jeu sur graphe}, noté $\mathcal{G}$ est la donnée d'une arène $\mathcal{A}$ et d'un \textit{objectif} pour chaque joueur.
\end{defi}

\begin{rem}
	Dans le cadre de ce document, les objectifs qui nous intéressent sont les \textit{objectifs d'atteignabilité}. Cette notion est abordée plus amplement dans la section \ref{sect:jeuxAtt}.
\end{rem}

\begin{rem}[Jeu initialisé]\label{sommetInit}
	Dans certains types de jeux sur graphe, on considère que le jeu commence à partir d'un \textit{sommet initial} donné. On le note communément $v_{0}$. Soit $\mathcal{G}$ un tel jeu, on note alors $(\mathcal{G},v_{0})$ le jeu ayant pour sommet initial $v_{0}$. On appelle $(\mathcal{G},v_{0})$ un \textit{jeu initialisé}.
\end{rem}

\begin{notations}
	Tout au long de ce rapport nous utilisons les conventions suivantes :
	\begin{enumerate}
		\item[$\bullet$] Soit $\mathcal{G}$ un jeu sur graphe, on note $(\mathcal{G},v_{0})$ le jeu ayant $v_{0}$ comme sommet initial .
		
		\item[$\bullet$] On note $J_{i}$  le joueur $i$.
		
		\item[$\bullet$] Pour $i\in \Pi$, on note : $-i\equiv \Pi\backslash \{ i\} $.
	\end{enumerate}
\end{notations}

%---------------------------------------
%Déroulement d'une partie d'un jeu sur graphe
%----------------------------------------


\noindent\textbf{Déroulement d'une partie}\label{derPar}

Nous pouvons imaginer le déroulement d'une partie d'un jeu sur graphe de la manière suivante: pour commencer un jeton est positionné sur un sommet $v_{0}$ du graphe (le sommet initial dans le cas de la remarque \ref{sommetInit}). Ensuite, comme ce sommet appartient à un certain $V_{i}$, le joueur $i$ choisit une arête $(v_{0},v_{1}) \in E$ et fait "glisser" le jeton le long de l'arc vers le sommet $v_{1}$. Ce sommet $v_{1}$ appartient à un certain $V_{j}$, c'est donc au joueur $j$ de choisir une arête $(v_{1},v_{2})\in E$ du graphe et faire "glisser" le jeton le long de cette arête. Le jeu se poursuit de la sorte infiniment. \\


\subsection{Stratégie}



%--------------------------------------
%Notions de chemin/jeu
%--------------------------------------

\noindent\textbf{Notions de chemin et de jeu}

\todo{Vérifier que j'utilise bien toutes les notions que je définis.}\\

Un \textit{jeu} $\rho \in V^{\omega}$(respectivement une \textit{histoire} $h \in V^{*}$) dans $\mathcal{A}$ est un chemin infini (respectivement fini) à travers le graphe. Nous noterons $\epsilon$ l'histoire vide, \textit{Plays} l'ensemble des jeux dans $\mathcal{A}$ et \textit{Hist} l'ensemble des histoires. Nous utiliserons les notations suivantes $\rho = \rho _{0}  \rho _{1} \rho _{2}\rho _{3} \ldots$ (où $\rho _{0},  \rho _{1},\ldots \in V$)  représentera un jeu et de manière similaire, pour une histoire $h$, $ h = h_{0} h_{1} h_{2} h_{3} ... h_{k}$ ( pour un certain $k \in \mathbb{N}$) où  $h_{0}, h_{1}, \ldots \in V$.
Un \textit{préfixe} de longueur $n+1$ (pour un certain $n\in \mathbb{N}$) d'un jeu $\rho = \rho _{0}  \rho _{1} \rho _{2}\rho _{3} \ldots$ est une histoire $\rho _{0}  \rho _{1} \rho _{2}\rho _{3} \ldots \rho _{n}$ et est notée $\rho[0,n]$. Soit $ h = h_{0} h_{1} \ldots h_{k}$ une histoire et soit $v \in V$ tel que $(h_{k},v)\in E$ on note $hv$ l'histoire $h_{0} h_{1} \ldots h_{k}v$. De même, étant donné une histoire $ h = h_{0} h_{1} h_{2} h_{3} ... h_{k}$ et un jeu $\rho = \rho _{0}  \rho _{1} \rho _{2} \ldots$ tels que $(h_{k},\rho_{0})\in E$ on note $h\rho$ le jeu $ h_{0} h_{1} \ldots h_{k}\rho _{0}  \rho _{1} \rho _{2} \ldots$ .

Etant donné une histoire $ h = h_{0} h_{1} h_{2} h_{3} ... h_{k}$,  on définit une fonction \textit{Last}(respectivement \textit{First}) qui prend comme argument l'histoire $h$ et qui retourne le dernier sommet $h_{k}$ (respectivement le premier sommet $h_{0}$).Nous définissons l'ensemble des histoires telles que c'est au tour du joueur $i \in \Pi$ de prendre une décision $Hist_{i} = \{ h \in Hist | Last(h) \in V_{i} \}$.

\begin{rem}
	Si un sommet initial $v_{0}$ a été fixé alors tous les jeux (et toutes les histoires) commencent par le sommet $v_{0}.$\\
\end{rem}

%--------------------------------------
% Définition de stratégie + stratégie consistante +  profil de stratégie + outcome
%--------------------------------------

\begin{defi}[Stratégie]
	Une \textit{stratégie} d'un joueur $i \in \Pi$ dans $\mathscr{A}$ est une fonction \mbox{$\sigma:Hist_{i} \rightarrow V$} telle que à chaque histoire $ h = h_{0} h_{1} h_{2} h_{3} ... h_{k}$ pour laquelle $h_{k} \in V_{i}$ est associée un sommet $v \in V$. De plus, on a: $(Last(h),\sigma(h))\in V$.
\end{defi}

\begin{defi}[Jeu consistant]	
	Un jeu $\rho = \rho _{0}  \rho _{1} \ldots$ est dit \textit{consistant} avec une stratégie $\sigma _{i}$ du joueur $i$ si pour tout préfixe $p = \rho _{0}\rho _{1}\ldots \rho _{k}$ (pour un certain $k \in \mathbb{N}$) tel que $p \in Hist_{i}$ on a : $\sigma _{i}(p) = \rho_{k+1}$.	
\end{defi}

\begin{rem}
	La notion de jeu consistant est facilement adaptable à la notion d'\textit{histoire consistante}.
\end{rem}

\begin{notations}
	Tout au long de ce rapport nous utilisons les conventions suivantes:
	\begin{enumerate}
		\item[$\bullet$] Un \textit{profil de stratégies} $(\sigma _{i})_{i \in \Pi}$ est un tuple tel que pour tout $i$ $\sigma _{i}$ désigne la stratégie du 	joueur $i$. 
				
		\item[$\bullet$] Soit  $(\sigma _{i})_{i \in \Pi}$ un profil de stratégies, pour un certain joueur $j\in \Pi $ on note $(\sigma _{i})_{i \in \Pi} = ( \sigma _{j},\sigma _{-j})$ .
		
		\item[$\bullet$] A un profil de stratégie $(\sigma _{i})_{i \in \Pi}$ et à un sommet initial $v_{0}$ est associé un unique jeu $\rho$ qui est consistant avec toutes les stratégies $\sigma _{i}$. Ce jeu est appelé \textit{outcome} de $\sigma _{i}$ et est noté $Outcome(v_{0},(\sigma _{i})_{i\in \Pi})$.
		
		\item[$\bullet$] On note $\Sigma _{i}$ l'ensemble des stratégies de $J_{i}$.
		
	\end{enumerate}
\end{notations}
		



%----------------------------------------
%Stratégie sans mémoire et avec mémoire
%----------------------------------------
\noindent\textbf{Stratégie avec mémoire vs. stratégie sans mémoire}\\
\todo{}

