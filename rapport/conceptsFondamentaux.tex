%!TEX root=main.tex

\section{Concepts fondamentaux de théorie des jeux}
\label{section:conceptsFond}

Dans le cadre de notre projet, nous sommes intéressés par les jeux sur graphe où tous les joueurs ont un \textit{objectif d'atteignabilité}.
Dans cette section, nous abordons la notion de jeux sur graphe de manière générale ainsi que les concepts fondamentaux liés à la théorie des jeux.

Cette section est essentiellement inspirée de l'article de Brihaye \emph{et al.}~\cite{DBLP:conf/lfcs/BrihayePS13} ainsi que de la thèse de Julie De~Pril~\cite{juliePhd}.

\subsection{Jeux sur graphe}


% Définition d'arène


\begin{defi}[Arène]
	Soit $\Pi$ un ensemble (fini) de joueurs. On appelle \textit{arène} le tuple suivant :\\
	 $\mathcal{A} = (\Pi,V , (V_{i})_{i\in{\Pi}}, E )$ où :
	\begin{enumerate}
		\item[$\bullet$] $G = (V,E)$ est un \textit{graphe orienté}  dont $V$ est l'ensemble (fini) de ses sommets (\textit{vertices}) et $E \subseteq V \times V$ est l'ensemble de ses arcs (\textit{edges}). De plus, pour tout $ v\in V $ il existe $v'\in V$ tel que $(v,v') \in E$ (\emph{i.e.,} pour tout sommet dans le graphe, il existe un arc sortant de ce sommet).
		\item[$\bullet$] $(V_{i})_{i\in\Pi}$ est une partition de l'ensemble des sommets du graphe $G$ telle que $V_{i}$ est l'ensemble des sommets du joueur $i$.
	\end{enumerate}
\end{defi}

%Définition de jeu sur graphe

\begin{defi}[Jeu sur graphe]
	Un \textit{jeu sur graphe}, noté $\mathcal{G}$ est la donnée d'une arène $\mathcal{A}$ et d'un \textit{objectif} pour chaque joueur.
\end{defi}


	Dans le cadre de ce document, les objectifs qui nous intéressent sont les \textit{objectifs d'atteignabilité}. Cette notion est abordée plus amplement dans la section \ref{sect:jeuxAtt}.


\begin{defi}[Jeu initialisé]\label{sommetInit}
	Dans certains types de jeux sur graphe, on considère que le jeu commence à partir d'un \textit{sommet initial} donné. On le note communément $v_{0}$. Soit $\mathcal{G}$ un tel jeu, on note alors $(\mathcal{G},v_{0})$ le jeu ayant pour sommet initial $v_{0}$. On appelle $(\mathcal{G},v_{0})$ un \textit{jeu initialisé}.
\end{defi}

\begin{notations}
	Tout au long de ce rapport nous utilisons les conventions suivantes :
	\begin{enumerate}
		\item[$\bullet$] Soit $\mathcal{G}$ un jeu sur graphe, on note $(\mathcal{G},v_{0})$ le jeu ayant $v_{0}$ comme sommet initial .
		
		\item[$\bullet$] On note $J_{i}$  le joueur $i$.
		
		\item[$\bullet$] Pour $i\in \Pi$, on note : $-i\equiv \Pi\backslash \{ i\} $.
	\end{enumerate}
\end{notations}

Nous pouvons maintenant expliquer comment se déroule une partie dans un jeu sur graphe.\\

%---------------------------------------
%Déroulement d'une partie d'un jeu sur graphe
%----------------------------------------


\noindent\textbf{Déroulement d'une partie}\label{derPar}

Nous pouvons imaginer le déroulement d'une partie d'un jeu sur graphe de la manière suivante: pour commencer un jeton est positionné sur un sommet $v_{0}$ du graphe (le sommet initial de la définition~\ref{sommetInit}). Ensuite, comme ce sommet appartient à un certain $V_{i}$, le joueur $i$ choisit une arête $(v_{0},v_{1}) \in E$ et fait \og glisser \fg~le jeton le long de l'arc vers le sommet $v_{1}$. Ce sommet $v_{1}$ appartient à un certain $V_{j}$, c'est donc au joueur $j$ de choisir une arête $(v_{1},v_{2})\in E$ du graphe et faire \og glisser\fg~le jeton le long de cette arête. Le jeu se poursuit de la sorte infiniment. Notons que cette procédure infinie est possible car nous imposant que tout noeud possède un arc sortant. \\

Les glissement successifs du jeton de noeud en noeud décrit un chemin infini dans le graphe du jeu et les choix effectués par chaque joueur  définit leur stratégie. Définissons formellement ces notions ainsi que certains concepts qui y sont relatifs.

\subsection{Stratégie}
%--------------------------------------
%Notions de chemin/partie
%--------------------------------------

\noindent\textbf{Notions de chemin et de jeu}

Une \textit{partie} $\rho \in V^{\omega}$(respectivement une \textit{histoire} $h \in V^{*}$) dans $\mathcal{A}$ est un chemin infini (respectivement fini) à travers le graphe. Nous notons $\epsilon$ l'histoire vide, \textit{Plays} l'ensemble des jeux dans $\mathcal{A}$ et \textit{Hist} l'ensemble des histoires. Nous utilisons les notations suivantes $\rho = \rho _{0}  \rho _{1} \rho _{2}\rho _{3} \ldots$ (où $\rho _{0},  \rho _{1},\ldots \in V$)  représente un jeu et de manière similaire, pour une histoire $h$, $ h = h_{0} h_{1} h_{2} h_{3} ... h_{k}$ ( pour un certain $k \in \mathbb{N}$) où  $h_{0}, h_{1}, \ldots \in V$.

 Soit $ h = h_{0} h_{1} \ldots h_{k}$ une histoire et soit $v \in V$ tel que $(h_{k},v)\in E$ on note $hv$ l'histoire $h_{0} h_{1} \ldots h_{k}v$. De même, étant donné une histoire $ h = h_{0} h_{1} h_{2} h_{3} ... h_{k}$ et un jeu $\rho = \rho _{0}  \rho _{1} \rho _{2} \ldots$ tels que $(h_{k},\rho_{0})\in E$ on note $h\rho$ le jeu $ h_{0} h_{1} \ldots h_{k}\rho _{0}  \rho _{1} \rho _{2} \ldots$ .

Etant donné une histoire $ h = h_{0} h_{1} h_{2} h_{3} ... h_{k}$,  on définit une fonction \textit{Last} (respectivement \textit{First}) qui prend comme argument l'histoire $h$ et qui retourne le dernier sommet $h_{k}$ (respectivement le premier sommet $h_{0}$). Nous définissons l'ensemble des histoires telles que c'est au tour du joueur $i \in \Pi$ de prendre une décision $Hist_{i} = \{ h \in Hist | Last(h) \in V_{i} \}$.

\begin{rem}
	Si un sommet initial $v_{0}$ a été fixé, alors tous les jeux (et toutes les histoires) commencent par le sommet $v_{0}.$\\
\end{rem}

%--------------------------------------
% Définition de stratégie + stratégie consistante +  profil de stratégie + outcome
%--------------------------------------

\begin{defi}[Stratégie]
	Une \textit{stratégie} d'un joueur $i \in \Pi$ dans $\mathcal{A}$ est une fonction \mbox{$\sigma _{i}: Hist_{i} \rightarrow V$} telle que à chaque histoire $ h = h_{0} h_{1} h_{2} h_{3} ... h_{k}$ pour laquelle $h_{k} \in V_{i}$ est associée un sommet $v \in V$. De plus, on a: $(Last(h),\sigma _{i}(h))\in E$.
\end{defi}

A la notion de stratégie, nous pouvons associer celle de partie consistante avec une stratégie. Une partie est dite consistante avec une stratégie d'un joueur  si à chaque fois que c'est au tour de ce joueur choisir l'action qu'il veut effectuer, ce choix est conforme à celui définit par sa stratégie. Nous définissons cela formellement:

\begin{defi}[Partie consistante]	
	Une partie $\rho = \rho _{0}  \rho _{1} \ldots$ est dite \textit{consistante} avec une stratégie $\sigma _{i}$ du joueur $i$ si pour tout préfixe $p = \rho _{0}\rho _{1}\ldots \rho _{k}$ (pour un certain $k \in \mathbb{N}$) tel que $p \in Hist_{i}$ on a : $\sigma _{i}(p) = \rho_{k+1}$.	
\end{defi}

	Notons que la notion de jeu consistant est facilement adaptable à la notion d'\textit{histoire consistante}.

\begin{notations}
	Tout au long de ce document nous utilisons les conventions suivantes:
	\begin{enumerate}
		\item[$\bullet$] Un \textit{profil de stratégies} $(\sigma _{i})_{i \in \Pi}$ est un tuple tel que pour tout $i$, $\sigma _{i}$ désigne la stratégie du 	joueur $i$. 
				
		\item[$\bullet$] Soit  $(\sigma _{i})_{i \in \Pi}$ un profil de stratégies, pour un certain joueur $j\in \Pi $ on note $(\sigma _{i})_{i \in \Pi} = ( \sigma _{j},\sigma _{-j})$ .
		
		\item[$\bullet$] A un profil de stratégies $(\sigma _{i})_{i \in \Pi}$ et à un sommet initial $v_{0}$ est associé un unique jeu $\rho$ qui est consistant avec toutes les stratégies $\sigma _{i}$. Ce jeu est appelé \textit{outcome} de $\sigma _{i}$ et est noté $\langle (\sigma _{i})_{i\in \Pi} \rangle_{v_0}$.
		
		\item[$\bullet$] On note $\Sigma _{i}$ l'ensemble des stratégies de $J_{i}$.
		
	\end{enumerate}
\end{notations}
		



%----------------------------------------
%Stratégie sans mémoire et avec mémoire
%----------------------------------------
\noindent\textbf{Stratégie avec mémoire vs. stratégie sans mémoire}\\



Lorsque l'on cherche des stratégies pour un joueur $J_{i}$, on distingue les \textit{stratégies sans mémoire}, les \textit{stratégies avec mémoire finie} et les \textit{stratégies avec mémoire infinie}. Nous définissons ci-dessous ces différents concepts.

\begin{defi}[Stratégie sans mémoire]
	
	Une stratégie $\sigma _{i} \in \Sigma _{i}$ est une \textit{stratégie sans mémoire} si le choix du prochain sommet dépend uniquement du sommet courant (\emph{i.e.,} $\sigma _{i}: V_{i} \rightarrow V$).
\end{defi}

\begin{defi}[Stratégie à mémoire finie]
	
	Une stratégie $\sigma _{i} \in \Sigma _{i}$ est une \textit{stratégie à mémoire finie} si on peut lui associer un \textit{automate de Mealy} $\mathcal{A} = (M, m_{0}, V, \delta, \nu)$ où:
	\begin{enumerate}
		\item[$\bullet$] $M$ est un ensemble fini non vide d'états de mémoire,
		\item[$\bullet$] $m_{0} \in M$ est l'état initial de la mémoire,
		\item[$\bullet$] $\delta : M \times V \rightarrow M$ est la fonction de mise à jour de la mémoire,
		\item[$\bullet$] $ \nu: M \times V_{i} \rightarrow V$ est la fonction de choix, telle que pour tout $m \in M$ et $v\in V_{i}$ $(v, \nu(m,v))\in E$.\end{enumerate}
		
		On peut étendre la fonction de mise à jour de la mémoire à une fonction $\delta ^{*}: M \times Hist \rightarrow M$ définie par récurrence sur la longueur de $h \in Hist$ de la manière suivante :\\\begin{center}
		 $\begin{cases}
																	\sigma^{*}(m,\epsilon) = m	\\
																	\sigma^{*}(m,hv)=\sigma(\sigma^{*}(m,h),v) & \text{pour tout } m\in M \text{ et } hv\in Hist
																	\end{cases}$ \end{center}
																	
		La stratégie $\sigma _{\mathcal{A}_{i}}$ calculée par un automate fini $\mathcal{A}_{i}$ est définie  par\\
 $\sigma _{\mathcal{A}_{i}}(hv) = \nu(\delta^{*}(m_{0},h),v)$ pour tout $hv \in Hist_{i}$ . Cela signifie qu'à chaque fois qu'une décision est prise, la mémoire est modifiée en conséquence et que chaque nouvelle décision est prise en fonction de la mémoire enregistrée jusque maintenant.
		Dès lors on dit que $\sigma _{i}$ est une stratégie à mémoire finie s'il existe un automate fini $\mathcal{A}_{i}$ tel que $\sigma _{i} = \sigma _{\mathcal{A}_{i}}$. De plus, on dit que $\sigma_i$ a une mémoire de taille $|M|$.

\end{defi}

Nous pouvons remarquer que le cas des stratégies sans mémoire est un cas particulier des stratégies à mémoire finie. En effet, dans ce cas l'automate de Mealy qui lui est associé est tel que $|M| = 1$.\\

\todo{Exemple?}

\begin{defi}[Stratégie à mémoire infinie]	
	Une stratégie $\sigma _{i} \in \Sigma _{i}$ est une \textit{stratégie à mémoire infinie } si elle n'est ni sans mémoire ni à mémoire finie.
\end{defi}

De même, nous pouvons définir ce que signifie la notion de mémoire pour une profil de stratégie.

\begin{defi}
	On appelle $(\sigma_i)_{i\in \Pi}$ un profil de stratégies de mémoire $m$ si pour tout $i\in \Pi$, la stratégie $\sigma_i$ a une mémoire d'au plus $m$.
\end{defi}

Mettons maintenant en lumière  ces notions par un simple exemple.

\begin{exemple}
	Soit le jeu illustré par la figure \ref{ex:jeuSurGraphe}. 
	L'arène de ce jeu est définie de la manière suivante:
	\begin{itemize}
		\item[$\bullet$] $\Pi = \{ 1, 2 \}$;
		\item[$\bullet$] $ V = \{ v_0, v_1, v_2, v_3, v_4 \}$;
		\item[$\bullet$] $V_1 = \{ v_0, v_1, v_3, v_4 \}$ et $V_2 = \{ v_2 \};$
		\item[$\bullet$] $E = \{(v_0, v_1), (v_1, v_2), (v_1, v_0), (v_2, v_4), (v_3, v_4), (v_4, v_2), (v_2, v_3),\\
		 (v_4, v_3),
		 (v_3, v_0) \}$
	\end{itemize} 
	$ $\\
	Si nous fixons $v_0$ comme sommet initial nous avons que $v_0 v_1 v_0 (v_2 v_4)^{\omega}$ ou encore $(v_0 v_1 v_2 v_3)^{\omega}$ sont des parties du jeu tandis que $v_0 v_1 v_0$ et $v_0 v_1 v_2 v_3$ sont des histoires du jeu. De plus, $v_0 v_1 v_0 $ appartient à $Hist_1$ et $v_0 v_1 v_2$ appartient à $Hist_2$.
	
	Illustrons désormais le concept de stratégie sur cet exemple. Définissons une stratégie sans mémoire possible pour chacun des deux joueurs.

	
	 \begin{minipage}[c]{0.4\linewidth}$$ \sigma_1(v) = \begin{cases}
					 v_1  &\text{ si } v = v_0\\
					v_2  &\text{ si } v = v_1\\
					v_3 &\text{ si } v = v_4\\
					v_0 &\text{ si } v = v_3
				\end{cases} $$\end{minipage} \hfill
	\begin{minipage}[c]{0.1\linewidth}\center{et}\end{minipage} \hfill \begin{minipage}[c]{0.30\linewidth}	\center{			$$ \sigma_2(v) = v_3$$
	} \end{minipage} \newline
	
	Nous remarquons que le jeu $(v_0 v_1 v_2 v_3)^{\omega}$ est consistant avec les stratégies $\sigma_1$ et $\sigma_2$ tandis que $v_0 v_1 v_0 (v_2 v_4)^{\omega}$ n'est pas consistant avec $\sigma_1$.
		
		
	
	\begin{figure}[ht!]
		\centering
		\begin{tikzpicture}
			\node[nR] (v3) at (4,-2){$v_{3}$};
			\node[nC] (v2) at (4,0){$v_{2}$};
			\node[nR] (v1) at (2,0){$v_{1}$};
			\node[nR] (v0) at (0,0){$v_{0}$};
			\node[nR] (v4) at (6,0){$v_{4}$};

			\draw[->,>=latex] (v0.north) to [out=95,in= 80] (v1.north);
			\draw[->,>=latex] (v1) to  (v2);
			\draw[->,>=latex] (v1) to  (v0);
			\draw[->,>=latex] (v2) to  (v4);

			\draw[->,>=latex] (v3.east) to [out=0,in= 0]  (v4.east);
			\draw[->,>=latex] (v4.north) to [out=95,in= 80]  (v2.north);

			\draw[->,>=latex] (v2) to (v3);
			\draw[->,>=latex] (v4) to (v3);

			\draw[->,>=latex] (v3) to (v0);
		\end{tikzpicture}
		\caption{Arène d'un jeu sur graphe}
		\label{ex:jeuSurGraphe}
	\end{figure}
\end{exemple}

Comme nous l'avons dit précédemment, lorsque l'on définit un jeu sur graphe nous devons préciser l'arène du jeu ainsi que les objectifs que chaque joueur doivent atteindre. Dans la section suivante nous allons nous intéresser à un certain type d'objectif: l'objectif d'atteignabilité.
