%!TEX root=main.tex

\subsection{Jeux quantitatifs }

Contrairement aux jeux à objectif qualitatif pour lesquels l'objectif d'un joueur est d'assurer qu'une certaine propriété soit vérifiée, pour le cas des \textit{jeux à objectif quantitatif} une certaine valeur quantifiée est associée à chaque jeu. Le but d'un joueur sera donc de \textit{maximiser} ou de \textit{minimiser} cette valeur afin que sa satisfaction soit maximale.\\

Nous introduisons cette section par un exemple, ensuite nous aborderons les notions essentielles aux jeux quantitatifs en distinguant les \textit{jeux multijoueurs} et les \textit{jeux à deux joueurs}.\\
Les définitions et les notions sont inspirées de l'article de Brihaye \emph{at al.} \cite{DBLP:conf/lfcs/BrihayePS13}.\\

\noindent\textbf {Exemple introductif} \\
\indent Antoine et Thomas désirent se rendre à l'école à pied. Le chemin étant long, Antoine propose à Thomas de jouer à un jeu. A chaque carrefour et à tour de rôle un des deux garçons choisit la route à emprunter. A chaque mètre parcouru Thomas devra donner un bonbon à Antoine. L'objectif de Thomas est donc clairement d'atteindre le plus rapidement possible l'école afin de minimiser son coût, tandis que l'objectif d'Antoine est d'emprunter le plus long chemin pour obtenir le plus de bonbons possible.\\

Au vu de cet exemple, il est clair que le modèle des jeux qualitatifs n'est pas suffisant pour modéliser cette situation. En effet, il ne permet pas de caractériser le fait que plus Antoine obtiendra de bonbons plus il sera satisfait et inversement pour Thomas. Pour ce faire, nous devons introduire les concepts de \textit{fonction de gain} (ou \textit{fonction de coût} en fonction du point de vue duquel on se place) ainsi qu'un nouveau concept de solution pour ces jeux appelés \textit{jeu avec coût (cost games)} : les \textit{équilibres de Nash}. Nous supposons que dans de tels jeux les joueurs sont \textit{rationnels} c'est-à-dire qu'ils jouent de telle sorte à maximiser leur gain ou minimiser leur coût.\\

%----------------------------
%Jeu multijoueur avec coût
%----------------------------
%!TEX root=main.tex


\subsubsection{Jeux multijoueurs avec coût}
% DEFINITION: mutliplayer cost game

\begin{defi}[Jeu multijoueur avec coût]
	Soit $\mathcal{A} = (\Pi, V, (V_{i})_{i \in \Pi},E)$ une arène,
	un \textit{jeu multijoueur avec coût} est un tuple $\mathcal{G} = (\mathcal{A},(Cost_{i})_{i \in \Pi})$ où
	\begin{enumerate}
		\item[$\bullet$] $\mathcal{A} = (\Pi ,V ,(V_{i})_{i \in \Pi} ,E )$ est l'arène d'un jeu sur graphe.
		\item[$\bullet$] $Cost_{i}: Plays \rightarrow \mathbb{R} \cup \{ +\infty , -\infty \} $ est la \textit{fonction de coût} de $J_{i}$. 
	\end{enumerate}
\end{defi}


	Pour chaque $\rho \in Plays$, $Cost_{i}(\rho)$ représente le montant que $J_{i}$ perd quand le jeu $\rho$ est joué.
	Le but de chaque joueur est donc de \textbf{minimiser} sa fonction de coût.

%EXEMPLE: fonctions de coût.
\begin{exemple}[Fonctions de coût]
	\label{ex:fonctionsCout}
  Dans le cadre de ce projet, nous nous intéressons aux jeux sur graphe tels que l'objectif des joueurs est un objectif quantitatif. De plus, nous souhaitons que l'objectif des joueurs soit atteint le plus rapidement possible. Les fonctions de coût qui nous intéressent sont donc les suivantes: \\
	
	Pour tout  $\rho = \rho _{0} \rho _{1} \rho _{2} \ldots $ où $\rho \in Plays$ on définit:
	\begin{enumerate}
	\item $Cost_{i}(\rho) = $ $\begin{cases} 
									\min \{ i | \rho _{i} \in Goal_{i} \} & \text{si } \exists i \text{ tq } \rho _{i} \in Goal_{i} \\
									+\infty & \text{ sinon}
									\end{cases}$
	\item $\varphi _{i}(\rho) = $ $\begin{cases}
									\sum_{i = 0}^{n-1} w(\rho_{i},\rho_{i+1}) & \text{ si } n \text{ est le plus petit indice tq } \rho_{n}\in 					  Goal_{i}\\
									+\infty & \text{sinon}
									\end{cases}$ \\
									où $w$ est une \textit{fonction de poids} (cf. définition \ref{def:fonctionPoids}).
	\end{enumerate}
\end{exemple}

\begin{rem}
	L'exemple 1 est un cas particulier de l'exemple 2 avec $w(\rho_{i},\rho_{i+1}) = 1$ pour tout $i$.
\end{rem}
%DEFINITION: poids d'un arc

\begin{defi}[Fonction de poids]
	\label{def:fonctionPoids}
	A chaque arc d'un graphe $G = (V,E)$ on peut y associer un \textit{poids} (\emph{i.e.,} une valeur chiffrée). On associe donc à $G$ une \textit{fonction de poids}  $w : E \rightarrow \mathbb{R}$. On dit alors que $G$ est un graphe \textit{pondéré}.
\end{defi}

\begin{exemple}
	Si chaque $v \in V$ représente une ville sur une carte, alors on peut imaginer une fonction de poids représentant une des valeurs suivantes:
	\begin{enumerate}
		\item [$\bullet$] le nombre de kilomètres entre deux villes,
		\item [$\bullet$] le temps pour aller d'une ville à l'autre,
		\item [$\bullet$] la consommation d'essence pour aller d'une ville à l'autre.
	\end{enumerate}
\end{exemple}


%DEFINITION: jeu d'atteignabilité multijoueur à objectif quantitatif

\begin{defi}[Jeu d'atteignabilité multijoueur à objectif quantitatif]
	
	Un \textit{jeu d'atteignabilité multijoueur à objectif quantitatif} est un jeu multijoueur avec coût $\mathcal{G} = (\Pi ,V ,(V_{i})_{i \in \Pi} ,E ,(Cost_{i})_{i \in \Pi})$ tel que pour tout joueur $i \in \Pi$ $Cost_{i} = \varphi _{i}$ pour un certain $Goal _{i} \subseteq V$.
	On note ces jeux $\mathcal{G} = (\mathcal{A},(\varphi _{i})_{i\in \Pi},(Goal_{i})_{i \in \Pi})$.
\end{defi}
	



%EXEMPLE: graphe pondéré



%!TEX root=main.tex

\begin{exemple}[Jeu avec un graphe pondéré]
	\label{ex:graphePond}
	
Soit un jeu $\mathcal{G} = ( \Pi, V, (V_{1},V_{2}), E, (Goal_{1},Goal_{2}))$ tel que $\Pi = {1,2}$, $V_{1} = \{ v_{0}, v_{1}, v_{3} \}$, $V_{2} = \{ v_{2}\}$, $Goal_{1} = \{ v_{3}\}$, $Goal_{2} = \{ v_{0} \}$, $w(v_{i},v_{j}) = 1$ pour tout $0 \leq i,j \leq 3 , i \neq j$. Cet exemple est illustré par la figure \ref{ex:graphePond1}.

\begin{figure}[ht!]
	\centering

	\begin{tikzpicture}
		
		\node[nRB] (v3) at (4,-2){$v_{3}$};
		\node[nC] (v2) at (4,0){$v_{2}$};
		\node[nR] (v1) at (2,0){$v_{1}$};
		\node[nRG] (v0) at (0,0){$v_{0}$};
	
		\draw[->,>=latex] (v0.north) to [out=95,in= 80] node[midway,above]{$1$}(v1.north);
		\draw[->,>=latex] (v1) to node[midway,above]{$1$} (v2);
		\draw[->,>=latex] (v1) to node[midway,above]{$1$} (v0);
		
		\draw[->,>=latex] (v2) to node[midway,right]{$1$} (v3);
		\draw[->,>=latex] (v3) to node[midway,above]{$1$} (v0);
		
	\end{tikzpicture}
	
	
	\caption{Jeu avec un graphe pondéré }
	\label{ex:graphePond1}
	

\end{figure}
\end{exemple}

\FloatBarrier


%DEFINITION: équilibre de Nash

\begin{defi}[Equilibre de Nash]
	
	Soit $(\mathcal{G}, v_{0})$ un \textit{jeu multijoueur avec coût et initialisé}, un profil de stratégie $(\sigma _{i})_{i \in \Pi}$ est un \textit{équilibre de Nash} dans $(\mathcal{G}, v_{0})$ si, pour chaque joueur $j \in \Pi$ et pour chaque stratégie $\tilde{\sigma}_{j}$ du joueur $j$, on a :
	\begin{center}$ Cost_{j}(\rho) \leq Cost_{j}(\tilde{\rho})$ \end{center}
	où $\rho = \langle (\sigma _{i})_{i \in \Pi}\rangle_{v_0}$ et $\tilde{\rho} = \langle \tilde{\sigma} _{j} ,\sigma _{-j}\rangle_{v_0}$.
\end{defi}	


%DEFINITION: déviation profitable

\begin{defi}[Déviation profitable]
	
	Soit $(\mathcal{G}, v_{0})$ un \textit{jeu multijoueur avec coût et initialisé}, soit $(\sigma _{i})_{i \in \Pi}$ un profil de stratégie, $\tilde{\sigma _{j}}$ est une \textit{déviation profitable} pour le joueur $j$ relativement à $(\sigma _{i})_{i \in \Pi}$ si:
	\begin{center} $ Cost_{j}(\rho) > Cost_{j}(\tilde{\rho})$ \end{center}
	où $\rho = \langle (\sigma _{i})_{i \in \Pi} \rangle_{v_0}$ et $\tilde{\rho} = \langle \tilde{\sigma} _{j} ,\sigma _{-j} \rangle_{v_0}$. 
\end{defi}

%REMARQUE: signification de la def d'EN
\begin{rem}
	$(\sigma _{i})_{i\in \Pi}$ est un équilibre de Nash si tout joueur $j \in \Pi$ n'a aucun intérêt à dévier de sa stratégie $\sigma _{j}$ si les autres joueurs suivent $\sigma _{-j}$. C'est-à-dire qu'aucun joueur n'a de déviation profitable. 
\end{rem}

%EXEMPLE D'EN

%!TEX root=main.tex

\begin{exemple}
	Considérons le jeu décrit dans l'exemple \ref{ex:graphePond} muni pour $J_{1}$ et $J_{2}$ de la fonction de coût $\varphi _{i}$ de l'exemple \ref{ex:fonctionsCout} (p.\pageref{ex:fonctionsCout}) et donnons un exemple d'équilibre de Nash de ce jeu.\\
	
	 \begin{minipage}[c]{0.30\linewidth}Soit $\sigma _{1}(v) =$ $\begin{cases}
						 v_{1} & \text{ si } v = v_{0}\\
						 v_{2} & \text{ si } v = v_{1}\\
						 v_{0} & \text{ si } v = v_{3}
						\end{cases}$ \end{minipage} \hfill
	\begin{minipage}[c]{0.30\linewidth}\center{et}\end{minipage} \hfill \begin{minipage}[c]{0.30\linewidth}	\center{$\sigma _{2}(v) = v_{3}$ si $v = v_{2}$} \end{minipage} \newline
		
\noindent	(rem: ces deux stratégies sont des stratégies sans mémoire), $(\sigma _{1}, \sigma _{2})$ est un équilibre de Nash de $(\mathcal{G}, v_{0})$. De plus, soit $\rho = Outcome(v_{0},(\sigma _{1},\sigma _{2}))$ nous avons : $Cost_{1}(\rho) = 3$ et $Cost_{2}(\rho) = 0$.\\
\begin{demonstration}	
	Les seules déviations à considérer sont celles quand $J_{1}$ est en $v_{1}$ car il a le choix de se rendre en $v_{0}$ ou en $v_{2}$.\\ Considérons $\tilde{\sigma _{1}}(v) = $ $\begin{cases}
						 v_{1} & \text{ si } v = v_{0}\\
						 v_{0} & \text{ si } v = v_{1}\\
						 v_{0} & \text{ si } v = v_{3}
						\end{cases}$ .
Dans ce cas, soit $\tilde{\rho} = Outcome(v_{0},(\tilde{\sigma _{1}},\sigma _{2}))$, nous avons : \mbox{$Cost_{1}(\tilde{\rho}) = + \infty$}. Ce n'est donc pas une déviation profitable.\\
Considérons maintenant la famille $(\sigma _{1}^{n})_{n \in \mathbb{N}}$ de stratégie de $J_{1}$ qui décrit le fait que $J_{1}$ fait passer $n$ fois le "jeton" par l'état $v_{1}$ avant de le faire glisser vers l'état $v_{2}$. \\

$\sigma _{1}^{n}(h) = $ $\begin{cases}
					 v_{1} & \text{ si } h = h'v_{0}\\
					 v_{0} & \text{ si } h = h'v_{3}\\
					 v_{2} & \text{ si } h = h'v_{1} \text{ et } \exists h_{1},\ldots,h_{n} \in h' \text{ tq } h_{1},\ldots,h_{n} = v_{1} \\
					v_{0} & \text{ sinon}
					
					\end{cases}$ .
					
\noindent Nous avons alors: soit $ p = Outcome(v_{0},(\sigma _{1}^{n},\sigma _{2}))$, $Cost_{1}(p) = 2n+1$. Et pour tout $n > 1$, on a que $3=Cost_{1}(\rho) < Cost_{1}(p) = 2n +1$. $\sigma _{1}^{n}$ n'est donc pas une déviation profitable pour $J_{1}$ et ce pour tout $n \in  \mathbb{N}$.\\

Comme $\sigma _{2}$ est la seule stratégie possible pour $J_{2}$, nous pouvons déduire qu'aucun des deux joueurs n'a de déviation profitable. Donc $(\sigma _{1}, \sigma _{2})$ est un équilibre de Nash de $(\mathcal{G},v_{0})$.

\end{demonstration}			 		
    
\end{exemple}



% Résultat d'existence d'un équilibre de Nash à petite mémoire

 Dans~\cite{DBLP:conf/lfcs/BrihayePS13}, le théorème suivant est énoncé et prouvé:

\begin{thm}
	Soient $\mathcal{A} = (\Pi, V, (V_{i})_{i \in \Pi}, E)$ et $\mathcal{G} = (\mathcal{A},(\varphi _{i})_{i \in \Pi}, (Goal_{i})_{i \in \Pi})$ un jeu d'atteignabilité multijoueur à objectif quantitatif, il existe un équilibre de Nash dans tout jeu initialisé ($\mathcal{G},v_{0}$) avec $v_{0}\in V$. De plus, cet équilibre possède une mémoire d'au plus $|V| + |\Pi|$.
\end{thm}



%----------------------------------------
% Jeu avec coût MIN-MAX
%---------------------------------------
%!TEX root=main.tex

\subsubsection{Jeux Min-Max avec coût}

% DEFINITION: jeu Min-Max avec coût

\begin{defi}[Jeu Min-Max avec coût] $\text{ }$\\
	Soit une arène $\mathcal{A} = (V, (V_{Min}, V_{Max}), E) $
	Un \textit{jeu Min-Max avec coût} est un tuple $\mathcal{G} = (\mathcal{A}, Cost_{Min}, Gain_{Max})$, où
	\begin{enumerate}
		\item[$\bullet$] $Cost_{Min}: Plays \rightarrow \mathbb{R} \cup \{+ \infty, -\infty \}$ est la \textit{fonction de coût} du joueur \textit{Min}.
		\item[$\bullet$] $Gain_{Max}: Plays \rightarrow \mathbb{R} \cup \{ + \infty, -\infty \}$ est la \textit{fonction de gain} du joueur \textit{Max}.
	\end{enumerate}
		
\end{defi}

\begin{rem}
	Dans cette définition, on sous-entend que $\Pi = \{ Min, Max \}$.
\end{rem}

Pour chaque $\rho \in Plays$, $Cost_{Min}(\rho)$ représente le montant que \textit{Min} perd quand le jeu $\rho$ est joué et $Gain_{Max}(\rho)$ représente le gain que \textit{Max} gagne quand le jeu $\rho$ est joué.
Le but de \textit{Min} (resp. \textit{Max}) est donc de \textbf{minimiser} (resp. \textbf{maximiser}) sa fonction de coût (resp. fonction de gain). Ce qui explique le choix des noms des joueurs.\\

%DEFINITION: jeu Min-Max à somme nulle

\begin{defi}[Jeu à somme nulle]
	Un jeu Min-Max avec coût est dit \textit{à somme nulle} si $Gain_{Max} = Cost_{Min}$.
\end{defi}

% DEFINITION: garantir le paiement

\begin{defi}[Garantir le paiement]$\text{}$\\
	
	\begin{center}On dit que le joueur \textit{Max} \textit{garantit le paiement} $d \in \mathbb{R}$\\ 
		ssi \\ 
	$\exists \sigma _{1}\Sigma _{Max}$ tq $\forall \sigma _{2} \in \Sigma _{Min}$ $ Gain_{Max}(\sigma _{1},\sigma _{2}) \geq d$\\\end{center}
	
	\begin{center} 
		On dit que le joueur \textit{Min} \textit{garantit le paiement} $d \in \mathbb{R}$\\		
		ssi	\\
		$\exists \sigma _{1}\in \Sigma _{Min}$ tq $\forall \sigma _{2} \in \Sigma _{Max}$ $ Cout_{Min}(\sigma _{1},\sigma _{2}) \leq d$\\
		\end{center}

\end{defi}
		

% DEFINITION: Valeur supérieure et valeur inférieure

\begin{defi}[Valeur inférieure/supérieure]
	
	Soit $\mathcal{G}$ un jeu Min-Max avec coût, on définit pour chaque sommet $v \in V$: 
	\begin{enumerate}
		\item[$\bullet$]\textit{Valeur supérieure:} $\overline{Val}(v) = \inf\limits_{\sigma _{1} \in \Sigma _{Min}} \sup\limits_{\sigma _{2} \in \Sigma_{Max}} Cost_{Min}(\rho)$ où $\rho = Outcome(v,(\sigma _{1},\sigma _{2}))$
		
		\item[$\bullet$]\textit{Valeur inférieure:} $\underline{Val}(v) = \sup\limits_{\sigma _{2} \in \Sigma_{Max}}  \inf\limits_{\sigma _{1} \in \Sigma _{Min}} Gain_{Max}(\rho)$  où $\rho = Outcome(v,(\sigma _{1},\sigma _{2}))$
	\end{enumerate}
\end{defi}
\begin{rem}
	La \textit{valeur supérieure}  $\overline{Val}(v)$ est la plus grande valeur que $J_{Min}$ peut perdre et la \textit{valeur inférieure} $\underline{Val}(v) $ est la plus grande valeur que $J_{Max}$ peut gagner.
\end{rem}

%PROPRIETE 
\begin{propriete}
	Pour tout $v \in V$, on a : $\underline{Val}(v) \leq \overline{Val}(v)$
\end{propriete}

%DEFINITION: jeu déterminé et valeur d'un jeu

\begin{defi}[Jeu déterminé et valeur d'un jeu]
		Soit $\mathcal{G}$ un jeu Min-Max avec coût, on dit que $\mathcal{G}$ est \textit{déterminé} si pour tout $v \in V$, $\overline{Val}(v) = \underline{Val}(v)$. On dit alors que le jeu $\mathcal{G}$ a une \textit{valeur} et pour tout $v \in V$ on note $Val(v) = \overline{Val}(v) = \underline{Val}(v)$.
\end{defi}

%DEFINITION: Stratégie optimale

\begin{defi}[Stratégie $\epsilon$-optimale]
	Soit $\epsilon > 0$,
	\begin{enumerate}
	\item[$\bullet$] $\sigma _{1} \in \Sigma _{Max}$ est une \textit{stratégie $\epsilon$-optimale} ssi $\forall v \in V $ $ \forall \sigma _{2}\in \Sigma_{Min}$ $ Gain_{Max}(\rho) \geq \underline{Val}(v) + \epsilon  $\\ où $\rho = Outcome(v, (\sigma _{1},\sigma _{2}))$.
	\item[$\bullet$] $\sigma _{2} \in \Sigma _{Min}$ est une \textit{stratégie $\epsilon$-optimale} ssi $\forall v \in V $ $ \forall \sigma _{1}\in \Sigma_{Max}$ $Cost_{Min}(\rho) \leq \overline{Val}(v) + \epsilon $\\ où $\rho = Outcome(v, (\sigma _{1},\sigma _{2}))$.
	\item[$\bullet$] Si $\epsilon = 0$, on dit que la stratégie $\sigma _{i}$ est \textit{optimale}
	\end{enumerate}
\end{defi}

%DEFINITION: reachability-price game

\begin{defi}[Reachability-price game]
	Soit $\mathcal{A} = (V, (V_{Min}, V_{Max}), E) $,soit $w: E \rightarrow \mathbb{R}$ une fonction de poids,
	un \textit{"reachability-price game"} est un jeu Min-Max avec coût $\mathcal{G} = (\mathcal{A},RP_{Min},RP_{Max})$\\ avec un objectif donné $Goal \subseteq V$, où pour tout $\rho \in Plays$ tq $\rho = \rho _{0}\rho _{1}...$:\\
	
	$RP_{Min}(\rho)=RP_{Max}(\rho) =$ $\begin{cases}
									\sum_{i = 0}^{n-1} w(\rho_{i},\rho_{i+1}) & \text{ si } n \text{ est le plus petit indice tq } \rho_{n}\in 					  Goal\\
									+\infty & \text{sinon}
									\end{cases}$ \\
									
  \noindent Ce jeu est un jeu à somme nulle.
\end{defi}

% EXEMPLE : reachability-price game + jeu déterminé + valeur + stratégie optimale

%!TEX root=main.tex

\begin{exemple}
	Soit $\mathcal{G} = (V,(V_{Min},V_{Max}),E,RP_{Min},RP_{Max})$ le \og\textit{reachability-price game}\fg décrit par la figure \ref{ex:reachPriceGame1} où les sommets contrôlés par $J_{Min}$ sont les sommets ronds et les sommets contrôlés par $J_{Max}$ sont les sommets carrés et $Goal = \{ v_{3} \}$.
	
	\begin{figure}[ht!]
		\centering

		\begin{tikzpicture}

			\node[nRG] (v3) at (4,-2){$v_{3}$};
			\node[nC] (v2) at (4,0){$v_{2}$};
			\node[nR] (v1) at (2,0){$v_{1}$};
			\node[nR] (v0) at (0,0){$v_{0}$};
			\node[nR] (v4) at (6,0){$v_{4}$};

			\draw[->,>=latex] (v0.north) to [out=95,in= 80] node[midway,above]{$1$}(v1.north);
			\draw[->,>=latex] (v1) to node[midway,above]{$1$} (v2);
			\draw[->,>=latex] (v1) to node[midway,above]{$1$} (v0);
			\draw[->,>=latex] (v2) to node[midway,above]{$1$} (v4);
			

			\draw[->,>=latex] (v2) to node[midway,right]{$1$} (v3);
			\draw[->,>=latex] (v4) to node[midway,right]{$1$} (v3);
			
			\draw[->,>=latex] (v3) to node[midway,above]{$1$} (v0);





		\end{tikzpicture}


		\caption{reachability-price game}
		\label{ex:reachPriceGame1}


	\end{figure}
	
Dans un premier temps, montrons que $\mathcal{G}$ est déterminé. Nous avons : $\underline{Val}(v_{0})=\overline{Val}(v_{0})=4$,  $\underline{Val}(v_{1})=\overline{Val}(v_{1}) =3$, $\underline{Val}(v_{2})=\overline{Val}(v_{2})= 2$,  $\underline{Val}(v_{3})=\overline{Val}(v_{3})= 0$, $\underline{Val}(v_{4})=\overline{Val}(v_{4})= 1$.\\


Ensuite, exhibons une stratégie optimale pour le joueur \textit{Min} et une stratégie optimale pour le joueur \textit{Max}.\\

\noindent Prenons la stratégie sans mémoire de $J_{Min}$ définie comme suit : $\sigma _{1}(v) = $ $\begin{cases}
																						v_{0} & \text{si }v = v_{3}\\
																						v_{1} & \text{si }v = v_{0}\\
																						v_{2} & \text{si }v = v_{1}\\
																						v_{3} & \text{si }v = v_{4}
																					\end{cases}$.

\noindent Pour $J_{Max}$ considérons la stratégie sans mémoire suivante : 																					$\sigma _{2}(v) = $ $\begin{cases}
					v_{4} & \text{si } v = v_{2}
																																										
																																										\end{cases}$.\\
																																										
						
\end{exemple}

Dans ~\cite{DBLP:conf/lfcs/BrihayePS13} le théorème suivant est énoncé:

\begin{thm}
	\label{thm:1}
	Les \og\textit{reachability-price games}\fg sont déterminés et ont des stratégies optimales sans mémoire.
\end{thm}


Comme dans le cas des jeux d'atteignabilité à objectif qualitatif et au vu du résultat ~\ref{thm:1} précédant nous nous interrogeons quant à la façon d'implémenter un algorithme pour résoudre les \textit \og \textit{reachability-price games} \fg.L'objectif de cet algorithme est de trouver une stratégie optimale pour chaque joueur ainsi que la valeur associée à chaque sommet du graphe. L'idée est la suivante: le joueur \textit{Min} a pour but d'emprunter un \textbf{plus court chemin} possible allant d'un noeud initial $v_{0}$ vers un noeud $v \in Goal$. On trouve dans la littérature différents algorithmes qui résolvent les problèmes de plus court chemin dans les graphes orientés.Toutefois, il n'y a pas de deuxième joueur qui agit de manière \textbf{antagoniste} face au joueur \textit{Min} qui entre en compte dans ces algorithmes. C'est pourquoi nous avons tenté une adaptation de l'algorithme de \textit{Dijkstra} (que nous rappelons dans l'annexe A (p.\pageref{algo:dijkstra})).

Comme dans le cas de l'algorithme de Dijkstra, la fonction de coût associée au graphe doit être de la forme $w : E \rightarrow \mathbb{R}^{+}$. Le but de l'algorithme est de calculer le paiement minimum que le joueur \textit{Min} peut garantir quelle que soit la stratégie adoptée par le joueur \textit{Max}. Du fait que le joueur \textit{Max} joue de manière antagoniste par rapport au joueur \textit{Min}, à chaque fois que c'est au joueur \textit{Max} de prendre une décision il voudra maximiser son gain. Pour ce faire, on aura besoin de connaître le chemin le plus coûteux allant du sommet du joueur \textit{Max} en cours de traitement vers un certain état objectif $o \in Goal$. Dès lors, contrairement à l'algorithme classique de Dijkstra qui part d'une source, l'algorithme s'exécutera à rebours à partir des états $o \in Goal$. Ci-dessous nous reprenons les idées essentielles de l'algorithme proposé ainsi que son pseudo-code.\\

\noindent \textbf{Idées de l'algorithme}\\

\begin{enumerate}
	
	\item[$\bullet$] A tout sommet $v \in V$ on associe une valeur $d$ qui représente l'estimation de la valeur $Val(v)$ (qui existe par le théorème ~\ref{thm:1}). Cette valeur est mise à jour en cours d'exécution de l'algorithme de sorte qu'à la fin de celle-ci on ait pour tout $v \in V$ $Val(v) = d$. Comme pour l'algorithme de Dijkstra on initialise la valeur des sommets à $+\infty$ sauf pour les sommets objectifs $o \in Goal$ pour lesquels on initialise la valeur à 0.
	
	\item[$\bullet$] De plus, pour tout sommet $v \in V$, on associe une \textit{file de priorité} $S$ dans laquelle on stocke chaque successeur $s$ de $v$ déjà relaxé (ie. dont on a fini le traitement). A chacun de ces successeurs on joint l'estimation de  $Val(v)$ si l'$Outcome$ associé à cette estimation est de la forme $v s ... o$ pour $o \in Goal$. On note les éléments de $S$: $(val,succ)$.
	
	\item[$\bullet$] Pour tout sommet $v \in V_{Max}$, on associe le nombre de successeurs que ce sommet possède. On note ce nombre : $nbrSucc$. De plus, si l'on désire reconstituer la stratégie optimale pour le joueur \textit{Max} on stockera dans une structure de données adéquate la liste des successeurs déjà testés pour la recherche du chemin le plus long.
	
		\item[$\bullet$] On utilise une \textit{file de priorité} $Q$ (structure de données permettant de stocker des éléments en fonction de la valeur d'une clef) qui permet de stocker les sommets classés par leur valeur $d$. Sur cette file de priorité on peut effectuer les opérations suivantes: insertion d'un élément, extraction d'un élément ayant la clef de la plus petit valeur,lecture de l'élément ayant la clef de la plus petite valeur,test de la présence ou non d'élément dans $Q$, augmentation ou diminution de la valeur de la clef associée à un sommet. A l'initialisation de l'algorithme les sommets présents dans $Q$ sont tous ceux présents dans $V$.
	
	\item[$\bullet$] On maintient $T \subseteq V$ un ensemble de sommets qui vérifient la propriété suivante: pour tout sommet $v \in V$ $Val(v) = d$. Il s'agit donc de l'ensemble des sommets dont le traitement est terminé. A l'initialisation de l'algorithme $T = \emptyset$.
	
	\item[$\bullet$] Tant que $Q \neq \emptyset$, l'algorithme procède de la manière suivante: 
	\begin{enumerate}
		\item On regarde la valeur du minimum de $Q$ et le sommet $s$ associé.
		\item Si cette valeur est $+\infty$, alors cela signifie qu'on a fini de traiter tous les sommets qui pouvaient atteindre un sommet objectif. On ajoute donc tous les sommets restants de $Q$ dans $T$.
		\item S'il s'agit d'un sommet du joueur \textit{Min} ou d'un sommet objectif $o$ alors on extrait le minimum de $Q$ et on l'ajoute à $T$, on \textit{relaxe} tous les arcs entrant de $s$.
		\item S'il s'agit d'un sommet du joueur \textit{Max}, on regarde la valeur de $nbrSucc$ si elle est égale à 1 alors le joueur \textit{Max} n'a pas le choix du chemin à emprunter. On ajoute donc $s$ à $T$ et on \textit{relaxe} tous les arcs entrant de $s$. Si $nbrSucc \neq 1$, alors \textit{Max} a plusieurs choix de chemin. Pour maximiser son gain, il ne choisira pas de passer par le successeur qui admet la plus petite valeur de $Val(s)$. On retire donc la plus petite valeur $S$, on met à joueur la clef de $s$ dans $Q$ et on décrémente $nbrSucc$ ainsi on sait qu'il y a un successeur de moins à traiter et on ajoute ce successeur à l'ensemble des successeurs déjà traités. En effet, au bout du compte, on désire que la dernière valeur restante dans $s.Q$ soit celle qui assure la maximisation du gain de \textit{Max}.
	\end{enumerate}
	
	\item[$\bullet$] La \textit{relaxation} des arcs entrants de $s$ consiste en la méthode suivante: pour tout $(p,s)\in E$ on lit le minimum de $s.S$(la file de priorité des valeurs associées aux successeurs du sommet $s$), on calcule la nouvelle estimation de la valeur du sommet $p$, on insère dans $p.S$ la valeur calculée associée à $s$; on décrémente la clef associée à $p$ dans $Q$.
	
\end{enumerate}

\noindent \textbf{Pseudo-code}\\

Soient $\mathcal{A} = (V, (V_{Min}, V_{Max}), E) $ une arène et $\mathcal{G} = (\mathcal{A},RP_{Min},RP_{Max})$ un \og \textit{reachability-prince game} \fg. Nous utilisons les notations suivante : 
\begin{enumerate} 
	\item[$\bullet$]$Q$ et $S$ sont les files de priorité décrites ci-dessus.
	\item[$\bullet$]$s.S$  correspond à la file de priorité $S$ du sommet $s$.
	\item[$\bullet$]$T$ est le sous-ensemble de $V$ décrit ci-dessus.
	\item[$\bullet$]Pour tout $v \in V$, $Pred(v)$ est l'ensemble des prédécesseurs de $v$.
	\item[$\bullet$]Pour tout $v \in V$, $v.d$ est l'estimation de $Val(v)$.
	\item[$\bullet$]Pour tout $v \in V_{Max}$, $v.nbrSucc$ est le nombre de successeurs de $v$ qu'il reste à traiter, $v.t$ est l'ensemble des successeurs de $v$ déjà testés.
\end{enumerate}

L'algorithme en lui même est explicité par l'algorithme ~\ref{algo:dijkMinMax} et les algorithmes \ref{algo:initS},\ref{algo:initQ},\ref{algo:relaxerMinMax},\ref{algo:traiterMax} sont appelés au sein de l'algorithme ~\ref{algo:dijkMinMax}.
Comme pour l'algorithme de Dijkstra, pour une meilleure complexité les files de priorités sont supposées implémentées par une structure de donnée telle que chaques opérations \textsc{Extraire-Min}() et \textsc{Lire-Min()} est en $\mathcal{O}(\log V)$ ainsi que chaque \textsc{DécrémenterClef}($Clef(v),nouvVal$) alors l'agorithme est en $\mathcal{O}((V + E) \log V)$.\\ 

%ALGO: DijkstraMinMax

\begin{algorithm}
	\caption{\textsc {DijkstraMinMax}(G,w,o)}
	 \label{algo:dijkMinMax}
	\begin{algorithmic}[1]
		\REQUIRE $G = (V,E)$ un graphe orienté pondéré où $E$ est représenté par sa matrice d'adjacence, $w: E \rightarrow \mathbb{R}^{+}$ une fonction de poids,$o$ le sommet objectif.
		\ENSURE / \textbf{\textsc{Effet(s) de bord :}} Calcule pour chaque noeud la valeur de ce noeud.
		
		\STATE $Q \leftarrow$\textsc{Initialiser-Q}($G,o$)
		\STATE $S \leftarrow$\textsc{Initialiser-S}($G,o$)
		
		\STATE $T \leftarrow \emptyset$
		\WHILE {$Q \neq \emptyset$}
			\STATE $s \leftarrow Q.\textsc{Lire-Min()}$
			\STATE $(val,succ) \leftarrow s.S.$\textsc{Lire-Min}$()$
			\IF{$val = +\infty$}
				\WHILE{$Q \neq \emptyset$}
					\STATE $u \leftarrow Q.$\textsc{Extraire-Min}$()$
					\STATE $T.$\textsc{Insérer}$(u)$
				\ENDWHILE
			
			
			\ELSE
				\IF{$s \in V_{Min} \cup \{ o \}$ }
					\STATE $s \leftarrow Q.$\textsc{Extraire-Min}$()$
					\STATE $T.$\textsc{Insérer}$(s)$
					\FORALL{$p \in Pred(s)$}
						\STATE \textsc{Relaxer}$(p,s,w)$
					\ENDFOR
				
				
				\ELSE
					\STATE \textsc{Traiter-Max}$(s)$
				\ENDIF
			\ENDIF
		\ENDWHILE
				
			
\end{algorithmic}
		
\end{algorithm}

%ALGO: Initialiser-S

\begin{algorithm}
	\caption{\textsc {Initialiser-S}($G,o)$}
	 \label{algo:initS}
	\begin{algorithmic}[1]
		\REQUIRE $G$ un graphe orienté pondéré et le noeud objectif $o$.
		\ENSURE / \textbf{\textsc{Effet(s) de bord :}} initialise les files de priorités de tous les sommets.
		
		\FORALL{$v \in G.V\backslash \{ Goal \}$}
			\STATE $S \leftarrow$ nouveau tas
			\STATE $v.S \leftarrow S$
			\STATE $v.S.$\textsc{Insérer}$(+\infty, NULL)$
		\ENDFOR
		\FORALL{$o \in Goal$}
			\STATE $S \leftarrow$ nouveau tas
			\STATE $o.S \leftarrow S$
			\STATE $o.S.$\textsc{Insérer}$(0, NULL)$
		\ENDFOR
	
			
\end{algorithmic}
		
\end{algorithm}

%ALGO: Initialiser-Q


\begin{algorithm}
	\caption{\textsc {Initialiser-Q}$(G,o)$}
	 \label{algo:initQ}
	\begin{algorithmic}[1]
		\REQUIRE $G$ un graphe orienté pondéré, l'état objectif $o$.
		\ENSURE Un tas $Q$ qui comprend tous les sommets de $V$ classés par leur valeur $d$.
		
		\STATE $Q \leftarrow$ nouveau tas 
		\FORALL{$v \in G.V \backslash \{ Goal \}$}
			\STATE $v.d \leftarrow +\infty$
			\STATE $Q.$\textsc{Insérer}(v)
		\ENDFOR
		\FORALL{$o \in Goal$}
			\STATE $o.d \leftarrow 0$
			\STATE $Q.$\textsc{Insérer}(o)	
		\ENDFOR
		
		\RETURN $Q$
	
			
\end{algorithmic}
		
\end{algorithm}

%ALGO:Relaxer

\begin{algorithm}
	\caption{\textsc {Relaxer}$(p,s,w)$}
	 \label{algo:relaxerMinMax}
	\begin{algorithmic}[1]
		\REQUIRE deux sommets $p$ et $s$, une fonction de poids $w : E \rightarrow \mathbb{R}^{+}$.
		\ENSURE / \textbf{\textsc{Effet(s) de bord :}} Ajoute à p.S une valeur pour $p$ et son successeur.
		
		\STATE $(sVal,succ)$ $\leftarrow$ \textsc{Lire-Min(s.S)}
		\STATE $pVal \leftarrow w(p,s) + sVal$
		\STATE $p.S.$\textsc{Insérer}$((pVal,s))$
		\STATE $Q.$\textsc{Décrémenter-Clef}$(p,pVal)$
			
\end{algorithmic}
		
\end{algorithm}

%ALGO:Traiter-Max

\begin{algorithm}
	\caption{\textsc {Traiter-Max}$(s)$}
	 \label{algo:traiterMax}
	\begin{algorithmic}[1]
		\REQUIRE un sommet $s$ appartenant au joueur \textit{Max}.
		\ENSURE / \textbf{\textsc{Effet(s) de bord :}} traite le sommet $s$ soit en relaxant ses arcs entrants soit en supprimant la plus petite valeur de s.S.
		
		\IF {$s.nbrSucc = 1$}
			\STATE $T.$\textsc{Insérer}(s)
			\FORALL{( $p \in Pred(s)$)}
				\STATE \textsc{Relaxer}$(p,s,w)$
			\ENDFOR
			
		\ELSE
			\STATE $(val,succ) \leftarrow$ $s.S.$\textsc{Extraire-Min}()
			\STATE $(nouvVal,nouvSucc) \leftarrow$ $s.S.$\textsc{Lire-Min}()
			\STATE $Q.$\textsc{Incrémenter-Clef}$(s,nouvVal)$
			\STATE $s.nbrSucc \leftarrow s.nbrSucc - 1 $
			\STATE $s.t.$\textsc{Insérer}$(succ)$
		\ENDIF
			
				
			
\end{algorithmic}
		
\end{algorithm}

On remarque que l'algorithme \textsc{DijkstraMinMax}  est un algorithme à effet de bord mais ne renvoie aucune valeur. Pour récupérer les valeurs de chaque sommet $s$, ainsi qu'une stratégie optimale pour $J_{Min}$ et une strategie optimale pour $J_{max}$ il suffit de récupérer la racine de $s.S$ qui comprend $Val(s)$ ainsi que le successeur qu'il faut emprunter pour obtenir cette valeur. On obtient alors les algorithmes \textsc{RécupéréStratégies} (qui récupère une stratégie optimale pour chaque joueur) et \textsc{RécupérerValeurs} qui récupère la valeur de chaque noeud).

\begin{rem}
	
	Dans le cas où $Val(v) = +\infty$, nous n'avons pas de successeur à notre disposition dans la file de priorité. Comme les fonctions de stratégie sont des fonctions totales, il faut toutefois définir $\sigma _{i} (v)$.
\begin{enumerate}
	\item[$1^{er} cas:$] Si $v \in V_{Min}$ alors cela signifie que pour tout $s \in V$ tq $(v,s) \in E$ on a : $Val(s) = +\infty$. En effet, s'il existe $s' \in V$ tq $(v,s')\in E$ et $Val(v') < +\infty$ alors $J_{Min}$ a tout intérêt à jouer $\sigma _{Min}(v) = s'$.
	\item[$2^{eme} cas:$] Si $v \in V_{Max}$ alors cela signifie qu'il existe $s \in V$ tq $(v,s) \in E$ tq : $Val(s) = +\infty$. En effet, sinon pour tout $s' \in V$ tq $(v,s') \in E$ et $Val(v') < +\infty$ alors $Val(v) \neq +\infty$. 
	
\end{enumerate}

\end{rem}

\begin{algorithm}
	\caption{\textsc{RécupérerStratégies}($G$)}
	\label{algo:recupStrat}
	\begin{algorithmic}[1]
	
	\STATE \textsc{Afficher("Stratégie optimale pour $J_{Max}$")}
	\FORALL{$v \in V_{Max}$} 
		\IF{$Val(v) = +\infty$}
			\STATE Choisir $s \in Succ(v) \backslash v.t$
			\STATE \textsc{Afficher( $v " \rightarrow " s$)}
		\ELSE
			\STATE $(val,succ) \leftarrow v.S.$\textsc{Lire-Min}$()$
			\STATE \textsc{Afficher( $v " \rightarrow " succ$)}
		\ENDIF
	\ENDFOR
	
	\STATE \textsc{Afficher("Stratégie optimale pour $J_{Min}$")}
	\FORALL{$v \in V_{Min}$} 
		\IF{$Val(v) = +\infty$}
			\STATE Choisir $s \in Succ(v)$
			\STATE \textsc{Afficher( $v " \rightarrow " s$)}
		\ELSE	
			\STATE $(val,succ) \leftarrow v.S.$\textsc{Lire-Min}$()$
			\STATE \textsc{Afficher( $v $"$ \rightarrow $" $succ$)}
		\ENDIF
	\ENDFOR
	
	\end{algorithmic}
\end{algorithm}

\begin{algorithm}
	\caption{\textsc{RécupérerValeurs}($G$)}
	\label{algo:recupVal}
	\begin{algorithmic}[1]
	
	\FORALL{$v \in V$} 
		\STATE $(val,succ) \leftarrow v.S.$\textsc{Lire-Min}$()$
		\STATE \textsc{Afficher( "La valeur de" $v$ "est " $val$)}
	\ENDFOR
	
	
	\end{algorithmic}
\end{algorithm}


\FloatBarrier
	

%EXEMPLE: application de l'algorithme DijkstraMinMax sur un exemple 

%!TEX root=main.tex

\begin{exemple}

	Soient $\mathcal{A} = (V, (V_{Min}, V_{Max}), E) $, $w: E \rightarrow \mathbb{R}$ une fonction de poids et $\mathcal{G} = (\mathcal{A}, g, Goal)$ le \og \textit{reachability-price game}\fg  associé, l'arène et la fonction de poids sont représentés sur le graphe de la figure~\ref{ex:reachPrice1} et $V_{Min}$ (resp. $V_{Max}$) est représenté par les sommets ronds (resp. les sommets carrés). $Goal =\{ v_{0} \}$. Pour les figures [9-15] , les états grisés représentent les états entièrement traités (i.e. les états dans $T$) et à l'intérieur de ceux-ci se trouve la valeur associée à l'état. De plus la table~\ref{tab:filePrior} (p.~\pageref{tab:filePrior}) reprend pour chaque étape et chaque noeud le contenu de la file de priorité $S$.
	
	\begin{figure}[ht!]
		\centering

		\begin{tikzpicture}

			\node[nC] (v7) at (0,0){$v_{7}$};
			\node[nR] (v6) at (2,0){$v_{6}$};
			\node[nC] (v5) at (4,0){$v_{5}$};
			\node[nR] (v4) at (6,0){$v_{4}$};
			\node[nC] (v2) at (8,0){$v_{2}$};
			\node[nC] (v0) at (10,0){$v_{0}$};
			\node[nR]  (v1) at (8,2){$v_{1}$};
			\node[nR] (v3) at (8,-2){$v_{3}$};

			\draw[->,>=latex] (v7) to [bend right] node[midway,above]{$1$} (v6);
			\draw[->,>=latex] (v6) to [bend right] node[midway,above]{$1$} (v7);
			\draw[->,>=latex] (v5) to node[midway,above]{$1$} (v6);
			\draw[->,>=latex] (v5) to node[midway,above]{$1$} (v4);


			\draw[->,>=latex] (v4) to node[midway,above]{$5$} (v2);
			\draw[->,>=latex] (v4) to node[midway,left]{$1$} (v3);

			\draw[->,>=latex] (v3) to node[midway,below]{$5$} (v0);
			\draw[->,>=latex] (v3) to node[midway,left]{$1$} (v2);

			\draw[->,>=latex] (v2) to node[midway,above]{$1$} (v0);
			\draw[->,>=latex] (v2) to node[midway,left]{$1$} (v1);

			\draw[->,>=latex] (v1) to node[midway,right]{$1$} (v0);

			\draw[->,>=latex] (v0) to [loop right] node[midway,right]{$1$} (v0);





		\end{tikzpicture}


		\caption{Arène du \og reachability-price game \fg }
		\label{ex:reachPrice1}


	\end{figure}

%ETAPE 1	
	\begin{figure}[ht!]
		\centering

		\begin{tikzpicture}

			\node[nC] (v7) at (0,0){$+\infty$};
			\node[nR] (v6) at (2,0){$+\infty$};
			\node[nC] (v5) at (4,0){$+\infty$};
			\node[nR] (v4) at (6,0){$+\infty$};
			\node[nC] (v2) at (8,0){$1$};
			\node[nCG] (v0) at (10,0){$0$};
			\node[nR]  (v1) at (8,2){$1$};
			\node[nR] (v3) at (8,-2){$5$};

			\draw[->,>=latex] (v7) to [bend right] node[midway,above]{$1$} (v6);
			\draw[->,>=latex] (v6) to [bend right] node[midway,above]{$1$} (v7);
			\draw[->,>=latex] (v5) to node[midway,above]{$1$} (v6);
			\draw[->,>=latex] (v5) to node[midway,above]{$1$} (v4);
			

			\draw[->,>=latex] (v4) to node[midway,above]{$5$} (v2);
			\draw[->,>=latex] (v4) to node[midway,left]{$1$} (v3);
			
			\draw[->,>=latex] (v3) to node[midway,below]{$5$} (v0);
			\draw[->,>=latex] (v3) to node[midway,left]{$1$} (v2);
			
			\draw[->,>=latex] (v2) to node[midway,above]{$1$} (v0);
			\draw[->,>=latex] (v2) to node[midway,left]{$1$} (v1);
			
			\draw[->,>=latex] (v1) to node[midway,right]{$1$} (v0);
			
			\draw[->,>=latex] (v0) to [loop right] node[midway,right]{$1$} (v0);


		\end{tikzpicture}


		\caption{Première étape -- Traitement du noeud $v_{0}$. On relaxe :($v_{1},v_{0}$), ($v_{2},v_{0}$), ($v_{3},v_{0}$) et ($v_{0z},v_{0}$). On grise $v_{0}$.}
		\label{ex:reachPrice2}


	\end{figure}
	
%ETAPE 2	
		\begin{figure}[ht!]
			\centering

			\begin{tikzpicture}

				\node[nC] (v7) at (0,0){$+\infty$};
				\node[nR] (v6) at (2,0){$+\infty$};
				\node[nC] (v5) at (4,0){$+\infty$};
				\node[nR] (v4) at (6,0){$+\infty$};
				\node[nC] (v2) at (8,0){$+\infty$};
				\node[nCG] (v0) at (10,0){$0$};
				\node[nR]  (v1) at (8,2){$1$};
				\node[nR] (v3) at (8,-2){$5$};

				\draw[->,>=latex] (v7) to [bend right] node[midway,above]{$1$} (v6);
				\draw[->,>=latex] (v6) to [bend right] node[midway,above]{$1$} (v7);
				\draw[->,>=latex] (v5) to node[midway,above]{$1$} (v6);
				\draw[->,>=latex] (v5) to node[midway,above]{$1$} (v4);


				\draw[->,>=latex] (v4) to node[midway,above]{$5$} (v2);
				\draw[->,>=latex] (v4) to node[midway,left]{$1$} (v3);

				\draw[->,>=latex] (v3) to node[midway,below]{$5$} (v0);
				\draw[->,>=latex] (v3) to node[midway,left]{$1$} (v2);

				\draw[->,>=latex] (v2) to node[midway,above]{$1$} (v0);
				\draw[->,>=latex] (v2) to node[midway,left]{$1$} (v1);

				\draw[->,>=latex] (v1) to node[midway,right]{$1$} (v0);

				\draw[->,>=latex] (v0) to [loop right] node[midway,right]{$1$} (v0);


			\end{tikzpicture}


			\caption{Deuxième étape -- Traitement du noeud $v_{2}$. $nbrSucc = 2$, on n'a donc pas encore testé tous les chemins possibles à partir de ce noeud. Décrémentation de $nbrSucc$, retrait de la plus petit valeur de $S$ et ajout de $v_{0}$ dans les successeurs déjà testés. }
			\label{ex:reachPrice3}


		\end{figure}
%ETAPE 3	
		\begin{figure}[ht!]
		\centering

		\begin{tikzpicture}

			\node[nC] (v7) at (0,0){$+\infty$};
			\node[nR] (v6) at (2,0){$+\infty$};
			\node[nC] (v5) at (4,0){$5$};
			\node[nR] (v4) at (6,0){$+\infty$};
			\node[nC] (v2) at (8,0){$2$};
			\node[nCG] (v0) at (10,0){$0$};
			\node[nRG]  (v1) at (8,2){$1$};
			\node[nR] (v3) at (8,-2){$+\infty$};

			\draw[->,>=latex] (v7) to [bend right] node[midway,above]{$1$} (v6);
			\draw[->,>=latex] (v6) to [bend right] node[midway,above]{$1$} (v7);
			\draw[->,>=latex] (v5) to node[midway,above]{$1$} (v6);
			\draw[->,>=latex] (v5) to node[midway,above]{$1$} (v4);

			\draw[->,>=latex] (v4) to node[midway,above]{$5$} (v2);
			\draw[->,>=latex] (v4) to node[midway,left]{$1$} (v3);

			\draw[->,>=latex] (v3) to node[midway,below]{$5$} (v0);
			\draw[->,>=latex] (v3) to node[midway,left]{$1$} (v2);

     		\draw[->,>=latex] (v2) to node[midway,above]{$1$} (v0);
			\draw[->,>=latex] (v2) to node[midway,left]{$1$} (v1);

			\draw[->,>=latex] (v1) to node[midway,right]{$1$} (v0);

			\draw[->,>=latex] (v0) to [loop right] node[midway,right]{$1$} (v0);


		\end{tikzpicture}


		\caption{Troisième étape -- Traitement du noeud $v_{1}$. On relaxe $(v_{2},v_{1})$. On grise $v_{1}$. }
		\label{ex:reachPrice4}


				\end{figure}
				
%ETAPE 4	
	\begin{figure}[ht!]
	\centering
	\begin{tikzpicture}

	\node[nC] (v7) at (0,0){$+\infty$};
	\node[nR] (v6) at (2,0){$+\infty$};
	\node[nC] (v5) at (4,0){$+\infty$};
	\node[nR] (v4) at (6,0){$7$};
	\node[nCG] (v2) at (8,0){$2$};
	\node[nCG] (v0) at (10,0){$0$};
	\node[nRG]  (v1) at (8,2){$1$};
	\node[nR] (v3) at (8,-2){$3$};

	\draw[->,>=latex] (v7) to [bend right] node[midway,above]{$1$} (v6);
	\draw[->,>=latex] (v6) to [bend right] node[midway,above]{$1$} (v7);
	\draw[->,>=latex] (v5) to node[midway,above]{$1$} (v6);
	\draw[->,>=latex] (v5) to node[midway,above]{$1$} (v4);

	\draw[->,>=latex] (v4) to node[midway,above]{$5$} (v2);
	\draw[->,>=latex] (v4) to node[midway,left]{$1$} (v3);

	\draw[->,>=latex] (v3) to node[midway,below]{$5$} (v0);
	\draw[->,>=latex] (v3) to node[midway,left]{$1$} (v2);

	\draw[->,>=latex] (v2) to node[midway,above]{$1$} (v0);
	\draw[->,>=latex] (v2) to node[midway,left]{$1$} (v1);

	\draw[->,>=latex] (v1) to node[midway,right]{$1$} (v0);

	\draw[->,>=latex] (v0) to [loop right] node[midway,right]{$1$} (v0);


	\end{tikzpicture}

	\caption{Quatrième étape -- Traitement du noeud $v_{2}$. $nbrSucc = 1$. On relaxe $(v_{4},v_{2})$ et $(v_{3},v_{2})$. On grise $v_{2}$. }
	\label{ex:reachPrice5}


	\end{figure}
	
	
%ETAPE 5	
		\begin{figure}[ht!]
		\centering
		\begin{tikzpicture}

		\node[nC] (v7) at (0,0){$+\infty$};
		\node[nR] (v6) at (2,0){$+\infty$};
		\node[nC] (v5) at (4,0){$+\infty$};
		\node[nR] (v4) at (6,0){$4$};
		\node[nCG] (v2) at (8,0){$2$};
		\node[nCG] (v0) at (10,0){$0$};
		\node[nRG]  (v1) at (8,2){$1$};
		\node[nRG] (v3) at (8,-2){$3$};

		\draw[->,>=latex] (v7) to [bend right] node[midway,above]{$1$} (v6);
		\draw[->,>=latex] (v6) to [bend right] node[midway,above]{$1$} (v7);
		\draw[->,>=latex] (v5) to node[midway,above]{$1$} (v6);
		\draw[->,>=latex] (v5) to node[midway,above]{$1$} (v4);

		\draw[->,>=latex] (v4) to node[midway,above]{$5$} (v2);
		\draw[->,>=latex] (v4) to node[midway,left]{$1$} (v3);

		\draw[->,>=latex] (v3) to node[midway,below]{$5$} (v0);
		\draw[->,>=latex] (v3) to node[midway,left]{$1$} (v2);

		\draw[->,>=latex] (v2) to node[midway,above]{$1$} (v0);
		\draw[->,>=latex] (v2) to node[midway,left]{$1$} (v1);

		\draw[->,>=latex] (v1) to node[midway,right]{$1$} (v0);

		\draw[->,>=latex] (v0) to [loop right] node[midway,right]{$1$} (v0);


		\end{tikzpicture}

		\caption{Cinquième étape -- Traitement du noeud $v_{3}$. On relaxe $(v_{4},v_{3})$. On grise $v_{3}$. }
		\label{ex:reachPrice6}


		\end{figure}
		
%ETAPE 6	
\begin{figure}[ht!]
	\centering
	\begin{tikzpicture}
	\node[nC] (v7) at (0,0){$+\infty$};
	\node[nR] (v6) at (2,0){$+\infty$};
	\node[nC] (v5) at (4,0){$5$};
	\node[nRG] (v4) at (6,0){$4$};
	\node[nCG] (v2) at (8,0){$2$};
	\node[nCG] (v0) at (10,0){$0$};
	\node[nRG]  (v1) at (8,2){$1$};
	\node[nRG] (v3) at (8,-2){$3$};

	\draw[->,>=latex] (v7) to [bend right] node[midway,above]{$1$} (v6);
	\draw[->,>=latex] (v6) to [bend right] node[midway,above]{$1$} (v7);
	\draw[->,>=latex] (v5) to node[midway,above]{$1$} (v6);
	\draw[->,>=latex] (v5) to node[midway,above]{$1$} (v4);

	\draw[->,>=latex] (v4) to node[midway,above]{$5$} (v2);
	\draw[->,>=latex] (v4) to node[midway,left]{$1$} (v3);

	\draw[->,>=latex] (v3) to node[midway,below]{$5$} (v0);
	\draw[->,>=latex] (v3) to node[midway,left]{$1$} (v2);

	\draw[->,>=latex] (v2) to node[midway,above]{$1$} (v0);
	\draw[->,>=latex] (v2) to node[midway,left]{$1$} (v1);

	\draw[->,>=latex] (v1) to node[midway,right]{$1$} (v0);

	\draw[->,>=latex] (v0) to [loop right] node[midway,right]{$1$} (v0);


	\end{tikzpicture}

	\caption{Sixième étape -- Traitement du noeud $v_{4}$. On relaxe $(v_{5},v_{4})$. On grise $v_{4}$. }
	\label{ex:reachPrice7}


\end{figure}

%ETAPE 7	
\begin{figure}[ht!]
	\centering
	\begin{tikzpicture}
	\node[nC] (v7) at (0,0){$+\infty$};
	\node[nR] (v6) at (2,0){$+\infty$};
	\node[nC] (v5) at (4,0){$+\infty$};
	\node[nRG] (v4) at (6,0){$4$};
	\node[nCG] (v2) at (8,0){$2$};
	\node[nCG] (v0) at (10,0){$0$};
	\node[nRG]  (v1) at (8,2){$1$};
	\node[nRG] (v3) at (8,-2){$3$};

	\draw[->,>=latex] (v7) to [bend right] node[midway,above]{$1$} (v6);
	\draw[->,>=latex] (v6) to [bend right] node[midway,above]{$1$} (v7);
	\draw[->,>=latex] (v5) to node[midway,above]{$1$} (v6);
	\draw[->,>=latex] (v5) to node[midway,above]{$1$} (v4);

	\draw[->,>=latex] (v4) to node[midway,above]{$5$} (v2);
	\draw[->,>=latex] (v4) to node[midway,left]{$1$} (v3);

	\draw[->,>=latex] (v3) to node[midway,below]{$5$} (v0);
	\draw[->,>=latex] (v3) to node[midway,left]{$1$} (v2);

	\draw[->,>=latex] (v2) to node[midway,above]{$1$} (v0);
	\draw[->,>=latex] (v2) to node[midway,left]{$1$} (v1);

	\draw[->,>=latex] (v1) to node[midway,right]{$1$} (v0);

	\draw[->,>=latex] (v0) to [loop right] node[midway,right]{$1$} (v0);


	\end{tikzpicture}

	\caption{Dernière étape -- Traitement du noeud $v_{4}$. $nbrSucc =2$.On retire la plus petite valeur de $S$. }
	\label{ex:reachPrice8}


\end{figure}

	
%Tableau des files de priorités$

\begin{table}[]
	
\caption{Sommets associés à leur file de priorité $S$ pour chaque étape de l'algorithme}
\label{tab:filePrior}

\begin{tabular}{l|l|l|l|l|l|l|l|l|}
\cline{2-5}
                               & Sommets &    &    &      \\ \hline
\multicolumn{1}{|l|}{}         & $v_{0}$ &$v_{1}$  & $v_{2}$ &$v_{3}$  \\ \hline
\multicolumn{1}{|l|}{Etape 1:} & $(0,null),(1,v_{0})$   &$(+\infty,null),(1,v_{0})$     &$(+\infty,null),(1,v_{0})$    &$(+\infty,null),(5,v_{0})$    \\ \hline
\multicolumn{1}{|l|}{Etape 2:} &$(0,null),(1,v_{0})$         &$(+\infty,null),(1,v_{0})$    &$(+\infty,null)$    &$(+\infty,null),(5,v_{0})$     \\ \hline
\multicolumn{1}{|l|}{Etape 3:} &$(0,null),(1,v_{0})$         &$(+\infty,null),(1,v_{0})$    &$(+\infty,null),(2,v_{1})$    &$(+\infty,null),(5,v_{0})$  \\ \hline
\multicolumn{1}{|l|}{Etape 4:} &$(0,null),(1,v_{0})$         &$(+\infty,null),(1,v_{0})$    &$(+\infty,null),(2,v_{1})$    &$(+\infty,null),(5,v_{0}),(3,v_{2})$   \\ \hline
\multicolumn{1}{|l|}{Etape 5:}   &$(0,null),(1,v_{0})$         &$(+\infty,null),(1,v_{0})$    &$(+\infty,null),(2,v_{1})$    &$(+\infty,null),(5,v_{0}),(3,v_{2})$    \\ \hline 
\multicolumn{1}{|l|}{Etape 6:}   &$(0,null),(1,v_{0})$         &$(+\infty,null),(1,v_{0})$    &$(+\infty,null),(2,v_{1})$    &$(+\infty,null),(5,v_{0}),(3,v_{2})$     \\ \hline
\multicolumn{1}{|l|}{Etape 7:}   &$(0,null),(1,v_{0})$         &$(+\infty,null),(1,v_{0})$    &$(+\infty,null),(2,v_{1})$    &$(+\infty,null),(5,v_{0}),(3,v_{2})$   \\ \hline

\multicolumn{1}{|l|}{}  		&  &    &    &      \\ \hline
\multicolumn{1}{|l|}{}         & $v_{4}$ &$v_{5}$  & $v_{6}$ &$v_{7}$  \\ \hline
\multicolumn{1}{|l|}{Etape 1:} & $(+\infty,null)$   &$(+\infty,null)$     &$(+\infty,null)$    &$(+\infty,null)$    \\ \hline
\multicolumn{1}{|l|}{Etape 2:} &$(+\infty,null)$    &$(+\infty,null)$    &$(+\infty,null)$    &$(+\infty,null)$    \\ \hline
\multicolumn{1}{|l|}{Etape 3:} &$(+\infty,null)$    &$(+\infty,null)$    &$(+\infty,null)$    &$(+\infty,null)$    \\ \hline
\multicolumn{1}{|l|}{Etape 4:} &$(+\infty,null),(7,v_{2})$    &$(+\infty,null)$    &$(+\infty,null)$    &$(+\infty,null)$    \\ \hline
\multicolumn{1}{|l|}{Etape 5:} &$(+\infty,null),(7,v_{2}),(4,v_{3})$    &$(+\infty,null)$    &$(+\infty,null)$    &$(+\infty,null)$    \\ \hline
\multicolumn{1}{|l|}{Etape 6:} &$(+\infty,null),(7,v_{2}),(4,v_{3})$     &$(+\infty,null),(5,v_{4})$    &$(+\infty,null)$    &$(+\infty,null)$    \\ \hline
\multicolumn{1}{|l|}{Etape 7:} &$(+\infty,null),(7,v_{2}),(4,v_{3})$     &$(+\infty,null)$    &$(+\infty,null)$    &$(+\infty,null)$    \\ \hline

\end{tabular}




\end{table}
	
\end{exemple}	 

\clearpage




 

%!TEX root=main.tex

\section{Questions posées}
\label{section:questionsPosees}

Dans cette section, nous allons expliciter les différentes questions que nous nous posons et que nous aimerions résoudre.\\


Tout d'abord, considérons le jeu $(\mathcal{G},v_{1})$ où $\mathcal{G} = ( \{ 1,2 \}, V, (V_{1}, V_{2}),E, (Cost _{1},Cost _{2}))$ où: \begin{enumerate}
\item[$\bullet$] Pour tout  $\rho = \rho _{0} \rho _{1} \rho _{2} \ldots $ où $\rho \in Plays$ $Cost_{i}(\rho) = $ $\begin{cases} 
								\min \{ i | \rho _{i} \in Goal_{i} \} & \text{si } \exists i \text{ tq } \rho _{i} \in Goal_{i} \\
								+\infty & \text{ sinon}
								\end{cases}$,
\item[$\bullet$] $Goal_{1} = \{ v_{3} \}$ et $Goal_{2} = \{ v_{0} \}$,
\item[$\bullet$]  $V_{1}$ (resp. $V_{2}$) est représenté par les noeuds ronds (resp. carrés) du graphe de la figure~\ref{ex:patologique}.

\end{enumerate}


\begin{figure}[ht!]
	\centering

	\begin{tikzpicture}
		
		\node[nRG] (v3) at (2,-2){$v_{3}$};
		\node[nC] (v2) at (2,0){$v_{2}$};
		\node[nR] (v1) at (0,0){$v_{1}$};
		\node[nRD] (v0) at (0,-2){$v_{0}$};
	
		\draw[->,>=latex] (v0) to [bend right] (v1);
		\draw[->,>=latex] (v1) to [bend right] (v0);
		
		\draw[->,>=latex] (v1) to [bend right] (v2);
		\draw[->,>=latex] (v2) to [bend right] (v1);
		
		\draw[->,>=latex] (v3) to [bend right] (v2);
		\draw[->,>=latex] (v2) to [bend right] (v3);
		
		
	\end{tikzpicture}
	
	\caption{Jeu d'atteignabilité avec coût}
	\label{ex:patologique}
	

\end{figure}

Soit $\sigma _{1}(v) =$ $\begin{cases}
						v_{2} & \text{si } v = v_{1} \\
						v_{1 } & \text{si } v = v_{0} \\
						v_{2} & \text{si } v = v_{3} 
						\end{cases}$
						
						
						
\noindent et soit $\sigma _{2}(v) = v_{1}$ alors $(\sigma _{1},\sigma _{2})$ est un équilibre de Nash du jeu $(\mathcal{G},v_{1})$ dont l'outcome est $(v_{1}v_{2})^{\omega}$. Nous remarquons qu'avec cet équilibre de Nash aucun des deux joueurs n'atteint son objectif. Nous pouvons également observer que si les deux joueurs coopéraient, ils pourraient tous deux minimiser leur coût. En effet, si les deux joueurs suivaient un profil de stratégie ayant comme outcome $\rho = v_{1}v_{0}v_{1}(v_{2}v_{3})^{\omega} $ nous aurions $Cost_{1}(\rho) = 4$ et $Cost_{2}(\rho) = 1$.

Nous nous posons alors les questions suivantes:

\begin{qst}
	
	Soit $G = (V,E)$ un graphe orienté fortement connexe \footnote{En théorie des graphes, un graphe $G = (V,E)$ est dit fortement connexe si pour tout $u$ et $v$ dans $V$, il existe un chemin de $u$ à $v$} qui représente l'arène d'un jeu d'atteignabilité multijoueur avec coût : $(\mathcal{G},v_{0})$ (pour un certain $v_{0} \in V$).
Existe-t'il un équilibre de Nash tel que chaque joueur atteigne son objectif?

\end{qst}
	

\begin{qst}
	\label{qst:2}
	Soit $(\mathcal{G},v_{0})$ où $\mathcal{G} = (V,(V_{Min},V_{Max}),E,RP_{Min},RP_{Max},Goal)$ est un \og reachability-price game\fg  et soit $\rho \in Plays$ un jeu sur $(\mathcal{G},v_{0})$, existe-t'il une procédure algorithmique pour déterminer si ce jeu $\rho$ correspond à l'outcome d'un équilibre de Nash $(\sigma _{1},\sigma _{2})$ pour certaines stratégies $\sigma _{1}\in \Sigma _{Min}$ et $\sigma _{2}\in \Sigma _{Max}$ ?
	
\end{qst}

A partir d'un jeu $(\mathcal{G},v_{0})$ où $\mathcal{G} =( \{ 1, 2 \}, V, (V_{1},V_{2}), E, (\varphi _{1},\varphi _{2}),(Goal_{1},Goal_{2}))$ est un jeu d'atteignabilité à deux joueurs avec coût nous pouvons y associer deux jeux à somme nulle du type \og reachability-price game \fg:
\begin{enumerate}
	\item $\mathcal{G}_{1} = (V,(V_{Min},V_{Max}),E,g,Goal)$ où $V_{Min} = V_{1}$, $V_{Max} = V_{2}$, $g = \varphi_{1}$ et $Goal = Goal_{1}$ (\emph{i.e.,} le jeu dans lequel $J_{1}$ tente d'atteindre au plus vite son objectif et où $J_{2}$ veut l'en empêcher),
	\item $\mathcal{G}_{2} = (V,(V_{Min},V_{Max}),E,g,Goal)$ où $V_{Min} = V_{2}$, $V_{Max} = V_{1}$,$g = \varphi_{2}$ et $Goal = Goal_{2}$ (\emph{i.e.,} le jeu dans lequel $J_{2}$ tente d'atteindre au plus vite son objectif et où $J_{1}$ veut l'en empêcher).
\end{enumerate}

Pour tout $v \in V$, nous notons alors $Val_{1}(v)$ la valeur de $Val(v)$ calculée dans $\mathcal{G}_{1}$ et $Val_{2}(v)$ la valeur de $Val(v)$ calculée dans $\mathcal{G}_{2}$.

\begin{qst}
	
	\label{qst:3}
	
	Soient $\mathcal{A} = (\Pi, V, (V_{1}, V_{2}), E)$ une arène et $(\mathcal{G} = (\mathcal{A}, (\varphi _{1}, \varphi _{2}), (Goal_{1}, Goal_{2}))$ un jeu d'atteignabilité à deux joueurs à objectif quantitatif, soit $(\mathcal{G}, v_{0})$ le jeu initialisé pour un certain $v_{0} \in V $ soit $\rho = s_{0}s_{1}... \in Plays$, on se demande s'il existe $(\sigma _{1},\sigma _{2})$ un équilibre de Nash dans $(\mathcal{G},v_{0})$ tel que $\rho = \langle \sigma _{1},\sigma _{2} \rangle_{v_0}$ et $(\sigma _{1},\sigma _{2}) \in \Sigma _{1} \times \Sigma _{2}.$
\end{qst}

Pour répondre à la question~\ref{qst:3} nous nous intéressons à la véracité de la propriété~\ref{prop:outEN2} :

\begin{propriete}
	\label{prop:outEN2}
	Soient $\mathcal{A} = (\Pi, V, (V_{1}, V_{2}), E)$ une arène et $\mathcal{G} = (\mathcal{A}, (\varphi _{1}, \varphi _{2}), (Goal_{1}, Goal_{2})$ un jeu d'atteignabilité à deux joueurs à objectif quantitatif, soit $(\mathcal{G}, v_{0})$ le jeu initialisé pour un certain $v_{0} \in V $ et soit $\rho = v_{0}v_{1}... \in Plays$. 
	
	Posons $(x,y) = (\varphi _{1}(\rho), \varphi _{2}(\rho))$ et pour $v_{j} \in \rho$ ($j \in \mathbb{N}$) nous définissons: $\varepsilon _{j} = \sum _{n= 0} ^{j-1} w(v_{n},v_{n+1})$ où $w$ est la fonction de poids associée à $G = (V,E)$ .
	
	\begin{center}Il existe $(\sigma _{1},\sigma _{2}) \in \Sigma _{1} \times \Sigma _{2}$ un équilibre de Nash dans $(\mathcal{G},v_{0})$ tq $\langle \sigma _{1},\sigma _{2}\rangle_{v_0} = \rho$\\ $\text{}$\\ si et seulement si\\$\text{}$\\ pour tout $j \in \mathbb{N}$, $\begin{cases}
													Val_{1}(v_{j}) + \varepsilon _{j} \geq x & \text{ si } v_{j} \in V_{1} \\
													Val_{2}(v_{j}) + \varepsilon _{j} \geq y & \text{ si } v_{j} \in V_{2} 
													\end{cases}$.\end{center}  
\end{propriete}
	
\subsection{Question 1}
\subsection{Question 2}
Afin de pouvoir répondre à cette question nous allons commencer par énoncer et prouver un résultat qui nous permettra de déterminer si un outcome donné correspond à un équilibre de Nash.

\begin{propriete}
	\label{prop:question2}
	 Soient $(\mathcal{G},v_0)$ tel que $\mathcal{G} = (\mathcal{A}, g, Goal)$ où $\mathcal{A}= ({Min,Max}, V, (V_{Min},V_{Max}))$ un \og reachability-price game\fg~initialisé, $\rho = v_0 \ldots v_k \ldots \in Plays$ tel que $g(\rho) = x$ pour un certain $x \in \mathbb{N}_0$ et $\varepsilon_k = \sum_{n=0}^{k-1} w(v_n,v_{n-1})$,
	\begin{center} $\exists (\sigma_1, \sigma_2) \in \Sigma_{Min} \times \Sigma_{Max}$ un équilibre de Nash tel que $\langle \sigma_1, \sigma_2 \rangle_{v_0} = \rho$\\ $\text{}$ \\ si et seulement si \\ $\text{}$ \\
		$ \forall v_k \in \rho$,  $\begin{cases} Val(v_k) \geq x - \varepsilon _k & \text{si } v_k \in V_{Min} \\
		 									 Val(v_k) \leq x - \varepsilon _k &  \text{si } v_k \in V_{Max}\end{cases}$ \end{center}

\end{propriete}
\setcounter{equation}{0}

\begin{demonstration}
	
	Nous savons qu'un tel jeu est déterminé et qu'il existe $(\sigma_{1}^* , \sigma_{2}^*) \in (\Sigma_{Min},\Sigma_{Max})$ des stratégies optimales. Nous avons donc qu'il existe $(\sigma_{1}^* , \sigma_{2}^*) \in (\Sigma_{Min},\Sigma_{Max})$ tel que pour tout $v \in V,\, g(\langle \sigma_1^*,\sigma_2^* \rangle_v) = Val(v)$. \\
	
	\begin{itemize}
		\item[($\Downarrow$)] Supposons $(\sigma_1, \sigma_2)$ soit un équilibre de Nash d'outcome $\rho$ et de paiement $x$. Supposons au contraire qu'il existe $v_k \in \rho$ tel que ($Val(v_k) < x - \varepsilon_k$ si $v_k \in V_{Min}$) ou ($Val(v_k) > x - \varepsilon $ si $v_k \in V_{Max}$).\\
		Sans perte de généralité, nous supposons:  
		\begin{align}
			\exists v_k \in \rho (v_k \in V_{Min}) \text{ tel que } Val(v_k) &< x - \varepsilon_k \notag \\
																			&= g(\langle \sigma_1, \sigma_2 \rangle_{v_k}) \label{eq:ENeq1}
		\end{align}
		
		De plus, nous avons :
		\begin{align} Val(v_k) = \sup_{\tau_2 \in \Sigma_{Max}} g(\langle \sigma_1^*, \tau_2 \rangle_{v_k}) \geq g(\langle \sigma_1^*,\sigma_2 \rangle_{v_k}). \label{eq:ENeq2}\end{align}
			
		De \eqref{eq:ENeq1} et \eqref{eq:ENeq2}, nous déduisons:
		\begin{align}
			g(\langle \sigma_1^*, \sigma_2 \rangle _{v_k}) < g (\langle \sigma_1, \sigma_2 \rangle_{v_k}) \label{eq:ENeq3}
		\end{align}
		Comme le joueur Min cherche à minimiser son gain, la relation \eqref{eq:ENeq3} signifie que le joueur Min a une déviation profitable  à partir de $v_k$.\\
		Ceci nous permet de conclure que $(\sigma_1,\sigma_2)$ n'est pas un équilibre de Nash.
		
		\item[($\Uparrow$)]
		Soit $(\tau_1, \tau_2) \in \Sigma_1 \times \Sigma_2$ un profil de stratégies qui permet d'obtenir l'outcome $\rho$ de paiement $x$.
		A partir de $(\tau_1, \tau_2)$ nous désirons construire un équilibre de Nash ayant le même outcome (et donc le même coût).
		L'idée est la suivante: dans un premier temps les deux joueurs suivent leur stratégie conformément au profil $(\tau_1,\tau_2)$. Si un des joueurs, notons le $i$,  dévie de sa stratégie alors l'autre joueur décide de le \og punir \fg~et joue en suivant sa stratégie optimale $\sigma_{-i}^*$\\ 
		
		
		Comme dans le papier \og Multiplayer Cost Games With Simple Nash Equilibria \fg~\cite{DBLP:conf/lfcs/BrihayePS13}, nous définissons une fonction de punition: $P : Hist \rightarrow \{ Min, Max \}\cup \{ \perp \}$ qui permet de définir quel est le premier joueur à avoir dévié du profil de stratégies initial $(\tau_1, \tau_2)$. Cette fonction est telle que $P(h) = \perp$ si aucun joueur n'a dévié le long de l'histoire $h$ et $P(h) = i$ pour un certain $i \in \{ Min, Max \}$ si le joueur $i$ a dévié le long de l'histoire $h$. Nous pouvons donc définir la fonction $P$ par récurrence sur la longueur des histoires : pour $v_0$, le noeud initial, $P(v_0) = \perp$  et pour $h \in Hist$ et $v\in V$ on a :
		$$
		P(hv) = \begin{cases}
				\perp & \text{ si } P(h) = \perp \text{ et } hv \text{ est un préfixe de } \rho \\
				i & \text{ si } P(h) = \perp ,\, hv \text{ n'est pas un préfixe de }\rho \text{ et } Last(h)\in V_i\\
				P(h) & \text{ sinon (\emph{i.e.,}}\, P(h)\neq \perp) \end{cases}
		$$\\
		
		Nous définissons pour tout $h \in Hist_{i}$:
		$\sigma_i(h) = \begin{cases} \tau_i(h) & \text{ si } P(h) = \perp \text{ ou } i \\
		\sigma_i^*(h) & \text{ sinon}
		
		\end{cases}$
		
		Nous avons clairement que $\langle \sigma_1, \sigma_2 \rangle_{v_{k}} = \rho$.\\
		Nous devons maintenant montrer qu'il s'agit d'un équilibre de Nash.
		Supposons au contre que le joueur Max possède une déviation profitable que nous notons $\tilde{\sigma_{2}}$. Comme $\sigma_2$ et $\tilde{\sigma_2}$ sont des stratégies du jeu $(\mathcal{G}, v_0)$ on a que:
		\begin{align} \tilde{\rho} &= \langle \sigma_1, \tilde{\sigma_2} \rangle_{v_0} 
								   = h. \langle \sigma_1, \tilde{\sigma_2} \rangle_{v_k} & \text{ car Max dévie donc Min le punit} \label{eq:ENeq4} \\
								\rho &= \langle \sigma_1, \sigma_2 \rangle_{v_0} 
								     = h. \langle \sigma_1, \sigma_2 \rangle_{v_k} \label{eq:ENeq5}
								\end{align} 
	où $h$ est le plus long préfixe commun et $J_{Max}$ dévie en $v_k$.
	Comme $\tilde{\sigma_2}$ est une déviation profitable et au vu de \eqref{eq:ENeq4} et \eqref{eq:ENeq5} on a:
	\begin{align}
		g(\tilde{\rho}) > g(\rho) && \text{ (Joueur Max maximise son gain}) \label{eq:ENeq6}
	\end{align}
	La relation~\eqref{eq:ENeq6} implique:
	\begin{align}
		g(\langle \sigma_1, \tilde{\sigma_2} \rangle _{v_k}) > g(\langle \sigma_1, \sigma_2 \rangle _{v_k})  \label{eq:ENeq7}
	\end{align}
	
	De plus, 
	\begin{align}
		g(\langle \sigma_1, \tilde{\sigma_2} \rangle_{v_k}) = g( \langle \sigma_1^*, \tilde{\sigma_2} \rangle_{v_k}) \leq Val(v_k) \label{eq:ENeq8}
	\end{align}
	Par hypothèse on a :
	\begin{align}
		Val(v_k) \leq x - \varepsilon_k = g(\langle \tau_1, \tau_2 \rangle_{v_k} ) = g(\langle \sigma_1, \sigma_2 \rangle_{v_k}) \label{eq:ENeq9}
	\end{align}
	Par \eqref{eq:ENeq8} et \eqref{eq:ENeq9} on a : 
	$$ g(\langle \sigma_1, \tilde{\sigma}_2 \rangle_{v_k}) \leq g( \langle \sigma_1, \sigma_2 \rangle_{v_k}).$$
	Ce qui contredit \eqref{eq:ENeq7} et termine notre preuve.
	\end{itemize}
	
\end{demonstration}

Grâce au résultat de la propriété~\ref{prop:question2} nous avons presque répondu à la question~\ref{qst:2}. Pour un outcome donné il suffirait en effet de vérifier que la propriété est vérifiée. Toutefois, l'outcome associé à un profil de stratégies est infini, il faut donc trouver un moyen de le représenter afin qu'un algorithme puisse s'appliquer dessus.

\subsection{Question 3}
Plutôt que de prouver la propriété~\ref{prop:outEN2} pour $|\Pi|= 2$, nous allons montrer qu'en fait nous pouvons la généralise pour $|\Pi| \geq 2 .$ Pour ce faire, nous avons besoin d'introduire quelques notions préliminaires.


\begin{defi}
	\label{defi:coalGame}
 Soient $\mathcal{A} = (\Pi, V, (V_{i})_{i\in\Pi}, E)$ une arène et $\mathcal{G} = (\mathcal{A}, (\varphi _{i})_{i\in\Pi}, (Goal_{i})_{i\in\Pi})$ un jeu d'atteignabilité à $|\Pi| \geq 2$ à objectif quantitatif.
Pour tout joueur $i \in \Pi$, nous pouvons y associer un jeu à somme nulle de type \og reachability-price game \fg~ noté $\mathcal{G}_{i}$.
On définit ce jeu de la manière suivante : 
$$ \displaystyle \mathcal{G}_{i}= (\mathcal{A}_{i}, g , Goal) \text{ où } \mathcal{A}_{i} = (\{i,\Pi\backslash{i}\}, V, (V_{i},V\backslash V_i,E) \text{, } g = \varphi_i \text{ et } Goal = Goal_i$$

\noindent De plus, pour tout $v\in V$, $Val_i(v)$ est la valeur du jeu $\mathcal{G}_i$ pour tout noeud $v\in V$. 
\end{defi} 

En d'autres mots, $G_i$ correspond au jeu où le joueur $i$ (joueur Min) joue contre la coalition $\Pi\backslash\{ i \}$ (joueur Max) . Cela signifie que le joueur $i$ tente d'atteindre son objectif le plus rapidement possible tandis que tous les autres joueurs veulent l'en empêcher (ou tout du moins maximiser son gain). Nous avons vu précédemment qu'un tel jeu est déterminé et que les deux joueurs possèdent une stratégie optimale ($\sigma^*_i$ et $\sigma^*_{-i}$) telles que:
$$ \inf_{\sigma _{i\in \Sigma _{Min}}} \varphi_i(\langle \sigma_i,\sigma^*_{-i}\rangle_v)= Val_i(v) = \sup _{\sigma_{-i}\in \Sigma_{Max}} \varphi_i(\langle \sigma^*_i, \sigma_{-i}\rangle_v).$$ De plus, de la stratégie optimale $\sigma^*_{-i}$ nous pouvons dériver une stratégie pour tout joueur $j \neq i$ que nous notons $\sigma_{j,i}$.\\

Ces considérations étant clairement établies, nous pouvons maintenant énoncer et prouver le résultat~\ref{prop:outEN3} qui nous intéresse.
\begin{propriete}
	\label{prop:outEN3}
	Soit $|\Pi| = n \geq 2$,
	soient $\mathcal{A} = (\Pi, V, (V_{i})_{i\in\Pi}, E)$ une arène et $\mathcal{G} = (\mathcal{A}, (\varphi _{i})_{i\in\Pi}, (Goal_{i})_{i\in\Pi})$ un jeu d'atteignabilité à $n$ joueurs à objectif quantitatif, soit $(\mathcal{G}, v_{0})$ le jeu initialisé pour un certain $v_{0} \in V $ et soit $\rho = v_{0}v_{1}... \in Plays$. 
	
	Posons $(x_{i})_{i\in\Pi} = (\varphi _{i}(\rho))_{i\in\Pi}$ le profil de paiement associé à la partie $\rho$ . Nous définissons pour $v_{k} \in \rho$ ($k \in \mathbb{N}$)  $\varepsilon _{k} := \sum _{n= 0} ^{k-1} w(v_{n},v_{n+1})$ où $w$ est la fonction de poids associée à $G = (V,E)$.
	
	\begin{center}Il existe $ (\sigma _{i})_{i\in\Pi} \in \prod_{i\in\Pi} \Sigma _{i}$ un équilibre de Nash dans $(\mathcal{G},v_{0})$ tq $\langle (\sigma _{i})_{i \in \Pi}\rangle_{v_0} = \rho$\\ $\text{}$\\ si et seulement si\\$\text{}$\\  $ \forall k \in \mathbb{N}, \forall j \in \Pi$, $Val_{j}(v_{k}) + \varepsilon _{k} \geq x_j \text{  si } v_{k} \in V_{j}$.\end{center}
	
\end{propriete}

\setcounter{equation}{0}

\begin{demonstration}
	Nous allons montrer les deux implications:\\
	\begin{itemize}
		\item[$(\Downarrow)$] Supposons au contraire qu'il existe $k\in \mathbb{N}$ et $j\in\Pi$ tels que $Val_j(v_k) + \varepsilon_k < x_j$,
		\begin{equation}
			\label{eq:questEq1}
			i.e., Val_j(v_k) < x_j + \varepsilon_k = \varphi_j(\langle (\sigma_i)_{i \in \Pi}\rangle_{v_k})
		\end{equation}
		où $\varphi_j(\langle (\sigma_i)_{i \in \Pi}\rangle_{v_k})$ est le coût de la partie pour le joueur $j$ si elle avait commencé en $v_k$.
		De plus, on a : 
		\begin{equation}
			\label{eq:questEq2}
			Val_j(v_k) = \sup_{\tau_{-j}\in \Sigma_{Max}} g(\langle \sigma^*_j,\tau_{-j} \rangle_{v_k}) \geq g (\langle \sigma^*_j,\sigma_{-j} \rangle_{v_k}) = \varphi_j(\langle \sigma^*_j,\sigma_{-j} \rangle_{v_k})
		\end{equation}
		où $\sigma^*_j$ est la stratégie optimale du joueur $j$ associée à $\mathcal{G}_j$ et $\sigma_{-j}$ dans l'expression $g (\langle \sigma^*_j,\sigma_{-j} \rangle_{v_k})$ est un abus de notation désignant la stratégie où la coalition $\Pi\backslash\{ j \}$ suit chacune des stratégies $\sigma_i$ pour tout $i \neq j$.\\
		
		Dès lors, \eqref{eq:questEq1} et \eqref{eq:questEq2} nous donnent:
		\begin{equation}
			\label{eq:questEq3}
			\varphi_j(\langle \sigma^*_j,\sigma_{-j} \rangle_{v_k}) < \varphi_j(\langle(\sigma_i)_{i\in \Pi}\rangle_{v_k})
		\end{equation}
		
		La relation~\eqref{eq:questEq3} signifie qu'à partir du noeud $v_k$ le joueur $j$ ferait mieux de suivre la stratégie $\sigma^*_j$. Il s'agit donc d'une déviation profitable pour le joueur $j$ par rapport au profil de stratégies $(\sigma_i)_{i\in \Pi}$. Cela implique que $(\sigma_i)_{i\in\Pi}$ n'est pas un équilibre de Nash. Nous avons donc la contradiction attendue.\\
		
		\item[$(\Uparrow)$] Soit $(\tau_i)_{i\in \Pi}$ un profil de stratégies qui permet d'obtenir l'outcome $\rho$ de paiement $(x_i)_{i\in\Pi}$.
		A partir de $(\tau_i)_{i\in \Pi}$ nous désirons construire un équilibre de Nash ayant le même outcome (et donc le même profil de coût).
		L'idée est la suivante: dans un premier temps tous les joueurs suivent leur stratégie conformément au profil $(\tau_i)_{i \in \Pi}$. Si un des joueurs, notons le $i$,  dévie de sa stratégie alors les autres joueurs se réunissent en une coalition $\Pi\backslash \{ i \}$ et jouent en suivant leur stratégie de punition dans $\mathcal{G}_i$ (\emph{i.e.,} pour tout j $\neq$ i, le joueur $j$ suit la stratégie $\sigma^*_{j,i}$).\\
		
	Comme dans le papier \og Multiplayer Cost Games With Simple Nash Equilibria \fg~\cite{DBLP:conf/lfcs/BrihayePS13}, nous définissons une fonction de punition: $P : Hist \rightarrow \Pi\cup \{ \perp \}$ qui permet de définir quel est le premier joueur à avoir dévié du profil de stratégies initial $(\tau_i)_{i\in\Pi}$. Cette fonction est telle que $P(h) = \perp$ si aucun joueur n'a dévié le long de l'histoire $h$ et $P(h) = i$ pour un certain $i \in \Pi$ si le joueur $i$ a dévié le long de l'histoire $h$. Nous pouvons donc définir la fonction $P$ par récurrence sur la longueur des histoires : pour $v_0$, le noeud initial, $P(v_0) = \perp$  et pour $h \in Hist$ et $v\in V$ on a :
	$$
	P(hv) = \begin{cases}
			\perp & \text{ si } P(h) = \perp \text{ et } hv \text{ est un préfixe de } \rho \\
			i & \text{ si } P(h) = \perp ,\, hv \text{ n'est pas un préfixe de }\rho \text{ et } Last(h)\in V_i\\
			P(h) & \text{ sinon (\emph{i.e.,}}\, P(h)\neq \perp) \end{cases}
	$$\\
	
	Nous pouvons maintenant définir notre équilibre de Nash potentiel dans $\mathcal{G}$. Pour tout $i\in \Pi$ et tout $h\in Hist$ tels que $Last(h)\in V_i$:
	$$\sigma_i(h)= \begin{cases}
					\tau_i(h) & \text{ si }P(h)= \perp \text{ ou }i \\
					\sigma^*_{i,P(h)}(h) & \text{ sinon }\end{cases}$$\\
					
	Nous devons maintenant montrer que le profil de stratégies $(\sigma_i)_{i\in\Pi}$ ainsi défini est un équilibre de Nash d'outcome $\rho$.\\
	Il est clair que $\langle (\sigma_i)_{i\in\Pi} \rangle_{v_0} = \rho$.\\
	Montrons maintenant qu'il s'agit bien d'un équilibre de Nash.\\
	\noindent Supposons au contraire que ce ne soit pas le cas. Cela signifie qu'il existe une déviation profitable pour un certain joueur $j \in\Pi$. Notons-la $\tilde{\sigma}_j$ .\\
		\noindent Soit $\tilde{\rho} = \langle \tilde{\sigma}_j , (\sigma_i)_{i \in \Pi \backslash \{j \}} \rangle_{v_0}$ l'outcome tel que le joueur $j$ joue sa déviation profitable et où les autres joueurs jouent conformément à leur ancienne stratégie.
		Puisque $\tilde{\sigma}_j$ est une déviation profitable nous avons: 
		\begin{equation}
			\label{eq:questEq4}
			\varphi_j(\tilde{\rho}) < \varphi_j(\rho)
		\end{equation}
		
		De plus, comme $\rho$ et $\tilde{\rho}$ commencent tous les deux à partir du noeud $v_0$, ils possèdent un préfixe commun. En d'autres termes, il existe une histoire $hv \in Hist$ telle que: 
		\begin{equation*}
			\rho = h. \langle (\sigma_i)_{i\in\Pi} \rangle_v \text{ et } \tilde{\rho} =  h.\langle \tilde{\sigma_j}, (\sigma)_{i\in\Pi\backslash \{ j \}} \rangle_v
		\end{equation*}
		 S'il en existe plusieurs nous en choisissons une de longueur maximale.
		Au vu de la définition de $\sigma_i$ , nous pouvons réécrire:
		
		\begin{equation*}
			\rho = h. \langle (\tau_i)_{i\in\Pi} \rangle_v \text{ et } \tilde{\rho} = h.\langle \tilde{\sigma_j}, (\sigma^*_{i,j})_{i\in\Pi \backslash\{ j \}}
		\end{equation*}
		En effet, le joueur $j$ dévie en $v$, donc à partir de $v$ tout joueur $i \neq j$ joue sa stratégie de punition. De plus, nous avons les relations suivantes : 
		\begin{align}
			Val_j(v) &= \inf_{\mu_j\in\Sigma_{Min}}\varphi_j(\langle \mu_j, \sigma^*_{-j}\rangle_v)\notag\\
					 &\leq \varphi_j(\langle \tilde{\sigma}_j, \sigma^*_{-j}\rangle_v)\notag\\
					& = \varphi_j(\langle \tilde{\sigma}_j, (\sigma^*_{i,j})_{i \in \Pi \backslash \{ j \}}\rangle_v). \label{eq:questEq5}
		\end{align}
		
	Supposons $h = v_0 \ldots v_k$ pour un certain $k \in \mathbb{N}$. Alors,
	\begin{equation}
		\label{eq:questEq6}
		\varphi_j(\tilde{\rho}) = \varepsilon_k + \varphi(\langle \tilde{\sigma}_j , (\sigma^*_{i,j})_{i\in\Pi\backslash\{ j \}}\rangle_v)
	\end{equation}
	Dès lors, \eqref{eq:questEq5} et~\eqref{eq:questEq6} nous donnent:
	\begin{equation}
		\label{eq:questEq7}
		Val_j(v) \leq \varphi_j(\tilde{\rho}) - \varepsilon_k
	\end{equation}
	
	Donc par~\eqref{eq:questEq4} et~\eqref{eq:questEq7}  nous avons:
	$$ Val_j(v) \leq \varphi_j(\tilde{\rho})- \varepsilon_k < \varphi_j(\rho)-\varepsilon_k = x_j-\varepsilon_k$$
	Ce qui contredit l'hypothèse et conclut notre preuve.
	\end{itemize}
\end{demonstration}

