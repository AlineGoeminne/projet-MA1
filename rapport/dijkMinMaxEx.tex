%!TEX root=main.tex

\begin{exemple}
	\label{ex:dernierExemple}

	Soient $\mathcal{A} = (V, (V_{Min}, V_{Max}), E) $, $w: E \rightarrow \mathbb{N}_0$ une fonction de poids et $\mathcal{G} = (\mathcal{A}, g, Goal)$ le \og \textit{reachability-price game}\fg~associé, l'arène et la fonction de poids sont représentées sur le graphe de la figure~\ref{ex:reachPrice1} et $V_{Min}$ (resp. $V_{Max}$) est représenté par les sommets ronds (resp. les sommets carrés). $Goal =\{ v_{0} \}$. Pour les figures [9-15] , les états grisés représentent les états entièrement traités (\emph{i.e.,} les états dans $T$) et à l'intérieur de ceux-ci se trouve la valeur associée à l'état. De plus, le tableau~\ref{tab:filePrior} (p.~\pageref{tab:filePrior}) reprend pour chaque étape et chaque noeud le contenu du champ $S$ (une file de priorité pour le joueur Max et un 2-uplet pour le joueur Min).
	
	\begin{figure}[ht!]
		\centering

		\begin{tikzpicture}

			\node[nC] (v7) at (0,0){$v_{7}$};
			\node[nR] (v6) at (2,0){$v_{6}$};
			\node[nC] (v5) at (4,0){$v_{5}$};
			\node[nR] (v4) at (6,0){$v_{4}$};
			\node[nC] (v2) at (8,0){$v_{2}$};
			\node[nC] (v0) at (10,0){$v_{0}$};
			\node[nR]  (v1) at (8,2){$v_{1}$};
			\node[nR] (v3) at (8,-2){$v_{3}$};

			\draw[->,>=latex] (v7) to [bend right] node[midway,above]{$1$} (v6);
			\draw[->,>=latex] (v6) to [bend right] node[midway,above]{$1$} (v7);
			\draw[->,>=latex] (v5) to node[midway,above]{$1$} (v6);
			\draw[->,>=latex] (v5) to node[midway,above]{$1$} (v4);


			\draw[->,>=latex] (v4) to node[midway,above]{$5$} (v2);
			\draw[->,>=latex] (v4) to node[midway,left]{$1$} (v3);

			\draw[->,>=latex] (v3) to node[midway,below]{$5$} (v0);
			\draw[->,>=latex] (v3) to node[midway,left]{$1$} (v2);

			\draw[->,>=latex] (v2) to node[midway,above]{$1$} (v0);
			\draw[->,>=latex] (v2) to node[midway,left]{$1$} (v1);

			\draw[->,>=latex] (v1) to node[midway,right]{$1$} (v0);

			\draw[->,>=latex] (v0) to [loop right] node[midway,right]{$1$} (v0);





		\end{tikzpicture}


		\caption{Arène du \og reachability-price game \fg }
		\label{ex:reachPrice1}


	\end{figure}

%ETAPE 1	
	\begin{figure}[ht!]
		\centering

		\begin{tikzpicture}

			\node[nC] (v7) at (0,0){$+\infty$};
			\node[nR] (v6) at (2,0){$+\infty$};
			\node[nC] (v5) at (4,0){$+\infty$};
			\node[nR] (v4) at (6,0){$+\infty$};
			\node[nC] (v2) at (8,0){$1$};
			\node[nCG] (v0) at (10,0){$0$};
			\node[nR]  (v1) at (8,2){$1$};
			\node[nR] (v3) at (8,-2){$5$};

			\draw[->,>=latex] (v7) to [bend right] node[midway,above]{$1$} (v6);
			\draw[->,>=latex] (v6) to [bend right] node[midway,above]{$1$} (v7);
			\draw[->,>=latex] (v5) to node[midway,above]{$1$} (v6);
			\draw[->,>=latex] (v5) to node[midway,above]{$1$} (v4);
			

			\draw[->,>=latex] (v4) to node[midway,above]{$5$} (v2);
			\draw[->,>=latex] (v4) to node[midway,left]{$1$} (v3);
			
			\draw[->,>=latex] (v3) to node[midway,below]{$5$} (v0);
			\draw[->,>=latex] (v3) to node[midway,left]{$1$} (v2);
			
			\draw[->,>=latex] (v2) to node[midway,above]{$1$} (v0);
			\draw[->,>=latex] (v2) to node[midway,left]{$1$} (v1);
			
			\draw[->,>=latex] (v1) to node[midway,right]{$1$} (v0);
			
			\draw[->,>=latex] (v0) to [loop right] node[midway,right]{$1$} (v0);


		\end{tikzpicture}


		\caption{Première étape -- Traitement du noeud $v_{0}$. On relaxe :($v_{1},v_{0}$), ($v_{2},v_{0}$), ($v_{3},v_{0}$) et ($v_{0z},v_{0}$). On grise $v_{0}$.}
		\label{ex:reachPrice2}


	\end{figure}
	
%ETAPE 2	
		\begin{figure}[ht!]
			\centering

			\begin{tikzpicture}

				\node[nC] (v7) at (0,0){$+\infty$};
				\node[nR] (v6) at (2,0){$+\infty$};
				\node[nC] (v5) at (4,0){$+\infty$};
				\node[nR] (v4) at (6,0){$+\infty$};
				\node[nC] (v2) at (8,0){$+\infty$};
				\node[nCG] (v0) at (10,0){$0$};
				\node[nR]  (v1) at (8,2){$1$};
				\node[nR] (v3) at (8,-2){$5$};

				\draw[->,>=latex] (v7) to [bend right] node[midway,above]{$1$} (v6);
				\draw[->,>=latex] (v6) to [bend right] node[midway,above]{$1$} (v7);
				\draw[->,>=latex] (v5) to node[midway,above]{$1$} (v6);
				\draw[->,>=latex] (v5) to node[midway,above]{$1$} (v4);


				\draw[->,>=latex] (v4) to node[midway,above]{$5$} (v2);
				\draw[->,>=latex] (v4) to node[midway,left]{$1$} (v3);

				\draw[->,>=latex] (v3) to node[midway,below]{$5$} (v0);
				\draw[->,>=latex] (v3) to node[midway,left]{$1$} (v2);

				\draw[->,>=latex] (v2) to node[midway,above]{$1$} (v0);
				\draw[->,>=latex] (v2) to node[midway,left]{$1$} (v1);

				\draw[->,>=latex] (v1) to node[midway,right]{$1$} (v0);

				\draw[->,>=latex] (v0) to [loop right] node[midway,right]{$1$} (v0);


			\end{tikzpicture}


			\caption{Deuxième étape -- Traitement du noeud $v_{2}$. $nbrSucc = 2$, on n'a donc pas encore testé tous les chemins possibles à partir de ce noeud. Décrémentation de $nbrSucc$, retrait de la plus petit valeur de $S$ et ajout de $v_{0}$ dans les successeurs déjà testés. }
			\label{ex:reachPrice3}


		\end{figure}
%ETAPE 3	
		\begin{figure}[ht!]
		\centering

		\begin{tikzpicture}

			\node[nC] (v7) at (0,0){$+\infty$};
			\node[nR] (v6) at (2,0){$+\infty$};
			\node[nC] (v5) at (4,0){$5$};
			\node[nR] (v4) at (6,0){$+\infty$};
			\node[nC] (v2) at (8,0){$2$};
			\node[nCG] (v0) at (10,0){$0$};
			\node[nRG]  (v1) at (8,2){$1$};
			\node[nR] (v3) at (8,-2){$+\infty$};

			\draw[->,>=latex] (v7) to [bend right] node[midway,above]{$1$} (v6);
			\draw[->,>=latex] (v6) to [bend right] node[midway,above]{$1$} (v7);
			\draw[->,>=latex] (v5) to node[midway,above]{$1$} (v6);
			\draw[->,>=latex] (v5) to node[midway,above]{$1$} (v4);

			\draw[->,>=latex] (v4) to node[midway,above]{$5$} (v2);
			\draw[->,>=latex] (v4) to node[midway,left]{$1$} (v3);

			\draw[->,>=latex] (v3) to node[midway,below]{$5$} (v0);
			\draw[->,>=latex] (v3) to node[midway,left]{$1$} (v2);

     		\draw[->,>=latex] (v2) to node[midway,above]{$1$} (v0);
			\draw[->,>=latex] (v2) to node[midway,left]{$1$} (v1);

			\draw[->,>=latex] (v1) to node[midway,right]{$1$} (v0);

			\draw[->,>=latex] (v0) to [loop right] node[midway,right]{$1$} (v0);


		\end{tikzpicture}


		\caption{Troisième étape -- Traitement du noeud $v_{1}$. On relaxe $(v_{2},v_{1})$. On grise $v_{1}$. }
		\label{ex:reachPrice4}


				\end{figure}
				
%ETAPE 4	
	\begin{figure}[ht!]
	\centering
	\begin{tikzpicture}

	\node[nC] (v7) at (0,0){$+\infty$};
	\node[nR] (v6) at (2,0){$+\infty$};
	\node[nC] (v5) at (4,0){$+\infty$};
	\node[nR] (v4) at (6,0){$7$};
	\node[nCG] (v2) at (8,0){$2$};
	\node[nCG] (v0) at (10,0){$0$};
	\node[nRG]  (v1) at (8,2){$1$};
	\node[nR] (v3) at (8,-2){$3$};

	\draw[->,>=latex] (v7) to [bend right] node[midway,above]{$1$} (v6);
	\draw[->,>=latex] (v6) to [bend right] node[midway,above]{$1$} (v7);
	\draw[->,>=latex] (v5) to node[midway,above]{$1$} (v6);
	\draw[->,>=latex] (v5) to node[midway,above]{$1$} (v4);

	\draw[->,>=latex] (v4) to node[midway,above]{$5$} (v2);
	\draw[->,>=latex] (v4) to node[midway,left]{$1$} (v3);

	\draw[->,>=latex] (v3) to node[midway,below]{$5$} (v0);
	\draw[->,>=latex] (v3) to node[midway,left]{$1$} (v2);

	\draw[->,>=latex] (v2) to node[midway,above]{$1$} (v0);
	\draw[->,>=latex] (v2) to node[midway,left]{$1$} (v1);

	\draw[->,>=latex] (v1) to node[midway,right]{$1$} (v0);

	\draw[->,>=latex] (v0) to [loop right] node[midway,right]{$1$} (v0);


	\end{tikzpicture}

	\caption{Quatrième étape -- Traitement du noeud $v_{2}$. $nbrSucc = 1$. On relaxe $(v_{4},v_{2})$ et $(v_{3},v_{2})$. On grise $v_{2}$. }
	\label{ex:reachPrice5}


	\end{figure}
	
	
%ETAPE 5	
		\begin{figure}[ht!]
		\centering
		\begin{tikzpicture}

		\node[nC] (v7) at (0,0){$+\infty$};
		\node[nR] (v6) at (2,0){$+\infty$};
		\node[nC] (v5) at (4,0){$+\infty$};
		\node[nR] (v4) at (6,0){$4$};
		\node[nCG] (v2) at (8,0){$2$};
		\node[nCG] (v0) at (10,0){$0$};
		\node[nRG]  (v1) at (8,2){$1$};
		\node[nRG] (v3) at (8,-2){$3$};

		\draw[->,>=latex] (v7) to [bend right] node[midway,above]{$1$} (v6);
		\draw[->,>=latex] (v6) to [bend right] node[midway,above]{$1$} (v7);
		\draw[->,>=latex] (v5) to node[midway,above]{$1$} (v6);
		\draw[->,>=latex] (v5) to node[midway,above]{$1$} (v4);

		\draw[->,>=latex] (v4) to node[midway,above]{$5$} (v2);
		\draw[->,>=latex] (v4) to node[midway,left]{$1$} (v3);

		\draw[->,>=latex] (v3) to node[midway,below]{$5$} (v0);
		\draw[->,>=latex] (v3) to node[midway,left]{$1$} (v2);

		\draw[->,>=latex] (v2) to node[midway,above]{$1$} (v0);
		\draw[->,>=latex] (v2) to node[midway,left]{$1$} (v1);

		\draw[->,>=latex] (v1) to node[midway,right]{$1$} (v0);

		\draw[->,>=latex] (v0) to [loop right] node[midway,right]{$1$} (v0);


		\end{tikzpicture}

		\caption{Cinquième étape -- Traitement du noeud $v_{3}$. On relaxe $(v_{4},v_{3})$. On grise $v_{3}$. }
		\label{ex:reachPrice6}


		\end{figure}
		
%ETAPE 6	
\begin{figure}[ht!]
	\centering
	\begin{tikzpicture}
	\node[nC] (v7) at (0,0){$+\infty$};
	\node[nR] (v6) at (2,0){$+\infty$};
	\node[nC] (v5) at (4,0){$5$};
	\node[nRG] (v4) at (6,0){$4$};
	\node[nCG] (v2) at (8,0){$2$};
	\node[nCG] (v0) at (10,0){$0$};
	\node[nRG]  (v1) at (8,2){$1$};
	\node[nRG] (v3) at (8,-2){$3$};

	\draw[->,>=latex] (v7) to [bend right] node[midway,above]{$1$} (v6);
	\draw[->,>=latex] (v6) to [bend right] node[midway,above]{$1$} (v7);
	\draw[->,>=latex] (v5) to node[midway,above]{$1$} (v6);
	\draw[->,>=latex] (v5) to node[midway,above]{$1$} (v4);

	\draw[->,>=latex] (v4) to node[midway,above]{$5$} (v2);
	\draw[->,>=latex] (v4) to node[midway,left]{$1$} (v3);

	\draw[->,>=latex] (v3) to node[midway,below]{$5$} (v0);
	\draw[->,>=latex] (v3) to node[midway,left]{$1$} (v2);

	\draw[->,>=latex] (v2) to node[midway,above]{$1$} (v0);
	\draw[->,>=latex] (v2) to node[midway,left]{$1$} (v1);

	\draw[->,>=latex] (v1) to node[midway,right]{$1$} (v0);

	\draw[->,>=latex] (v0) to [loop right] node[midway,right]{$1$} (v0);


	\end{tikzpicture}

	\caption{Sixième étape -- Traitement du noeud $v_{4}$. On relaxe $(v_{5},v_{4})$. On grise $v_{4}$. }
	\label{ex:reachPrice7}


\end{figure}

%ETAPE 7	
\begin{figure}[ht!]
	\centering
	\begin{tikzpicture}
	\node[nC] (v7) at (0,0){$+\infty$};
	\node[nR] (v6) at (2,0){$+\infty$};
	\node[nC] (v5) at (4,0){$+\infty$};
	\node[nRG] (v4) at (6,0){$4$};
	\node[nCG] (v2) at (8,0){$2$};
	\node[nCG] (v0) at (10,0){$0$};
	\node[nRG]  (v1) at (8,2){$1$};
	\node[nRG] (v3) at (8,-2){$3$};

	\draw[->,>=latex] (v7) to [bend right] node[midway,above]{$1$} (v6);
	\draw[->,>=latex] (v6) to [bend right] node[midway,above]{$1$} (v7);
	\draw[->,>=latex] (v5) to node[midway,above]{$1$} (v6);
	\draw[->,>=latex] (v5) to node[midway,above]{$1$} (v4);

	\draw[->,>=latex] (v4) to node[midway,above]{$5$} (v2);
	\draw[->,>=latex] (v4) to node[midway,left]{$1$} (v3);

	\draw[->,>=latex] (v3) to node[midway,below]{$5$} (v0);
	\draw[->,>=latex] (v3) to node[midway,left]{$1$} (v2);

	\draw[->,>=latex] (v2) to node[midway,above]{$1$} (v0);
	\draw[->,>=latex] (v2) to node[midway,left]{$1$} (v1);

	\draw[->,>=latex] (v1) to node[midway,right]{$1$} (v0);

	\draw[->,>=latex] (v0) to [loop right] node[midway,right]{$1$} (v0);


	\end{tikzpicture}

	\caption{Dernière étape -- Traitement du noeud $v_{4}$. $nbrSucc =2$.On retire la plus petite valeur de $S$. }
	\label{ex:reachPrice8}


\end{figure}

	
%Tableau des files de priorités$
\setlength{\overfullrule}{0pt}

\begin{table}[]
	
\caption{Données stockées dans le champ $S$ pour chaque sommet et chaque étape de l'algorithme}
\label{tab:filePrior}

\begin{tabular}{l|l|l|l|l|l|l|l|l|}
\cline{2-5}
                               & Sommets &    &    &      \\ \hline
\multicolumn{1}{|l|}{}         & $v_{0}$ &$v_{1}$  & $v_{2}$ &$v_{3}$  \\ \hline
\multicolumn{1}{|l|}{Etape 1:} & $(0,null)$   &$(1,v_{0})$     &$(+\infty,null),(1,v_{0})$        &$(5,v_{0})$    \\ \hline
\multicolumn{1}{|l|}{Etape 2:} &$(0,null)$         &$(1,v_{0})$    &$(+\infty,null)$              &$(5,v_{0})$     \\ \hline
\multicolumn{1}{|l|}{Etape 3:} &$(0,null)$         &$(1,v_{0})$    &$(+\infty,null),(2,v_{1})$    &$(5,v_{0})$  \\ \hline
\multicolumn{1}{|l|}{Etape 4:} &$(0,null)$         &$(1,v_{0})$    &$(+\infty,null),(2,v_{1})$    &$(3,v_{2})$   \\ \hline
\multicolumn{1}{|l|}{Etape 5:}   &$(0, null)$         &$(1,v_{0})$    &$(+\infty,null),(2,v_{1})$    &$(3,v_{2})$    \\ \hline 
\multicolumn{1}{|l|}{Etape 6:}   &$(0,null)$         &$(1,v_{0})$    &$(+\infty,null),(2,v_{1})$    &$(3,v_{2})$     \\ \hline
\multicolumn{1}{|l|}{Etape 7:}   &$(0, null)$         &$(1,v_{0})$    &$(+\infty,null),(2,v_{1})$    &$(3,v_{2})$   \\ \hline

\multicolumn{1}{|l|}{}  		&  &    &    &      \\ \hline
\multicolumn{1}{|l|}{}         & $v_{4}$ &$v_{5}$  & $v_{6}$ &$v_{7}$  \\ \hline
\multicolumn{1}{|l|}{Etape 1:} & $(+\infty,null)$   &$(+\infty,null)$     &$(+\infty,null)$    &$(+\infty,null)$    \\ \hline
\multicolumn{1}{|l|}{Etape 2:} &$(+\infty,null)$    &$(+\infty,null)$    &$(+\infty,null)$    &$(+\infty,null)$    \\ \hline
\multicolumn{1}{|l|}{Etape 3:} &$(+\infty,null)$    &$(+\infty,null)$    &$(+\infty,null)$    &$(+\infty,null)$    \\ \hline
\multicolumn{1}{|l|}{Etape 4:} &$(7,v_{2})$    &$(+\infty,null)$    &$(+\infty,null)$    &$(+\infty,null)$    \\ \hline
\multicolumn{1}{|l|}{Etape 5:} &$(4,v_{3})$    &$(+\infty,null)$    &$(+\infty,null)$    &$(+\infty,null)$    \\ \hline
\multicolumn{1}{|l|}{Etape 6:} &$(4,v_{3})$     &$(+\infty,null),(5,v_{4})$    &$(+\infty,null)$    &$(+\infty,null)$    \\ \hline
\multicolumn{1}{|l|}{Etape 7:} &$(4,v_{3})$     &$(+\infty,null)$    &$(+\infty,null)$    &$(+\infty,null)$    \\ \hline

\end{tabular}




\end{table}
\setlength{\overfullrule}{10pt}

\end{exemple}	 

\clearpage