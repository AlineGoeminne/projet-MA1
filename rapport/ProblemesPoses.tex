%!TEX root=main.tex

\subsubsection{Questions posées}
\label{section:questionsPosees}

Dans cette section, nous allons expliciter les différentes questions que nous nous posons et que nous aimerions résoudre.\\


Tout d'abord, considérons le jeu $(\mathcal{G},v_{1})$ où $\mathcal{G} = ( \{ 1,2 \}, V, (V_{1}, V_{2}),E, (Cost _{1},Cost _{2}))$ où: \begin{enumerate}
\item[$\bullet$] Pour tout  $\rho = \rho _{0} \rho _{1} \rho _{2} \ldots $ où $\rho \in Plays$ $Cost_{i}(\rho) = $ $\begin{cases} 
								\min \{ i | \rho _{i} \in Goal_{i} \} & \text{si } \exists i \text{ tq } \rho _{i} \in Goal_{i} \\
								+\infty & \text{ sinon}
								\end{cases}$,
\item[$\bullet$] $Goal_{1} = \{ v_{3} \}$ et $Goal_{2} = \{ v_{0} \}$,
\item[$\bullet$]  $V_{1}$ (resp. $V_{2}$) est représenté par les noeuds ronds (resp. carrés) du graphe de la figure~\ref{ex:patologique}.

\end{enumerate}


\begin{figure}[ht!]
	\centering

	\begin{tikzpicture}
		
		\node[nRG] (v3) at (2,-2){$v_{3}$};
		\node[nC] (v2) at (2,0){$v_{2}$};
		\node[nR] (v1) at (0,0){$v_{1}$};
		\node[nRD] (v0) at (0,-2){$v_{0}$};
	
		\draw[->,>=latex] (v0) to [bend right] (v1);
		\draw[->,>=latex] (v1) to [bend right] (v0);
		
		\draw[->,>=latex] (v1) to [bend right] (v2);
		\draw[->,>=latex] (v2) to [bend right] (v1);
		
		\draw[->,>=latex] (v3) to [bend right] (v2);
		\draw[->,>=latex] (v2) to [bend right] (v3);
		
		
	\end{tikzpicture}
	
	\caption{Jeu d'atteignabilité avec coût}
	\label{ex:patologique}
	

\end{figure}

Soit $\sigma _{1}(v) =$ $\begin{cases}
						v_{2} & \text{si } v = v_{1} \\
						v_{1 } & \text{si } v = v_{0} \\
						v_{2} & \text{si } v = v_{3} 
						\end{cases}$
						
						
						
\noindent et soit $\sigma _{2}(v) = v_{1}$ alors $(\sigma _{1},\sigma _{2})$ est un équilibre de Nash du jeu $(\mathcal{G},v_{1})$ dont l'outcome est $(v_{1}v_{2})^{\omega}$. Nous remarquons qu'avec cet équilibre de Nash aucun des deux joueurs n'atteint son objectif. Nous pouvons également observer que si les deux joueurs coopéraient, ils pourraient tous deux minimiser leur coût. En effet, si les deux joueurs suivaient un profil de stratégie ayant comme outcome $\rho = v_{1}v_{0}v_{1}(v_{2}v_{3})^{\omega} $ nous aurions $Cost_{1}(\rho) = 4$ et $Cost_{2}(\rho) = 1$.

Nous nous posons alors les questions suivantes:

\begin{qst}
	
	Soit $G = (V,E)$ un graphe orienté fortement connexe \footnote{En théorie des graphes, un graphe $G = (V,E)$ est dit fortement connexe si pour tout $u$ et $v$ dans $V$, il existe un chemin de $u$ à $v$} qui représente l'arène d'un jeu d'atteignabilité multijoueur avec coût : $(\mathcal{G},v_{0})$ (pour un certain $v_{0} \in V$).
Existe-t'il un équilibre de Nash tel que chaque joueur atteigne son objectif?

\end{qst}
	

\begin{qst}
	Soit $(\mathcal{G},v_{0})$ où $\mathcal{G} = (V,(V_{Min},V_{Max}),E,RP_{Min},RP_{Max},Goal)$ est un \og reachability-price game\fg  et soit $\rho \in Plays$ un jeu sur $(\mathcal{G},v_{0})$, existe-t'il une procédure algorithmique pour déterminer si ce jeu $\rho$ correspond à l'outcome d'un équilibre de Nash $(\sigma _{1},\sigma _{2})$ pour certaines stratégies $\sigma _{1}\in \Sigma _{Min}$ et $\sigma _{2}\in \Sigma _{Max}$ ?
	
\end{qst}

A partir d'un jeu $(\mathcal{G},v_{0})$ où $\mathcal{G} =( \{ 1, 2 \}, V, (V_{1},V_{2}), E, (\varphi _{1},\varphi _{2}),(Goal_{1},Goal_{2}))$ est un jeu d'atteignabilité à deux joueurs avec coût nous pouvons y associer deux jeux à somme nulle du type \og reachability-price game \fg:
\begin{enumerate}
	\item $\mathcal{G}_{1} = (V,(V_{Min},V_{Max}),E,RP_{Min},RP_{Max},Goal)$ où $V_{Min} = V_{1}$, $V_{Max} = V_{2}$ et $Goal = Goal_{1}$ (i.e. le jeu dans lequel $J_{1}$ tente d'atteindre au plus vite son objectif et où $J_{2}$ veut l'en empêcher),
	\item $\mathcal{G}_{2} = (V,(V_{Min},V_{Max}),E,RP_{Min},RP_{Max},Goal)$ où $V_{Min} = V_{2}$, $V_{Max} = V_{1}$ et $Goal = Goal_{2}$ (i.e. le jeu dans lequel $J_{2}$ tente d'atteindre au plus vite son objectif et où $J_{1}$ veut l'en empêcher).
\end{enumerate}

Pour tout $v \in V$, nous notons alors $Val_{1}(v)$ la valeur de $Val(v)$ calculée dans $\mathcal{G}_{1}$ et $Val_{2}(v)$ la valeur de $Val(v)$ calculée dans $\mathcal{G}_{2}$.

\begin{qst}
	
	\label{qst:3}
	
	Soient $\mathcal{A} = (\Pi, V, (V_{1}, V_{2}), E)$ une arène et $(\mathcal{G} = (\mathcal{A}, (\varphi _{1}, \varphi _{2}), (Goal_{1}, Goal_{2}))$ un jeu d'atteignabilité à deux joueurs à objectif quantitatif, soit $(\mathcal{G}, v_{0})$ le jeu initialisé pour un certain $v_{0} \in V $ soit $\rho = s_{0}s_{1}... \in Plays$, on se demande s'il existe $(\sigma _{1},\sigma _{2})$ un équilibre de Nash dans $(\mathcal{G},v_{0})$ tel que $\rho = Outcome(v_{0},(\sigma _{1},\sigma _{2}))$ et $(\sigma _{1},\sigma _{2}) \in \Sigma _{1} \times \Sigma _{2}.$
\end{qst}

Pour répondre à la question~\ref{qst:3} nous nous intéressons à la véracité de la propriété~\ref{prop:outEN2} :

\begin{propriete}
	\label{prop:outEN2}
	Soient $\mathcal{A} = (\Pi, V, (V_{1}, V_{2}), E)$ une arène et $\mathcal{G} = (\mathcal{A}, (\varphi _{1}, \varphi _{2}), (Goal_{1}, Goal_{2})$ un jeu d'atteignabilité à deux joueurs à objectif quantitatif, soit $(\mathcal{G}, v_{0})$ le jeu initialisé pour un certain $v_{0} \in V $ et soit $\rho = v_{0}v_{1}... \in Plays$. 
	
	Posons $(x,y) = (\varphi _{1}(\rho), \varphi _{2}(\rho))$ et pour $v_{j} \in \rho$ ($j \in \mathbb{N}$) nous définissons: $\epsilon _{j} = \sum _{n= 0} ^{j-1} w(v_{n},v_{n+1})$ où $w$ est la fonction de poids associée à $G = (V,E)$ .
	
	\begin{center}Il existe $(\sigma _{1},\sigma _{2}) \in \Sigma _{1} \times \Sigma _{2}$ un équilibre de Nash dans $(\mathcal{G},v_{0})$ tq $Outcome(v_{0},(\sigma _{1},\sigma _{2})) = \rho$\\ $\text{}$\\ si et seulement si\\$\text{}$\\ pour tout $j \in \mathbb{N}$, $\begin{cases}
													Val_{1}(v_{j}) + \epsilon _{j} \geq x & \text{ si } v_{j} \in V_{1} \\
													Val_{2}(v_{j}) + \epsilon _{j} \geq y & \text{ si } v_{j} \in V_{2} 
													\end{cases}$.\end{center}  
\end{propriete}
	
 						