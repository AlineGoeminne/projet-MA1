%!TEX root=main.tex


\subsection{Jeux qualitatifs}

%DEFINITION: jeu d'atteignabilité à objectif qualitatif
	
	\begin{defi}[Jeu d'atteignabilité à objectif qualitatif]
		Un \textit{jeu d'atteignabilité à objectif qualitatif} est un jeu sur graphe $\mathcal{G} = (\mathcal{A}, (Goal_{i})_{i},(\Omega _{i})_{i \in \Pi})$ où :
		\begin{enumerate}
			\item[$\bullet$] $\mathcal{A} = (\Pi,V,(V_{i})_{i \in \Pi}, E)$ est l'arène d'un jeu sur graphe,
			\item[$\bullet$] Pour tout $i \in \Pi$, $Goal_{i\in \Pi} \subseteq V $ est l'ensemble des sommets de $V$ que $J_{i}$ essaie d'atteindre,
			\item[$\bullet$] Pour tout $i \in \Pi$, $\Omega _{i} = \{(u_{j})_{j \in \mathbb{N}}\in V^{\omega}| \exists k \in \mathbb{N}$  tel que $u_{k}\in Goal_{i}\}$. C'est l'ensemble des jeux $\rho$ sur $\mathcal{G}$ pour lesquels $J_{i}$ gagne le jeu.
		\end{enumerate}	
	\end{defi}
	
% DEFINITION: stratégie gagnante
	\label{strategieGagnante}
	\begin{defi}[Stratégie gagnante]
		Soit $v \in V$, soit $\sigma _{i}$ une stratégie du joueur $i$, on dit que $\sigma _{i}$ est \textit{gagnante pour $J_{i}$} à partir de $v$ si $Outcome(v,(\sigma _{i}, \sigma _{-i})) \subseteq \Omega _{1}$.
	\end{defi}
	
	\begin{rem}
		Dans le cadre de la définition \ref{strategieGagnante} , $Outcome(v,(\sigma _{i}, \sigma _{-i}))$ ne représente pas un seul jeu mais bien un ensemble de jeux. En effet, dans ce cas $\sigma _{-i}$ n'est pas fixé.
	\end{rem}
	
% DEFINITION: ensemble des états gagnants		
	
	\begin{defi}[Ensemble des états gagnants]
		Soit $\mathcal{G} = (\Pi,(V,E),(V_{i})_{i \in \Pi}, (Goal_{i})_{i \in \Pi},(\Omega _{i})_{i \in \Pi})$,\\
		\mbox{$W_{i} = \{ u_{j} |j\in \mathbb{N}$ et il existe une stratégie gagnante $\sigma _{i}$ pour $J_{i}$ à partir de $u_{j}\}$} est \textit{l'ensemble des états gagnants} de $J_{i}$. C'est l'ensemble des sommets de $\mathcal{G}$ à partir desquels $J_{i}$ est assuré de gagner.
	\end{defi}
	
	
	
	Une fois le concept de jeu d'atteignabilité clairement établi, nous pouvons nous poser les questions suivantes : "Quels joueurs peuvent-ils gagner le jeu?" et "Quelle stratégie doivent adopter les joueurs pour atteindre leur objectif quelle que soit la stratégie jouée par les autres joueurs?". \\
	
	Intéressons nous au cas de ces jeux restreints à deux joueurs.
%-------------------------------------
%Cas des jeux à deux joueurs et à somme nulle
%-------------------------------------
	
	\subsubsection{Jeux à deux joueurs et à somme nulle}
	Nous sommes intéressés par l'étude des jeux d'atteignabilité à objectif qualitatif dans le cadre des jeux à deux joueurs. Dans ce cadre, nous notons $\Pi = \{ 1,2\}$. Nous supposons que $\Omega _{2} = V^{\omega}\backslash \Omega _{1}$, on dit alors que le jeu est \textit{à somme nulle}. Ceci signifie que dans le cas du jeu d'atteignabilité à deux joueurs le but de $J_{2}$ est d'empêcher $J_{1}$ d'atteindre son objectif. Nous allons expliciter une méthode permettant de déterminer à partir de quels sommets $J_{1}$ (respectivement $J_{2}$) est assuré de gagner le jeu (respectivement d'empêcher $J_{1}$ d'atteindre son objectif). Dans ce cas, nous posons $F$ l'ensemble des sommets objectifs de $J_{1}$.
	
%PROPRIETE	
	\label{Wempty}
	\begin{propriete}
		Soit $\mathcal{G}$ un jeu, on a : $W_{1}\cap W_{2} = \emptyset$.
	\end{propriete}
	\begin{demonstration}
		Supposons au contraire que $W_{1}\cap W_{2} \neq \emptyset$. Cela signifie qu'il existe $s \in W_{1}$ tel que $s \in W_{2}$.\\
		$s \in W_{1}$ si et seulement si il existe $\sigma _{1}$ une stratégie de $J_{1}$ telle que pour toute $\sigma _{2}$ stratégie de $J_{2}$ nous avons : $Outcome(s,(\sigma _{1},\sigma _{2})) \in \Omega _{1}$.\\
		$s \in W_{2}$ si et seulement si il existe $\tilde{\sigma} _{2}$ une stratégie de $J_{2}$ telle que pour toute $\tilde{\sigma}_{1}$ stratégie de $J_{1}$ nous avons : $Outcome(s,(\tilde{\sigma}_{1},\tilde{\sigma}_{2})) \in \Omega _{2}$.\\
		Dès lors, on obtient : $Outcome(s,(\sigma _{1},\tilde{\sigma}_{2})) \in \Omega _{1} \cap \Omega _{2}$. Or $\Omega _{1} \cap \Omega _{2} = \emptyset$, ce qui amène la contradiction.\\
	\end{demonstration}

%DEFINITION: jeu déterminé	
	\begin{defi}[Jeu déterminé]
		Soit $\mathcal{G}$ un jeu, on dit que ce jeu est \textit{déterminé} si et seulement si $W_{1} = V \backslash W_{2}$.
	\end{defi}

%DEFINITION: predecesseur + ensembles attracteurs	
	\begin{defi}
		\label{def:predecesseur}
		 Soit $X \subseteq V$.\\ 
		Posons $Pre(X) = \{ v \in V_{1}| \exists v'((v,v')\in E) \wedge (v' \in X)\} \cup \{ v \in V_{2}|\forall v' ((v,v')\in E) \Rightarrow (v' \in X)\}$.
		Définissons $(X_{k})_{k \in \mathbb{N}}$ la suite de sous-ensembles de $V$ suivante: \\
		
			$$\left\lbrace
			  \begin{array}{c}
			   X_{0} = F \\
			   X_{k+1} = X_{k} \cup Pre(X_{k})
		       \end{array}
			\right. $$
		
	\end{defi}
	
%PROPRIETE
	
	\begin{propriete}
		\label{prop:suiteUltConst}
		La suite $(X_{k})_{k \in \mathbb{N}}$ est ultimement constante. 
	\end{propriete}
	\begin{demonstration}
		Premièrement, nous avons clairement que  $\forall k \in \mathbb{N}, X_{k} \subseteq X_{k+1}$.\\
		Deuxièmement, nous avons : $\forall k \in \mathbb{N}, |X_{k}| \leq |V| $.\\
		Dès lors, vu que la suite $(X_{k})_{k \in \mathbb{N}}$ est une suite croissante dont la cardinalité des ensembles est bornée par celle de $V$, elle est ultimement constante.\\
		
	\end{demonstration}
	
	
% DEFINITION: attracteur
	
	\begin{defi}
		La limite de la suite $(X_{k})_{k \in \mathbb{N}}$ est appelée \textit{attracteur de F} et est notée $Attr(F)$.
	\end{defi}
	
% PROPRIETE

	\begin{propriete}
		\label{prop:attracteur}
	\begin{equation}
		W_{1} = Attr(F) \label{line1}
	\end{equation}
	\begin{equation}
		W_{2} = V \backslash Attr(F) \label{line2}
	\end{equation}
		
	\end{propriete}
	\begin{demonstration}
		Pour prouver ~\eqref{line1} et ~\eqref{line2} nous allons procéder en plusieurs étapes.\\
		
		\noindent$\mathbf{Attr(F) \subseteq W_{1}}$: Soit $v \in Attr(F)$ alors par la propriétés \ref{prop:suiteUltConst} on a : $Attr(F) = X_{N}$ pour un certain $N \in \mathbb{N}$. Montrons par récurrence sur $n$ que dans ce cas, pour tout $n \in \mathbb{N}$ tel que $X_{n} \subseteq Attr(F)$ on peut construire une stratégie $\sigma _{1}$ pour $J_{1}$ telle que $Outcome(v,(\sigma _{1},\sigma _{2})) \subseteq \Omega _{1}$.
		\begin{enumerate}
			\item[$\star$] Pour $n=0$: alors $v \in X_{0} = F$ et l'objectif est atteint par $J_{1}$.
			\item[$\star$] Supposons que la propriété soit vérifiée pour tout $ 0 \leq n \leq k $ et montrons qu'elle est toujours satisfaite pour $n = k + 1 \leq N$. \\
			Soit $v \in X_{k+1} = X_{k} \cup Pre(X_{k})$. \\
			Si $v \in X_{k}$ alors par hypothèse de récurrence, il existe $\sigma _{1}$ telle que $Outcome(v,(\sigma _{1},\sigma _{2})) \subseteq \Omega _{1}$.\\
			Si $v \in Pre(X_{k})$, alors si $v \in V_{1}$ par définition de $Pre(X_{k})$ on sait qu'il existe $v'\in V_{k}$ tel que $(v,v')\in E$. Ainsi, on définit $\sigma _{1}(v) = v'$. Tandis que si $v \in V_{2}$, par définition de $Pre(X)$, quelle que soit la stratégie $\sigma _{2}$ adoptée par $J_{2}$  nous sommes assurés que $\sigma _{2}(v) \in X_{k}$. Dès lors, le résultat $Outcome(v,(\sigma _{1},\sigma _{2})) \subseteq \Omega _{1}$ est assuré.
		\end{enumerate}
		Par cette preuve par récurrence, l'assertion est bien vérifiée.\\
		
		\noindent$\mathbf{V \backslash Attr(F) \subseteq W_{2}}$: Soit $v \in V \backslash Attr(F)$. Une stratégie gagnante pour $J_{2}$ est une stratégie telle que à chaque tour de jeu le sommet $s$ considéré soit dans l'ensemble $V\backslash Attr(F)$. Sinon, au vu de la preuve précédente, il existerait à partir du sommet $s$ une stratégie gagnante pour $J_{1}$ et $J_{1}$ n'aurait donc qu'à utiliser cette stratégie pour s'assurer la victoire. Donc, si $v \in V_{1}$ par définition de $V \backslash Attr(F)$ on est assuré que pour tout $v'\in V $ tel que $(v,v')\in E$ $v' \in V \backslash Attr(F)$. De plus, si $ v \in V_{2}$ par définition de $V \backslash Attr(F)$ il existe $v' \in V$ tel que $(v,v')\in E$ et $v' \in V \backslash Attr(F)$. La stratégie $\sigma _{2}$ adoptée par $J_{2}$ sera donc $\sigma _{2}(v)= v'$.Procéder de la sorte nous assure $Outcome(v,(\sigma _{1},\sigma _{2})) \subseteq \Omega _{2}$. Ce que nous voulions démontrer.\\
		
		\noindent $\mathbf{W_{1} \subseteq Attr(F)}$: Supposons au contraire : $W_{1} \not\subseteq Attr(F)$. Cela signifie qu'il existe $v \in W_{1}$ tel que $v \notin Attr(F)$. D'où $v \in V\backslash Attr(F)$ et comme $V \backslash Attr(F) \subseteq W_{2}$, on a $v \in W_{2}$. Or par la propriété \ref{Wempty}, $W_{1} \cap W_{2} = \emptyset $ et ici $v \in W_{1}$ et $v \in W_{2}$. Ce qui amène la contradiction.\\
		
		\noindent $\mathbf{W_{2} \subseteq V\backslash Attr(F)}$ : La preuve est similaire à celle de $W_{1} \subseteq Attr(F)$.\\
		
		Ces quatre inclusions d'ensemble démontrent donc ~\eqref{line1} et ~\eqref{line2}.
	\end{demonstration}
	\begin{rem}
		Cette propriété nous montre que les jeux d'atteignabilité à objectif qualitatif et à deux joueurs sont déterminés.
	\end{rem}
	
%EXEMPLE

\begin{exemple}
Soit $\mathcal{G} = ((V,E),V_{1},V_{2},\Omega _{1}, \Omega _{2},F)$ le jeu d'atteignabilité représenté par le graphe ci-dessous. Pour lequel $V = \{ v_{0},v_{1},v_{2},v_{3},v_{4},v_{5},v_{6},v_{7},v_{8} \}$, $E$ est l'ensemble des arcs représentés sur le graphe, $V_{1}$ est représenté par les sommets de forme ronde, $V_{2}$ est représenté par les sommets de forme carrée et $F$ est l'ensemble des sommets grisés.\\
Appliquons sur l'exemple ci-dessous, le principe de l'attracteur.

%!TEX root=main.tex

\begin{figure}[!ht]

	\centering

	\begin{tikzpicture}
		
		\node[nR] (v5)at(2,4){$v_{5}$};
		\node[nC] (v4) at (4,4){$v_{4}$};
		\node[nC] (v2) at (6,4){$v_{2}$};
		\node[nR] (v1) at (8,4){$v_{1}$};
		\node[nRG] (v0) at (8,2){$v_{0}$};
		\node[nR] (v3) at (6,2){$v_{3}$};
		\node[nC] (v6) at (2,2){$v_{6}$};
		\node[nR] (v7) at (3.5,0){$v_{7}$};
		\node[nR] (v8) at (0.5,0){$v_{8}$};
	
		\draw[fleche] (v5)--(v4);
		\draw[fleche] (v4)--(v2);
		\draw[fleche] (v2)--(v1);
		\draw[fleche] (v1)--(v0);
		\draw[->,>=latex] (v0.south) to [out=-95,in=-45](v3.south);
		\draw[fleche] (v3)--(v0);
		\draw[fleche] (v3)--(v2);
		\draw[fleche] (v4)--(v3);
		\draw[fleche] (v5)--(v4);
		\draw[fleche] (v6)--(v8);
		\draw[fleche] (v5)--(v6);
		\draw[fleche] (v8)--(v7);
		\draw[fleche] (v7)--(v6);

	\end{tikzpicture}
	
	
	\caption{$X_{0} = \{ v_{0} \}$ -Situation initiale, le seul état grisé est l'état objectif.}

\end{figure}

\begin{figure}[!ht]
	\centering

	\begin{tikzpicture}
		
		\node[nR] (v5)at(2,4){$v_{5}$};
		\node[nC] (v4) at (4,4){$v_{4}$};
		\node[nC] (v2) at (6,4){$v_{2}$};
		\node[nRG] (v1) at (8,4){$v_{1}$};
		\node[nRG] (v0) at (8,2){$v_{0}$};
		\node[nRG] (v3) at (6,2){$v_{3}$};
		\node[nC] (v6) at (2,2){$v_{6}$};
		\node[nR] (v7) at (3.5,0){$v_{7}$};
		\node[nR] (v8) at (0.5,0){$v_{8}$};
	
		\draw[fleche] (v5)--(v4);
		\draw[fleche] (v4)--(v2);
		\draw[fleche] (v2)--(v1);
		\draw[fleche] (v1)--(v0);
		\draw[->,>=latex] (v0.south) to [out=-95,in=-45](v3.south);
		\draw[fleche] (v3)--(v0);
		\draw[fleche] (v3)--(v2);
		\draw[fleche] (v4)--(v3);
		\draw[fleche] (v5)--(v4);
		\draw[fleche] (v6)--(v8);
		\draw[fleche] (v5)--(v6);
		\draw[fleche] (v8)--(v7);
		\draw[fleche] (v7)--(v6);

	\end{tikzpicture}
	
	
	\caption{$X_{1} = \{ v_{0},v_{1},v_{3}\}$ -- Première étape, $v_{1},v_{3} \in Pre(X_{0})$ car $v_{1},v_{3} \in V_{1}$ et il existe un arc entre $v_{1}$ et $v_{0}$ et entre $v_{3}$ et $v_{0}$.}

\end{figure}


\begin{figure}[!ht]
	\centering

	\begin{tikzpicture}
		
		\node[nR] (v5)at(2,4){$v_{5}$};
		\node[nC] (v4) at (4,4){$v_{4}$};
		\node[nCG] (v2) at (6,4){$v_{2}$};
		\node[nRG] (v1) at (8,4){$v_{1}$};
		\node[nRG] (v0) at (8,2){$v_{0}$};
		\node[nRG] (v3) at (6,2){$v_{3}$};
		\node[nC] (v6) at (2,2){$v_{6}$};
		\node[nR] (v7) at (3.5,0){$v_{7}$};
		\node[nR] (v8) at (0.5,0){$v_{8}$};
	
		\draw[fleche] (v5)--(v4);
		\draw[fleche] (v4)--(v2);
		\draw[fleche] (v2)--(v1);
		\draw[fleche] (v1)--(v0);
		\draw[->,>=latex] (v0.south) to [out=-95,in=-45](v3.south);
		\draw[fleche] (v3)--(v0);
		\draw[fleche] (v3)--(v2);
		\draw[fleche] (v4)--(v3);
		\draw[fleche] (v5)--(v4);
		\draw[fleche] (v6)--(v8);
		\draw[fleche] (v5)--(v6);
		\draw[fleche] (v8)--(v7);
		\draw[fleche] (v7)--(v6);

	\end{tikzpicture}
	
	
	\caption{$X_{2} = \{ v_{0},v_{1},v_{2},v_{3} \}$ -- Deuxième étape, $v_{2} \in Pre(X_{1})$ car $v_{2} \in V_{2}$ et tous les arcs sortant de $v_{2}$ atteignent un état de $X_{1}$.}

\end{figure}

\begin{figure}[!ht]
	\centering

	\begin{tikzpicture}
		
		\node[nR] (v5)at(2,4){$v_{5}$};
		\node[nCG] (v4) at (4,4){$v_{4}$};
		\node[nCG] (v2) at (6,4){$v_{2}$};
		\node[nRG] (v1) at (8,4){$v_{1}$};
		\node[nRG] (v0) at (8,2){$v_{0}$};
		\node[nRG] (v3) at (6,2){$v_{3}$};
		\node[nC] (v6) at (2,2){$v_{6}$};
		\node[nR] (v7) at (3.5,0){$v_{7}$};
		\node[nR] (v8) at (0.5,0){$v_{8}$};
	
		\draw[fleche] (v5)--(v4);
		\draw[fleche] (v4)--(v2);
		\draw[fleche] (v2)--(v1);
		\draw[fleche] (v1)--(v0);
		\draw[->,>=latex] (v0.south) to [out=-95,in=-45](v3.south);
		\draw[fleche] (v3)--(v0);
		\draw[fleche] (v3)--(v2);
		\draw[fleche] (v4)--(v3);
		\draw[fleche] (v5)--(v4);
		\draw[fleche] (v6)--(v8);
		\draw[fleche] (v5)--(v6);
		\draw[fleche] (v8)--(v7);
		\draw[fleche] (v7)--(v6);

	\end{tikzpicture}
	
	
	\caption{$X_{3} = \{ v_{0},v_{1},v_{2},v_{3},v_{4}\} $ -- Troisième étape, $v_{4} \in Pre(X_{2})$ car $v_{4} \in V_{2}$ et tous les arcs sortants de $v_{2}$ atteignent un état de $X_{2}$ ( $v_{2}$ et $v_{3}$).}

\end{figure}


\begin{figure}[!ht]
	\centering

	\begin{tikzpicture}
		
		\node[nRG] (v5)at(2,4){$v_{5}$};
		\node[nCG] (v4) at (4,4){$v_{4}$};
		\node[nCG] (v2) at (6,4){$v_{2}$};
		\node[nRG] (v1) at (8,4){$v_{1}$};
		\node[nRG] (v0) at (8,2){$v_{0}$};
		\node[nRG] (v3) at (6,2){$v_{3}$};
		\node[nC] (v6) at (2,2){$v_{6}$};
		\node[nR] (v7) at (3.5,0){$v_{7}$};
		\node[nR] (v8) at (0.5,0){$v_{8}$};
	
		\draw[fleche] (v5)--(v4);
		\draw[fleche] (v4)--(v2);
		\draw[fleche] (v2)--(v1);
		\draw[fleche] (v1)--(v0);
		\draw[->,>=latex] (v0.south) to [out=-95,in=-45](v3.south);
		\draw[fleche] (v3)--(v0);
		\draw[fleche] (v3)--(v2);
		\draw[fleche] (v4)--(v3);
		\draw[fleche] (v5)--(v4);
		\draw[fleche] (v6)--(v8);
		\draw[fleche] (v5)--(v6);
		\draw[fleche] (v8)--(v7);
		\draw[fleche] (v7)--(v6);

	\end{tikzpicture}
	
	
	\caption{$X_{4}= Attr(F) = \{ v_{0},v_{1},v_{2},v_{3},v_{4}, v_{5} \} $ -- Dernière étape, $v_{5} \in Pre(X_{3})$ car $v_{4} \in V_{1}$ et il existe un arc sortant de $v_{5}$ vers $v_{4} \in Pre(X_{3})$.}

\end{figure}









\FloatBarrier
\end{exemple}

\noindent\textbf{Implémentation}:\\

Au vu de la définition de l'attracteur d'un ensemble, l'implémentation de la résolution d'un jeu d'atteignabilité à deux joueurs avec objectif qualitatif en découle aisément. Nous donnons ci-après le pseudo-code d'un algorithme permettant de résoudre ce type de jeu. \\

Soit $\mathcal{G} = ((V,E),V_{1},V_{2}, F, \Omega _{1}, \Omega _{2})$, nous supposons que $V$ comprend $n$ sommets numérotés de $1$ à $n$ et que le graphe $G = (V,E)$ est encodé sur base de sa \textit{matrice d'adjacence}: matAdj. \emph{I.e.,} soit $M$ une matrice $ n \times n $ telle que $ \forall 1 \leq i \leq n, \forall 1\leq j \leq n , M_{i,j} = 1 \Leftrightarrow (v_{i},v_{j})\in E $ et $M_{i,j} = 0$ sinon.\\

Un premier algorithme calcule à partir d'un sous ensemble $X \subseteq V$ de $V$ l'ensemble $Pre(X)$ comme défini en \ref{def:predecesseur}. Le principe de cet algorithme est le suivant: pour tout sommet $v\in V_{1}$ on teste s'il existe un arc sortant de $v$ vers un sommet de $X$. Si c'est le cas, alors $v\in Pre(X)$. On traite ensuite tous les sommets $v \in V_{2}$. Pour qu'un tel sommet $v$ appartienne à $Pre(X)$ il faut que tous les arcs sortant de $v$ atteignent un sommet de $X$.

\begin{algorithm}
	\caption{PreX}
	\begin{algorithmic}[1]
		\REQUIRE Un sous ensemble $X$ de sommets de $V$.
		\ENSURE Pre(X)
		
		\STATE preX $\leftarrow$ un ensemble vide
		
		\FORALL { $v_{i} \in V_{1}$}	
			\STATE ind $\leftarrow$ 0
			\STATE existeArc $\leftarrow$ faux
			\WHILE{$\neg existeArc$ et $ind \leq n$}
				\IF {($matAdj_{i,ind} = 1$) et ($v_{ind}\in X$)}
					\STATE existeArc $\leftarrow$ vrai
				\ELSE
					\STATE ind $\leftarrow$ ind + 1
				\ENDIF
			\ENDWHILE
			\IF {(existeArc = vrai)}
				\STATE ajouter $v_{i}$ à $preX$
			\ENDIF
		\ENDFOR
		
		\FORALL { $v_{i}\in V_{2}$}
			\STATE ind $\leftarrow$ 0
			\STATE tousArcs $\leftarrow$ vrai
			\WHILE{tousArcs et $ind \leq n$}
				\IF {($matAdj_{i,ind} = 1$) et ($v_{ind}\not\in X$)}
					\STATE tousArcs $\leftarrow$ faux
				\ELSE
					\STATE ind $\leftarrow$ ind + 1
				\ENDIF
			\ENDWHILE
			\IF {(tousArcs = vrai)}
				\STATE ajouter $v_{i}$ à $preX$
			\ENDIF
		\ENDFOR
		
		\RETURN preX
			
\end{algorithmic}
		
\end{algorithm}

Un second algorithme permet de calculer les états gagnants de $J_{1}$ en utilisant les résultats \ref{prop:suiteUltConst} et \ref{prop:attracteur}. En effet, on y construit itérativement la suite $(X_{k})_{k \in \mathbb{N}}$ jusqu'à ce qu'elle devienne ultimement constante (\emph{i.e.,} jusqu'à ce que la cardinalité des ensembles $X_{k}$ ne varie plus).

\begin{algorithm}
	\caption{Attr(F)}
	\begin{algorithmic}[1]
		\REQUIRE $F$ l'ensemble des états objectifs de $J_{1}$
		\ENSURE $Attr(F)$ l'ensemble des états gagnants de $J_{1}$
		
		\STATE $X_{k}$ $\leftarrow$ un ensemble vide
		\STATE tailleDiff $\leftarrow$ vrai
		
		\WHILE {(tailleDiff = vrai)}
			\STATE ancCard $\leftarrow$ $|X_{k}|$
			\STATE $preX_{k}$ $\leftarrow PreX(X_{k})$
			\STATE Ajouter à $X_{k}$ tous les éléments de $preX_{k}$
			\STATE nouvCard $\leftarrow$ $|X_{k}|$
			
			\IF {ancCard = nouvCard}
				\STATE tailleDiff $\leftarrow$ faux
			\ENDIF
		\ENDWHILE
		
		\RETURN $X_{k}$
\end{algorithmic}
\end{algorithm}

%\FloatBarrier
\clearpage


