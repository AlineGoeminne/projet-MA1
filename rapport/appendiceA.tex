%!Tex root=main.tex

\section{AnnexeA}
	\label{algo:dijkstra}

Cette section se base sur le livre de référence de Cormen \emph{et al.} ~\cite{Cormen:2009:IA:580470} (pages 658--662).\\

Soit $G = (v,E)$ un graphe orienté et pondéré,à partir d'un sommet $s$ donné l'algorithme de Dijkstra permet de calculer les plus courts chemins à partir de $s$ vers les autres sommets du graphe. Cet algorithme s'applique sur des graphes orientés pondérés tels que la fonction de poids $w$ associée au graphe vérifie la propriété suivante: pour tout $(u,v)\in E$ on a que $w(u,v) \geq 0$.\\

\noindent \textbf{Idées de l'algorithme:}\\
\begin{enumerate}
	
	\item[$\bullet$] A tout sommet $s'$ de $V$ on associe une valeur $d$ qui représente l'estimation du plus court chemin de $s$ à $s'$. Cette valeur est mise à jour en court d'exécution de l'algorithme afin qu'à la fin de celle-ci $d$ soit exactement le poids du plus court chemin de $s$ à $s'$. On initialise l'algorithme de Dijkstra en mettant la valeur $+\infty$ à tous les sommets et $0$ à $s$. En effet, le plus court chemin pour aller de $s$ à $s$ est de rester en $s$.
	
	\item[$\bullet$] On utilise une file de priorité $Q$ (structure de données permettant de stocker des éléments en fonction de la valeur d'une clef) qui permet de stocker les sommets classés par leur valeur $d$. Sur cette file de priorité on peut effectuer les opérations suivantes: insertion d'un élément, extraction d'un élément ayant la clef de la plus petit valeur,test de la présence ou non d'élément dans $Q$, augmentation ou diminution de la valeur de la clef associée à un sommet. A l'initialisation de l'algorithme $Q = V$.
	
	\item[$\bullet$] Afin de pouvoir retrouver un plus court chemin de $s$ à un autre sommet $s'$ chaque sommet stocke le prédecesseur qui a permis de constituer ce plus court chemin.
	
	\item[$\bullet$] On maintient $S \subseteq V$ un ensemble de sommets qui vérifient la propriété suivante: pour tout sommet $s'\in V$ le plus court chemin de $s$ à $s'$ a déjà été calculé. A l'initialisation de l'algorithme $V = \emptyset$.
	
	\item[$\bullet$]De manière répétée: \begin{enumerate}
										\item On sélectionne un sommet $u \in V \backslash S$ associé à l'approximation minimum du plus court chemin de $s$ à $u$.
										\item On ajoute $u$ à $S$.
										\item On \textit{relaxe} tous les arcs sortant de $u$.
									\end{enumerate}
	\item[$\bullet$] La \textit{relaxation} des arcs sortant de $u$ consiste à vérifier pour tout $u'$ tq $(u,u')\in E$ qu'il n'existe pas un plus court chemin de $s$ vers $u'$ que celui potentiellement déjà calculé et tel que ce nouveau chemin est de la forme $s ... uu'$. Si on trouve un tel nouveau chemin, alors on procède à la mise à jour du prédécesseur de $u'$ ( qui devient en fait $u$).
\end{enumerate}