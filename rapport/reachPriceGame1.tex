%!TEX root=main.tex

\begin{exemple}
	Soit $\mathcal{G} = (V,(V_{Min},V_{Max}),E,g,Goal)$ le \og\textit{reachability-price game}\fg  décrit par la figure \ref{ex:reachPriceGame1} où les sommets contrôlés par $J_{Min}$ sont les sommets ronds et les sommets contrôlés par $J_{Max}$ sont les sommets carrés et $Goal = \{ v_{3} \}$.
	
	\begin{figure}[ht!]
		\centering

		\begin{tikzpicture}

			\node[nRG] (v3) at (4,-2){$v_{3}$};
			\node[nC] (v2) at (4,0){$v_{2}$};
			\node[nR] (v1) at (2,0){$v_{1}$};
			\node[nR] (v0) at (0,0){$v_{0}$};
			\node[nR] (v4) at (6,0){$v_{4}$};

			\draw[->,>=latex] (v0.north) to [out=95,in= 80] node[midway,above]{$1$}(v1.north);
			\draw[->,>=latex] (v1) to node[midway,above]{$1$} (v2);
			\draw[->,>=latex] (v1) to node[midway,above]{$1$} (v0);
			\draw[->,>=latex] (v2) to node[midway,above]{$1$} (v4);
			

			\draw[->,>=latex] (v2) to node[midway,right]{$1$} (v3);
			\draw[->,>=latex] (v4) to node[midway,right]{$1$} (v3);
			
			\draw[->,>=latex] (v3) to node[midway,above]{$1$} (v0);





		\end{tikzpicture}


		\caption{reachability-price game}
		\label{ex:reachPriceGame1}


	\end{figure}
	
Dans un premier temps, montrons que $\mathcal{G}$ est déterminé. Nous avons : $\underline{Val}(v_{0})=\overline{Val}(v_{0})=4$,  $\underline{Val}(v_{1})=\overline{Val}(v_{1}) =3$, $\underline{Val}(v_{2})=\overline{Val}(v_{2})= 2$,  $\underline{Val}(v_{3})=\overline{Val}(v_{3})= 0$, $\underline{Val}(v_{4})=\overline{Val}(v_{4})= 1$.\\


Ensuite, exhibons une stratégie optimale pour le joueur \textit{Min} et une stratégie optimale pour le joueur \textit{Max}.\\

\noindent Prenons la stratégie sans mémoire de $J_{Min}$ définie comme suit : $\sigma _{1}(v) = $ $\begin{cases}
																						v_{0} & \text{si }v = v_{3}\\
																						v_{1} & \text{si }v = v_{0}\\
																						v_{2} & \text{si }v = v_{1}\\
																						v_{3} & \text{si }v = v_{4}
																					\end{cases}.$

\noindent Pour $J_{Max}$, considérons la stratégie sans mémoire suivante :$$\sigma _{2}(v) = v_{4}  \text{ si } v = v_{2}$$.\\
																																										
						
\end{exemple}