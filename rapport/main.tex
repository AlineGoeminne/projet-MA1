\documentclass[12pt,a4paper]{memoire-umons}

\usepackage[utf8]{inputenc}
\usepackage[T1]{fontenc}
\usepackage[francais]{babel}
\usepackage{amssymb,amsmath,amsthm}

\usepackage{float}
\usepackage{graphicx}
\usepackage{caption}
\usepackage{subcaption}
\usepackage{array}
%\usepackage[in,headings]{fullpage}
\usepackage{xcolor}
\usepackage{hyperref}
\usepackage{lscape}


\usepackage{mathrsfs}
\usepackage{amsthm}
\usepackage{amssymb}
\usepackage{amsmath}

\usepackage{tikz}
\usetikzlibrary{decorations.markings}
\usetikzlibrary{positioning}

\usepackage{pgf,tikz}
\usepackage{mathrsfs}
\usetikzlibrary{arrows}

\usepackage{placeins}
\usepackage{algorithm}
\usepackage{algorithmic}

\usepackage{comment}
\usepackage{adjustbox}

\setcounter{secnumdepth}{3}
\setcounter{tocdepth}{3}

\hypersetup{
    colorlinks,
    linkcolor={black!50!black},
    citecolor={black!50!black},
    urlcolor={black!80!black}
}

%---------------------------------------------------------
% Personnal commands
%---------------------------------------------------------
\newcommand\todo[1]{\textcolor{red}{TODO: #1}\PackageWarning{TODO:}{#1!}}

\definecolor{gris}{gray}{0.60}

\setlength{\parindent}{1cm}
\newcommand{\textcalli}[1]{{\small{\textbf{$\negmedspace$\calligra #1}}}}

\tikzset{nRG/.style={draw,circle,fill=gris,minimum height=0.7cm}}
\tikzset{nR/.style={draw,circle,minimum height=0.7cm}}
\tikzset{nC/.style={draw,rectangle,minimum height=0.7cm,minimum width=0.7cm}}
\tikzset{nCG/.style={draw,rectangle,fill=gris,minimum height=0.7cm,minimum width=0.7cm}}
\tikzset{nCD/.style={draw,rectangle,dashed,minimum height=0.7cm,minimum width=0.7cm}}
\tikzset{nCB/.style={draw,rectangle,fill=blue!40,minimum height=0.7cm,minimum width=0.7cm}}
\tikzset{nRB/.style={draw,circle,fill=blue!40,minimum height=0.7cm}}



\tikzset{fleche/.style={->,>=latex}}
\theoremstyle{definition}
\newtheorem{defi}{Définition}[section]
\newtheorem{note}{Note}[section]
\newtheorem{propriete}{Propriété}[section]
\newtheorem{exemple}{Exemple}[section]
\newtheorem{corollaire}{Corollaire}[section]
\theoremstyle{remark}
\newtheorem{rem}{Remarque}[section]
\newtheorem{thm}{Théorème}[section]
\newtheorem{illustration}{Illustration}[section]
\newenvironment{demonstration}{\begin{proof}[\textnormal{\textbf{Preuve.}}]}{\end{proof}}
\theoremstyle{definition}
\newtheorem*{notation}{Notation}
\theoremstyle{definition}
\newtheorem*{notations}{Notations}
%%% francisation des algorithmes
\renewcommand{\algorithmicrequire} {\textbf{\textsc{Entrée(s):}}}
\renewcommand{\algorithmicensure}  {\textbf{\textsc{Sortie(s):}}}
\renewcommand{\algorithmicwhile}   {\textbf{tant que}}
\renewcommand{\algorithmicdo}      {\textbf{faire}}
\renewcommand{\algorithmicendwhile}{\textbf{fin tant que}}
\renewcommand{\algorithmicend}     {\textbf{fin}}
\renewcommand{\algorithmicif}      {\textbf{si}}
\renewcommand{\algorithmicendif}   {\textbf{fin si}}
\renewcommand{\algorithmicelse}    {\textbf{sinon}}
\renewcommand{\algorithmicthen}    {\textbf{alors}}
\renewcommand{\algorithmicfor}     {\textbf{pour}}
\renewcommand{\algorithmicforall}  {\textbf{pour tout}}
\renewcommand{\algorithmicdo}      {\textbf{faire}}
\renewcommand{\algorithmicendfor}  {\textbf{fin pour}}
\renewcommand{\algorithmicloop}    {\textbf{boucler}}
\renewcommand{\algorithmicendloop} {\textbf{fin boucle}}
\renewcommand{\algorithmicrepeat}  {\textbf{répéter}}
\renewcommand{\algorithmicuntil}   {\textbf{jusqu'à}}
\renewcommand{\algorithmicreturn}   {\textbf{retourner}}

\floatname{algorithm}{Algorithme}

\let\mylistof\listof
\renewcommand\listof[2]{\mylistof{algorithm}{Liste des algorithmes}}

% pour palier au problème de niveau des algos
\makeatletter
\providecommand*{\toclevel@algorithm}{0}
\makeatother

%\listofalgorithms % pour lister les algos (après la toc)

\DeclareMathOperator*{\argmin}{arg\,min}

\newcommand{\refrangeconj}[2]{[\ref{#1}--\ref{#2}]}

%\usepackage{hyperref}% hyperliens dans le PDF, pas pour impression
\title{Objectif d'atteignabilité et équilibres de Nash dans les jeux sur graphe}

\author{Aline \textsc{Goeminne}}
\date{2015--2016}
%\directeur{Nom du directeur}
\directeurs{Brihaye Thomas\\ Mélot Hadrien}
%\codirecteurs{}
\service{Service de Mathématiques effectives \\ Service d'Algorithmique}
%\rapporteurs{
 % Rapporteur 1\\
 % Rapporteur 2}
%\discipline{math\'ematiques}

%%%%%%%%%%%%%%%%%%%%%%%%%%%%%%%%%%%%%%%%%%%%%%%%%%%%%%%%%%%%%%%%%%%%%%%%
%% Vos macros


%%%%%%%%%%%%%%%%%%%%%%%%%%%%%%%%%%%%%%%%%%%%%%%%%%%%%%%%%%%%%%%%%%%%%%%%

% Compile uniquement certains morceaux sans perdre les références
% automatiques et la table des matières des parties déjà compilées :
%\includeonly{introduction,chapitre1}

\begin{document}
% Éventuellement utiliser l'environnement « preface » pour avoir une
% numérotation des pages en chiffres romains.
\tableofcontents


%!TEX root=main.tex

\section{Introduction}

\textsc{Théorie des jeux} 

\textsc{Travaux relatifs} 

\textsc{But du projet}
\newpage
%!TEX root=main.tex

\section{Concepts fondamentaux de théorie des jeux}
Dans le cadre de notre projet, nous sommes intéressés par les jeux sur graphe où tous les joueurs ont un \textit{objectif d'atteignabilité}.
Dans cette section, nous abordons la notion de jeux sur graphe de manière générale ainsi que les concepts fondamentaux liés à la théorie des jeux.\\
Cette section est essentiellement inspirée de l'article de Brihaye \emph{et al.} \cite{DBLP:conf/lfcs/BrihayePS13}.

\subsection{Jeux sur graphe}


% Définition d'arène


\begin{defi}[Arène]
	Soit $\Pi$ un ensemble (fini) de joueurs. On appelle \textit{arène} le tuple suivant :\\
	 $\mathcal{A} = (\Pi,V , (V_{i})_{i\in{\Pi}}, E )$ où :
	\begin{enumerate}
		\item[$\bullet$] $G = (V,E)$ est un \textit{graphe orienté}  dont $V$ est l'ensemble (fini) de ses sommets (\textit{vertex}) et $E \subseteq V \times V$ est l'ensemble de ses arcs (\textit{edge}). De plus, pour tout $ v\in V $ il existe $v'\in V$ tel que $(v,v') \in E$ (ie: pour tout sommet dans le graphe, il existe un arc sortant de ce sommet).
		\item[$\bullet$] $(V_{i})_{i\in\Pi}$ est une partition de l'ensemble des sommets du graphe $G$ telle que $V_{i}$ est l'ensemble des sommets du joueur $i$.
	\end{enumerate}
\end{defi}

%Définition de jeu sur graphe

\begin{defi}[Jeu sur graphe]
	Un \textit{jeu sur graphe}, noté $\mathcal{G}$ est la donnée d'une arène $\mathcal{A}$ et d'un \textit{objectif} pour chaque joueur.
\end{defi}

\begin{rem}
	Dans le cadre de ce document, les objectifs qui nous intéressent sont les \textit{objectifs d'atteignabilité}. Cette notion est abordée plus amplement dans la section \ref{sect:jeuxAtt}.
\end{rem}

\begin{defi}[Jeu initialisé]\label{sommetInit}
	Dans certains types de jeux sur graphe, on considère que le jeu commence à partir d'un \textit{sommet initial} donné. On le note communément $v_{0}$. Soit $\mathcal{G}$ un tel jeu, on note alors $(\mathcal{G},v_{0})$ le jeu ayant pour sommet initial $v_{0}$. On appelle $(\mathcal{G},v_{0})$ un \textit{jeu initialisé}.
\end{defi}

\begin{notations}
	Tout au long de ce rapport nous utilisons les conventions suivantes :
	\begin{enumerate}
		\item[$\bullet$] Soit $\mathcal{G}$ un jeu sur graphe, on note $(\mathcal{G},v_{0})$ le jeu ayant $v_{0}$ comme sommet initial .
		
		\item[$\bullet$] On note $J_{i}$  le joueur $i$.
		
		\item[$\bullet$] Pour $i\in \Pi$, on note : $-i\equiv \Pi\backslash \{ i\} $.
	\end{enumerate}
\end{notations}

%---------------------------------------
%Déroulement d'une partie d'un jeu sur graphe
%----------------------------------------


\noindent\textbf{Déroulement d'une partie}\label{derPar}

Nous pouvons imaginer le déroulement d'une partie d'un jeu sur graphe de la manière suivante: pour commencer un jeton est positionné sur un sommet $v_{0}$ du graphe (le sommet initial dans le cas de la remarque \ref{sommetInit}). Ensuite, comme ce sommet appartient à un certain $V_{i}$, le joueur $i$ choisit une arête $(v_{0},v_{1}) \in E$ et fait \og glisser \fg le jeton le long de l'arc vers le sommet $v_{1}$. Ce sommet $v_{1}$ appartient à un certain $V_{j}$, c'est donc au joueur $j$ de choisir une arête $(v_{1},v_{2})\in E$ du graphe et faire \og glisser\fg le jeton le long de cette arête. Le jeu se poursuit de la sorte infiniment. \\


\subsection{Stratégie}



%--------------------------------------
%Notions de chemin/jeu
%--------------------------------------

\noindent\textbf{Notions de chemin et de jeu}

%\todo{Vérifier que j'utilise bien toutes les notions que je définis.}\\

Un \textit{jeu} $\rho \in V^{\omega}$(respectivement une \textit{histoire} $h \in V^{*}$) dans $\mathcal{A}$ est un chemin infini (respectivement fini) à travers le graphe. Nous notons $\epsilon$ l'histoire vide, \textit{Plays} l'ensemble des jeux dans $\mathcal{A}$ et \textit{Hist} l'ensemble des histoires. Nous utilisons les notations suivantes $\rho = \rho _{0}  \rho _{1} \rho _{2}\rho _{3} \ldots$ (où $\rho _{0},  \rho _{1},\ldots \in V$)  représente un jeu et de manière similaire, pour une histoire $h$, $ h = h_{0} h_{1} h_{2} h_{3} ... h_{k}$ ( pour un certain $k \in \mathbb{N}$) où  $h_{0}, h_{1}, \ldots \in V$.
Un \textit{préfixe} de longueur $n+1$ (pour un certain $n\in \mathbb{N}$) d'un jeu $\rho = \rho _{0}  \rho _{1} \rho _{2}\rho _{3} \ldots$ est une histoire $\rho _{0}  \rho _{1} \rho _{2}\rho _{3} \ldots \rho _{n}$ et est notée $\rho[0,n]$. Soit $ h = h_{0} h_{1} \ldots h_{k}$ une histoire et soit $v \in V$ tel que $(h_{k},v)\in E$ on note $hv$ l'histoire $h_{0} h_{1} \ldots h_{k}v$. De même, étant donné une histoire $ h = h_{0} h_{1} h_{2} h_{3} ... h_{k}$ et un jeu $\rho = \rho _{0}  \rho _{1} \rho _{2} \ldots$ tels que $(h_{k},\rho_{0})\in E$ on note $h\rho$ le jeu $ h_{0} h_{1} \ldots h_{k}\rho _{0}  \rho _{1} \rho _{2} \ldots$ .

Etant donné une histoire $ h = h_{0} h_{1} h_{2} h_{3} ... h_{k}$,  on définit une fonction \textit{Last} (respectivement \textit{First}) qui prend comme argument l'histoire $h$ et qui retourne le dernier sommet $h_{k}$ (respectivement le premier sommet $h_{0}$). Nous définissons l'ensemble des histoires telles que c'est au tour du joueur $i \in \Pi$ de prendre une décision $Hist_{i} = \{ h \in Hist | Last(h) \in V_{i} \}$.

\begin{rem}
	Si un sommet initial $v_{0}$ a été fixé, alors tous les jeux (et toutes les histoires) commencent par le sommet $v_{0}.$\\
\end{rem}

%--------------------------------------
% Définition de stratégie + stratégie consistante +  profil de stratégie + outcome
%--------------------------------------

\begin{defi}[Stratégie]
	Une \textit{stratégie} d'un joueur $i \in \Pi$ dans $\mathcal{A}$ est une fonction \mbox{$\sigma _{i}: Hist_{i} \rightarrow V$} telle que à chaque histoire $ h = h_{0} h_{1} h_{2} h_{3} ... h_{k}$ pour laquelle $h_{k} \in V_{i}$ est associée un sommet $v \in V$. De plus, on a: $(Last(h),\sigma _{i}(h))\in E$.
\end{defi}

\begin{defi}[Jeu consistant]	
	Un jeu $\rho = \rho _{0}  \rho _{1} \ldots$ est dit \textit{consistant} avec une stratégie $\sigma _{i}$ du joueur $i$ si pour tout préfixe $p = \rho _{0}\rho _{1}\ldots \rho _{k}$ (pour un certain $k \in \mathbb{N}$) tel que $p \in Hist_{i}$ on a : $\sigma _{i}(p) = \rho_{k+1}$.	
\end{defi}

\begin{rem}
	La notion de jeu consistant est facilement adaptable à la notion d'\textit{histoire consistante}.
\end{rem}

\begin{notations}
	Tout au long de ce document nous utilisons les conventions suivantes:
	\begin{enumerate}
		\item[$\bullet$] Un \textit{profil de stratégies} $(\sigma _{i})_{i \in \Pi}$ est un tuple tel que pour tout $i$ $\sigma _{i}$ désigne la stratégie du 	joueur $i$. 
				
		\item[$\bullet$] Soit  $(\sigma _{i})_{i \in \Pi}$ un profil de stratégies, pour un certain joueur $j\in \Pi $ on note $(\sigma _{i})_{i \in \Pi} = ( \sigma _{j},\sigma _{-j})$ .
		
		\item[$\bullet$] A un profil de stratégie $(\sigma _{i})_{i \in \Pi}$ et à un sommet initial $v_{0}$ est associé un unique jeu $\rho$ qui est consistant avec toutes les stratégies $\sigma _{i}$. Ce jeu est appelé \textit{outcome} de $\sigma _{i}$ et est noté $Outcome(v_{0},(\sigma _{i})_{i\in \Pi})$.
		
		\item[$\bullet$] On note $\Sigma _{i}$ l'ensemble des stratégies de $J_{i}$.
		
	\end{enumerate}
\end{notations}
		



%----------------------------------------
%Stratégie sans mémoire et avec mémoire
%----------------------------------------
\noindent\textbf{Stratégie avec mémoire vs. stratégie sans mémoire}\\



Lorsque l'on cherche des stratégies pour un joueur $J_{i}$, on distingue les \textit{stratégies sans mémoire},les \textit{stratégies avec mémoire finie} et les \textit{stratégies avec mémoire infinie}.

\begin{defi}[Stratégie sans mémoire]
	
	Une stratégie $\sigma _{i} \in \Sigma _{i}$ est une \textit{stratégie sans mémoire} si le choix du prochain sommet dépend uniquement du sommet courant (ie. $\sigma _{i}: V_{i} \rightarrow V$).
\end{defi}

\begin{defi}[Stratégie à mémoire finie]
	
	Une stratégie $\sigma _{i} \in \Sigma _{i}$ est une \textit{stratégie à mémoire finie} si on peut lui associer un \textit{automate de Mealy} $\mathcal{A} = (M, m_{0}, V, \delta, \nu)$ où:
	\begin{enumerate}
		\item[$\bullet$] $M$ est un ensemble fini non vide d'états de mémoire,
		\item[$\bullet$] $m_{0} \in M$ est l'état initial de la mémoire,
		\item[$\bullet$] $\delta : M \times V \rightarrow M$ est la fonction de mise à jour de la mémoire,
		\item[$\bullet$] $ \nu: M \times V_{i} \rightarrow V$ est la fonction de choix, telle que pour tout $m \in M$ et $v\in V_{i}$ $(v, \nu(m,v))\in E$.\end{enumerate}
		
		On peut étendre la fonction de mise à jour de la mémoire à une fonction $\delta ^{*}: M \times Hist \rightarrow M$ définie par récurrence sur la longueur de $h \in Hist$ de la manière suivante :\\ $\begin{cases}
																	\sigma^{*}(m,\epsilon) = m	\\
																	\sigma^{*}(m,hv)=\sigma(\sigma^{*}(m,h),v) & \text{pour tout } m\in M \text{ et } hv\in Hist
																	\end{cases}$ 
																	
		La stratégie $\sigma _{\mathcal{A}_{i}}$ calculée par un automate fini $\mathcal{A}_{i}$ est définie par $\sigma _{\mathcal{A}_{i}}(hv) = \nu(\delta^{*}(m_{0},h),v)$ pour tout $hv \in Hist_{i}$ . Cela signifie qu'à chaque fois qu'une décision est prise, la mémoire est modifiée en conséquence et que chaque nouvelle décision est prise en fonction de la mémoire enregistrée jusque maintenant.
		Dès lors on dit que $\sigma _{i}$ est une stratégie à mémoire finie s'il existe un automate fini $\mathcal{A}_{i}$ tel que $\sigma = \sigma _{\mathcal{A}_{i}}$.		

\end{defi}

\begin{defi}[Stratégie à mémoire infinie]
	
	Une stratégie $\sigma _{i} \in \Sigma _{i}$ est une \textit{stratégie à mémoire finie } si elle n'est ni sans mémoire ni à mémoire finie.
\end{defi}

\newpage
%!TEX root=main.tex

\section{Jeux d'atteignabilité}

Un jeu d'atteignabilité est un jeu sur graphe particulier. Chaque joueur possède un ensemble objectif qu'il souhaite atteindre. Le déroulement d'une partie d'un jeu d'atteignabilité se déroule comme un jeu sur graphe (cf \ref{derPar}) sauf que le but de chaque joueur est d'atteindre un élément de son ensemble objectif. On peut considérer les jeux d'atteignabilité selon deux points de vue: les jeux qualitatifs et les jeux quantitatifs. Nous développons dans les sections suivantes ces deux notions.

%-----------------------------
%-----------------------------
%Jeux qualtitatifs
%-----------------------------
%-----------------------------

\subsection{Jeux qualitatifs}

%DEFINITION: jeu d'atteignabilité à objectif qualitatif
	
	\begin{defi}[Jeu d'atteignabilité à objectif qualitatif]
		Un \textit{jeu d'atteignabilité à objectif qualitatif} est un jeu sur graphe $\mathcal{G} = (\Pi,(V,E),(V_{i})_{i \in \Pi}, (Goal_{i})_{i},(\Omega _{i})_{i \in \Pi})$ où :
		\begin{enumerate}
			\item[$\bullet$] Pour tout $i \in \Pi$, $Goal_{i\in \Pi} \subseteq V $ est l'ensemble des sommets de $V$ que $J_{i}$ essaie d'atteindre.
			\item[$\bullet$] Pour tout $i \in \Pi$, $\Omega _{i} = \{(u_{j})_{j \in \mathbb{N}}\in V^{\omega}| \exists k \in \mathbb{N}$  tel que $u_{k}\in Goal_{i}\}$. C'est l'ensemble des jeux $\rho$ sur $\mathcal{G}$ pour lesquels $J_{i}$ gagne le jeu.
		\end{enumerate}	
	\end{defi}
	
% DEFINITION: stratégie gagnante
	\label{strategieGagnante}
	\begin{defi}[Stratégie gagnante]
		Soit $v \in V$, soit $\sigma _{i}$ une stratégie du joueur $i$, on dit que $\sigma _{i}$ est \textit{gagnante pour $J_{i}$} à partir de $v$ si $Outcome(v,(\sigma _{i}, \sigma _{-i})) \subseteq \Omega _{1}$.
	\end{defi}
	
	\begin{rem}
		Dans le cadre de la définition \ref{strategieGagnante} , $Outcome(v,(\sigma _{i}, \sigma _{-i}))$ ne représente pas un seul jeu mais bien un ensemble de jeux. En effet, dans ce cas $\sigma _{-i}$ n'est pas fixé.
	\end{rem}
	
% DEFINITION: ensemble des états gagnants		
	
	\begin{defi}[Ensemble des états gagnants]
		Soit $\mathcal{G} = (\Pi,(V,E),(V_{i})_{i \in \Pi}, (Goal_{i})_{i \in \Pi},(\Omega _{i})_{i \in \Pi})$,\\
		\mbox{$W_{i} = \{ u_{j} |j\in \mathbb{N}$ et il existe une stratégie gagnante $\sigma _{i}$ pour $J_{i}$ à partir de $u_{j}\}$} est \textit{l'ensemble des états gagnants} de $J_{i}$. C'est l'ensemble des sommets de $\mathcal{G}$ à partir desquels $J_{i}$ est assuré de gagner.
	\end{defi}
	
	
	
	Une fois le concept de jeu d'atteignabilité clairement établi, nous pouvons nous poser les questions suivantes : "Quels joueurs peuvent-ils gagner le jeu?" et "Quelle stratégie doivent adopter les joueurs pour atteindre leur objectif quelle que soit la stratégie jouée par les autres joueurs?". \\
	
	Intéressons nous au cas de ces jeux restreints à deux joueurs.
%-------------------------------------
%Cas des jeux à deux joueurs
%-------------------------------------
	
	\subsubsection{Cas particulier des jeux à deux joueurs}
	Nous sommes intéressés à étudier les jeux d'atteignabilité à objectif qualitatif dans le cadre des jeux à deux joueurs. Dans ce cadre, nous notons $\Pi = \{1,2\}$ et nous avons que $\Omega _{2} = V^{\omega}\backslash \Omega _{1}$. Ceci signifie que dans le cas du jeu d'atteignabilité à deux joueurs le but de $J_{2}$ est d'empêcher $J_{1}$ d'atteindre son objectif. Nous allons expliciter une méthode permettant de déterminer à partir de quels sommets $J_{1}$ (respectivement $J_{2}$) est assuré de gagner le jeu (respectivement d'empêcher $J_{1}$ d'atteindre son objectif).Dans ce cas nous posons $F$ l'ensemble des sommets objectifs de $J_{1}$.
	
	\begin{rem}
		Ce jeu est un exemple de \textit{jeu combinatoire}.
	\end{rem}

%PROPRIETE	
	\label{Wempty}
	\begin{propriete}
		Soit $\mathcal{G}$ un jeu, on a : $W_{1}\cap W_{2} = \emptyset$.
	\end{propriete}
	\begin{demonstration}
		Supposons au contraire que $W_{1}\cap W_{2} \neq \emptyset$. Cela signifie qu'il existe $s \in W_{1}$ tel que $s \in W_{2}$.\\
		$s \in W_{1}$ si et seulement si il existe $\sigma _{1}$ une stratégie de $J_{1}$ telle que pour toute $\sigma {2}$ stratégie de $J_{2}$ nous avons : $Outcome(s,(\sigma _{1},\sigma _{2})) \in \Omega _{1}$.\\
		$s \in W_{2}$ si et seulement si il existe $\tilde{\sigma} _{2}$ une stratégie de $J_{2}$ telle que pour toute $\tilde{\sigma}_{1}$ stratégie de $J_{1}$ nous avons : $Outcome(s,(\tilde{\sigma}_{1},\tilde{\sigma}_{2})) \in \Omega _{2}$.\\
		Dès lors, on obtient : $Outcome(s,(\sigma _{1},\tilde{\sigma}_{2})) \in \Omega _{1} \cap \Omega _{2}$. Or $\Omega _{1} \cap \Omega _{2} = \emptyset$, ce qui amène la contradiction.\\
	\end{demonstration}

%DEFINITION: jeu déterminé	
	\begin{defi}[Jeu déterminé]
		Soit $\mathcal{G}$ un jeu, on dit que ce jeu est \textit{déterminé} si et seulement si $W_{1} = V \backslash W_{2}$.
	\end{defi}

%DEFINITION: predecesseur + ensembles attracteurs	
	\begin{defi}
		 Soit $X \subseteq V$.\\ 
		Posons $Pre(X) = \{ v \in V_{1}| \exists v'((v,v')\in E) \wedge (v' \in X)\} \cup \{ v \in V_{2}|\forall v' ((v,v')\in E) \Rightarrow (v' \in X)\}$
		Définissons $(X_{k})_{k \in \mathbb{N}}$ la suite de sous-ensembles de $V$ suivante: \\
		
			$$\left\lbrace
			  \begin{array}{c}
			   X_{0} = F \\
			   X_{k+1} = X_{k} \cup Pre(X_{k})
		       \end{array}
			\right. $$
		
	\end{defi}
	
%PROPRIETE
	\label{suiteUltConst}
	\begin{propriete}
		
		La suite $(X_{k})_{k \in \mathbb{N}}$ est ultimement constante. 
	\end{propriete}
	\begin{demonstration}
		Premièrement, nous avons clairement que  $\forall k \in \mathbb{N}, X_{k} \subseteq X_{k+1}$.\\
		Deuxièmement, nous avons : $\forall k \in \mathbb{N}, |X_{k}| \leq |V| $.\\
		Dès lors, vu que la suite $(X_{k})_{k \in \mathbb{N}}$ est une suite croissante dont la cardinalité des ensembles est bornée par celle de $V$, elle est ultimement constante.\\
		
	\end{demonstration}
	
	
% DEFINITION: attracteur
	
	\begin{defi}
		La limite de la suite $(X_{k})_{k \in \mathbb{N}}$ est appelée \textit{attracteur de F} et sera notée $Attr(F)$.
	\end{defi}
	
% PROPRIETE

	\begin{propriete}
	\begin{equation}
		W_{1} = Attr(F) \label{line1}
	\end{equation}
	\begin{equation}
		W_{2} = V \backslash Attr(F) \label{line2}
	\end{equation}
		
	\end{propriete}
	\begin{demonstration}
		Pour prouver ~\eqref{line1} et ~\eqref{line2} nous allons procéder en plusieurs étapes.\\
		
		\noindent$\mathbf{Attr(F) \subseteq W_{1}}$: Soit $v \in Attr(F)$ alors par la propriétés \ref{suiteUltConst} on a : $Attr(F) = X_{N}$ pour un certain $N \in \mathbb{N}$. Montrons par récurrence sur $n$ que dans ce cas, pour tout $n \in \mathbb{N}$ tel que $X_{n} \subseteq Attr(F)$ on peut construire une stratégie $\sigma _{1}$ pour $J_{1}$ telle que $Outcome(v,(\sigma _{1},\sigma _{2})) \subseteq \Omega _{1}$.
		\begin{enumerate}
			\item[$\star$] Pour $n=0$: alors $v \in X_{0} = F$ et l'objectif est atteint par $J_{1}$.
			\item[$\star$] Supposons que la propriété soit vérifiée pour tout $ 0 \leq n \leq k $ et montrons qu'elle est toujours satisfaite pour $n = k + 1 \leq N$. \\
			Soit $v \in X_{k+1} = X_{k} \cup Pre(X_{k})$. \\
			Si $v \in X_{k}$ alors par hypothèse de récurrence il existe $\sigma _{1}$ telle que $Outcome(v,(\sigma _{1},\sigma _{2})) \subseteq \Omega _{1}$.\\
			Si $v \in Pre(X_{k})$, alors si $v \in V_{1}$ par définition de $Pre(X_{k})$ on sait qu'il existe $v'\in V_{k}$ tel que $(v,v')\in E$. Ainsi, on définit $\sigma _{1}(v) = v'$. Tandis que si $v \in V_{2}$, par définition de $Pre(X)$, quelle que soit la stratégie $\sigma _{2}$ adoptée par $J_{2}$  nous sommes assurés que $\sigma _{2}(v) \in X_{k}$. Dès lors le résultat $Outcome(v,(\sigma _{1},\sigma _{2})) \subseteq \Omega _{1}$ est assuré.
		\end{enumerate}
		Dès lors l'assertion est bien vérifiée.\\
		
		\noindent$\mathbf{V \backslash Attr(F) \subseteq W_{2}}$: Soit $v \in V \backslash Attr(F)$. Une stratégie gagnante pour $J_{2}$ est une stratégie telle que à chaque tour de jeu le sommet $s$ considéré soit dans l'ensemble $V\backslash Attr(F)$. En effet, sinon, au vu de la preuve précédent il existerait à partir du sommet $s$ une stratégie gagnante pour $J_{1}$ et $J_{1}$ n'aurait donc qu'à utiliser cette stratégie pour s'assurer la victoire. Donc, si $v \in V_{1}$ par définition de 	$V \backslash Attr(F)$ on est assuré que pour tout $v'\in V $ tel que $(v,v')\in E$ $v' \in V \backslash Attr(F)$. De plus, si $ v \in V_{2}$ par définition de $V \backslash Attr(F)$ il existe $v' \in V$ tel que $(v,v')\in E$ et $v' \in V \backslash Attr(F)$. La stratégie $\sigma _{2}$ adoptée par $J_{2}$ sera donc $\sigma _{2}(v)= v'$.Procéder de la sorte nous assure $Outcome(v,(\sigma _{1},\sigma _{2})) \subseteq \Omega _{2}$. Ce que nous voulions démontrer.\\
		
		\noindent $\mathbf{W_{1} \subseteq Attr(F)}$: Supposons au contraire : $W_{1} \not\subseteq Attr(F)$. Cela signifie qu'il existe $v \in W_{1}$ tel que $v \notin Attr(F)$. D'où $v \in V\backslash Attr(F)$ et comme $V \backslash Attr(F) \subseteq W_{2}$, on a $v \in W_{2}$. Or par la propriété \ref{Wempty}, $W_{1} \cap W_{2} = \emptyset $ et ici $v \in W_{1}$ et $v \in W_{2}$. Ce qui amène la contradiction.\\
		
		\noindent $\mathbf{W_{2} \subseteq V\backslash Attr(F)}$ : La preuve est similaire à celle de $W_{1} \subseteq Attr(F)$.\\
		
		Ces quatre inclusions d'ensemble démontrent donc ~\eqref{line1} et ~\eqref{line2}.
	\end{demonstration}
	\begin{rem}
		Cette propriété nous montre que les jeux d'atteignabilité à objectif qualitatif et à deux joueurs sont déterminés.
	\end{rem}
	
%EXEMPLE
		
		
\subsection{Jeux quantitatifs }
	blabla

\newpage
%%!TEX root=main.tex

\section{Questions posées}
\label{section:questionsPosees}

Dans cette section, nous allons expliciter les différentes questions que nous nous posons et que nous aimerions résoudre.\\


Tout d'abord, considérons le jeu $(\mathcal{G},v_{1})$ où $\mathcal{G} = ( \{ 1,2 \}, V, (V_{1}, V_{2}),E, (Cost _{1},Cost _{2}))$ où: \begin{enumerate}
\item[$\bullet$] Pour tout  $\rho = \rho _{0} \rho _{1} \rho _{2} \ldots $ où $\rho \in Plays$ $Cost_{i}(\rho) = $ $\begin{cases} 
								\min \{ i | \rho _{i} \in Goal_{i} \} & \text{si } \exists i \text{ tq } \rho _{i} \in Goal_{i} \\
								+\infty & \text{ sinon}
								\end{cases}$,
\item[$\bullet$] $Goal_{1} = \{ v_{3} \}$ et $Goal_{2} = \{ v_{0} \}$,
\item[$\bullet$]  $V_{1}$ (resp. $V_{2}$) est représenté par les noeuds ronds (resp. carrés) du graphe de la figure~\ref{ex:patologique}.

\end{enumerate}


\begin{figure}[ht!]
	\centering

	\begin{tikzpicture}
		
		\node[nRG] (v3) at (2,-2){$v_{3}$};
		\node[nC] (v2) at (2,0){$v_{2}$};
		\node[nR] (v1) at (0,0){$v_{1}$};
		\node[nRD] (v0) at (0,-2){$v_{0}$};
	
		\draw[->,>=latex] (v0) to [bend right] (v1);
		\draw[->,>=latex] (v1) to [bend right] (v0);
		
		\draw[->,>=latex] (v1) to [bend right] (v2);
		\draw[->,>=latex] (v2) to [bend right] (v1);
		
		\draw[->,>=latex] (v3) to [bend right] (v2);
		\draw[->,>=latex] (v2) to [bend right] (v3);
		
		
	\end{tikzpicture}
	
	\caption{Jeu d'atteignabilité avec coût}
	\label{ex:patologique}
	

\end{figure}

Soit $\sigma _{1}(v) =$ $\begin{cases}
						v_{2} & \text{si } v = v_{1} \\
						v_{1 } & \text{si } v = v_{0} \\
						v_{2} & \text{si } v = v_{3} 
						\end{cases}$
						
						
						
\noindent et soit $\sigma _{2}(v) = v_{1}$ alors $(\sigma _{1},\sigma _{2})$ est un équilibre de Nash du jeu $(\mathcal{G},v_{1})$ dont l'outcome est $(v_{1}v_{2})^{\omega}$. Nous remarquons qu'avec cet équilibre de Nash aucun des deux joueurs n'atteint son objectif. Nous pouvons également observer que si les deux joueurs coopéraient, ils pourraient tous deux minimiser leur coût. En effet, si les deux joueurs suivaient un profil de stratégie ayant comme outcome $\rho = v_{1}v_{0}v_{1}(v_{2}v_{3})^{\omega} $ nous aurions $Cost_{1}(\rho) = 4$ et $Cost_{2}(\rho) = 1$.

Nous nous posons alors les questions suivantes:

\begin{qst}
	
	Soit $G = (V,E)$ un graphe orienté fortement connexe \footnote{En théorie des graphes, un graphe $G = (V,E)$ est dit fortement connexe si pour tout $u$ et $v$ dans $V$, il existe un chemin de $u$ à $v$} qui représente l'arène d'un jeu d'atteignabilité multijoueur avec coût : $(\mathcal{G},v_{0})$ (pour un certain $v_{0} \in V$).
Existe-t'il un équilibre de Nash tel que chaque joueur atteigne son objectif?

\end{qst}
	

\begin{qst}
	\label{qst:2}
	Soit $(\mathcal{G},v_{0})$ où $\mathcal{G} = (V,(V_{Min},V_{Max}),E,RP_{Min},RP_{Max},Goal)$ est un \og reachability-price game\fg  et soit $\rho \in Plays$ un jeu sur $(\mathcal{G},v_{0})$, existe-t'il une procédure algorithmique pour déterminer si ce jeu $\rho$ correspond à l'outcome d'un équilibre de Nash $(\sigma _{1},\sigma _{2})$ pour certaines stratégies $\sigma _{1}\in \Sigma _{Min}$ et $\sigma _{2}\in \Sigma _{Max}$ ?
	
\end{qst}

A partir d'un jeu $(\mathcal{G},v_{0})$ où $\mathcal{G} =( \{ 1, 2 \}, V, (V_{1},V_{2}), E, (\varphi _{1},\varphi _{2}),(Goal_{1},Goal_{2}))$ est un jeu d'atteignabilité à deux joueurs avec coût nous pouvons y associer deux jeux à somme nulle du type \og reachability-price game \fg:
\begin{enumerate}
	\item $\mathcal{G}_{1} = (V,(V_{Min},V_{Max}),E,g,Goal)$ où $V_{Min} = V_{1}$, $V_{Max} = V_{2}$, $g = \varphi_{1}$ et $Goal = Goal_{1}$ (\emph{i.e.,} le jeu dans lequel $J_{1}$ tente d'atteindre au plus vite son objectif et où $J_{2}$ veut l'en empêcher),
	\item $\mathcal{G}_{2} = (V,(V_{Min},V_{Max}),E,g,Goal)$ où $V_{Min} = V_{2}$, $V_{Max} = V_{1}$,$g = \varphi_{2}$ et $Goal = Goal_{2}$ (\emph{i.e.,} le jeu dans lequel $J_{2}$ tente d'atteindre au plus vite son objectif et où $J_{1}$ veut l'en empêcher).
\end{enumerate}

Pour tout $v \in V$, nous notons alors $Val_{1}(v)$ la valeur de $Val(v)$ calculée dans $\mathcal{G}_{1}$ et $Val_{2}(v)$ la valeur de $Val(v)$ calculée dans $\mathcal{G}_{2}$.

\begin{qst}
	
	\label{qst:3}
	
	Soient $\mathcal{A} = (\Pi, V, (V_{1}, V_{2}), E)$ une arène et $(\mathcal{G} = (\mathcal{A}, (\varphi _{1}, \varphi _{2}), (Goal_{1}, Goal_{2}))$ un jeu d'atteignabilité à deux joueurs à objectif quantitatif, soit $(\mathcal{G}, v_{0})$ le jeu initialisé pour un certain $v_{0} \in V $ soit $\rho = s_{0}s_{1}... \in Plays$, on se demande s'il existe $(\sigma _{1},\sigma _{2})$ un équilibre de Nash dans $(\mathcal{G},v_{0})$ tel que $\rho = \langle \sigma _{1},\sigma _{2} \rangle_{v_0}$ et $(\sigma _{1},\sigma _{2}) \in \Sigma _{1} \times \Sigma _{2}.$
\end{qst}

Pour répondre à la question~\ref{qst:3} nous nous intéressons à la véracité de la propriété~\ref{prop:outEN2} :

\begin{propriete}
	\label{prop:outEN2}
	Soient $\mathcal{A} = (\Pi, V, (V_{1}, V_{2}), E)$ une arène et $\mathcal{G} = (\mathcal{A}, (\varphi _{1}, \varphi _{2}), (Goal_{1}, Goal_{2})$ un jeu d'atteignabilité à deux joueurs à objectif quantitatif, soit $(\mathcal{G}, v_{0})$ le jeu initialisé pour un certain $v_{0} \in V $ et soit $\rho = v_{0}v_{1}... \in Plays$. 
	
	Posons $(x,y) = (\varphi _{1}(\rho), \varphi _{2}(\rho))$ et pour $v_{j} \in \rho$ ($j \in \mathbb{N}$) nous définissons: $\varepsilon _{j} = \sum _{n= 0} ^{j-1} w(v_{n},v_{n+1})$ où $w$ est la fonction de poids associée à $G = (V,E)$ .
	
	\begin{center}Il existe $(\sigma _{1},\sigma _{2}) \in \Sigma _{1} \times \Sigma _{2}$ un équilibre de Nash dans $(\mathcal{G},v_{0})$ tq $\langle \sigma _{1},\sigma _{2}\rangle_{v_0} = \rho$\\ $\text{}$\\ si et seulement si\\$\text{}$\\ pour tout $j \in \mathbb{N}$, $\begin{cases}
													Val_{1}(v_{j}) + \varepsilon _{j} \geq x & \text{ si } v_{j} \in V_{1} \\
													Val_{2}(v_{j}) + \varepsilon _{j} \geq y & \text{ si } v_{j} \in V_{2} 
													\end{cases}$.\end{center}  
\end{propriete}
	
\subsection{Question 1}
\subsection{Question 2}
Afin de pouvoir répondre à cette question nous allons commencer par énoncer et prouver un résultat qui nous permettra de déterminer si un outcome donné correspond à un équilibre de Nash.

\begin{propriete}
	\label{prop:question2}
	 Soient $(\mathcal{G},v_0)$ tel que $\mathcal{G} = (\mathcal{A}, g, Goal)$ où $\mathcal{A}= ({Min,Max}, V, (V_{Min},V_{Max}))$ un \og reachability-price game\fg~initialisé, $\rho = v_0 \ldots v_k \ldots \in Plays$ tel que $g(\rho) = x$ pour un certain $x \in \mathbb{N}_0$ et $\varepsilon_k = \sum_{n=0}^{k-1} w(v_n,v_{n-1})$,
	\begin{center} $\exists (\sigma_1, \sigma_2) \in \Sigma_{Min} \times \Sigma_{Max}$ un équilibre de Nash tel que $\langle \sigma_1, \sigma_2 \rangle_{v_0} = \rho$\\ $\text{}$ \\ si et seulement si \\ $\text{}$ \\
		$ \forall v_k \in \rho$,  $\begin{cases} Val(v_k) \geq x - \varepsilon _k & \text{si } v_k \in V_{Min} \\
		 									 Val(v_k) \leq x - \varepsilon _k &  \text{si } v_k \in V_{Max}\end{cases}$ \end{center}

\end{propriete}
\setcounter{equation}{0}

\begin{demonstration}
	
	Nous savons qu'un tel jeu est déterminé et qu'il existe $(\sigma_{1}^* , \sigma_{2}^*) \in (\Sigma_{Min},\Sigma_{Max})$ des stratégies optimales. Nous avons donc qu'il existe $(\sigma_{1}^* , \sigma_{2}^*) \in (\Sigma_{Min},\Sigma_{Max})$ tel que pour tout $v \in V,\, g(\langle \sigma_1^*,\sigma_2^* \rangle_v) = Val(v)$. \\
	
	\begin{itemize}
		\item[($\Downarrow$)] Supposons $(\sigma_1, \sigma_2)$ soit un équilibre de Nash d'outcome $\rho$ et de paiement $x$. Supposons au contraire qu'il existe $v_k \in \rho$ tel que ($Val(v_k) < x - \varepsilon_k$ si $v_k \in V_{Min}$) ou ($Val(v_k) > x - \varepsilon $ si $v_k \in V_{Max}$).\\
		Sans perte de généralité, nous supposons:  
		\begin{align}
			\exists v_k \in \rho (v_k \in V_{Min}) \text{ tel que } Val(v_k) &< x - \varepsilon_k \notag \\
																			&= g(\langle \sigma_1, \sigma_2 \rangle_{v_k}) \label{eq:ENeq1}
		\end{align}
		
		De plus, nous avons :
		\begin{align} Val(v_k) = \sup_{\tau_2 \in \Sigma_{Max}} g(\langle \sigma_1^*, \tau_2 \rangle_{v_k}) \geq g(\langle \sigma_1^*,\sigma_2 \rangle_{v_k}). \label{eq:ENeq2}\end{align}
			
		De \eqref{eq:ENeq1} et \eqref{eq:ENeq2}, nous déduisons:
		\begin{align}
			g(\langle \sigma_1^*, \sigma_2 \rangle _{v_k}) < g (\langle \sigma_1, \sigma_2 \rangle_{v_k}) \label{eq:ENeq3}
		\end{align}
		Comme le joueur Min cherche à minimiser son gain, la relation \eqref{eq:ENeq3} signifie que le joueur Min a une déviation profitable  à partir de $v_k$.\\
		Ceci nous permet de conclure que $(\sigma_1,\sigma_2)$ n'est pas un équilibre de Nash.
		
		\item[($\Uparrow$)]
		Soit $(\tau_1, \tau_2) \in \Sigma_1 \times \Sigma_2$ un profil de stratégies qui permet d'obtenir l'outcome $\rho$ de paiement $x$.
		A partir de $(\tau_1, \tau_2)$ nous désirons construire un équilibre de Nash ayant le même outcome (et donc le même coût).
		L'idée est la suivante: dans un premier temps les deux joueurs suivent leur stratégie conformément au profil $(\tau_1,\tau_2)$. Si un des joueurs, notons le $i$,  dévie de sa stratégie alors l'autre joueur décide de le \og punir \fg~et joue en suivant sa stratégie optimale $\sigma_{-i}^*$\\ 
		
		
		Comme dans le papier \og Multiplayer Cost Games With Simple Nash Equilibria \fg~\cite{DBLP:conf/lfcs/BrihayePS13}, nous définissons une fonction de punition: $P : Hist \rightarrow \{ Min, Max \}\cup \{ \perp \}$ qui permet de définir quel est le premier joueur à avoir dévié du profil de stratégies initial $(\tau_1, \tau_2)$. Cette fonction est telle que $P(h) = \perp$ si aucun joueur n'a dévié le long de l'histoire $h$ et $P(h) = i$ pour un certain $i \in \{ Min, Max \}$ si le joueur $i$ a dévié le long de l'histoire $h$. Nous pouvons donc définir la fonction $P$ par récurrence sur la longueur des histoires : pour $v_0$, le noeud initial, $P(v_0) = \perp$  et pour $h \in Hist$ et $v\in V$ on a :
		$$
		P(hv) = \begin{cases}
				\perp & \text{ si } P(h) = \perp \text{ et } hv \text{ est un préfixe de } \rho \\
				i & \text{ si } P(h) = \perp ,\, hv \text{ n'est pas un préfixe de }\rho \text{ et } Last(h)\in V_i\\
				P(h) & \text{ sinon (\emph{i.e.,}}\, P(h)\neq \perp) \end{cases}
		$$\\
		
		Nous définissons pour tout $h \in Hist_{i}$:
		$\sigma_i(h) = \begin{cases} \tau_i(h) & \text{ si } P(h) = \perp \text{ ou } i \\
		\sigma_i^*(h) & \text{ sinon}
		
		\end{cases}$
		
		Nous avons clairement que $\langle \sigma_1, \sigma_2 \rangle_{v_{k}} = \rho$.\\
		Nous devons maintenant montrer qu'il s'agit d'un équilibre de Nash.
		Supposons au contre que le joueur Max possède une déviation profitable que nous notons $\tilde{\sigma_{2}}$. Comme $\sigma_2$ et $\tilde{\sigma_2}$ sont des stratégies du jeu $(\mathcal{G}, v_0)$ on a que:
		\begin{align} \tilde{\rho} &= \langle \sigma_1, \tilde{\sigma_2} \rangle_{v_0} 
								   = h. \langle \sigma_1, \tilde{\sigma_2} \rangle_{v_k} & \text{ car Max dévie donc Min le punit} \label{eq:ENeq4} \\
								\rho &= \langle \sigma_1, \sigma_2 \rangle_{v_0} 
								     = h. \langle \sigma_1, \sigma_2 \rangle_{v_k} \label{eq:ENeq5}
								\end{align} 
	où $h$ est le plus long préfixe commun et $J_{Max}$ dévie en $v_k$.
	Comme $\tilde{\sigma_2}$ est une déviation profitable et au vu de \eqref{eq:ENeq4} et \eqref{eq:ENeq5} on a:
	\begin{align}
		g(\tilde{\rho}) > g(\rho) && \text{ (Joueur Max maximise son gain}) \label{eq:ENeq6}
	\end{align}
	La relation~\eqref{eq:ENeq6} implique:
	\begin{align}
		g(\langle \sigma_1, \tilde{\sigma_2} \rangle _{v_k}) > g(\langle \sigma_1, \sigma_2 \rangle _{v_k})  \label{eq:ENeq7}
	\end{align}
	
	De plus, 
	\begin{align}
		g(\langle \sigma_1, \tilde{\sigma_2} \rangle_{v_k}) = g( \langle \sigma_1^*, \tilde{\sigma_2} \rangle_{v_k}) \leq Val(v_k) \label{eq:ENeq8}
	\end{align}
	Par hypothèse on a :
	\begin{align}
		Val(v_k) \leq x - \varepsilon_k = g(\langle \tau_1, \tau_2 \rangle_{v_k} ) = g(\langle \sigma_1, \sigma_2 \rangle_{v_k}) \label{eq:ENeq9}
	\end{align}
	Par \eqref{eq:ENeq8} et \eqref{eq:ENeq9} on a : 
	$$ g(\langle \sigma_1, \tilde{\sigma}_2 \rangle_{v_k}) \leq g( \langle \sigma_1, \sigma_2 \rangle_{v_k}).$$
	Ce qui contredit \eqref{eq:ENeq7} et termine notre preuve.
	\end{itemize}
	
\end{demonstration}

Grâce au résultat de la propriété~\ref{prop:question2} nous avons presque répondu à la question~\ref{qst:2}. Pour un outcome donné il suffirait en effet de vérifier que la propriété est vérifiée. Toutefois, l'outcome associé à un profil de stratégies est infini, il faut donc trouver un moyen de le représenter afin qu'un algorithme puisse s'appliquer dessus.

\subsection{Question 3}
Plutôt que de prouver la propriété~\ref{prop:outEN2} pour $|\Pi|= 2$, nous allons montrer qu'en fait nous pouvons la généralise pour $|\Pi| \geq 2 .$ Pour ce faire, nous avons besoin d'introduire quelques notions préliminaires.


\begin{defi}
	\label{defi:coalGame}
 Soient $\mathcal{A} = (\Pi, V, (V_{i})_{i\in\Pi}, E)$ une arène et $\mathcal{G} = (\mathcal{A}, (\varphi _{i})_{i\in\Pi}, (Goal_{i})_{i\in\Pi})$ un jeu d'atteignabilité à $|\Pi| \geq 2$ à objectif quantitatif.
Pour tout joueur $i \in \Pi$, nous pouvons y associer un jeu à somme nulle de type \og reachability-price game \fg~ noté $\mathcal{G}_{i}$.
On définit ce jeu de la manière suivante : 
$$ \displaystyle \mathcal{G}_{i}= (\mathcal{A}_{i}, g , Goal) \text{ où } \mathcal{A}_{i} = (\{i,\Pi\backslash{i}\}, V, (V_{i},V\backslash V_i,E) \text{, } g = \varphi_i \text{ et } Goal = Goal_i$$

\noindent De plus, pour tout $v\in V$, $Val_i(v)$ est la valeur du jeu $\mathcal{G}_i$ pour tout noeud $v\in V$. 
\end{defi} 

En d'autres mots, $G_i$ correspond au jeu où le joueur $i$ (joueur Min) joue contre la coalition $\Pi\backslash\{ i \}$ (joueur Max) . Cela signifie que le joueur $i$ tente d'atteindre son objectif le plus rapidement possible tandis que tous les autres joueurs veulent l'en empêcher (ou tout du moins maximiser son gain). Nous avons vu précédemment qu'un tel jeu est déterminé et que les deux joueurs possèdent une stratégie optimale ($\sigma^*_i$ et $\sigma^*_{-i}$) telles que:
$$ \inf_{\sigma _{i\in \Sigma _{Min}}} \varphi_i(\langle \sigma_i,\sigma^*_{-i}\rangle_v)= Val_i(v) = \sup _{\sigma_{-i}\in \Sigma_{Max}} \varphi_i(\langle \sigma^*_i, \sigma_{-i}\rangle_v).$$ De plus, de la stratégie optimale $\sigma^*_{-i}$ nous pouvons dériver une stratégie pour tout joueur $j \neq i$ que nous notons $\sigma_{j,i}$.\\

Ces considérations étant clairement établies, nous pouvons maintenant énoncer et prouver le résultat~\ref{prop:outEN3} qui nous intéresse.
\begin{propriete}
	\label{prop:outEN3}
	Soit $|\Pi| = n \geq 2$,
	soient $\mathcal{A} = (\Pi, V, (V_{i})_{i\in\Pi}, E)$ une arène et $\mathcal{G} = (\mathcal{A}, (\varphi _{i})_{i\in\Pi}, (Goal_{i})_{i\in\Pi})$ un jeu d'atteignabilité à $n$ joueurs à objectif quantitatif, soit $(\mathcal{G}, v_{0})$ le jeu initialisé pour un certain $v_{0} \in V $ et soit $\rho = v_{0}v_{1}... \in Plays$. 
	
	Posons $(x_{i})_{i\in\Pi} = (\varphi _{i}(\rho))_{i\in\Pi}$ le profil de paiement associé à la partie $\rho$ . Nous définissons pour $v_{k} \in \rho$ ($k \in \mathbb{N}$)  $\varepsilon _{k} := \sum _{n= 0} ^{k-1} w(v_{n},v_{n+1})$ où $w$ est la fonction de poids associée à $G = (V,E)$.
	
	\begin{center}Il existe $ (\sigma _{i})_{i\in\Pi} \in \prod_{i\in\Pi} \Sigma _{i}$ un équilibre de Nash dans $(\mathcal{G},v_{0})$ tq $\langle (\sigma _{i})_{i \in \Pi}\rangle_{v_0} = \rho$\\ $\text{}$\\ si et seulement si\\$\text{}$\\  $ \forall k \in \mathbb{N}, \forall j \in \Pi$, $Val_{j}(v_{k}) + \varepsilon _{k} \geq x_j \text{  si } v_{k} \in V_{j}$.\end{center}
	
\end{propriete}

\setcounter{equation}{0}

\begin{demonstration}
	Nous allons montrer les deux implications:\\
	\begin{itemize}
		\item[$(\Downarrow)$] Supposons au contraire qu'il existe $k\in \mathbb{N}$ et $j\in\Pi$ tels que $Val_j(v_k) + \varepsilon_k < x_j$,
		\begin{equation}
			\label{eq:questEq1}
			i.e., Val_j(v_k) < x_j + \varepsilon_k = \varphi_j(\langle (\sigma_i)_{i \in \Pi}\rangle_{v_k})
		\end{equation}
		où $\varphi_j(\langle (\sigma_i)_{i \in \Pi}\rangle_{v_k})$ est le coût de la partie pour le joueur $j$ si elle avait commencé en $v_k$.
		De plus, on a : 
		\begin{equation}
			\label{eq:questEq2}
			Val_j(v_k) = \sup_{\tau_{-j}\in \Sigma_{Max}} g(\langle \sigma^*_j,\tau_{-j} \rangle_{v_k}) \geq g (\langle \sigma^*_j,\sigma_{-j} \rangle_{v_k}) = \varphi_j(\langle \sigma^*_j,\sigma_{-j} \rangle_{v_k})
		\end{equation}
		où $\sigma^*_j$ est la stratégie optimale du joueur $j$ associée à $\mathcal{G}_j$ et $\sigma_{-j}$ dans l'expression $g (\langle \sigma^*_j,\sigma_{-j} \rangle_{v_k})$ est un abus de notation désignant la stratégie où la coalition $\Pi\backslash\{ j \}$ suit chacune des stratégies $\sigma_i$ pour tout $i \neq j$.\\
		
		Dès lors, \eqref{eq:questEq1} et \eqref{eq:questEq2} nous donnent:
		\begin{equation}
			\label{eq:questEq3}
			\varphi_j(\langle \sigma^*_j,\sigma_{-j} \rangle_{v_k}) < \varphi_j(\langle(\sigma_i)_{i\in \Pi}\rangle_{v_k})
		\end{equation}
		
		La relation~\eqref{eq:questEq3} signifie qu'à partir du noeud $v_k$ le joueur $j$ ferait mieux de suivre la stratégie $\sigma^*_j$. Il s'agit donc d'une déviation profitable pour le joueur $j$ par rapport au profil de stratégies $(\sigma_i)_{i\in \Pi}$. Cela implique que $(\sigma_i)_{i\in\Pi}$ n'est pas un équilibre de Nash. Nous avons donc la contradiction attendue.\\
		
		\item[$(\Uparrow)$] Soit $(\tau_i)_{i\in \Pi}$ un profil de stratégies qui permet d'obtenir l'outcome $\rho$ de paiement $(x_i)_{i\in\Pi}$.
		A partir de $(\tau_i)_{i\in \Pi}$ nous désirons construire un équilibre de Nash ayant le même outcome (et donc le même profil de coût).
		L'idée est la suivante: dans un premier temps tous les joueurs suivent leur stratégie conformément au profil $(\tau_i)_{i \in \Pi}$. Si un des joueurs, notons le $i$,  dévie de sa stratégie alors les autres joueurs se réunissent en une coalition $\Pi\backslash \{ i \}$ et jouent en suivant leur stratégie de punition dans $\mathcal{G}_i$ (\emph{i.e.,} pour tout j $\neq$ i, le joueur $j$ suit la stratégie $\sigma^*_{j,i}$).\\
		
	Comme dans le papier \og Multiplayer Cost Games With Simple Nash Equilibria \fg~\cite{DBLP:conf/lfcs/BrihayePS13}, nous définissons une fonction de punition: $P : Hist \rightarrow \Pi\cup \{ \perp \}$ qui permet de définir quel est le premier joueur à avoir dévié du profil de stratégies initial $(\tau_i)_{i\in\Pi}$. Cette fonction est telle que $P(h) = \perp$ si aucun joueur n'a dévié le long de l'histoire $h$ et $P(h) = i$ pour un certain $i \in \Pi$ si le joueur $i$ a dévié le long de l'histoire $h$. Nous pouvons donc définir la fonction $P$ par récurrence sur la longueur des histoires : pour $v_0$, le noeud initial, $P(v_0) = \perp$  et pour $h \in Hist$ et $v\in V$ on a :
	$$
	P(hv) = \begin{cases}
			\perp & \text{ si } P(h) = \perp \text{ et } hv \text{ est un préfixe de } \rho \\
			i & \text{ si } P(h) = \perp ,\, hv \text{ n'est pas un préfixe de }\rho \text{ et } Last(h)\in V_i\\
			P(h) & \text{ sinon (\emph{i.e.,}}\, P(h)\neq \perp) \end{cases}
	$$\\
	
	Nous pouvons maintenant définir notre équilibre de Nash potentiel dans $\mathcal{G}$. Pour tout $i\in \Pi$ et tout $h\in Hist$ tels que $Last(h)\in V_i$:
	$$\sigma_i(h)= \begin{cases}
					\tau_i(h) & \text{ si }P(h)= \perp \text{ ou }i \\
					\sigma^*_{i,P(h)}(h) & \text{ sinon }\end{cases}$$\\
					
	Nous devons maintenant montrer que le profil de stratégies $(\sigma_i)_{i\in\Pi}$ ainsi défini est un équilibre de Nash d'outcome $\rho$.\\
	Il est clair que $\langle (\sigma_i)_{i\in\Pi} \rangle_{v_0} = \rho$.\\
	Montrons maintenant qu'il s'agit bien d'un équilibre de Nash.\\
	\noindent Supposons au contraire que ce ne soit pas le cas. Cela signifie qu'il existe une déviation profitable pour un certain joueur $j \in\Pi$. Notons-la $\tilde{\sigma}_j$ .\\
		\noindent Soit $\tilde{\rho} = \langle \tilde{\sigma}_j , (\sigma_i)_{i \in \Pi \backslash \{j \}} \rangle_{v_0}$ l'outcome tel que le joueur $j$ joue sa déviation profitable et où les autres joueurs jouent conformément à leur ancienne stratégie.
		Puisque $\tilde{\sigma}_j$ est une déviation profitable nous avons: 
		\begin{equation}
			\label{eq:questEq4}
			\varphi_j(\tilde{\rho}) < \varphi_j(\rho)
		\end{equation}
		
		De plus, comme $\rho$ et $\tilde{\rho}$ commencent tous les deux à partir du noeud $v_0$, ils possèdent un préfixe commun. En d'autres termes, il existe une histoire $hv \in Hist$ telle que: 
		\begin{equation*}
			\rho = h. \langle (\sigma_i)_{i\in\Pi} \rangle_v \text{ et } \tilde{\rho} =  h.\langle \tilde{\sigma_j}, (\sigma)_{i\in\Pi\backslash \{ j \}} \rangle_v
		\end{equation*}
		 S'il en existe plusieurs nous en choisissons une de longueur maximale.
		Au vu de la définition de $\sigma_i$ , nous pouvons réécrire:
		
		\begin{equation*}
			\rho = h. \langle (\tau_i)_{i\in\Pi} \rangle_v \text{ et } \tilde{\rho} = h.\langle \tilde{\sigma_j}, (\sigma^*_{i,j})_{i\in\Pi \backslash\{ j \}}
		\end{equation*}
		En effet, le joueur $j$ dévie en $v$, donc à partir de $v$ tout joueur $i \neq j$ joue sa stratégie de punition. De plus, nous avons les relations suivantes : 
		\begin{align}
			Val_j(v) &= \inf_{\mu_j\in\Sigma_{Min}}\varphi_j(\langle \mu_j, \sigma^*_{-j}\rangle_v)\notag\\
					 &\leq \varphi_j(\langle \tilde{\sigma}_j, \sigma^*_{-j}\rangle_v)\notag\\
					& = \varphi_j(\langle \tilde{\sigma}_j, (\sigma^*_{i,j})_{i \in \Pi \backslash \{ j \}}\rangle_v). \label{eq:questEq5}
		\end{align}
		
	Supposons $h = v_0 \ldots v_k$ pour un certain $k \in \mathbb{N}$. Alors,
	\begin{equation}
		\label{eq:questEq6}
		\varphi_j(\tilde{\rho}) = \varepsilon_k + \varphi(\langle \tilde{\sigma}_j , (\sigma^*_{i,j})_{i\in\Pi\backslash\{ j \}}\rangle_v)
	\end{equation}
	Dès lors, \eqref{eq:questEq5} et~\eqref{eq:questEq6} nous donnent:
	\begin{equation}
		\label{eq:questEq7}
		Val_j(v) \leq \varphi_j(\tilde{\rho}) - \varepsilon_k
	\end{equation}
	
	Donc par~\eqref{eq:questEq4} et~\eqref{eq:questEq7}  nous avons:
	$$ Val_j(v) \leq \varphi_j(\tilde{\rho})- \varepsilon_k < \varphi_j(\rho)-\varepsilon_k = x_j-\varepsilon_k$$
	Ce qui contredit l'hypothèse et conclut notre preuve.
	\end{itemize}
\end{demonstration}


\newpage
%!TEX root=main.tex

\section{Recherche d'équilibres de Nash pertinents}
\label{section:equilibrePert}

Les prérequis sur les jeux d'atteignabilité ayant été clairement expliqué, nous nous intéressons à la recherche - de manière algorithmique - d'équilibres de Nash \og pertinents \fg. Dans un premier temps, nous définissons clairement l'objectif que nous désirons atteindre ainsi que ce signifie pour nous des équilibres de Nash pertinents. Ensuite, nous établissons quelques remarques et propriétés qui nous sont utiles afin de mener à bien notre raisonnement. Enfin, nous expliquons de quelle manière nous désirons mettre en oeuvre des procédés permettant de résoudre cette problématique.

\subsection{Définition du problème et des équilibres pertinents}
\label{subsection:defEqPert}

Tout d'abord, soit $\mathcal{G} = ( \{ 1,2 \}, V, (V_{1}, V_{2}),E, (Cost _{1},Cost _{2}))$, considérons le jeu $(\mathcal{G},v_{1})$ où:\begin{enumerate}
\item[$\bullet$] Pour tout  $\rho = \rho _{0} \rho _{1} \rho _{2} \ldots $ où $\rho \in Plays$ :\\$Cost_{i}(\rho) = $ $\begin{cases} 
								\min \{ i | \rho _{i} \in Goal_{i} \} & \text{si } \exists i \text{ tq } \rho _{i} \in Goal_{i} \\
								+\infty & \text{ sinon}
								\end{cases}$,
\item[$\bullet$] $Goal_{1} = \{ v_{3} \}$ et $Goal_{2} = \{ v_{0} \}$,
\item[$\bullet$]  $V_{1}$ (resp. $V_{2}$) est représenté par les noeuds ronds (resp. carrés) du graphe de la figure~\ref{ex:patologique}.

\end{enumerate}


\begin{figure}[ht!]
	\centering

	\begin{tikzpicture}
		
		\node[nRG] (v3) at (2,-2){$v_{3}$};
		\node[nC] (v2) at (2,0){$v_{2}$};
		\node[nR] (v1) at (0,0){$v_{1}$};
		\node[nRD] (v0) at (0,-2){$v_{0}$};
	
		\draw[->,>=latex] (v0) to [bend right] (v1);
		\draw[->,>=latex] (v1) to [bend right] (v0);
		
		\draw[->,>=latex] (v1) to [bend right] (v2);
		\draw[->,>=latex] (v2) to [bend right] (v1);
		
		\draw[->,>=latex] (v3) to [bend right] (v2);
		\draw[->,>=latex] (v2) to [bend right] (v3);
		
		
	\end{tikzpicture}
	
	\caption{Jeu d'atteignabilité avec coût}
	\label{ex:patologique}
	

\end{figure}

Soit $\sigma _{1}(v) =$ $\begin{cases}
						v_{2} & \text{si } v = v_{1} \\
						v_{1 } & \text{si } v = v_{0} \\
						v_{2} & \text{si } v = v_{3} 
						\end{cases}$
						
						
						
\noindent et soit $\sigma _{2}(v) = v_{1}$ alors $(\sigma _{1},\sigma _{2})$ est un équilibre de Nash du jeu $(\mathcal{G},v_{1})$ dont l'outcome est $(v_{1}v_{2})^{\omega}$. Nous remarquons qu'avec cet équilibre de Nash aucun des deux joueurs n'atteint son objectif. Nous pouvons également observer que si les deux joueurs coopéraient, ils pourraient tous deux minimiser leur coût. En effet, si les deux joueurs suivaient un profil de stratégie ayant comme outcome $\rho = v_{1}v_{0}v_{1}(v_{2}v_{3})^{\omega} $ nous aurions $Cost_{1}(\rho) = 4$ et $Cost_{2}(\rho) = 1$.

La question que nous nous posons alors est la suivante : \og Etant donné un jeu d'atteignabilité à objectifs quantitatifs et multijoueur, nous souhaitons trouver de manière rapide un équilibre de Nash pertinent \fg. Nous avons donc dû nous interroger au sens à donner au concept d'équilibre de Nash pertinent.

Au vu de l'exemple ci-dessus, nous avons mis en lumière le fait que pour certains équilibres de Nash, aucun joueur ne voyait son objectif atteint. Dès lors nous pouvons établir qu'un équilibre de Nash pertinent vérifie:

\begin{itemize}
	\item[$\bullet$] Si on est dans un type de jeu tel qu'il existe (au moins) un équilibre de Nash tel que tous les joueurs voient leur objectif atteint alors on souhaite que si $\rho$ est l'outcome de l'équilibre de Nash trouvé on ait que
	$ \sum_{i \in \Pi} \varphi_i(\rho)$ soit \textbf{minimale}.
	\item[$\bullet$]S'il n'est pas certain qu'il existe un équilibre de Nash du type de celui décrit ci-dessus alors on désire \textbf{maximiser} le nombre de joueurs qui atteignent leur objectif, notons cet ensemble de joueurs $Visit(\rho)$. De plus, on souhaite \textbf{minimiser} $\sum_{i \in \visit(\rho)} \varphi_i(\rho)$.
\end{itemize}

\subsection{Caractérisation de l'outcome d'un équilibre de Nash}

Une des premières questions que nous nous sommes alors posée est la suivante:
\begin{qst}
	\label{qst:1}
	
	Soit $G = (V,E)$ un graphe orienté fortement connexe \footnote{En théorie des graphes, un graphe $G = (V,E)$ est dit fortement connexe si pour tout $u$ et $v$ dans $V$, il existe un chemin de $u$ à $v$} qui représente l'arène d'un jeu d'atteignabilité multijoueur avec coût : $(\mathcal{G},v_{0})$ (pour un certain $v_{0} \in V$).
Existe-t'il un équilibre de Nash tel que chaque joueur atteigne son objectif?

\end{qst}

Nous n'avons pas encore de réponse claire à cette question. Toutefois, pour un jeu à $n$ joueurs nous avons une bonne intuition quant à la manière de passer d'un équilibre de Nash où $n-1$ joueurs atteignent leur objectif à un équilibre de Nash où $n$ joueurs atteignent leur objectif. En effet, à partir du dernier état objectif atteint, si $j$ est le joueur qui n'a pas atteint son objectif, il suffit que tous les joueurs s'allient afin d'atteindre un objectif du joueur $j$.\\\textbf{Cette question est dès lors toujours une question ouverte.}
\\

Nous remarquons que si nous répondons positivement à cette question alors nous rentrons dans les conditions explicitées dans le premier point de la section précédente.


Maintenant, nous nous demandons s'il existe un processus algorithmique qui permet de déterminer si un outcome particulier correspond à l'outcome d'un équilibre de Nash. Dans le cas échéant, nous désirons expliciter ce procédé. Dans la suite de cette section nous répondons positivement à cette question. Nous énonçons d'abord une propriété (propriété~\ref{prop:rechEqpert1}) qui nous permet de déterminer une condition nécessaire et suffisante pour qu'un outcome soit l'outcome d'un certain équilibre de Nash. Toutefois, pour pouvoir effectuer une procédure algorithmique sur les outcomes, il faut que nous trouvions un moyen de représenter ceux-ci car nous travaillons avec des mots infinis. Nous nous convainquons donc par la suite que nous pouvons nous restreindre à l'étude d'équilibres de Nash dont l'outcome est de la forme $\alpha \beta^{\omega}$ où $\alpha$ et $\beta$ sont des mots finis. 

Avant de formuler notre propriété, nous devons aborder quelques notions qui nous sont nécessaires.


\begin{defi}
	\label{defi:coalGame}
 Soient $\mathcal{A} = (\Pi, V, (V_{i})_{i\in\Pi}, E)$ une arène,\\
et $\mathcal{G} = (\mathcal{A}, (\varphi _{i})_{i\in\Pi}, (Goal_{i})_{i\in\Pi})$ un jeu d'atteignabilité où $|\Pi| \geq 2$ à objectif quantitatif.
Pour tout joueur $i \in \Pi$, nous pouvons y associer un jeu à somme nulle de type \og reachability-price game \fg~noté $\mathcal{G}_{i}$.
On définit ce jeu de la manière suivante : 
$ \displaystyle \mathcal{G}_{i}= (\mathcal{A}_{i}, g , Goal) \text{ où }$:
\begin{itemize}
	\item[$\bullet$] $\mathcal{A}_{i} = (\{i,-i \}, V, (V_{i},V\backslash V_i),E)$
	\item[$\bullet$] $g = \varphi_i$ 
	\item[$\bullet$] $Goal = Goal_i$
\end{itemize}

\noindent De plus, pour tout $v\in V$, $Val_i(v)$ est la valeur du jeu $\mathcal{G}_i$ pour tout noeud $v\in V$. 
\end{defi} 

En d'autres mots, $G_i$ correspond au jeu où le joueur $i$ (joueur Min) joue contre la coalition $\Pi\backslash\{ i \}$ (joueur Max). Cela signifie que le joueur $i$ tente d'atteindre son objectif le plus rapidement possible tandis que tous les autres joueurs veulent l'en empêcher (ou tout du moins maximiser son gain). Nous avons vu précédemment qu'un tel jeu est déterminé et que les deux joueurs possèdent une stratégie optimale ($\sigma^*_i$ et $\sigma^*_{-i}$) telles que:
$$ \inf_{\sigma _{i\in \Sigma _{Min}}} \varphi_i(\langle \sigma_i,\sigma^*_{-i}\rangle_v)= Val_i(v) = \sup _{\sigma_{-i}\in \Sigma_{Max}} \varphi_i(\langle \sigma^*_i, \sigma_{-i}\rangle_v).$$ De plus, de la stratégie optimale $\sigma^*_{-i}$ nous pouvons dériver une stratégie pour tout joueur $j \neq i$ que nous notons $\sigma_{j,i}$.\\

Nous sommes maintenant aptes à énoncer notre propriété. La preuve de celle-ci a été effectuée par nos soins, mais nous faisons remarquer qu'une preuve similaire dans le cas des jeux concurrents à informations parfaites a déjà été effectuée par Haddad~\cite{characNashEq}. 

\begin{propriete}
	\label{prop:rechEqpert1}
	Soient $|\Pi| = n \geq 2$, $\mathcal{A} = (\Pi, V, (V_{i})_{i\in\Pi}, E)$ une arène et $\mathcal{G} = (\mathcal{A}, (\varphi _{i})_{i\in\Pi}, (Goal_{i})_{i\in\Pi})$ un jeu d'atteignabilité à $n$ joueurs à objectif quantitatif, on considère $(\mathcal{G}, v_{0})$ le jeu initialisé pour un certain $v_{0} \in V $.\\ Soit \ $\rho = v_{0}v_{1}... \in Plays$, posons $(x_{i})_{i\in\Pi} = (\varphi _{i}(\rho))_{i\in\Pi}$ le profil de paiements associé à la partie $\rho$. Nous définissons pour $v_{k} \in \rho$: $\varepsilon _{k} := \sum _{n= 0} ^{k-1} w(v_{n},v_{n+1})$ où $w$ est la fonction de poids associée à $G = (V,E)$.
	
	\begin{center}Il existe un  profil de stratégies $ (\sigma _{i})_{i\in\Pi} \in \prod_{i\in\Pi} \Sigma _{i}$ qui est un équilibre de Nash dans $(\mathcal{G},v_{0})$ et tel que $\langle (\sigma _{i})_{i \in \Pi}\rangle_{v_0} = \rho$\\ $\text{}$\\ si et seulement si\\$\text{}$\\  $ \forall k \in \mathbb{N}, \forall j \in \Pi$, $Val_{j}(v_{k}) + \varepsilon _{k} \geq x_j \text{  si } v_{k} \in V_{j}$.\end{center}
	
\end{propriete}

\setcounter{equation}{0}

\begin{demonstration}
	Nous allons montrer les deux implications:\\
	\begin{itemize}
		\item[$(\Downarrow)$] Supposons au contraire qu'il existe $k\in \mathbb{N}$ et $j\in\Pi$ tels que $Val_j(v_k) + \varepsilon_k < x_j$,
		\begin{equation}
			\label{eq:questEq1}
			i.e., Val_j(v_k) < x_j - \varepsilon_k = \varphi_j(\langle (\sigma_i)_{i \in \Pi}\rangle_{v_k})
		\end{equation}
		où $\varphi_j(\langle (\sigma_i)_{i \in \Pi}\rangle_{v_k})$ est le coût de la partie pour le joueur $j$ si elle avait commencé en $v_k$.
		De plus, on a : 
		\begin{align}
			\label{eq:questEq2}
			Val_j(v_k) &= \sup_{\tau_{-j}\in \Sigma_{Max}} g(\langle \sigma^*_j,\tau_{-j} \rangle_{v_k}) \notag\\
			           &\geq g (\langle \sigma^*_j,\sigma_{-j} \rangle_{v_k}) = \varphi_j(\langle \sigma^*_j,\sigma_{-j} \rangle_{v_k})
		\end{align}
		où $\sigma^*_j$ est la stratégie optimale du joueur $j$ associée à $\mathcal{G}_j$ et $\sigma_{-j}$ dans l'expression $g (\langle \sigma^*_j,\sigma_{-j} \rangle_{v_k})$ est un abus de notation désignant la stratégie où la coalition $\Pi\backslash\{ j \}$ suit chacune des stratégies $\sigma_i$ pour tout $i \neq j$.\\
		
		Dès lors, \eqref{eq:questEq1} et \eqref{eq:questEq2} nous donnent:
		\begin{equation}
			\label{eq:questEq3}
			\varphi_j(\langle \sigma^*_j,\sigma_{-j} \rangle_{v_k}) < \varphi_j(\langle(\sigma_i)_{i\in \Pi}\rangle_{v_k})
		\end{equation}
		
		La relation~\eqref{eq:questEq3} signifie qu'à partir du noeud $v_k$ le joueur $j$ ferait mieux de suivre la stratégie $\sigma^*_j$. Il s'agit donc d'une déviation profitable pour le joueur $j$ par rapport au profil de stratégies $(\sigma_i)_{i\in \Pi}$. Cela implique que $(\sigma_i)_{i\in\Pi}$ n'est pas un équilibre de Nash. Nous avons donc la contradiction attendue.\\
		
		\item[$(\Uparrow)$] Soit $(\tau_i)_{i\in \Pi}$ un profil de stratégies qui permet d'obtenir l'outcome $\rho$ de paiement $(x_i)_{i\in\Pi}$.
		A partir de $(\tau_i)_{i\in \Pi}$ nous désirons construire un équilibre de Nash ayant le même outcome (et donc le même profil de coût).
		L'idée est la suivante: dans un premier temps tous les joueurs suivent leur stratégie conformément au profil $(\tau_i)_{i \in \Pi}$. Si un des joueurs, notons le $i$,  dévie de sa stratégie alors les autres joueurs se réunissent en une coalition $\Pi\backslash \{ i \}$ et jouent en suivant leur stratégie de punition dans $\mathcal{G}_i$ (\emph{i.e.,} pour tout j $\neq$ i, le joueur $j$ suit la stratégie $\sigma^*_{j,i}$).\\
		
	Comme dans le papier \og Multiplayer Cost Games With Simple Nash Equilibria \fg~\cite{DBLP:conf/lfcs/BrihayePS13}, nous définissons une fonction de punition \linebreak\mbox{$P:Hist \rightarrow \Pi\cup \{ \perp \}$} qui permet de définir quel est le premier joueur à avoir dévié du profil de stratégies initial $(\tau_i)_{i\in\Pi}$. Cette fonction est telle que $P(h) = \perp$ si aucun joueur n'a dévié le long de l'histoire $h$ et $P(h) = i$ pour un certain $i \in \Pi$ si le joueur $i$ a dévié le long de l'histoire $h$. Nous pouvons donc définir la fonction $P$ par récurrence sur la longueur des histoires : pour $v_0$, le noeud initial, $P(v_0) = \perp$  et pour $h \in Hist$ et $v\in V$ on a :

\setlength{\overfullrule}{0pt}
	$$
	P(hv) = \begin{cases}
			\perp & \text{ si } P(h) = \perp \text{ et } hv \text{ est un préfixe de } \rho \\
			\multirow{2}{*}{$i$} & \text{ si } P(h) = \perp ,\, hv \text{ n'est pas un préfixe de }\rho \\
			                   & \text{ et } Last(h)\in V_i \\
			P(h) & \text{ sinon (\emph{i.e.,}}\, P(h)\neq \perp) \end{cases}
	$$\\
	
	
\setlength{\overfullrule}{10pt}	
	Nous pouvons maintenant définir notre équilibre de Nash potentiel dans $\mathcal{G}$. Pour tout $i\in \Pi$ et tout $h\in Hist$ tels que $Last(h)\in V_i$:
	$$\sigma_i(h)= \begin{cases}
					\tau_i(h) & \text{ si }P(h)= \perp \text{ ou }i \\
					\sigma^*_{i,P(h)}(h) & \text{ sinon }\end{cases}$$\\
					
	Nous devons désormais montrer que le profil de stratégies $(\sigma_i)_{i\in\Pi}$ ainsi défini est un équilibre de Nash d'outcome $\rho$.\\
	Il est clair que $\langle (\sigma_i)_{i\in\Pi} \rangle_{v_0} = \rho$.\\
	Montrons maintenant qu'il s'agit bien d'un équilibre de Nash.\\
	\noindent Supposons au contraire que ce ne soit pas le cas. Cela signifie qu'il existe une déviation profitable (notons-la $\tilde{\sigma}_j$) pour un certain joueur $j \in\Pi$.\\ 
		\noindent Soit $\tilde{\rho} = \langle \tilde{\sigma}_j , (\sigma_i)_{i \in \Pi \backslash \{j \}} \rangle_{v_0}$ l'outcome tel que le joueur $j$ joue sa déviation profitable et où les autres joueurs jouent conformément à leur ancienne stratégie.
		Puisque $\tilde{\sigma}_j$ est une déviation profitable nous avons: 
		\begin{equation}
			\label{eq:questEq4}
			\varphi_j(\tilde{\rho}) < \varphi_j(\rho)
		\end{equation}
		
		De plus, comme $\rho$ et $\tilde{\rho}$ commencent tous les deux à partir du noeud $v_0$, ils possèdent un préfixe commun. En d'autres termes, il existe une histoire $hv \in Hist$ telle que: 
		\begin{equation*}
			\rho = h. \langle (\sigma_i)_{i\in\Pi} \rangle_v \text{ et } \tilde{\rho} =  h.\langle \tilde{\sigma_j}, (\sigma)_{i\in\Pi\backslash \{ j \}} \rangle_v
		\end{equation*}
		 S'il en existe plusieurs nous en choisissons une de longueur maximale.
		Au vu de la définition de $\sigma_i$, nous pouvons réécrire:
		
		\begin{equation*}
			\rho = h. \langle (\tau_i)_{i\in\Pi} \rangle_v \text{ et } \tilde{\rho} = h.\langle \tilde{\sigma_j}, (\sigma^*_{i,j})_{i\in\Pi \backslash\{ j \} }\rangle_v
		\end{equation*}
		En effet, le joueur $j$ dévie en $v$, donc à partir de $v$ tout joueur $i \neq j$ joue sa stratégie de punition. De plus, nous avons les relations suivantes : 
		\begin{align}
			Val_j(v) &= \inf_{\mu_j\in\Sigma_{Min}}\varphi_j(\langle \mu_j, \sigma^*_{-j}\rangle_v)\notag\\
					 &\leq \varphi_j(\langle \tilde{\sigma}_j, \sigma^*_{-j}\rangle_v)\notag\\
					& = \varphi_j(\langle \tilde{\sigma}_j, (\sigma^*_{i,j})_{i \in \Pi \backslash \{ j \}}\rangle_v). \label{eq:questEq5}
		\end{align}
		
	Supposons $h = v_0 \ldots v_k$ pour un certain $k \in \mathbb{N}$. Alors,
	\begin{equation}
		\label{eq:questEq6}
		\varphi_j(\tilde{\rho}) = \varepsilon_k + \varphi(\langle \tilde{\sigma}_j , (\sigma^*_{i,j})_{i\in\Pi\backslash\{ j \}}\rangle_v)
	\end{equation}
	Dès lors, \eqref{eq:questEq5} et~\eqref{eq:questEq6} nous donnent:
	\begin{equation}
		\label{eq:questEq7}
		Val_j(v) \leq \varphi_j(\tilde{\rho}) - \varepsilon_k
	\end{equation}
	
	Donc par~\eqref{eq:questEq4} et~\eqref{eq:questEq7}  nous avons:
	$$ Val_j(v) \leq \varphi_j(\tilde{\rho})- \varepsilon_k < \varphi_j(\rho)-\varepsilon_k = x_j-\varepsilon_k$$
	Ce qui contredit l'hypothèse et conclut notre preuve.
	\end{itemize}
\end{demonstration}

Dans sa thèse~\cite{juliePhd}, Julie De Pril explicite une procédure afin de construire à partir d'un équilibre de Nash un équilibre de Nash du même \emph{type} tel que toutes les stratégies sont à mémoire finie. Le type d'un profil de stratégies est l'ensemble des joueurs qui ont visité leur objectif en suivant cet équilibre. Si le profil de stratégies est $(\sigma_i)_{i \in \Pi}$ alors on note le type de ce profil $\type((\sigma_i)_{i\in \Pi})$. Le théorème suivant est donc énoncé:

\begin{thm}
	Etant donné un équilibre de Nash dans un jeu multijoueur initialisé à objectif quantitatif, il existe un équilibre de Nash du même type
\end{thm}

Ce procédé consiste en deux étapes. Etant donné un équilibre de Nash, on commence par construire un second équilibre de Nash du même type duquel on a supprimé les cycles inutiles. Un cycle inutile est un cycle tel que lors de son parcourt aucun joueur qui n'avait pas encore visité son objectif n'atteint celui-ci mais qu'ensuite un nouvel objectif est atteint. Ensuite, on construit un équilibre de Nash $(\sigma_i)_{i\in \Pi}$, toujours avec le même type, tel qu'à partir de l'outcome $\langle (\sigma_i)_{i\in \Pi}\rangle_{v_0}$ on puisse identifier un préfixe $\alpha\beta$ pour lequel on peut répéter infiniment $\beta$. De plus, si nous définissons la notation $\visit(\alpha)$ comme étant l'ensemble des joueurs qui on vu leur objectif atteint lors du parcourt de $\alpha$, nous retrouvons dès lors le résultat suivant:

\begin{propriete}
	\label{prop: rechEqPert1}
	Soit $(\sigma_i)_{i\in \Pi}$ un équilibre de Nash dans le jeu d'atteignabilité multijoueur à objectifs quantitatifs et initialisé $(\mathcal{G}, v_0)$, il existe un équilibre de Nash $(\tau_i)_{i\in \Pi}$ avec le même type et tel que $\langle (\tau_i)_{i\in \Pi} \rangle_{v_0} = \alpha \beta^{\omega}$, où \linebreak $\visit(\alpha) = \type((\sigma_i)_{i \in \Pi})$ et $|\alpha\beta| < (\Pi + 1)\cdot |V|$.
\end{propriete}
Remarquons toutefois que dans le cadre de ces preuves, le fonction de poids $w: E \rightarrow \mathbb{R}$ est la fonction telle que pour tout $e \in E$, $w(e) = 1$. Les preuves s'adaptent toutefois si la fonction de poids est de la forme \linebreak $w : E \rightarrow \mathbb{N}_{0}$. En effet, s'il existe un arc $(v,v')$ tel que $w(v,v') = c$ pour un certain $c \in \mathbb{N}_{0}$, il suffit de rajouter autant d'arc de poids 1 qu'il est nécessaire pour que la somme de ces poids vaille $c$ (cf. figure~\ref{fig:transfoGraphPoids} avec $c = 4$). Remarquons que si le graphe d'origine présente $|V|$ noeud, le graphe transformé, dont l'ensemble des noeuds est $V'$, possède $|V'| = |V| + \sum_{e \in E} (w(e) - 1)$ noeuds. Il semble donc raisonnable que nous nous contentions de considérer des outcomes de longueur $(\Pi + 1)\cdot |V'|.$\\

\textbf{Une étude attentive de la preuve effectuée pour \linebreak la propriété~\ref{prop: rechEqPert1} permettrait peut-être d'exhiber une meilleure bor\-ne sur la longueur maximale des outcomes testés.}

\begin{figure}[!h]
	\centering
	\begin{tikzpicture}
		\node[nR] (v) at (0,0){$v$};
		\node[nR] (v') at (2,0){$v'$};
		\node[nR] (0) at (6,0){$v$};
		\node[nR] (1) at (7.5,0){};
		\node[nR] (2) at (9,0){};
		\node[nR] (3) at (10.5,0){};
		\node[nR] (4) at (12,0){$v'$};
		
		
		
		
		

		\draw[->,>=latex] (v) to node[midway,above]{$4$} (v');
		\draw[-latex] (3,0) -- (5,0);
		\draw[->,>=latex] (0) to node[midway,above]{$1$} (1);
		\draw[->,>=latex] (1) to node[midway,above]{$1$} (2);
		\draw[->,>=latex] (2) to node[midway,above]{$1$} (3);
		\draw[->,>=latex] (3) to node[midway,above]{$1$} (4);
		
		
		
		
		
	\end{tikzpicture}
	\caption{Transformation d'une arête de poids différent de 1.}
	\label{fig:transfoGraphPoids}
\end{figure}

Nous pouvons donc conclure qu'il est correct de se restreindre à le recherche d'équilibres dont l'outcome est de cette forme. En effet, vu qu'ils possèdent le même type cela n'influence pas notre intention de maximiser le nombre de joueurs qui atteignent leur objectif. De plus, puisque les poids sur les arcs sont tous des poids positifs, la suppression des cycles lors de la première étape de la procédure ne fait que diminuer le coût des joueurs pour cet équilibre si un cycle est présent. Cette modification est à notre avantage, en effet, cela diminue la somme des coûts des joueurs qui est la valeur que nous désirons minimiser pour trouver un équilibre pertinent.\\

Nous avons désormais à notre disposition:
\begin{itemize}
	\item[$\bullet$] Une manière de tester si un outcome correspond à l'outcome d'un équilibre de Nash.
	\item[$\bullet$] Un résultat permettant d'affirmer que nous pouvons nous contenter d'examiner les équilibres de Nash de la forme $\alpha \beta^{\omega}$ où $\alpha$ et $\beta$ sont des éléments de $V^{+}$. Nous pouvons donc travailler uniquement à partir de chemins de longueur au plus $(\Pi + 1)\cdot |V'|$. De là, nous pouvons appliquer le point précédant sur $\alpha\beta$ et comme ce mot est un mot fini, un algorithme peut effectuer cette tâche. 
	\item[$\bullet$] Un algorithme (\verb|DijkstraMinMax|) qui permet des récupérer les valeurs de chaque noeud pour tous les jeux  où un joueur joue contre la coalition des autres joueurs. Ces valeurs permettent de vérifier si la propriété du point précédent est respectée.
\end{itemize}
\smallskip
\indent Il nous reste donc à déterminer un algorithme qui nous permet de trouver rapidement un équilibre de Nash pertinent. La phase d'implémentation de ce projet ayant commencé relativement tard les méthodes testées restent naïves et peu efficaces. Avant d'aborder ces dernières avec plus de précisions, nous expliquons certains types d'algorithmes d'exploration desquels nous nous sommes inspirés.

\subsection{Algorithmes d'exploration}

Cette section se base sur le livre de Russel et Norvig~\cite{artInt} ainsi que sur les notes du cours d'Intelligence Artificielle enseigné par Hadrien Mélot à l'université de Mons (notes de l'année 2014--2015).\\


Lorsque l'on recherche une solution à un problème plusieurs méthodes sont applicables en fonction de la nature du problème. Nous nous sommes penchés sur la résolution de notre problème via des méthodes d'exploration. C'est-à-dire qu'à un problème donné est associé un \emph{espace d'états}. Ces états sont parcourus jusqu'à atteindre l'objectif attendu, celui-ci correspond à la solution du problème.

Dans notre cas, l'objectif que nous désirons atteindre est de trouver un chemin dans le graphe du jeu qui corresponde à l'outcome d'un équilibre de Nash. La pertinence de cet équilibre est jaugée via les critères évoqués dans la section~\ref{subsection:defEqPert}. Le but ultime est donc d'atteindre l'équilibre le plus pertinent.

Différents types d'exploration sont possibles, deux grandes familles sont à évoquer:

\begin{itemize}
	\item[$\bullet$] Algorithme d'exploration non informée: seule la définition du problème est connue (\emph{blind search}).
	\item[$\bullet$] Algorithme d'exploration informée: la recherche est guidée grâce à des données supplémentaires (\emph{heuristic search}).
\end{itemize}

\begin{exemple}
	Un voyageur qui se situe dans une ville A souhaite rejoindre une ville B. Pour ce faire, il aimerait emprunter le chemin le plus court pour relier A et B. Une exploration non informée testerait les différentes villes voisines de la ville A et ainsi de suite jusqu'à atteindre la ville B. Une approche plus éclairée serait donc d'utiliser la distance à vol d'oiseau séparant une ville C de la ville B afin de pouvoir choisir à chaque fois quelle ville voisine C visiter en priorité. Il s'agit alors d'une exploration informée.
\end{exemple}

L'objectif à atteindre étant établi, le problème doit maintenant être défini. Les cinq composants suivant sont essentiels:

\begin{description}
	\item[L'état initial] est l'état à partir duquel l'exploration commence. Dans notre cas, puis que nous travaillons avec des jeux initialisés par un certain sommet $v_0$, l'état initial est le chemin $v_0$.
	\item[Les actions possibles] à partir d'un état donné afin de pouvoir continuer l'exploration. Ici, si le dernier noeud du chemin courant est la noeud $v_i$, les actions possibles sont de choisir un noeud parmi ceux de $Succ(v_i) = \{ v\, |\, (v_i, v) \in E \}.$
	\item[Le modèle de transition] associe à partir d'un état courant et d'une action l'état successeur. Si on est à l'état $h = h_1 ... h_k$ et que l'action choisie est de sélectionner le noeud $v$, alors l'état successeur est l'état $hv$.
\end{description}

L'état initial, les actions possibles ainsi que le modèle de transition définissent \emph{l'espace d'états} qui est l'ensemble de tous les états possibles associés au problème.

\begin{description}
	\item[Un test] détermine si un état correspond à un état objectif. Dans le cas présent, ce test consiste soit à vérifier si la longueur du chemin correspond à la longueur maximale de l'outcome d'un équilibre de Nash et dans le cas échéant de tester si ce chemin en est bien un équilibre de Nash via le critère de la propriété~\ref{prop:rechEqpert1}, soit tous les joueurs ont atteint leur objectif et nous vérifions via ce même critère si ce chemin est un équilibre de Nash.
	
	\item[Une fonction de coût] modélise la qualité du chemin courant. Une solution optimale est telle que le coût associé à l'état objectif solution est minimal.
\end{description}

Maintenant que le modèle est explicité, nous expliquons de quelle manière la recherche est effectuée. Tous les algorithmes d'exploration utilisent cette structure. La différence entre ces différents algorithmes est la manière dont les états successifs sont sélectionnés, cette notion est nommée \emph{stratégie d'exploration}.

La recherche se fait en parcourant un arbre dont la racine est l'état initial, les branches les actions possibles et les noeuds les états. A chaque fois qu'un état est sélectionné tous ses successeurs sont générés et sont ajoutés à l'ensemble des états qui attendent d'être visités. Cette ensemble est appelée \emph{frontière} tandis que ce processus de génération des successeurs d'un état est appelé \emph{expansion}. Tant qu'un état objectif n'est pas atteint, un nouvel état est sélectionné dans la frontière et ses successeurs y sont ajoutés. Cette procédure est appelée \emph{tree search}.

L'inconvénient de cette approche est que l'arbre généré peut-être infini. En effet, supposons qu'il existe un arc $(v_0, v_1)$ et un arc $(v_1, v_0)$ dans le graphe du jeu, alors le chemin $ (v_0v_1)^{\omega}$ peut être généré et le processus pourrait ne jamais s'arrêter. Dans notre cas, puisque nous nous restreignant à la recherche de chemin d'une certaine longueur maximale ce comportement n'est pas dérangeant. En effet, une fois la longueur maximale du chemin atteinte, nous pouvons couper la recherche à partir de ce noeud. De plus, empêcher l'apparition de cycle n'est pas envisageable car même pour la recherche d'équilibre de Nash pertinent des cycles peuvent être nécessaires (cf. exemple de la section~\ref{prop:rechEqpert1} pour lequel un outcome  d'équilibre de Nash pertinent est $\rho = v_{1}v_{0}v_{1}(v_{2}v_{3})^{\omega} $).\\

Comme nous l'avons déjà précisé, les différents algorithmes d'exploration différent entre eux par leur stratégie d'exploration. La structure de donnée utilisée pour représenter la frontière influence donc l'ordre de traitement des noeuds de l'arbre d'exploration. Les trois structures suivantes sont utilisées : les piles (le dernier élément ajouté est le premier retiré), les files (la premier élément ajouté est le premier élément retiré) et les files de priorités (les éléments sont triés selon une certaine préférence).

\begin{comment}

Les stratégies d'exploration sont évaluées selon quatre critères:
\begin{itemize}
	\item[\textbf{Complétude}] La stratégie trouve-t-elle toujours une solution si elle existe?
	\item[\textbf{Complexité en temps}] Quel est le nombre de noeuds de l'arbre d'exploration qui sont générés lors de celle-ci?
	\item[\textbf{Complexité en espace}] Quel est le nombre de noeuds maximal gardé en mémoire?
	\item[\textbf{Optimalité}] La solution retournée est-elle une solution optimale?
\end{itemize}
	
Pour exprimer la complexité en temps et en mémoire, les notations suivantes sont utilisées:

\begin{itemize}
	\item[$\bullet$] \emph{d}: distance minimale entre l'état initial et l'état objectif.
	\item[$\bullet$] \emph{b}: nombre maximum d'enfants d'un noeud dans l'arbre d'exploration.
	\item[$\bullet$] \emph{m}: profondeur maximale de l'espace d'état.	
\end{itemize}
\end{comment}

L'utilisation d'une file permet, par exemple, une exploration en largeur de l'arbre - appelée \emph{breadth-first search}. Cette exploration visite en premier les noeuds de l'arbre qui sont les moins profonds. Tandis qu'une file assure prioritairement l'expansion des noeuds les plus profonds. Cette exploration est appelée \emph{depth-first search}. Ces deux approches sont des exemples d'explorations non informées. Elles ne sont, en général, pas optimales car elles retournent le premier état objectif trouvé. De plus, leur complexité en temps est exponentielle. Si \emph{d} est distance minimale entre l'état initial et l'état objectif, \emph{b} est le nombre maximum d'enfants d'un noeud dans l'arbre d'exploration et \emph{m} est la profondeur maximale de l'espace d'état, alors breadth-first search a une complexité en temps en $\mathcal{O}(b^d)$ tandis que le le depth-first search est en $\mathcal{O}(b^m)$.\\


Ces deux approches étant assez naïves et non exploitables en pratique, nous aimerions guider notre recherche afin que celle-ci trouve le plus rapidement possible une solution de bonne qualité. Les stratégies d'exploration informées sont utilisées dans ce but et ce type d'exploration est appélé \emph{best-first search}.

Dans ce genre de stratégie, les noeuds de la frontière sont ordonnés grâce à une \emph{fonction d'évaluation} $f$ qui permet d'estimer à quel point ce noeud est souhaitable. De plus, pour la plupart des algorithmes de type best-first search, une \emph{fonction heuristique} $h$ est utilisée dans l'expression de la fonction $f$. En fait, $h(n)$ estime le plus petit coût nécessaire pour rejoindre l'état objectif le plus proche à partir du noeud $n$.

Terminons en expliquant le cas particulier de l'algorithme $A^*$ ainsi que la manière dont nous l'avons utilisé afin de trouver une solution à notre problème.

\subsubsection{Algorithme $\mathbf{A^*}$}
\label{subsubsection:aStar}

L'algorithme $A^*$ est un exemple d'algorithme best-first search qui est caractérisé par sa fonction d'évaluation. Dans ce cas $f$ est défini de la manière suivante:

$$f(n) = g(n) + h(n)$$
où 
\begin{itemize}
	\item[$\bullet$] $g(n)$ est le coût nécessaire pour aller du noeud initial au noeud $n$.
	\item[$\bullet$] $h(n)$ est l'estimation du coût minimum nécessaire pour aller du noeud $n$ à un état objectif.
	\item[$\bullet$] $f(n)$ est donc l'estimation du coût minimum pour pour aller du noeud initial à un noeud objectif et ce en passant par le noeud $n$.
\end{itemize}

Pour notre part, la fonction que nous désirons minimiser est la suivante:

$$ F(\rho) = \sum_{i \in \visit(\rho)} \varphi_i(\rho) + |\Pi \backslash \visit(\rho)| \cdot p$$

où nous avons:

\begin{itemize}
	\item[$\bullet$] $\rho = v_0 \rho_1 ... \rho_l$ où $v_0$ est le noeud initial du jeu et $l$ est la longueur maximale des chemins à tester.
	\item[$\bullet$] $p$ est le poids maximal que peut atteindre un chemin tel que sa longueur est la longueur maximale à tester.\\
	 Dans le cadre de nos tests, nous fixons cette valeur à :\linebreak $((\Pi + 1) \cdot (|V| + \sum_{e \in E} (w(e) - 1)))\cdot \max_{e \in E} w(e)$.\\ \textbf{Une étude plus approfondie du problème permettrait-elle \linebreak également de trouver une borne plus fine?} \textbf{Améliorerait-elle vraiment les résultats obtenus?}\\
\end{itemize}

Nous utilisons donc pour $g$ et $h$ les fonctions suivantes:

$$ g(n) = \sum_{i \in \visit(h)} \varphi_i(h) + |\Pi\backslash \visit(h)| \cdot \epsilon_k$$



$$h(n) = \sum_{i \in \visit(h)} \min\{ p - \epsilon_k ,\, c_i \}$$


où nous avons:

\begin{itemize}
	\item[$\bullet$] $h$ est le chemin $h = h_0 h_1 ... h_k$ (où $h_0$ = $v_0$ le noeud initial du jeu) stocké dans le noeud $n$.
	\item[$\bullet$] $\displaystyle \epsilon_k = \sum_{j = 0}^{k-1} w(v_j, v_{j+1})$.
	\item[$\bullet$] $c_i$ est le poids du plus court chemin pour rejoindre un objectif du joueur $i$ à partir du sommet $h_k$.
\end{itemize}
$ $\\
\textbf{Nous savons que sous certaines conditions l'algorithme $A^*$ est optimal. Est-il possible dans ce cas d'arriver à une telle conclusion? L'est-il sous certaines conditions? Quelles sont les performances réelles de cette approche? Quelles en sont les limites, les types de graphes de jeu sur lesquels elle est applicable?}\\
	
	
	
Toutes les notions théoriques qui nous sont nécessaires pour l'implémentation d'un procédé visant à rechercher un équilibre de Nash pertinent sont maintenant expliquées. Nous pouvons maintenant expliquer la manière dont nous les avons mises en \oe uvre.











\newpage
%!TEX root=main.tex

\section{Conclusion}
\newpage



\bibliographystyle{acm}
\bibliography{biblio}
\addtocounter{section}{1}
        \addcontentsline{toc}{section}{\protect\numberline{\thesection}Références}
		\addcontentsline{toc}{section}{Annexe A}


%\addcontentsline{toc}{section}{\protect\numberline{\thesection}{biblio}
\newpage
\appendix
%!TEX root=main.tex


\section{Annexe A: Algorithme de Dijkstra}
\label{algo:dijkstra}
Cette section se base sur le livre de référence de Cormen \emph{et al.} ~\cite{Cormen:2009:IA:580470} (pages 658--662).\\

Soit $G = (V,E)$ un graphe orienté et pondéré,à partir d'un sommet $s$ donné (appelé \textit{source}) l'algorithme de Dijkstra permet de calculer les plus courts chemins à partir de $s$ vers les autres sommets du graphe. Cet algorithme s'applique sur des graphes orientés pondérés tels que la fonction de poids $w$ associée au graphe vérifie la propriété suivante: pour tout $(u,v)\in E$ on a que $w(u,v) \geq 0$.\\

%Idée de l'algorithme

\noindent \textbf{Idées de l'algorithme:}\\
\begin{enumerate}
	
	\item[$\bullet$] A tout sommet $s'$ de $V$ on associe une \textit{valeur} $d$ qui représente l'estimation du plus court chemin de $s$ à $s'$. Cette valeur est mise à jour en court d'exécution de l'algorithme afin qu'à la fin de celle-ci $d$ soit exactement le poids du plus court chemin de $s$ à $s'$. On initialise l'algorithme de Dijkstra en mettant la valeur $+\infty$ à tous les sommets et $0$ à $s$. En effet, le plus court chemin pour aller de $s$ à $s$ est de rester en $s$.
	
	\item[$\bullet$] On utilise une \textit{file de priorité} $Q$ (structure de données permettant de stocker des éléments en fonction de la valeur d'une clef) qui permet de stocker les sommets classés par leur valeur $d$. Sur cette file de priorité on peut effectuer les opérations suivantes: insertion d'un élément, extraction d'un élément ayant la clef de la plus petit valeur,test de la présence ou non d'élément dans $Q$, augmentation ou diminution de la valeur de la clef associée à un sommet. A l'initialisation de l'algorithme les sommets présents dans $Q$ sont tous ceux présents dans $V$.
	
	\item[$\bullet$] Afin de pouvoir retrouver un plus court chemin de $s$ à un autre sommet $s'$ chaque sommet stocke le \textit{prédécesseur} qui a permis de constituer ce plus court chemin.
	
	\item[$\bullet$] On maintient $S \subseteq V$ un ensemble de sommets qui vérifient la propriété suivante: pour tout sommet $s'\in V$ le plus court chemin de $s$ à $s'$ a déjà été calculé. A l'initialisation de l'algorithme $V = \emptyset$.
	
	\item[$\bullet$]De manière répétée: \begin{enumerate}
										\item On sélectionne un sommet $u \in V \backslash S$ associé à l'approximation minimum du plus court chemin de $s$ à $u$.
										\item On ajoute $u$ à $S$.
										\item On \textit{relaxe} tous les arcs sortant de $u$.
									\end{enumerate}
	\item[$\bullet$] La \textit{relaxation} des arcs sortant de $u$ consiste à vérifier pour tout $u'$ tq $(u,u')\in E$ qu'il n'existe pas un plus court chemin de $s$ vers $u'$ que celui potentiellement déjà calculé et tel que ce nouveau chemin est de la forme $s ... uu'$. Si on trouve un tel nouveau chemin, alors on procède à la mise à jour du prédécesseur de $u'$ ( qui devient en fait $u$).
\end{enumerate}
$\text{}$\\

%Pseudo-code

\noindent \textbf{Pseudo-code}

Maintenant que les grandes idées de l'algorithme ont été expliquées, retranscrivons le pseudo-code. Pour des questions de complexité, l'ensemble des arcs du graphe sont représentés sous la forme d'une \textit{liste d'adjacence} \footnote{Une liste d'adjacence,Adj, est définit comme tel: à chaque case d'un tableau est associé un sommet de $V$ et à chacun de ces sommets est associé la liste de ses successeurs. $Adj[u]$ permet de récupérer la liste des successeurs du sommet $u$. } et le file de priorité est implémentée par un \textit{tas (heap min)}. L'algorithme ~\ref{algo:dijk} représente le pseudo-code de l'algorithme proprement dit, comme explicité dans le livre de référence de Cormen \emph{et al.} ~\cite{Cormen:2009:IA:580470}. Les algorithmes \ref{algo:initSU},\ref{algo:initTas},\ref{algo:relaxer} décrivent les algorithmes qui sont appelés au sein de l'algorithme ~\ref{algo:dijk}.

\begin{algorithm}
	\caption{\textsc {Dijkstra(G,w,s)}}
	 \label{algo:dijk}
	\begin{algorithmic}[1]
		\REQUIRE $G = (V,E)$ un graphe orienté pondéré où $E$ est représenté par sa liste d'adjacence, $w: E \rightarrow \mathbb{R}^{+}$ une fonction de poids,$s$ le sommet source.
		\ENSURE / \textbf{\textsc{Effet(s) de bord :}} Calcule un plus court chemin de $s$ vers les autres sommets du graphe.
		
		\STATE \textsc{Initialiser-Source-Unique}($G,s$)
		\STATE $S \leftarrow \emptyset$
		\STATE $Q \leftarrow $\textsc{Initialiser-Tas}($G$)
		\WHILE {$Q \neq \emptyset$}
			\STATE $u \leftarrow Q.$\textsc{Extraire-Min}()
			\STATE $S \leftarrow S \cup \{ u \} $
			
			\FORALL{$v \in Adj[u]$}
				\STATE \textsc{Relaxer}$(u,v,w)$
			\ENDFOR
		\ENDWHILE
	
			
\end{algorithmic}
		
\end{algorithm}

%\todo{Regarder comment on implémente un tas, pour l'initialisation}

% ALGO: Initialiser-Source-Unique

\begin{algorithm}
	\caption{\textsc {Initialiser-Source-Unique}($G,s)$}
	 \label{algo:initSU}
	\begin{algorithmic}[1]
		\REQUIRE $G$ un graphe orienté pondéré
		\ENSURE / \textbf{\textsc{Effet(s) de bord :}} initialise les valeurs de tous les sommets.
		
		\FORALL{$v \in G.V$}
			\STATE $v.d \leftarrow +\infty$
			\STATE $v.pred \leftarrow NULL$
		\ENDFOR
		
		\STATE $s.d = 0$
	
			
\end{algorithmic}
		
\end{algorithm}

%ALGO: Initialiser-Tas

\begin{algorithm}
	\caption{\textsc {Initialiser-Tas}$(G)$}
	 \label{algo:initTas}
	\begin{algorithmic}[1]
		\REQUIRE $G$ un graphe orienté pondéré
		\ENSURE Un tas $Q$ qui comprend tous les sommets de $V$ classés par leur valeur $d$.
		
		\STATE $Q \leftarrow$ nouveau tas 
		\FORALL{$v \in G.V$}
			\STATE $Q.$\textsc{Insérer}(v)
		\ENDFOR
		
		\RETURN $Q$
	
			
\end{algorithmic}
		
\end{algorithm}

%ALGO:Relaxer

\begin{algorithm}
	\caption{\textsc {Relaxer}$(u,v,w)$}
	 \label{algo:relaxer}
	\begin{algorithmic}[1]
		\REQUIRE deux sommets $u$ et $v$, une fonction de poids $w : E \rightarrow \mathbb{R}^{+}$.
		\ENSURE / \textbf{\textsc{Effet(s) de bord :}} Met potentiellement à jour la  valeur de $v$ et son prédécesseur.
		
		\STATE nouvVal $\leftarrow$ $w(u,v) + u.d$
		\IF{nouvVal $< v.d$}
			\STATE $Q.$\textsc{DécrémenterClef}($Clef(v),nouvVal$)
			\STATE $v.p \leftarrow u$
		\ENDIF
	
			
\end{algorithmic}
		
\end{algorithm}

La complexité de l'algorithme de Dijkstra dépend de la manière dont est implémentée la file de priorité. Si on implémente celle-ci de telle sorte que chaque opération \textsc{Extraire-Min}() est en $\mathcal{O}(\log V)$ ainsi que chaque \textsc{DécrémenterClef}($,Clef(v),nouvVal$) alors l'agorithme est en $\mathcal{O}((V + E) \log V)$.\\
Les preuves d'exactitude et de complexité de l'agorithme Dijkstra ne sont pas abordées ici mais se trouve dans le livre de référence de Cormen \emph{et al.} ~\cite{Cormen:2009:IA:580470}.



\clearpage





\end{document}
%%% Local Variables: 
%%% mode: latex
%%% TeX-master: t
%%% TeX-PDF-mode: t
%%% End: 
