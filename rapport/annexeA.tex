%!TEX root=main.tex

\section*{Annexe A: Algorithme de Dijkstra}
\label{algo:dijkstra}
Cette section se base sur le livre de Cormen \emph{et al.}~\cite{Cormen:2009:IA:580470} (pages 658--662).\\

Soit $G = (V,E)$ un graphe orienté et pondéré, à partir d'un sommet $s$ donné (appelé \textit{source}) l'algorithme de Dijkstra permet de calculer les plus courts chemins à partir de $s$ vers les autres sommets du graphe. Cet algorithme s'applique sur des graphes orientés pondérés tels que la fonction de poids $w$ associée au graphe vérifie la propriété suivante: pour tout $(u,v)\in E$ on a que $w(u,v) \geq 0$.\\

%Idée de l'algorithme

\noindent \textbf{Idées de l'algorithme:}\\
\begin{enumerate}
	
	\item[$\bullet$] A tout sommet $s'$ de $V$ on associe une \textit{valeur} $d$ qui représente l'estimation du plus court chemin de $s$ à $s'$. Cette valeur est mise à jour en court d'exécution de l'algorithme afin qu'à la fin de celle-ci $d$ soit exactement le poids du plus court chemin de $s$ à $s'$. On initialise l'algorithme de Dijkstra en mettant la valeur $+\infty$ à tous les sommets et $0$ à $s$. En effet, le plus court chemin pour aller de $s$ à $s$ est de rester en $s$.
	
	\item[$\bullet$] On utilise une \textit{file de priorité} $Q$ (structure de données permettant de stocker des éléments en fonction de la valeur d'une clef) qui permet de stocker les sommets classés par leur valeur $d$. Sur cette file de priorité, on peut effectuer les opérations suivantes: insertion d'un élément, extraction d'un élément ayant la clef de la plus petite valeur, test de la présence ou non d'élément dans $Q$, augmentation ou diminution de la valeur de la clef associée à un sommet. A l'initialisation de l'algorithme les sommets présents dans $Q$ sont tous ceux présents dans $V$.
	
	\item[$\bullet$] Afin de pouvoir retrouver un plus court chemin de $s$ à un autre sommet $s'$, chaque sommet stocke le \textit{prédécesseur} qui a permis de constituer ce plus court chemin.
	
	\item[$\bullet$] On maintient $S \subseteq V$ un ensemble de sommets qui vérifient la propriété suivante: pour tout sommet $s'\in V$ le plus court chemin de $s$ à $s'$ a déjà été calculé. A l'initialisation de l'algorithme $V = \emptyset$.
	
	\item[$\bullet$]De manière répétée: \begin{enumerate}
										\item On sélectionne un sommet $u \in V \backslash S$ associé à l'approximation minimum du plus court chemin de $s$ à $u$.
										\item On ajoute $u$ à $S$.
										\item On \textit{relaxe} tous les arcs sortant de $u$.
									\end{enumerate}
	\item[$\bullet$] La \textit{relaxation} des arcs sortant de $u$ consiste à vérifier pour tout $u'$ tq $(u,u')\in E$ qu'il n'existe pas un plus court chemin de $s$ vers $u'$ que celui potentiellement déjà calculé et tel que ce nouveau chemin est de la forme $s ... uu'$. Si on trouve un tel nouveau chemin, alors on procède à la mise à jour du prédécesseur de $u'$ ( qui devient en fait $u$).
\end{enumerate}
$\text{}$\\

%Pseudo-code

\noindent \textbf{Pseudo-code}

Maintenant que les grandes idées de l'algorithme ont été expliquées, retranscrivons le pseudo-code. Pour des questions de complexité, l'ensemble des arcs du graphe sont représentés sous la forme d'une \textit{liste d'adjacence} \footnote{Une liste d'adjacence,Adj, est définie comme telle: à chaque case d'un tableau est associé un sommet de $V$ et à chacun de ces sommets est associée la liste de ses successeurs. $Adj[u]$ permet de récupérer la liste des successeurs du sommet $u$. } et la file de priorité est implémentée par un \textit{tas (heap min)}. L'algorithme ~\ref{algo:dijk} représente le pseudo-code de l'algorithme proprement dit, comme explicité dans le livre de Cormen \emph{et al.} ~\cite{Cormen:2009:IA:580470}. Les algorithmes \ref{algo:initSU},\ref{algo:initTas},\ref{algo:relaxer} décrivent les algorithmes qui sont appelés au sein de l'algorithme ~\ref{algo:dijk}.

\begin{algorithm}
	\caption{\textsc {Dijkstra(G,w,s)}}
	 \label{algo:dijk}
	\begin{algorithmic}[1]
		\REQUIRE $G = (V,E)$ un graphe orienté pondéré où $E$ est représenté par sa liste d'adjacence, $w: E \rightarrow \mathbb{R}^{+}$ une fonction de poids,$s$ le sommet source.
		\ENSURE / \textbf{\textsc{Effet(s) de bord :}} Calcule un plus court chemin de $s$ vers les autres sommets du graphe.
		
		\STATE \textsc{Initialiser-Source-Unique}($G,s$)
		\STATE $S \leftarrow \emptyset$
		\STATE $Q \leftarrow $\textsc{Initialiser-Tas}($G$)
		\WHILE {$Q \neq \emptyset$}
			\STATE $u \leftarrow Q.$\textsc{Extraire-Min}()
			\STATE $S \leftarrow S \cup \{ u \} $
			
			\FORALL{$v \in Adj[u]$}
				\STATE \textsc{Relaxer}$(u,v,w)$
			\ENDFOR
		\ENDWHILE
	
			
\end{algorithmic}
		
\end{algorithm}

% ALGO: Initialiser-Source-Unique

\begin{algorithm}
	\caption{\textsc {Initialiser-Source-Unique}($G,s)$}
	 \label{algo:initSU}
	\begin{algorithmic}[1]
		\REQUIRE $G$ un graphe orienté pondéré
		\ENSURE / \textbf{\textsc{Effet(s) de bord :}} initialise les valeurs de tous les sommets.
		
		\FORALL{$v \in G.V$}
			\STATE $v.d \leftarrow +\infty$
			\STATE $v.pred \leftarrow NULL$
		\ENDFOR
		
		\STATE $s.d = 0$
	
			
\end{algorithmic}
		
\end{algorithm}

%ALGO: Initialiser-Tas

\begin{algorithm}
	\caption{\textsc {Initialiser-Tas}$(G)$}
	 \label{algo:initTas}
	\begin{algorithmic}[1]
		\REQUIRE $G$ un graphe orienté pondéré
		\ENSURE Un tas $Q$ qui comprend tous les sommets de $V$ classés par leur valeur $d$.
		
		\STATE $Q \leftarrow$ nouveau tas 
		\FORALL{$v \in G.V$}
			\STATE $Q.$\textsc{Insérer}(v)
		\ENDFOR
		
		\RETURN $Q$
	
			
\end{algorithmic}
		
\end{algorithm}

%ALGO:Relaxer

\begin{algorithm}
	\caption{\textsc {Relaxer}$(u,v,w)$}
	 \label{algo:relaxer}
	\begin{algorithmic}[1]
		\REQUIRE deux sommets $u$ et $v$, une fonction de poids $w : E \rightarrow \mathbb{R}^{+}$.
		\ENSURE / \textbf{\textsc{Effet(s) de bord :}} Met potentiellement à jour la  valeur de $v$ et son prédécesseur.
		
		\STATE nouvVal $\leftarrow$ $w(u,v) + u.d$
		\IF{nouvVal $< v.d$}
			\STATE $Q.$\textsc{DécrémenterClef}($Clef(v),nouvVal$)
			\STATE $v.p \leftarrow u$
		\ENDIF
	
			
\end{algorithmic}
		
\end{algorithm}

La complexité de l'algorithme de Dijkstra dépend de la manière dont est implémentée la file de priorité. Si on implémente celle-ci de telle sorte que chaque opération \textsc{Extraire-Min}() est en $\mathcal{O}(\log V)$ ainsi que chaque \textsc{DécrémenterClef}($Clef(v),nouvVal$) alors l'agorithme est en \mbox{$\mathcal{O}((V + E) \log V)$}.\\
Les preuves d'exactitude et de complexité de l'agorithme Dijkstra ne sont pas abordées ici mais se trouvent dans le livre de Cormen \emph{et al.}~\cite{Cormen:2009:IA:580470}.


\clearpage

