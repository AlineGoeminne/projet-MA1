%!TEX root=main.tex


\subsubsection{Jeux multijoueurs avec coût}
% DEFINITION: mutliplayer cost game

\begin{defi}[Jeu multijoueur avec coût]
	Soit $\mathcal{A} = (\Pi, V, (V_{i})_{i \in \Pi},E)$ une arène,
	un \textit{jeu multijoueur avec coût} est un tuple $\mathcal{G} = (\mathcal{A},(Cost_{i})_{i \in \Pi})$ où
	\begin{enumerate}
		\item[$\bullet$] $\mathcal{A} = (\Pi ,V ,(V_{i})_{i \in \Pi} ,E )$ est l'arène d'un jeu sur graphe.
		\item[$\bullet$] $Cost_{i}: Plays \rightarrow \mathbb{R} \cup \{ +\infty , -\infty \} $ est la \textit{fonction de coût} de $J_{i}$. 
	\end{enumerate}
\end{defi}


	Pour chaque $\rho \in Plays$, $Cost_{i}(\rho)$ représente le montant que $J_{i}$ perd quand le jeu $\rho$ est joué.
	Le but de chaque joueur est donc de \textbf{minimiser} sa fonction de coût.

%EXEMPLE: fonctions de coût.
\begin{exemple}[Fonctions de coût]
	\label{ex:fonctionsCout}
  Dans le cadre de ce projet, nous nous intéressons aux jeux sur graphe tels que l'objectif des joueurs est un objectif quantitatif. De plus, nous souhaitons que l'objectif des joueurs soit atteint le plus rapidement possible. Les fonctions de coût qui nous intéressent sont donc les suivantes: \\
	
	Pour tout  $\rho = \rho _{0} \rho _{1} \rho _{2} \ldots $ où $\rho \in Plays$ on définit:
	\begin{enumerate}
	\item $Cost_{i}(\rho) = $ $\begin{cases} 
									\min \{ i | \rho _{i} \in Goal_{i} \} & \text{si } \exists i \text{ tq } \rho _{i} \in Goal_{i} \\
									+\infty & \text{ sinon}
									\end{cases}$
	\item $\varphi _{i}(\rho) = $ $\begin{cases}
									\sum_{i = 0}^{n-1} w(\rho_{i},\rho_{i+1}) & \text{ si } n \text{ est le plus petit indice tq } \rho_{n}\in 					  Goal_{i}\\
									+\infty & \text{sinon}
									\end{cases}$ \\
									où $w$ est une \textit{fonction de poids} (cf. définition \ref{def:fonctionPoids}).
	\end{enumerate}
\end{exemple}

\begin{rem}
	L'exemple 1 est un cas particulier de l'exemple 2 avec $w(\rho_{i},\rho_{i+1}) = 1$ pour tout $i$.
\end{rem}
%DEFINITION: poids d'un arc

\begin{defi}[Fonction de poids]
	\label{def:fonctionPoids}
	A chaque arc d'un graphe $G = (V,E)$ on peut y associer un \textit{poids} (\emph{i.e.,} une valeur chiffrée). On associe donc à $G$ une \textit{fonction de poids}  $w : E \rightarrow \mathbb{R}$. On dit alors que $G$ est un graphe \textit{pondéré}.
\end{defi}
\begin{rem}
	Même si par définition, considérer une fonction de poids à valeurs réelles est correct, dans notre cas nous ne nous intéresserons qu'à des fonctions de poids à valeurs dans $\mathbb{N}_0$.
\end{rem}

\begin{exemple}
	Si chaque $v \in V$ représente une ville sur une carte, alors on peut imaginer une fonction de poids représentant une des valeurs suivantes:
	\begin{enumerate}
		\item [$\bullet$] le nombre de kilomètres entre deux villes,
		\item [$\bullet$] le temps pour aller d'une ville à l'autre,
		\item [$\bullet$] la consommation d'essence pour aller d'une ville à l'autre.
	\end{enumerate}
\end{exemple}


%DEFINITION: jeu d'atteignabilité multijoueur à objectif quantitatif

\begin{defi}[Jeu d'atteignabilité multijoueur à objectif quantitatif]
	
	Un \textit{jeu d'atteignabilité multijoueur à objectif quantitatif} est un jeu multijoueur avec coût $\mathcal{G} = (\Pi ,V ,(V_{i})_{i \in \Pi} ,E ,(Cost_{i})_{i \in \Pi})$ tel que pour tout joueur $i \in \Pi$ $Cost_{i} = \varphi _{i}$ pour un certain $Goal _{i} \subseteq V$.
	On note ces jeux $\mathcal{G} = (\mathcal{A},(\varphi _{i})_{i\in \Pi},(Goal_{i})_{i \in \Pi})$.
\end{defi}
	



%EXEMPLE: graphe pondéré



%!TEX root=main.tex

\begin{exemple}[Jeu avec un graphe pondéré]
	\label{ex:graphePond}
	
Soit un jeu $\mathcal{G} = ( \Pi, V, (V_{1},V_{2}), E, (Goal_{1},Goal_{2}))$ tel que $\Pi = {1,2}$, $V_{1} = \{ v_{0}, v_{1}, v_{3} \}$, $V_{2} = \{ v_{2}\}$, $Goal_{1} = \{ v_{3}\}$, $Goal_{2} = \{ v_{0} \}$, $w(v_{i},v_{j}) = 1$ pour tout $0 \leq i,j \leq 3 , i \neq j$. Cet exemple est illustré par la figure \ref{ex:graphePond1}.

\begin{figure}[ht!]
	\centering

	\begin{tikzpicture}
		
		\node[nRB] (v3) at (4,-2){$v_{3}$};
		\node[nC] (v2) at (4,0){$v_{2}$};
		\node[nR] (v1) at (2,0){$v_{1}$};
		\node[nRG] (v0) at (0,0){$v_{0}$};
	
		\draw[->,>=latex] (v0.north) to [out=95,in= 80] node[midway,above]{$1$}(v1.north);
		\draw[->,>=latex] (v1) to node[midway,above]{$1$} (v2);
		\draw[->,>=latex] (v1) to node[midway,above]{$1$} (v0);
		
		\draw[->,>=latex] (v2) to node[midway,right]{$1$} (v3);
		\draw[->,>=latex] (v3) to node[midway,above]{$1$} (v0);
		
	\end{tikzpicture}
	
	
	\caption{Jeu avec un graphe pondéré }
	\label{ex:graphePond1}
	

\end{figure}
\end{exemple}

\FloatBarrier


%DEFINITION: équilibre de Nash

\begin{defi}[Equilibre de Nash]
	
	Soit $(\mathcal{G}, v_{0})$ un \textit{jeu multijoueur avec coût et initialisé}, un profil de stratégie $(\sigma _{i})_{i \in \Pi}$ est un \textit{équilibre de Nash} dans $(\mathcal{G}, v_{0})$ si, pour chaque joueur $j \in \Pi$ et pour chaque stratégie $\tilde{\sigma}_{j}$ du joueur $j$, on a :
	\begin{center}$ Cost_{j}(\rho) \leq Cost_{j}(\tilde{\rho})$ \end{center}
	où $\rho = \langle (\sigma _{i})_{i \in \Pi}\rangle_{v_0}$ et $\tilde{\rho} = \langle \tilde{\sigma} _{j} ,\sigma _{-j}\rangle_{v_0}$.
\end{defi}	


%DEFINITION: déviation profitable

\begin{defi}[Déviation profitable]
	
	Soit $(\mathcal{G}, v_{0})$ un \textit{jeu multijoueur avec coût et initialisé}, soit $(\sigma _{i})_{i \in \Pi}$ un profil de stratégie, $\tilde{\sigma _{j}}$ est une \textit{déviation profitable} pour le joueur $j$ relativement à $(\sigma _{i})_{i \in \Pi}$ si:
	\begin{center} $ Cost_{j}(\rho) > Cost_{j}(\tilde{\rho})$ \end{center}
	où $\rho = \langle (\sigma _{i})_{i \in \Pi} \rangle_{v_0}$ et $\tilde{\rho} = \langle \tilde{\sigma} _{j} ,\sigma _{-j} \rangle_{v_0}$. 
\end{defi}

%REMARQUE: signification de la def d'EN
\begin{rem}
	$(\sigma _{i})_{i\in \Pi}$ est un équilibre de Nash si tout joueur $j \in \Pi$ n'a aucun intérêt à dévier de sa stratégie $\sigma _{j}$ si les autres joueurs suivent $\sigma _{-j}$. C'est-à-dire qu'aucun joueur n'a de déviation profitable. 
\end{rem}

%EXEMPLE D'EN

%!TEX root=main.tex

\begin{exemple}
	Considérons le jeu décrit dans l'exemple \ref{ex:graphePond} muni pour $J_{1}$ et $J_{2}$ de la fonction de coût $\varphi _{i}$ de l'exemple \ref{ex:fonctionsCout} (p.\pageref{ex:fonctionsCout}) et donnons un exemple d'équilibre de Nash de ce jeu.\\
	
	 \begin{minipage}[c]{0.30\linewidth}Soit $\sigma _{1}(v) =$ $\begin{cases}
						 v_{1} & \text{ si } v = v_{0}\\
						 v_{2} & \text{ si } v = v_{1}\\
						 v_{0} & \text{ si } v = v_{3}
						\end{cases}$ \end{minipage} \hfill
	\begin{minipage}[c]{0.30\linewidth}\center{et}\end{minipage} \hfill \begin{minipage}[c]{0.30\linewidth}	\center{$\sigma _{2}(v) = v_{3}$ si $v = v_{2}$} \end{minipage} \newline
		
\noindent	(rem: ces deux stratégies sont des stratégies sans mémoire), $(\sigma _{1}, \sigma _{2})$ est un équilibre de Nash de $(\mathcal{G}, v_{0})$. De plus, soit $\rho = Outcome(v_{0},(\sigma _{1},\sigma _{2}))$ nous avons : $Cost_{1}(\rho) = 3$ et $Cost_{2}(\rho) = 0$.\\
\begin{demonstration}	
	Les seules déviations à considérer sont celles quand $J_{1}$ est en $v_{1}$ car il a le choix de se rendre en $v_{0}$ ou en $v_{2}$.\\ Considérons $\tilde{\sigma _{1}}(v) = $ $\begin{cases}
						 v_{1} & \text{ si } v = v_{0}\\
						 v_{0} & \text{ si } v = v_{1}\\
						 v_{0} & \text{ si } v = v_{3}
						\end{cases}$ .
Dans ce cas, soit $\tilde{\rho} = Outcome(v_{0},(\tilde{\sigma _{1}},\sigma _{2}))$, nous avons : \mbox{$Cost_{1}(\tilde{\rho}) = + \infty$}. Ce n'est donc pas une déviation profitable.\\
Considérons maintenant la famille $(\sigma _{1}^{n})_{n \in \mathbb{N}}$ de stratégie de $J_{1}$ qui décrit le fait que $J_{1}$ fait passer $n$ fois le "jeton" par l'état $v_{1}$ avant de le faire glisser vers l'état $v_{2}$. \\

$\sigma _{1}^{n}(h) = $ $\begin{cases}
					 v_{1} & \text{ si } h = h'v_{0}\\
					 v_{0} & \text{ si } h = h'v_{3}\\
					 v_{2} & \text{ si } h = h'v_{1} \text{ et } \exists h_{1},\ldots,h_{n} \in h' \text{ tq } h_{1},\ldots,h_{n} = v_{1} \\
					v_{0} & \text{ sinon}
					
					\end{cases}$ .
					
\noindent Nous avons alors: soit $ p = Outcome(v_{0},(\sigma _{1}^{n},\sigma _{2}))$, $Cost_{1}(p) = 2n+1$. Et pour tout $n > 1$, on a que $3=Cost_{1}(\rho) < Cost_{1}(p) = 2n +1$. $\sigma _{1}^{n}$ n'est donc pas une déviation profitable pour $J_{1}$ et ce pour tout $n \in  \mathbb{N}$.\\

Comme $\sigma _{2}$ est la seule stratégie possible pour $J_{2}$, nous pouvons déduire qu'aucun des deux joueurs n'a de déviation profitable. Donc $(\sigma _{1}, \sigma _{2})$ est un équilibre de Nash de $(\mathcal{G},v_{0})$.

\end{demonstration}			 		
    
\end{exemple}



% Résultat d'existence d'un équilibre de Nash à petite mémoire

 Dans~\cite{DBLP:conf/lfcs/BrihayePS13}, le théorème suivant est énoncé et prouvé:

\begin{thm}
	Soient $\mathcal{A} = (\Pi, V, (V_{i})_{i \in \Pi}, E)$ et $\mathcal{G} = (\mathcal{A},(\varphi _{i})_{i \in \Pi}, (Goal_{i})_{i \in \Pi})$ un jeu d'atteignabilité multijoueur à objectif quantitatif, si la fonction de poids associée au graphe du jeu est une fonction positive alors il existe un équilibre de Nash dans tout jeu initialisé ($\mathcal{G},v_{0}$) avec $v_{0}\in V$. De plus, cet équilibre possède une mémoire d'au plus $|V| + |\Pi|$.
\end{thm}

