%!TEX root=main.tex

\begin{exemple}
	Considérons le jeu décrit dans l'exemple \ref{ex:graphePond} et donnons un exemple d'équilibre de Nash de ce jeu.\\
	
	 \begin{minipage}[c]{0.4\linewidth}Soit $\sigma _{1}(v) =\begin{cases}
						 v_{1} & \text{ si } v = v_{0}\\
						 v_{2} & \text{ si } v = v_{1}\\
						 v_{0} & \text{ si } v = v_{3}\\
						v_{3} & \text{ si } v = v_4
						\end{cases}$ \end{minipage} \hfill
	\begin{minipage}[c]{0.1\linewidth}\center{et}\end{minipage} \hfill \begin{minipage}[c]{0.30\linewidth}	\center{$\sigma _{2}(v) = v_{3}$ si $v = v_{2}$} \end{minipage} \newline
		
\noindent	(rem: ces deux stratégies sont des stratégies sans mémoire), $(\sigma _{1}, \sigma _{2})$ est un équilibre de Nash de $(\mathcal{G}, v_{0})$. De plus, soit $\rho = \langle \sigma _{1},\sigma _{2} \rangle_{v_0}$ nous avons : $\varphi_{1}(\rho) = 3$ et $\varphi_{2}(\rho) = 0$.\\
	Montrons que $J_1$ ne possède pas de déviation profitable. Nous devrions pour cela le montrer pour toutes les stratégies possibles pour $J_1$. Nous nous contentons de le montrer pour les stratégies sans mémoire. \\
	 Considérons $\tilde{\sigma _{1}}(v) = $ $\begin{cases}
						 v_{1} & \text{ si } v = v_{0}\\
						 v_{0} & \text{ si } v = v_{1}\\
						 v_{0} & \text{ si } v = v_{3}\\
						v_{3} & \text{ si } v = v_4						
						\end{cases}$ .\\
Dans ce cas, nous avons : \mbox{$\varphi_{1}(\langle \tilde{\sigma _{1}},\sigma _{2} \rangle_{v_0}) = + \infty$}. C'est en fait le cas pour toutes les déviations $\tilde{\sigma_1}$ telles que $\tilde{\sigma_1}(v_1) = v_0$.  Ce ne sont donc pas des déviations profitables.\\
Considérons maintenant $\tilde{\sigma _{1}}(v) = $ $\begin{cases}
					 v_{1} & \text{ si } v = v_{0}\\
					 v_{2} & \text{ si } v = v_{1}\\
					 v_{0} & \text{ si } v = v_{3}\\
					v_{2} & \text{ si } v = v_4						
					\end{cases}$.\\
On a alors \mbox{$\varphi_{1}(\langle \tilde{\sigma _{1}},\sigma _{2} \rangle_{v_0}) = 3 = \varphi_1(\langle \sigma _{1},\sigma _{2}\rangle_{v_0})$}. Ce n'est donc pas non plus une déviation profitable.
Les déviations ci-dessous ont également un coût de 3 et ne sont pas des déviations profitables.
\begin{figure}[!h]
   \begin{minipage}[c]{.46\linewidth}
	$\tilde{\sigma _{1}}(v) = $ $\begin{cases}
						 v_{1} & \text{ si } v = v_{0}\\
						 v_{2} & \text{ si } v = v_{1}\\
						 v_{4} & \text{ si } v = v_{3}\\
						v_{3} & \text{ si } v = v_4						
						\end{cases}$.\\   \end{minipage} \hfill
   \begin{minipage}[c]{.46\linewidth}
	$\tilde{\sigma _{1}}(v) = $ $\begin{cases}
						 v_{1} & \text{ si } v = v_{0}\\
						 v_{2} & \text{ si } v = v_{1}\\
						 v_{4} & \text{ si } v = v_{3}\\
						v_{2} & \text{ si } v = v_4						
						\end{cases}$.\\   \end{minipage}
\end{figure}

Nous avons ainsi testé toutes les déviations sans mémoire pour le $J_1$.

De plus, comme le jeu est initialisé en $v_0$ et qu'il s'agit de l'objectif du second joueur, celui-ci a un coût de 0 quelle que soit la stratégie qu'il adopte. $J_2$ n'a donc pas non plus de déviation profitable. Nous pouvons dès lors conclure que $(\sigma_1, \sigma_2)$ est un équilibre Nash.
\begin{comment}
Considérons maintenant la famille $(\sigma _{1}^{n})_{n \in \mathbb{N}}$ de stratégie de $J_{1}$ qui décrit le fait que $J_{1}$ fait passer $n$ fois le \og jeton \fg~par l'état $v_{1}$ avant de le faire glisser vers l'état $v_{2}$. \\

$\sigma _{1}^{n}(h) = $ $\begin{cases}
					 v_{1} & \text{ si } h = h'v_{0}\\
					 v_{0} & \text{ si } h = h'v_{3}\\
					 v_{2} & \text{ si } h = h'v_{1} \text{ et } \exists h_{1},\ldots,h_{n} \in h' \text{ tq } h_{1},\ldots,h_{n} = v_{1} \\
					v_{0} & \text{ sinon}
					
					\end{cases}$ .
					
\noindent Nous avons alors: soit $ p = \langle \sigma _{1}^{n},\sigma _{2} \rangle_{v_0}$, $Cost_{1}(p) = 2n+1$. Et pour tout $n > 1$, on a que: \\\mbox{$3=Cost_{1}(\rho)~< Cost_{1}(p)~= 2n +1$. $\sigma _{1}^{n}$} n'est donc pas une déviation profitable pour $J_{1}$ et ce pour tout $n \in  \mathbb{N}$.\\

Comme $\sigma _{2}$ est la seule stratégie possible pour $J_{2}$, nous pouvons déduire qu'aucun des deux joueurs n'a de déviation profitable. Donc $(\sigma _{1}, \sigma _{2})$ est un équilibre de Nash de $(\mathcal{G},v_{0})$.
\end{comment}
\end{exemple}


