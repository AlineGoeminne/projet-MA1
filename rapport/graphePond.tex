%!TEX root=main.tex

\begin{exemple}[Jeu d'atteignabilité multijoueur à objectif quantitatif]
	\label{ex:graphePond}
	
$ $\\
Soit un jeu $\mathcal{G} = ( \Pi, V, (V_{1},V_{2}), E,(\varphi _{1}, \varphi _{2}), (Goal_{1},Goal_{2}))$ tel que $\Pi = \{ 1,2 \} $, $V_{1} = \{ v_{0}, v_{1}, v_{3}, v_{4} \}$, $V_{2} = \{ v_{2}\}$, $Goal_{1} = \{ v_{3}\}$, $Goal_{2} = \{ v_{0} \}$. Cet exemple est illustré par la figure \ref{ex:graphePond1}.

\begin{figure}[ht!]
	\centering

	\begin{tikzpicture}
		
		\node[nRD] (v3) at (4,-2){$v_{3}$};
		\node[nC] (v2) at (4,0){$v_{2}$};
		\node[nR] (v1) at (2,0){$v_{1}$};
		\node[nRG] (v0) at (0,0){$v_{0}$};
		\node[nR] (v4) at (6,0){$v_{4}$};
		
	
		\draw[->,>=latex] (v0.north) to [out=95,in= 80] node[midway,above]{$1$}(v1.north);
		\draw[->,>=latex] (v1) to node[midway,above]{$1$} (v2);
		\draw[->,>=latex] (v1) to node[midway,above]{$1$} (v0);
		\draw[->,>=latex] (v2) to node[midway,above]{$4$} (v4);
		
		\draw[->,>=latex] (v2) to node[midway,right]{$1$} (v3);
		\draw[->,>=latex] (v4) to node[midway,right]{$1$} (v3);
		
		\draw[->,>=latex] (v3) to node[midway,above]{$1$} (v0);
		
		\draw[->,>=latex] (v3.east) to [out=0,in= 0] node[midway,right]{$1$} (v4.east);
		\draw[->,>=latex] (v4.north) to [out=95,in= 80] node[midway,above]{$2$} (v2.north);
		
	\end{tikzpicture}
		
	\caption{Jeu d'atteignabilité multijoueur à objectif quantitatif}
	\label{ex:graphePond1}
	

\end{figure}
\end{exemple}

\FloatBarrier