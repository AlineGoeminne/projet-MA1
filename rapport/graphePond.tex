%!TEX root=main.tex
\begin{exemple}
	Exemple de jeu dont le graphe est pondéré et tel que $V_{1} = \{ v_{0}, v_{1}, v_{3} \}$, $V_{2} = \{ v_{2}\}$, $Goal_{1} = \{ v_{2}\}$,$Goal_{2} = \{ v_{0} \}$, $w(v_{i},v_{j}) = 1$ pour tout $0 \leq i,j \leq 3$ :

\begin{figure}[ht!]
	\centering

	\begin{tikzpicture}
		
		\node[nR] (v3) at (4,-2){$v_{3}$};
		\node[nCD] (v2) at (4,0){$v_{2}$};
		\node[nR] (v1) at (2,0){$v_{1}$};
		\node[nRG] (v0) at (0,0){$v_{0}$};
	
		\draw[->,>=latex] (v0.north) to [out=95,in= 80](v1.north);
		\draw[->,>=latex] (v1) to (v2);
		\draw[->,>=latex] (v1) to (v0);
		
		\draw[->,>=latex] (v2) to (v3);
		\draw[->,>=latex] (v3) to (v0);
		
		
		
		
		
	\end{tikzpicture}
	
	
	\caption{Jeu avec arcs pondérés }

\end{figure}
\end{exemple}